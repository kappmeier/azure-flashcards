\subsectioncard{Implementieren von sicheren Lösungen}

\subsubsectioncard{encrypt and decrypt data at rest and in transit}

\begin{flashcard}[Definition]{Datenverschlüsselung}
    \vspace*{\stretch{1}}
    Ruhende Daten:
    \begin{itemize}
        \item symmetrische Verschlüsselung von gespeicherten Daten (Storage Account, \ldots)
        \item gestohlene Daten sind nutzlos
        \item Schlüssel wird in Key Vault gespeichert
        \item Azure-Verwaltete Schlüssel, oder Kundenschlüssel importieren
        \item Aktivieren nach Modell:
            \begin{itemize}
                \item SaaS: aktivieren in der Software
                \item Paas: Speicher (BLob, \ldots) kann verschlüsselt werden
                \item IaaS: Azure Disk Encryption für VHDs und VMs, und Speicher\newline
                    Standardmäßig verschlüsselt
            \end{itemize}
    \end{itemize}
    \vspace*{\stretch{1}}
\end{flashcard}

\begin{flashcard}[Definition]{Verschlüsselung Speicherkonto}
    \vspace*{\stretch{1}}
    \begin{itemize}
        \item Daten im Speicherkonto sind standardmäßig verschlüsselt
        \item Microsoft verwaltet Schlüssel standardmäßig (in Microsoft Key Store)
        \item Benutzer-Verwaltet: benötigt Azure Key Vault.\newline
            Nur Blob und Azure Files
        \item vom Benutzer bereitgestellt: eigener Key Store des nutzers\newline
            Nur Blob!
    \end{itemize}
    \vspace*{\stretch{1}}
\end{flashcard}

\begin{flashcard}[Definition]{Verschlüsselung in Transit}
    \vspace*{\stretch{1}}
    \begin{itemize}
        \item Zugriff auf Endpunkte nur verschlüsselt
        \item HTTPS-Protokoll
        \item aktivieren in Eigenschaften von Speicherkonto
    \end{itemize}
    \vspace*{\stretch{1}}
\end{flashcard}

\begin{flashcard}[Definition]{Transparent Data Encryption}
    \vspace*{\stretch{1}}
    \begin{itemize}
        \item Data at rest-Verschlüsselung von SQL-Datenbanken
        \item schützt vor Offline-Angriffen
        \item verschlüsselt die Datenbank, Backups, und Transaction logs
        \item[$\Rightarrow$] transparent für Anwendungen
        \item standardmäßig aktiviert
    \end{itemize}
    \vspace{1cm}
    Im Unterschied zu Always Encrypted kennt die Datenbank (bzw. der Azure Service) den Schlüssel (DEK)
    \vspace*{\stretch{1}}
\end{flashcard}

\subsubsectioncard{encrypt data with Always Encoded}

\begin{flashcard}[Definition]{Always Encrypted}
    \vspace*{\stretch{1}}
    \begin{itemize}
        \item Verschlüsselung in Transit
        \item Verschlüsselung von Daten in SQL-Datenbank
        \item Entschlüsselung der Daten erst in Client-Anwendung
        \item Durchführung:
            \begin{itemize}
             \item PowerShell/T-SQL (nicht vollständig unterstützt)
            \end{itemize}
    \end{itemize}
    \vspace*{\stretch{1}}
\end{flashcard}

\begin{flashcard}[Definition]{Always Encrypted und Windows Certificate Store}
    \vspace*{\stretch{1}}
    Speichert den Schlüssel auf der lokalen Maschine

    \vspace{1cm}
    $\Rightarrow$ nur in der Anwendung sind Daten verfügbar
    \vspace*{\stretch{1}}
\end{flashcard}

\subsubsectioncard{implement Azure Confidential Compute}

\begin{flashcard}[Definition]{Azure Confidential Computing}
    \vspace*{\stretch{1}}
    \begin{itemize}
        \item geschützte Verarbeitung von Daten während der Berechnung\newline
            (während Daten normalerweise nach Lesen vom Ruhezustand entschlüsselt sind)
        \item benötigt Trusted Execution Environment (TEE)
        \item Code wird in geschütztem Container ausgeführt
        \item benötigt eine VM mit Intel XGX
    \end{itemize}
    \vspace*{\stretch{1}}
\end{flashcard}

\subsubsectioncard{implement SSL/TLS communications}

\begin{flashcard}[Definition]{SSL/TLS-Verbindungen}
    \vspace*{\stretch{1}}
    \begin{itemize}
        \item Speicherkonto: in Einstellungen aktivieren
        \item Andere Endpunkte: aktivieren
        \item[$\Rightarrow$] muss erzwungen werden
        \item VMs: können Schlüsselpaar erstellen bei der Erstellung, oder eigenen Schlüssel
    \end{itemize}
    \vspace*{\stretch{1}}
\end{flashcard}

\begin{flashcard}[Definition]{Gegenseitige TLS-Authentifizierung}
    \vspace*{\stretch{1}}
    \begin{itemize}
        \item benötigt mindestens Basic-Tarif
        \item Web App vorbereiten:
            \begin{itemize}
                \item Im Portal:\newline
                    \texttt{Eingehende Zertifikate anfordern}
                \item CLI:\newline
                    \texttt{az webapp update --set clientCertEnabled=true --name \ldots\\--resource-group \ldots}
            \end{itemize}
        \item Abrufen des Zertifikats
            \begin{itemize}
                \item ASP.NET: als Eigenschaft \texttt{HttpRequest.ClientCertificate}
                \item Andere Apps: HTML-Header \texttt{X-ARR-ClientCert}, base64 kodiert
            \end{itemize}
    \end{itemize}
    \vspace*{\stretch{1}}
\end{flashcard}

\subsubsectioncard{create, read, update, and delete keys, secrets, and certificates by using the KeyVault API}

\begin{flashcard}[Definition]{Key Vault}
    \vspace*{\stretch{1}}
    \begin{itemize}
        \item Speicher für verschlüsselte Daten
        \item Hardware (FIPS-2) oder Software
        \item Signierungen werden in sicherem Bereich durchgeführt, Schlüssel verlässt den Tresor nicht
        \item Verwaltung von Zertifikaten, inklusive Verlängerung
        \item Integration in Azure-Dienste: App Service, Virtuelle Maschinen, \ldots
    \end{itemize}
    \vspace*{\stretch{1}}
\end{flashcard}

\begin{flashcard}[Definition]{Key Vault-Zugriff}
    \vspace*{\stretch{1}}
    \begin{itemize}
        \item Geheimnis-URL:\newline
            \texttt{https://vault-name.vault.azure.net/secrets/secret-name}\newline
        \item Schlüssel URL: (POST)\newline
            \texttt{https://vault-name.vault.azure.net/keys/key-name}\newline
        \item Zertifikat-URL:\newline
            \texttt{https://vault-name.vault.azure.net/certificates/cert-name}\newline
        \item Versionen: können an die URL angehängt werden
    \end{itemize}
    Operationen:
    \begin{itemize}
        \item GET (abrufen), POST (erstellen), DELETE (löschen), PATCH (aktualisieren)
    \end{itemize}
    \vspace*{\stretch{1}}
\end{flashcard}


\begin{flashcard}[Definition]{Löschen und Wiederherstellen}
    \vspace*{\stretch{1}}
    \begin{itemize}
        \item wenn ein Wert gelöscht wird, ist er zunächst als gelöschtes Geheimnis weiterhin vorhanden
        \item das gelöschte Geheimnis kann wiederhergestellt, oder wirklich gelöscht werden
        \item Wiederherstellen:\newline
            \texttt{https://vault-name.vault.azure.net/deletedsecrets/secret-name/recover}\newline
            POST, da das alte Gemeimnis wieder hergestellt wird
        \item Löschen:\newline
            \texttt{https://vault-name.vault.azure.net/deletedsecrets/secret-name}\newline
            DELETE (löschen)
    \end{itemize}
    Analog: \texttt{deletedkeys}, \texttt{deletedcertificates}
    \vspace*{\stretch{1}}
\end{flashcard}
