\subsectioncard{Konfigurieren von Serverless}

\begin{flashcard}[Definition]{Logic App}
    \vspace*{\stretch{1}}
    \begin{itemize}
        \item Kombination von vielen Diensten mit vorkonfigurierten Modulen
        \item Modellierung von Work-Flows in einem Design-First-Ansatz
        \item Erstellt im Logic-App-Designer im Portal
        \item Alternativ: JSON-Format
    \end{itemize}
    \vspace*{\stretch{1}}
\end{flashcard}

\begin{flashcard}[Definition]{Logic App-Komponenten}
    \vspace*{\stretch{1}}
    \begin{itemize}
        \item Trigger:\newline
            Auslösen einer Logic App
        \item Aktionen:\newline
            Aufruf eines Dienstes mit Daten: SQL, Machine Learning, \ldots
        \item Steureung:\newline
            Entscheidungen, Abfragen, Schleifen, \ldots
    \end{itemize}
    \vspace{1cm}
    Connector: Kombination von verknüpften Triggern und Aktionen
    \vspace*{\stretch{1}}
\end{flashcard}

\begin{flashcard}[Definition]{Trigger}
    \vspace*{\stretch{1}}
    Typen:
    \begin{itemize}
        \item Daten: Email, Tween, \ldots
            \begin{itemize}
                \item Abfragemodell: \emph{Häufigkeit (Einheit)}, \emph{Intervall (Anzahl)}
                \item Push-getrieben: webhooks
            \end{itemize}
            $\Rightarrow$ Abfragen verursachen Kosten!
        \item Serientrigger: zeitgesteuert
        \item manuelle Anforderung: webhooks, \ldots
    \end{itemize}
    Eigenschaften:
    \begin{itemize}
        \item Parameter: optional und notwendig
        \item Rückgabewerte: ein Objekt, oder mehrere ($\rightarrow$ Schleifen), teilen und parallel ausführen
    \end{itemize}

    \vspace*{\stretch{1}}
\end{flashcard}

\begin{flashcard}[Definition]{Aktionen}
    \vspace*{\stretch{1}}
    \begin{itemize}
        \item externe Dienste: Zugriff auf Dienste, z.\,B. mit Verbindungszeichenfolgen, Logins, \ldots
        \item Bearbeiten von Daten: z.\,B. Konkatenation, Array-Auswahl, \ldots
        \item Steuerung: Bedingungen, Schleifen, \ldots
    \end{itemize}
    \vspace{1cm}
    Eigenschaften:
    \begin{itemize}
        \item Parameter: Eingaben für die Aktion, statisch oder dynamisch
        \item Rückgabewerte:
    \end{itemize}
    \vspace*{\stretch{1}}
\end{flashcard}

\begin{flashcard}[Definition]{Logic App erstellen}
    \vspace*{\stretch{1}}
    Logic-App-Designer
    \begin{itemize}
        \item Benötigt:
            \begin{itemize}
                \item Name
                \item Abonnement + Ressourcengruppe
                \item Standort
            \end{itemize}
        \item im Portal: \texttt{+ Ressource erstellen | Logik-App}\newline
        \item Katalog von Konnectoren, Triggern, Aktionen
        \item \texttt{Logik-App-Designer | Vorlagen} im Logik-App-Blade, und entweder vorlage oder \texttt{Leere Logik-App}
        \item grafische Modellierung im Portal: \texttt{Logik-App-Designer} im Logik-App-Blade
    \end{itemize}
    \vspace*{\stretch{1}}
\end{flashcard}

\begin{flashcard}[Definition]{Logic-App-Designer}
    \vspace*{\stretch{1}}
    \begin{enumerate}
        \item Konnektor/Aktion suchen
        \item Parameter setzen
        \item neue Hinzufügen
    \end{enumerate}
    \vspace*{\stretch{1}}
\end{flashcard}

\begin{flashcard}[Definition]{}
    \vspace*{\stretch{1}}
    \begin{itemize}
        \item
    \end{itemize}
    \vspace*{\stretch{1}}
\end{flashcard}

\begin{flashcard}[Definition]{Azure Funktion}
    \vspace*{\stretch{1}}
    \begin{itemize}
        \item Code First
        \item Code im Portal oder in Quellcode-Verwaltung
        \item zustandslos (Alternative: Durable Functions)
        \item ereignisgesteuert: werden als Reaktion auf Trigger ausgeführt
        \item beschränkte Ausführungszeit: standard 5 Minuten, maximal 10
        \item bei häufiger Ausführung: VM kann günstiger sein
    \end{itemize}
    \vspace*{\stretch{1}}
\end{flashcard}

\begin{flashcard}[Definition]{App-Serviceplan}
    \vspace*{\stretch{1}}
    \begin{itemize}
        \item verbrauchsbasiert: serverless mit automatischer Skalierung, bezahlen nach Nutzung
        \item Azure App Service-Plan: Ausführung auf definierter VM, ermöglicht längere Funktionsausführung
    \end{itemize}
    \vspace*{\stretch{1}}
\end{flashcard}

\begin{flashcard}[Definition]{Funktions-App erstellen}
    \vspace*{\stretch{1}}
    Funktions-App: Container zum Verwalten von Funktionen
    \begin{itemize}
        \item Benötigt:
            \begin{itemize}
                \item Abonnement + Resourcengruppe
                \item Name (global eindeutig)
                \item Runtime/Sprache, Version
                \item Region
                \item Speicherkonto
                \item Betriebssystem
                \item Serviceplan
            \end{itemize}
        \item Im Portal: \texttt{+ Neue Ressource erstellen | Compute | Funktions-App}
    \end{itemize}
    \vspace*{\stretch{1}}
\end{flashcard}

\begin{flashcard}[Definition]{Konfigurieren einer Funktions-App}
    \vspace*{\stretch{1}}
    \begin{itemize}
        \item Ereignisse/Trigger zur Ausführung:
        \item Bindung
        \item Autorisierung
    \end{itemize}
    \vspace*{\stretch{1}}
\end{flashcard}

\begin{flashcard}[Definition]{Funktions-App-Trigger}
    \vspace*{\stretch{1}}
    \begin{itemize}
        \item Blob-Speicher: bei Änderungen an Blobs
        \item Azure Cosmos DB: Einfügung/Aktualisierung
        \item Event Grid: bei empfangenem Ereignis
        \item HTTP: bei hook
        \item Microsoft-Graph webhooks
        \item Queue Storage: bei Elementen in Warteschlangen, Element als Eingabe
        \item Service Bus: bei Nachrichten aus einer Service Bus-Schlange
        \item Timer: regelmäßig
    \end{itemize}
    \vspace*{\stretch{1}}
\end{flashcard}

\begin{flashcard}[Definition]{Funktions-App-Bindung}
    \vspace*{\stretch{1}}
    \begin{itemize}
        \item Kommunikation mit Diensten
        \item wird automatisch bereitgestellt
        \item Richtung
        \begin{itemize}
            \item Eingabebindung: zum Lesen von Daten
            \item Ausgabebindung: zum Ausgeben von Daten
        \end{itemize}
    \end{itemize}
    Beispiel:
    \begin{enumerate}
        \item \emph{Azure Queue Storage Trigger}: liest Daten aus Queue
        \item \emph{Azure Table Storage-Ausgabebindung}: schreibt Daten in Tabelle
    \end{enumerate}

    \vspace*{\stretch{1}}
\end{flashcard}

\begin{flashcard}[Definition]{Funktion App-Autorisierungsstufen}
    \vspace*{\stretch{1}}
    \begin{itemize}
        \item Funktion:\newline
            spezifischer API-Schlüssel
        \item Administrator\newline
            globaler Schlüssel
        \item anonym\newline
            keine Autorisierung erforderlich
    \end{itemize}
    Im Bereich \texttt{Entwickler | Funktionsschlüssel} im Funktions-Blade
    \vspace*{\stretch{1}}
\end{flashcard}

\begin{flashcard}[Definition]{Funktion erstellen}
    \vspace*{\stretch{1}}
    \begin{itemize}
        \item Im Portal:\newline
            \texttt{Function-App | Funktionen | + Hinzufügen}
        \item Vorlage auswählen
    \end{itemize}
    \vspace*{\stretch{1}}
\end{flashcard}

