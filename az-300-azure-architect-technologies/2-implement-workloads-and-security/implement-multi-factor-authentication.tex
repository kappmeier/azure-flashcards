\subsectioncard{Mehrstufige Authentifikation einrichten}

\begin{flashcard}[Definition]{Mehrstufige Authentifikation}
    \vspace*{\stretch{1}}
    \begin{itemize}
        \item Mindestens zwei Methoden zum Login
            \begin{itemize}
                \item Passwort (immer)
                \item eine zweite Methode
            \end{itemize}
        \item Für Administratoren verpflichtend
        \item Kann aus, für einige Gruppen, für alle Mitglieder im Active Directory an sein
    \end{itemize}
    \vspace{1cm}
    Möglichkeit, mit Legacy-Login MFA zu umgehen!
    \vspace*{\stretch{1}}
\end{flashcard}

\subsubsectioncard{configure user accounts for MFA}

\begin{flashcard}[Definition]{Zeitraum zwischen Abfragen des zweiten Faktors}
    \vspace*{\stretch{1}}
    \begin{itemize}
        \item nicht jeder Authentifizierungs-Versuch muss mit zwei Methoden abgesichert sein
        \item Caching-Regel erlaubt, eine Periode, in der es nicht notwendig ist, anzugeben
    \end{itemize}
    \vspace*{\stretch{1}}
\end{flashcard}

\subsubsectioncard{configure fraud alerts}

\begin{flashcard}[Definition]{Betrugs-Benachrichtigung}
    \vspace*{\stretch{1}}
    Benutzer können betrügerische Anmeldeverusche melden
    \begin{itemize}
        \item Betrugsversuch:\newline
            falls eine MFA-Aufforderung ohne Anmeldeversuch eingeht
        \item Automatische Blockierung:\newline
            Sperrung von Benutzerkonten, die Betrug melden für 90 Tage oder \emph{bis ein Administrator die Sperre aufhebt}
        \item Code zur Meldung über Telefonanrufe (standard 0)
    \end{itemize}
    \vspace*{\stretch{1}}
\end{flashcard}

\subsubsectioncard{configure bypass methods}

\begin{flashcard}[Definition]{Organisations-Mitglieder}
    \vspace*{\stretch{1}}
    \begin{itemize}
        \item Nutzer innerhalb der Organisation können MFA überspringen
        \item benötigt einen Nachweis durch Active Directory Federation Services (AD FS)
        \item bestimmte Ip-Bereiche können übersprungen werden (Trusted IP)
    \end{itemize}
    \vspace*{\stretch{1}}
\end{flashcard}

\begin{flashcard}[Definition]{One-time bypass}
    \vspace*{\stretch{1}}
    \begin{itemize}
        \item Für eine einmalige Anmeldung wird der zweite Faktor deaktiviert
        \item Bypass ist zeitlich begrenzt für einige Sekunden
    \end{itemize}
    \vspace*{\stretch{1}}
\end{flashcard}

\subsubsectioncard{configure Trusted IPs}

\begin{flashcard}[Definition]{Trusted IPs}
    \vspace*{\stretch{1}}
    \begin{itemize}
        \item IP-Bereiche, für die keine mehrstufige Authentifizierung notwendig ist
        \item[!] für das intranet der Organisation\newline
            für generelle Standort-Freigate: Bedingter Zugriff
        \item Bis zu 50 CIDR-Bereiche
    \end{itemize}
    Zwei Möglichkeiten:
    \begin{itemize}
        \item Einstellung für mehrstufige Authentifizierung:\newline
            \texttt{Azure Active Directory | Benutzer | Multi-Factor Authentication | service Settings | Trusted IPs}
        \item Über Bedingten Zugriff:\newline
            \texttt{Azure Active Directory | Sicherheit | Conditional Access | Named Locations}
    \end{itemize}

    \vspace*{\stretch{1}}
\end{flashcard}

\subsubsectioncard{configure verification methods}

\begin{flashcard}[Definition]{Authentifizierungsmethoden}
    \vspace*{\stretch{1}}
    \begin{itemize}
        \item Passwort
        \item Textnachricht
        \item (Microsoft-)App
        \item Sicherheitsfragen
        \item FIDO
        \item Biometrie
    \end{itemize}
    \vspace{1cm}
    ! Sicherheitsfragen können nur in der Self-Service-Kennwortrücksetzung verwendet werden
    \vspace*{\stretch{1}}
\end{flashcard}

