\subsectioncard{Einrichten von Anwendungs-Load-Balancing}

\begin{flashcard}[Definition]{Lastenausgleichs-Lösungen in Azure}
    \vspace*{\stretch{1}}
    \begin{itemize}
        \item Application Gateway\newline
            Level 7, HTTP-Basierter Ausgleich
        \item Load Balancer\newline
            Hochperformanter Level 4-Load-Balancer (IP-Basiert)
        \item Traffic Manager\newline
            DNS lastenausgleich
    \end{itemize}
    \vspace*{\stretch{1}}
\end{flashcard}   

\subsubsectioncard{configure application gateway}

\begin{flashcard}[Definition]{Application Gateway}
    \vspace*{\stretch{1}}
    \begin{itemize}
        \item Weiterleitung abhängig von URL
        \item Anwendungsschichtrouting
        \item Ziele: VMs, Skalierungsgruppen, lokale Server\newline
            genaugenommen: nicht nur lokale Server sondern beliebige Endpunkte! (Externe URLs)
        \item Routing
            \begin{itemize}
                \item Pfadbasiert:\newline
                    abhängig von URL, z.\,B. \emph{/videos/}
                \item mehrere Webseiten
            \end{itemize}

    \end{itemize}
    \vspace*{\stretch{1}}
\end{flashcard}   

\begin{flashcard}[Definition]{Lastenausgleich im Application Gateway}
    \vspace*{\stretch{1}}
    \begin{itemize}
        \item Round robin
        \item möglich: Persistenz
        \item auf Anwendngsschicht (OSI 7)\newline
            $\Rightarrow$ im Unterschied zu Load Balancer (OSI 4, IP-Adressen)
        \item z.\,B. Ausgleich von Traffic zum Webserver
    \end{itemize}
    \vspace*{\stretch{1}}
\end{flashcard}

\begin{flashcard}[Definition]{Application Gateway-Komponenten}
    \vspace*{\stretch{1}}
    \begin{itemize}
        \item Front-End-IP-Adresse: öffentliche oder private Adresse
        \item Listener: empfängt Daten nach \emph{Protokoll,Port,Host,IP}
        \item Routing-Regeln: binden Listener an Back-End-Pools
        \item Back-End-Pools: mehrere Webserver, z.\,B. auch Skalierungsgruppen oder App Service App
        \item Web Application Firewall: optional, blockiert eingehende Anfragen vor Listener
        \item Integritätstest: tested Server im Back-End-Pool\newline
            Integritätstest für Port und Pool an den weitergeleitet wird
    \end{itemize}
    \vspace*{\stretch{1}}
\end{flashcard}

\begin{flashcard}[Definition]{Application Gateway-Listener}
    \vspace*{\stretch{1}}
    Basislistener
    \begin{itemize}
        \item beachtet nur URL
    \end{itemize}
    \vspace{1cm}
    Listener für mehrere Standorte
    \begin{itemize}
        \item nutzt auch Hostname
        \item für Unter-Domänen
    \end{itemize}
    \vspace*{\stretch{1}}
\end{flashcard}

\begin{flashcard}[Definition]{Application Gateway erstellen}
    \vspace*{\stretch{1}}
    Eigenes Subnetz im VNET für skalierung benötigt!
    \begin{itemize}
        \item Benötigt:
            \begin{itemize}
                \item Resourcegruppe + Name
                \item SKU: Standard, WAF, jeweils v1 und v2
                \item Kapazität
                \item VNET + Subnet
                \item Frontend-IP + Port
                \item HTTP-Einstellungen: Protokoll, Port, \ldots
            \end{itemize}
        \item CLI: \texttt{az network application-gateway create}
        \item Portal
    \end{itemize}
    \vspace*{\stretch{1}}
\end{flashcard}

\begin{flashcard}[Definition]{Application Gateway Leistungsstufen}
    \vspace*{\stretch{1}}
    \begin{itemize}
        \item Standard v1
            \begin{itemize}
             \item Web-Protokoll-Balancing
            \end{itemize}
        \item Standard v2 (empfohlen)
            \begin{itemize}
                \item Autoscaling
                \item Zonenredundanz (Verfügbarkeitszonen)
                \item Statische IPs
            \end{itemize}
        \item WAF v1: features von Web Application Firewall
        \item WAF v2: zusätzlich benutzerdefinierte Regeln
    \end{itemize}
    \vspace*{\stretch{1}}
\end{flashcard}

\begin{flashcard}[Definition]{Web Application Firewall}
    \vspace*{\stretch{1}}
    Managed Service, der eine HTTP Firewall darstellt
    \begin{itemize}
        \item Eingebundne in weitere Systeme:
            \begin{itemize}
                \item Application Gateway
                \item Front Door
                \item CDN
            \end{itemize}
        \item SQL injection
        \item Cross site scripting
        \item viele weitere Regeln
    \end{itemize}
    \vspace*{\stretch{1}}
\end{flashcard}

\begin{flashcard}[Definition]{Application Gateway Back-End-Pool erstellen}
    \vspace*{\stretch{1}}
    \begin{itemize}
        \item Benötigt:
            \begin{itemize}
                \item Gateway-Name
                \item Resourcegruppe + Name
                \item Server (als IPs)
            \end{itemize}
        \item CLI: \texttt{az network application-gateway address-pool create}
        \item Portal
    \end{itemize}
    \vspace*{\stretch{1}}
\end{flashcard}

\begin{flashcard}[Definition]{Application Gateway Frontend erstellen}
    \vspace*{\stretch{1}}
    \begin{itemize}
        \item Benötigt:
            \begin{itemize}
                \item Gateway-Name
                \item Resourcegruppe + Name
                \item Port
            \end{itemize}
        \item CLI: \texttt{az network application-gateway frontend-port create}
        \item Portal
    \end{itemize}
    \vspace*{\stretch{1}}
\end{flashcard}

\begin{flashcard}[Definition]{Application Gateway Listener erstellen}
    \vspace*{\stretch{1}}
    \begin{itemize}
        \item Benötigt:
            \begin{itemize}
                \item Gateway-Name
                \item Resourcegruppe
                \item Name
                \item Frontend-Port-Name
            \end{itemize}
        \item CLI: \texttt{az network application-gateway http-listener create}
        \item Portal
    \end{itemize}
    \vspace*{\stretch{1}}
\end{flashcard}

\begin{flashcard}[Definition]{Application Gateway Integritätstest}
    \vspace*{\stretch{1}}
    \begin{itemize}
        \item Ziel: testen, ob VMs im Backend erreichbar sind
        \item über HTTP-Anfragen
        \item mögliche Tests:
            \begin{itemize}
                \item HTTP-Result-Code:\newline
                    z.\,B. \texttt{200-399}
                \item return body:\newline
                    z.\,B. \texttt{Status: OK}
            \end{itemize}
    \end{itemize}
    \vspace*{\stretch{1}}
\end{flashcard}

\begin{flashcard}[Definition]{Application Gateway Integritätstest erstellen}
    \vspace*{\stretch{1}}
    \begin{itemize}
        \item Benötigt:
            \begin{itemize}
                \item Gateway-Name + Resourcegruppe
                \item Name
                \item Pfad
                \item Intervall, Threshold, Timeout
                \item Protokoll: \emph{HTTP}
            \end{itemize}
        \item CLI: \texttt{az network application-gateway probe create}
        \item Portal
    \end{itemize}
    \vspace*{\stretch{1}}
\end{flashcard}

\begin{flashcard}[Definition]{Application Gateway Regeln erstellen}
    \vspace*{\stretch{1}}
    Schritt 1: URL-Pfad-Zuordnung
    \begin{itemize}
        \item Benötigt:
            \begin{itemize}
                \item Gateway-Name + Resourcegruppe
                \item Name
                \item Pfade
                \item HTTP-Einstellungen
                \item zu benutzender Backend-Pool
            \end{itemize}
        \item CLI: \texttt{az network application-gateway url-path-map create}
        \item Portal:
    \end{itemize}
    \vspace*{\stretch{1}}
\end{flashcard}

\begin{flashcard}[Definition]{Application Gateway Regeln erstellen}
    \vspace*{\stretch{1}}
    Schritt 2: URL-Pfad-Zuordnuns-Regel erstellen
    \begin{itemize}
        \item Benötigt:
            \begin{itemize}
                \item Gateway-Name + Resourcegruppe
                \item Name
                \item Pfade
                \item HTTP-Einstellungen
                \item zu benutzender Backend-Pool
                \item zu benutzende Pfad-Map
            \end{itemize}
        \item CLI: \texttt{az network application-gateway url-path-map rule reate}
        \item Portal
    \end{itemize}
    \vspace*{\stretch{1}}
\end{flashcard}

\begin{flashcard}[Definition]{Application Gateway Regeln erstellen}
    \vspace*{\stretch{1}}
    Schritt 3: Routingregel mit der Pfadzuordnung erstellen
    \begin{itemize}
        \item Benötigt:
            \begin{itemize}
                \item Gateway-Name + Resourcegruppe
                \item Name
                \item Listener
                \item Regeltyp: \emph{PathBasedRouting}
                \item zu benutzender Backend-Pool
                \item zu benutzende Pfad-Map
            \end{itemize}
        \item CLI: \texttt{az network application-gateway rule reate}
        \item Portal
    \end{itemize}
    Achtung: standardmäßig wird eine Regel erstellt
    \vspace*{\stretch{1}}
\end{flashcard}

\begin{flashcard}[Definition]{Konfiguration von Application Gateway-Komponenten}
    \vspace*{\stretch{1}}
    \begin{itemize}
        \item Listener: \texttt{az network application-gateway http-listener}
        \item Regeln: \texttt{az network application-gateway rule}
        \item Front-End-IP: \texttt{az network application-gateway front-end-port}
        \item Back-End-Pools: \texttt{az network application-gateway address-pool}
        \item Integritätstest: \texttt{az network application-gateway http-settings}
    \end{itemize}
    \vspace*{\stretch{1}}
\end{flashcard}

\subsubsectioncard{configure Azure Front Door service}

\begin{flashcard}[Definition]{Front Door}
    \vspace*{\stretch{1}}
    \begin{itemize}
        \item Anwendungs-Auslieferungs-Netz, also level 7 Lastausgleich
        \item Beschleunigung
        \item Realzeit-Failover
        \item hochverfügbar und skalierbar, vollständig von Azure verwaltet
        \item unterstützte Protokolle: HTTP, HTTPS, HTTP/2 (nur Front End)
    \end{itemize}
    \vspace*{\stretch{1}}
\end{flashcard}

\begin{flashcard}[Definition]{Vergleich Front Door $\leftrightarrow$ Application Gateway}
    \vspace*{\stretch{1}}
    \begin{itemize}
        \item Front Door ist global\newline
            $\Rightarrow$ Ausgleich zwischen verschiedenen Clustern über Regionen
        \item Application Gateway ist lokal\newline
            $\Rightarrow$ Ausgleich zwischen VMs, Containern in einer Skalierungsgruppe
    \end{itemize}
    \vspace*{\stretch{1}}
\end{flashcard}

\begin{flashcard}[Definition]{Zusammenspiel mit Azure Load Balancer}
    \vspace*{\stretch{1}}
    \begin{itemize}
        \item Mögliche Verbindungen von/zu Front Door:
            \begin{itemize}
                \item eine öffentliche IP als Eingang
                \item öffentlicher DNS-Name als Ausgang
            \end{itemize}
        \item Azure Load Balancer kann intern eingesetzt werden
    \end{itemize}
    $\Rightarrow$ Load Balancer hinter Front Door ist möglich
    \vspace*{\stretch{1}}
\end{flashcard}

\begin{flashcard}[Definition]{Routing-Methoden}
    \vspace*{\stretch{1}}
    \begin{itemize}
        \item Latency: das am schnellsten zu erreichende Backend wird genutzt
        \item Priority: Priorisierung von Backends mit Backup-Backend
        \item Weighted: Gewichtung für die verschiedneen Backends
        \item Session Affinity: nachfolgende Anforderungen werden an das gleiche Backend geliefert
    \end{itemize}
    \vspace*{\stretch{1}}
\end{flashcard}

\begin{flashcard}[Definition]{Health-Probe}
    \vspace*{\stretch{1}}
    \begin{itemize}
        \item GET: ruft Information von URL ab
        \item HEAD: nur Header, kein Message Body
        \item Status: muss 200 sein
        \item Latency: zeit zwischen Senden der Probe und erhalt der Antwort
    \end{itemize}
    \vspace*{\stretch{1}}
\end{flashcard}

\begin{flashcard}[Definition]{Traffic-Weiterleitung}
    \vspace*{\stretch{1}}
    Front Door kann als Redirection-Service eingesetzt werden
    \begin{itemize}
        \item Untertsützt Forward/Redirect (301)
        \item Protokolle: HTTPS, HTTP, wie angefordert
    \end{itemize}
    \vspace*{\stretch{1}}
\end{flashcard}

\subsubsectioncard{configure Azure Traffic Manager}

\begin{flashcard}[Definition]{Azure Traffic Manager}
    \vspace*{\stretch{1}}
    \begin{itemize}
        \item DNS-basierendes Load Balancing
        \item unterstützt Hochverfügbarkeit, Resilienz, Reaktionszeit:
            \begin{itemize}
                \item automatisches Failover
                \item regionale Umleitung von Client-Anforderungen
            \end{itemize}
    \end{itemize}
    \vspace*{\stretch{1}}
\end{flashcard}

\begin{flashcard}[Definition]{Funktion von Traffic Manager}
    \vspace*{\stretch{1}}
    \begin{itemize}
        \item DNS-basiert, Traffic Manager hat keinen Zugriff auf Daten
        \item stellt passende IP-Adresse zur Verfügung (Klienten verbinden sich direkt dahin)\newline
            $\rightarrow$ \emph{Endpunkt}
        \item \emph{nicht}: Proxy, Gateway
    \end{itemize}
    \vspace{1cm}
    Endpunkte
    \begin{itemize}
        \item Azure: Dienste oder VMs in Azure
        \item Extern: \emph{externe} Dienste, per IP oder FQDN
        \item geschachtelt: kombination von Profilen für komplexe Anforderungen
    \end{itemize}
    \vspace*{\stretch{1}}
\end{flashcard}

\begin{flashcard}[Definition]{Traffic Manager Routing-Optionen}
    \vspace*{\stretch{1}}
    \begin{itemize}
        \item gewichtetes Routing
            \begin{itemize}
                \item gleichmäßig (ggf. gewichtete) Verteilung\newline
                    (von 1 - 1000)
                \item Auswahl zufällig anhand Gewichtung
            \end{itemize}
        \item leistungsbasiertes Routing
            \begin{itemize}
                \item möglich, wenn verschiedene geographische Standorte existierende
                \item Auswahl nach Internetlatenztabelle
            \end{itemize}
        \item geographisches Routing
            \begin{itemize}
                \item verbindung nach Geographieen (aber nicht Latenz)
                \item z.\,B. für Complience-Anforderungen
            \end{itemize}
    \end{itemize}
    \vspace*{\stretch{1}}
\end{flashcard}

\begin{flashcard}[Definition]{Traffic Manager Routing-Optionen 2}
    \vspace*{\stretch{1}}
    \begin{itemize}
        \item mehrere Endpunkte
            \begin{itemize}
                \item mehrere Endpunkte in einer DNS-Abfrage
                \item erhöht Verfügbarkeit und verringert Latenz im Fehlerfall
            \end{itemize}
        \item prioritätsbasiertes Routing
            \begin{itemize}
                \item Endpunkte sortiert nach Priorität
                \item es wird immer der mit der höchsten Priorität verwendet, der Verfügbar ist
            \end{itemize}
        \item Subnetz-Routing
            \begin{itemize}
                \item Endpunkte per IP-Adresse bestimmen
                \item Filtern z.\,  B. nach Nutzern, Anbietern, \ldots
            \end{itemize}
    \end{itemize}
    \vspace*{\stretch{1}}
\end{flashcard}

\begin{flashcard}[Definition]{Konfiguration von Traffic Manager}
    \vspace*{\stretch{1}}
    \begin{enumerate}
        \item Traffic Manager-Profil erstellen
        \item Endpunkt hinzufügen
    \end{enumerate}
    \vspace{1cm}
    Endpunkte können deaktiviert werden
    \vspace*{\stretch{1}}
\end{flashcard}

\begin{flashcard}[Definition]{Traffic Manager-Profil erstellen}
    \vspace*{\stretch{1}}
        \begin{itemize}
            \item Benötigte Parameter:
                \begin{itemize}
                    \item Name
                    \item eindeutiger DNS-Name\newline
                        DNS-Name: FQDN=name.trafficmanager.net
                    \item Abonnement
                    \item Resourcengruppe
                    \item Routingmethode: \emph{Performance}, \emph{Priority}
                    \item Standort
                \end{itemize}
            \item CLI: \texttt{az network traffic-manager profile create}\newline
                eindeutiger DNS-Name kann selbst vergeben werden
            \item Portal: Traffic Manager-Profil-Resource\newline
                Name ist gleichzeitig eindeutiger DNS-Name
        \end{itemize}
    \vspace*{\stretch{1}}
\end{flashcard}

\begin{flashcard}[Definition]{Traffic Manager Endpunkt erstellen}
    \vspace*{\stretch{1}}
    \begin{itemize}
        \item Benötigte Parameter:
            \begin{itemize}
                \item Resourcegruppe
                \item Profilname: das Traffic Manager-Profil
                \item Name
                \item Typ: Azure, Extern, geschachtelt
                \item Ziel-ID: z.\,B. IP-Adresse, Azure Dienst, \ldots
                \item Parameter für Routingmethode, z.\,B. Priorität, nichts für Performance
            \end{itemize}
        \item CLI: \texttt{az network traffic-manager endpoint create}
        \item Portal: \texttt{Einstellungen | Endpunkte | Hinzufügen} im Traffic-Manager-Blade
    \end{itemize}
    \vspace*{\stretch{1}}
\end{flashcard}
