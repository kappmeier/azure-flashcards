\subsectioncard{Verbinden eines lokalen Netzwerks mit einem Azure VNET}

\subsubsectioncard{create and configure Azure VPN Gateway}

\begin{flashcard}[Definition]{VPN Gateways}
    \vspace*{\stretch{1}}
    Szenarios:
    \begin{itemize}
        \item Standord-zu-Standord (Site-to-Site) verbinden lokale Rechenzentren mit einem virtuellen Netzwerk in Azure
        \item Point-to-Site-Gateways verbinden ein einzelnes Gerät mit einem virtuellen Netzwerk in Azure
        \item Netzwerk-zu-Netzwerk-Verbindung zwischen zwei virtuellen Netzwerken in Azure
    \end{itemize}

    Maximal ein VPN-Gateway pro virtuellem Netzwerk

    \vspace*{\stretch{1}}
\end{flashcard}

\begin{flashcard}[Definition]{VPN Gateway Routing}
    \vspace*{\stretch{1}}
    Typen:
    \begin{itemize}
        \item richtlinienbasiert (statisches Routing)
        \item routenbasiert (dynamisches Routen)
    \end{itemize}

    \vspace{1cm}
    Wechsel nicht ohne Neu-Initialisierung möglich\newline
    $\Rightarrow$ IP-Adresse wechselt, bis 1 Stunde Einrichtungszeit

    \vspace*{\stretch{1}}
\end{flashcard}

\begin{flashcard}[Definition]{Richtlinienbasiertes VPN Gateway}
    \vspace*{\stretch{1}}
    \begin{itemize}
        \item nur IKEv1-Veschlüsselung
        \item statisches Routing über Routingtabelle und Adresspräfixen
        \item unterstützt gegebenfalls ältere lokale VPN-Hardware
    \end{itemize}
    \vspace*{\stretch{1}}
\end{flashcard}

\begin{flashcard}[Definition]{Routenbasiertes VPN Gateway}
    \vspace*{\stretch{1}}
    \begin{itemize}
        \item IKEv2
        \item Versand von Datenpaketen anhand von Netzwerkroutingtabellen
        \item $n$:$n$-Datenverkehrsselektoren
    \end{itemize}

    Scenarios:
    \begin{itemize}
        \item Netwerk-zu-Netzwerk-Verbindungen
        \item Point-to-Site
        \item mehrere Standorte
        \item parallele Nutzung mit ExpressRoute-Gateway
    \end{itemize}
    \vspace*{\stretch{1}}
\end{flashcard}

\begin{flashcard}[Definition]{Erstellen und Konfigurieren eines VPN Gateways}
    \vspace*{\stretch{1}}
    Benötigte Resourcen (in Azure!):
    \begin{itemize}
        \item Virtuelles Netzwerk\newline
            benötigt genügend Platz für zusätzliches Subnetzdes VPN Gateways
        \item Gateway Subnetz\newline
            groß genug, mindestens $/27$. Kann nicht für andere Zwecke verwendet werden
        \item öffentliche IP-Adresse: dynamisch, aber konstant solange VPN Gateway existiert
        \item Gateway im lokalen Netz (in Azure!)
        \item Gateway im virtuellen Netz (virtuelles Netzwerkgateway)
        \item Verbindung: von öffentlicher IP-Adresse mit IPv4-Adresse des lokalen Netzwerkgateways
    \end{itemize}

    Benötigte lokale Resourcen
    \begin{itemize}
        \item VPN-Gerät
        \item IPv4-Adresse (siehe lokales Netzwerkgateway)
    \end{itemize}
    \vspace*{\stretch{1}}
\end{flashcard}

\begin{flashcard}[Definition]{Konfiguration}
    \vspace*{\stretch{1}}
    \begin{itemize}
        \item Größenauswahl/SKU:
            \begin{itemize}
                \item Anzahl Tunnel, Durchsatz und BGP-Unterstützung
                \item wechsel von Basic in höherwertige SKUs erfordet erneute Bereitstellung
            \end{itemize}
        \item Hochverfügbarkeit:
            \begin{itemize}
                \item zwei VPN Gateways mit Active/Standby\newline
                    (automatisch aktiv)
                \item Active/Active (mit zwei öffentlichen IP-Adressen)\newline
                    (kann erweitert werden mit zweitem Lokalen VPN-Gerät)
                \item als Backup/Failover für ExpressRoute
                \item Zonenredundanz
            \end{itemize}
    \end{itemize}
    \vspace*{\stretch{1}}
\end{flashcard}

\begin{flashcard}[Definition]{Erstellen der Voraussetzungen}
    \vspace*{\stretch{1}}
    \begin{itemize}
        \item Erstellung eines VNETs:\newline
            \texttt{az network vnet create }
        \item Erstellen eines Subnets: (gateway subnet)\newline
            \texttt{az network vnet subnet create}
        \item Erstellen eines lokalen Gateways:\newline
            \texttt{az network local-gateway create}
        \item Erstellen einer öffentlichen IP:\newline
            \texttt{az network public-ip create}
    \end{itemize}

    Weiter: Erstellen eines VPN-Gateways
    \vspace*{\stretch{1}}
\end{flashcard}

\begin{flashcard}[Definition]{Erstellen eines VPN-Gateways}
    \vspace*{\stretch{1}}
    \begin{itemize}
        \item Erstellen eines VPN-Gateways: (langsam!)\newline
            \texttt{az network vnet-gateway create}
            \begin{itemize}
                \item \texttt{--public-ip-address PIP-VNG-HQ-Network}
                \item \texttt{--vnet HQ-Network}
                \item \texttt{--gateway-type Vpn}
                \item \texttt{--vpn-type RouteBased}
                \item \texttt{--sku VpnGw1}
            \end{itemize}
        \item Erstellen der VPN-Verbindung:\newline
            \texttt{az network vpn-connection create}
    \end{itemize}
    \vspace*{\stretch{1}}
\end{flashcard}

\subsubsectioncard{create and configure site to site VPN}

\subsubsectioncard{configure ExpressRoute}

\begin{flashcard}[Definition]{ExpressRoute-Voraussetzungen}
    \vspace*{\stretch{1}}
    \begin{itemize}
        \item Partner zum Einrichten der Verbindung
        \item Azure-Abonnement mit registriefung beim Partner
        \item Microsoft-Konto
        \item Optional: Office 365-Abonnement
        \item BGP-fähiger Router
        \item öffentliche IP-Adressen und NAT
        \item Weiterleitung des Datenverkehrs über Range /29 oder /30
    \end{itemize}
    \vspace*{\stretch{1}}
\end{flashcard}

\begin{flashcard}[Definition]{Express Route einrichten}
    \vspace*{\stretch{1}}
    Express Route ist eine direkte Verbindung! (Kein VPN)
    \begin{itemize}
        \item Verbindungstypen:
            \begin{itemize}
                \item Cloud Exchange: ISP, Server-Anbieter, etc verbinden
                \item Point-to-Point: lokales Rechenzenter verbinden
                \item Any-to-any: Einbinden ins WAN
            \end{itemize}
        \item Peering-Schema:
            \begin{itemize}
                \item Microsoft-Peering: für Office 365, \ldots
                \item privates Azure-Peering: für Azure services, vnets, \ldots
            \end{itemize}
    \end{itemize}
    \vspace*{\stretch{1}}
\end{flashcard}

\begin{flashcard}[Definition]{Express-Route-Anbieter}
    \vspace*{\stretch{1}}
    \begin{itemize}
        \item einer der verfügbaren Anbieter, mit Registrierung
        \item benötigte Informationen:
            \begin{itemize}
                \item Name
                \item Peeringstandorte: vom Anbieter unterstützte Orte
                \item Bandbreite
            \end{itemize}
        \item Abrufen der Informationen:
            \begin{itemize}
                \item CLI:\newline
                    \texttt{az network express-route list-service-providers}
                \item PowerShell:\newline
                    \texttt{Get-AzExpressRouteServiceProvider}
            \end{itemize}
    \end{itemize}
    \vspace*{\stretch{1}}
\end{flashcard}

\begin{flashcard}[Definition]{Express-Route im Portal erstellen}
    \vspace*{\stretch{1}}
    \begin{itemize}
        \item Resourcen-Definition
            \begin{itemize}
                \item Abonnement + Resourcengruppe
                \item Name: name ohne Leer- und Sonderzeichen
            \end{itemize}
        \item Anbieter
        \item SKU: Lokal, Standard, Premium: Premium für > 10 virtuelle Netzwerke
        \item Abrechnungsmodell: getaktet oder unbegrenzt\newline
            eingehende Datenübertragungen sind immer kostenlos
        \item Standort: Standord (in Azure) für die Leitung
    \end{itemize}
    Sobald die Leitung in Azure verfügbar ist, muss der \emph{Dienstschlüssel} an den Anbieter gesendet werden

    \vspace{1cm}
    Zum aktivieren (wenn alles erledigt ist), muss ein \emph{Gateway für virtuelle Netze} mit der Option \emph{ExpressRoute} erstellt werden
    \vspace*{\stretch{1}}
\end{flashcard}

\begin{flashcard}[Definition]{Privates ExpressRoute-Peering konfigurieren}
    \vspace*{\stretch{1}}
    \begin{itemize}
        \item Peer-ASN: private oder öffentliche autonome Systemnummer
        \item primäres Subnetz: ein /30-Subnetz, IP\#1 für router, IP\#2 für Microsoft-Router\newline
            Die anderen zwei?
        \item sekundäres Subnetz: ein /30-Subnetz, aus sicherheitsgründen
        \item VLAN-ID: VLAN in dem das Peering hergestellt wird
        \item Schlüssel: optionaler MD5-Hash für Codierung von Nachrichten
    \end{itemize}
    \vspace*{\stretch{1}}
\end{flashcard}

\begin{flashcard}[Definition]{Microsoft ExpressRoute-Peering konfigurieren}
    \vspace*{\stretch{1}}
    \begin{itemize}
        \item gleiche Daten wie für privates Peering: \newline
            (Peer-ASN, primäres Subnetz, sekundäres Subnetz, VLAN-ID, Schlüssel)
        \item Angekündigte öffentliche Präfixe: für BGP verwendete Adresspräfixe
        \item Kunden-ASN: optionale autonome Systemnummer
        \item Name: Registrierungsname für Kunden-ASN
    \end{itemize}
    \vspace*{\stretch{1}}
\end{flashcard}

\begin{flashcard}[Definition]{Vergleich VPN Gateway und ExpressRoute}
    \vspace*{\stretch{1}}
    \begin{tabular}{l|lll}
        Funktion       &  VPN Gateway               & ExpressRoute                    \\
        \hline
        Azure-Dienste  &  Azure-Dienste \& VMs      & Microsoft Cloud                 \\
        Bandbreite     &  10 GBit/s                 & 10 GBit/s, 100 GBit/S           \\
        Protokoll      &  SSTP, IPsec               & VLAN oder MPLS                  \\
        Routing        &  Statisch, dynamisch       & BGP (Border Gateway)            \\
        Resilienz      &  Aktiv-passiv, aktiv-aktiv & Aktiv-passiv, aktiv-aktiv       \\
        Use-Cases      & Prototyp, Dev              & Unternehmenskritsiche Workloads \\
        SLA            & 99,95\% -- 99,99\%         & 99,95\%                         \\
    \end{tabular}
    \vspace*{\stretch{1}}
\end{flashcard}

\subsubsectioncard{configure Virtual WAN}

\begin{flashcard}[Definition]{Virtuelles WAN}
    \vspace*{\stretch{1}}
    Azure-Dienst, der alle (viele) Netzwerkfunktionen in einer Oberfläche vereint
    \begin{itemize}
        \item virtuelle Netzwerke
        \item VPN Gateways
        \item Lastenausgleich
        \item ExpressRoute
        \item Site-2-Site
        \item Point-2-Site
    \end{itemize}
    \vspace*{\stretch{1}}
\end{flashcard}

\subsubsectioncard{verify on premises connectivity}

\subsubsectioncard{troubleshoot on premises connectivity with Azure}
