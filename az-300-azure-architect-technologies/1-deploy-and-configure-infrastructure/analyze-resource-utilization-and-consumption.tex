\subsectioncard{Analysieren von Resourcen-Nutzung und -Verbrauch}

\subsubsectioncard{configure diagnostic settings on resources}

\begin{flashcard}[Definition]{Diagnoseeinstellungen überprüfen}
    \vspace*{\stretch{1}}
    \begin{itemize}
        \item \texttt{Azure Monitor | Diagnoseeinstellungen}: zeigt für bis 5 Abonnements alle Resourcen
        \item \texttt{Diagnoseeinstellungen} für eine Ressourcengruppen zeigt Einstellungen für alle Ressourcen
         \item \texttt{Diagnoseeinstellungen} für eine einzelne Ressource
    \end{itemize}
    \vspace*{\stretch{1}}
\end{flashcard}

\begin{flashcard}[Definition]{Diagnoseeinstellungen}
    \vspace*{\stretch{1}}
    \begin{itemize}
        \item Anwendungslogs: Protokollierung der Anwendung
            \begin{itemize}
                \item Kritisch
                \item Verbose
            \end{itemize}
        \item Systemlogs: Systemmeldungen
        \item Absturzabbild: erstellt einen Speicherabbild, wenn eine Anwendung abstürzt
    \end{itemize}
    \vspace*{\stretch{1}}
\end{flashcard}

\subsubsectioncard{create baseline for resources}

\begin{flashcard}[Definition]{Dynamische Regeln}
    \vspace*{\stretch{1}}
    \begin{itemize}
        \item für eine Metrik-Regel typ \texttt{Dynamisch} auswählen
        \item von Azure wird ein dynamischer Schwellwert aus der Vergangenheit berechnet
        \item Regeln abhängig von diesem dynamischen Schwellwert ausgelöst
    \end{itemize}
    \vspace*{\stretch{1}}
\end{flashcard}

\subsubsectioncard{create and test alerts}

\begin{flashcard}[Definition]{Warnungstypen in Azure Monitor}
    \vspace*{\stretch{1}}
    \begin{itemize}
        \item Metrik-Warnungen:\newline
            basierend auf Schwellwerten von Metriken (CPU-Nutzung, \ldots)
        \item Activity Log-Warnungen:\newline
            basierend auf Zustandsänderungen von Ressourcen (löschen, \ldots)
        \item Log-Warnungen:\newline
            basierend auf Analyse von Log-Dateien (Fehlermeldungen, \ldots)
    \end{itemize}
    \vspace*{\stretch{1}}
\end{flashcard}

\begin{flashcard}[Definition]{Warnungsregel erstellen}
    \vspace*{\stretch{1}}
    Benötigte Einstellungen
    \begin{itemize}
        \item Ressource/Ziel-Ressource:\newline
            Die Ressource für die Regel (auch mehrere). Bestimmt auch den Signaltyp.
        \item Bedingung:\newline
            Typ des Signals (Metrik, Activity Log, Log) und angewendete Prüflogik
        \item Aktion:\newline
            Ausgeführte Tätigkeit (z.\,B. Email versenden) an eine Aktionsgruppe
        \item Details:\newline
            Textuelle Beschreibung und Schweregrad (0 bis 4)
    \end{itemize}
    \vspace{1cm}
    \vspace*{\stretch{1}}
\end{flashcard}

\begin{flashcard}[Definition]{Metrik-Warnung erstellen}
    \vspace*{\stretch{1}}
    \begin{itemize}
        \item CLI:\newline
            \texttt{az monitor metrics alert create \
            -name \ldots\
            --resource-group \ldots\\
            --scopes <vm>\\
            --condition "max percentage CPU > 80"\\
            --description \ldots\\
            --evaluation-frequency \ldots\ --window-size \ldots\\
            --severity <0-4>}
        \item Im Portal:\newline
            \texttt{Monitor | Warnungen | + Regel erstellen}
    \end{itemize}
    \vspace*{\stretch{1}}
\end{flashcard}

\subsubsectioncard{analyze alerts across subscription}

\begin{flashcard}[Definition]{Warnungszusammenfassung}
    \vspace*{\stretch{1}}
    \begin{itemize}
        \item \texttt{Montior | Warnungen}
        \item Liste der Warnungen z.\,B. nach Abonnements, Ressourcengruppen, Zeitbereich
        \item Zustände von Warnungen
            \begin{enumerate}
                \item Neu: initialer Zustand
                \item Bestätigt: tatsächlich ein Problem wurde festgestellt
                \item Geschlossen: wenn das Problem gelöst worden ist
            \end{enumerate}
    \end{itemize}
    \vspace*{\stretch{1}}
\end{flashcard}

\subsubsectioncard{analyze metrics across subscription}

\subsubsectioncard{create action group}

\begin{flashcard}[Definition]{Aktionsgruppe}
    \vspace*{\stretch{1}}
    \begin{itemize}
        \item Aktionen, die bei Auslösung einer Warnung ausgeführt werden
        \item mehrere Aktionen in einer Gruppe
        \item verfügbare Aktionen
            \begin{itemize}
                \item Nachricht senden (Email, SMS, Push, Sprachanruf)
                \item Azure-Funktion/Logik-App
                \item Webhook
                \item Ticket erstellen
                \item Runbook (zum Neustarten oder Skalieren von VMs)
            \end{itemize}
        \item Aktionsgruppen können mit mehreren Regeln verknüpft werden
    \end{itemize}
    \vspace*{\stretch{1}}
\end{flashcard}

\begin{flashcard}[Definition]{Aktionsgruppe erstellen}
    \vspace*{\stretch{1}}
    \begin{itemize}
        \item Im Portal:\newline
            \texttt{Metriken | + Neue Warnungsregel | AKTIONEN | Aktionsgruppe erstellen}
    \end{itemize}
    \vspace*{\stretch{1}}
\end{flashcard}

\subsubsectioncard{monitor for unused resources}

\begin{flashcard}[Definition]{Azure Advisor}
    \vspace*{\stretch{1}}
    integrierter Dienst in Azure mit Empfehlungen, z.\,B. zu Kosten
    \begin{itemize}
        \item nicht bereitgestellte ExpressRoute-Verbindungen
        \item erwerb von reservierten VM-Instanzen
        \item ungenutzte/überdimensionerte virtuelle Maschinen
    \end{itemize}
    Im Portal: \texttt{Verwaltung + Governance | Advisor | Kosten}
    oder \texttt{Kostenverwaltung + Abrechnung | Kostenverwaltung | Ratgeberempfehlungen }
    \vspace*{\stretch{1}}
\end{flashcard}

\subsubsectioncard{monitor spend}

\begin{flashcard}[Definition]{Azure Cost Management}
    \vspace*{\stretch{1}}

    \begin{itemize}
        \item Übersicht über Kosten
        \item Filtern und Gruppieren nach Service-Art, Resourcegruppe, Region, \ldots
        \item verschiedene Zeiträume, z.\,B. aktueller Monat (Rechnungszeitraum)
    \end{itemize}
    Im Portal: \texttt{Kostenverwaltung + Abrechnung | Kostenverwaltung | Kostenanalyse}
    \vspace*{\stretch{1}}
\end{flashcard}

\begin{flashcard}[Definition]{Budgetwarnungen}
    \vspace*{\stretch{1}}
    \begin{itemize}
        \item automatische Warnung sobald monatliches Budget erreicht ist
        \item für Kosten und Verbrauch
        \item Warnung im Portal bei Kostenwarnungen oder per Email
        \item auch Warnungen pro Abteilung
    \end{itemize}
    \vspace*{\stretch{1}}
\end{flashcard}

\subsubsectioncard{report on spend}

\begin{flashcard}[Definition]{Export von Kostenübersicht}
    \vspace*{\stretch{1}}
    \begin{itemize}
        \item Exporte nach: Abonnements, Resourcengruppen, Konten, Abteilungen, Registrierungen
        \item Zeitpläne: täglich aktueller Monat, wöchentlich, monatlich, benutzerdefiniert
        \item export in Speicherkonto
        \item Im Portal:\newline
            \texttt{Kostenanalyse | Einstellungen | Konfiguration | Exporte }
    \end{itemize}
    \vspace*{\stretch{1}}
\end{flashcard}

\begin{flashcard}[Definition]{Rechnung}
    \vspace*{\stretch{1}}
    Die (monatliche) Rechnung zeigt detaillierte Übersichten, welche Ressourcen welche Kosten verursachen.
    \vspace*{\stretch{1}}
\end{flashcard}

\subsubsectioncard{utilize Log Search query functions}

\subsubsectioncard{view Alerts in Azure Monitor logs}

\subsubsectioncard{visualize diagnostics data using Azure Monitor Workbooks}
