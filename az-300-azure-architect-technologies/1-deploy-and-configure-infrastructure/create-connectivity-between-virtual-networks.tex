\subsectioncard{Verbindungen zwischen virtuellen Netzwerken}

\subsubsectioncard{create and configure Vnet peering}

\begin{flashcard}[Definition]{VNet-Peering}
    \vspace*{\stretch{1}}
    \begin{itemize}
        \item peering virtueller Netzwerke: VNETs in der gleichen Region
        \item peering globaler virtueller Netzwerke: VNETs in unterschiedlichen Regionen
        \item Peering ist \emph{nicht} transitiv
        \item überlappende Adressräume sind nicht möglich
        \item Limit: 500 peerings pro Netzwerk
    \end{itemize}
    \vspace*{\stretch{1}}
\end{flashcard}

\begin{flashcard}[Definition]{VNET Peering-Anwendungsfälle}
    \vspace*{\stretch{1}}
    \begin{itemize}
        \item einfachste Möglichkeit, Azure-Netzwerke zu verbinden\newline
            einfacher, als z.\,B. Verbindung mit lokalen Netzwerken
        \item über Peering und Gateways verbundene Netze \emph{bevorzugen} die Peering-Verbindung
        \item Geräte in VNETs als wenn sie im gleichen Netz wären
        \item Verbindung in Azure: schnell, nicht von Verbindung, Gateways, \ldots abhängig
    \end{itemize}
    \vspace*{\stretch{1}}
\end{flashcard}

\begin{flashcard}[Definition]{VNET Peering herstellen}
    \vspace*{\stretch{1}}
    Voraussetzungen:
    \begin{itemize}
        \item wenn in verschiedenen Abonnements, Rolle \emph{Netzwerkmitwirkender} notwendig
        \item beide Netzwerke müssen geegenseitig verbunden werden\newline
            sonst: Status ist \emph{Initiiert}
    \end{itemize}

    \begin{itemize}
        \item CLI: \texttt{az network vnet peering create}\newline
            CLI erfordert zwei Peerings zu erstellen
        \item Azure Portal: Im VNet-Blade \texttt{Einstellungen | Peerings | Hinzufügen}
        \begin{itemize}
            \item Name
            \item Resourcen-ID oder Abonnement + VNET
            \item VNET-Zugriff zulassen
        \end{itemize}
        Im Portal wird die Verbindung automatisch reziprok erstellt!
    \end{itemize}

    \vspace*{\stretch{1}}
\end{flashcard}

\begin{flashcard}[Definition]{VNET-Peering konfigurieren}
    \vspace*{\stretch{1}}
    Gatewaytransit
    \begin{itemize}
        \item erlaubt weiterleitung von Daten (also ähnlich wie Transitivität)
        \item Netzwerk mit Gateway: \emph{Gatewaytransit zulassen}\newline
            Transit-Verkehr darf das Gateway nutzen.
        \item anderes Netzwerk: \emph{Remotegateways benutzen}\newline
            Verkehr darf das Remote-Gateway benutzen
    \end{itemize}
    \vspace*{\stretch{1}}
\end{flashcard}

\subsubsectioncard{create and configure Vnet to Vnet connections}

\begin{flashcard}[Definition]{Transit über virtuelle Netzwerke}
    \vspace*{\stretch{1}}
    Eigenschaften einer Peer-Verbindung:
    \begin{itemize}
        \item weitergeleiteten Verkehr zulassen
        \item[$\Rightarrow$] erlaubt Verkehr aus anderen Netzwerken als dem Quell-Netzwerk weiterzuleiten
        \item ermöglicht Reihenschaltung von Netzwerken
    \end{itemize}
    \vspace*{\stretch{1}}
\end{flashcard}

\begin{flashcard}[Definition]{Globale VNet-Peerings}
    \vspace*{\stretch{1}}
    \begin{itemize}
        \item Resourcen in virtuellem netz können \emph{nicht} mit der Front-end-IP eines internen Load Balancers im Basic SKU kommunizieren
        \item Load Balancer kann generell eingeschränkt sein bei globalem Peering
    \end{itemize}
    \vspace*{\stretch{1}}
\end{flashcard}

\subsubsectioncard{verify virtual network connectivity}

\begin{flashcard}[Definition]{Verifizieren der Verbindung}
    \vspace*{\stretch{1}}
    \begin{itemize}
        \item CLI Liste der Peerings:\newline
            \texttt{az network vnet peering list --resource-group \ldots --vnet-name \ldots}
        \item Routen:\newline
            \texttt{az network nic show-effective-route-table --resource-group \ldots --name \ldots}
        \item Routen im Portal:\newline
            \texttt{VM | Netzwerkschnittstelle | Support + Problembehandlung | Effektive Routen}
        \item Ausgabe:
            \texttt{Default   Active   10.2.0.0/16       VNetPeering} falls verbunden
    \end{itemize}
    \vspace*{\stretch{1}}
\end{flashcard}

\subsubsectioncard{create virtual network gateway}
