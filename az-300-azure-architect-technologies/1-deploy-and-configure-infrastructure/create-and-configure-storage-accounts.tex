\subsectioncard{Erstellen und Konfigurieren von Speicherkonten}

\begin{flashcard}[Definition]{Datenklassifizierung}
    \vspace*{\stretch{1}}
    \begin{itemize}
        \item strukturiert\newline
        $\Rightarrow$ relationale Daten in Tabellen, Business-Daten, SQL
        \item teilweise strukturiert\newline
        $\Rightarrow$ getaggte Daten, unterschiedliche Klassifizierer, JSON
        \item unstrukturiert\newline
        $\Rightarrow$ Texte, Bilder, Videos, \ldots
    \end{itemize}
    \vspace*{\stretch{1}}
\end{flashcard}

\begin{flashcard}[Definition]{Anforderungen an Speicher ermitteln}
    \vspace*{\stretch{1}}
    \begin{itemize}
        \item Latenz
        \item Anzahl Operation pro Zeiteinheit
        \item Komplexität einzelner Anfragen
        \item notwendige Filtermöglichkeiten
    \end{itemize}
    \vspace*{\stretch{1}}
\end{flashcard}

\subsubsectioncard{configure network access to the storage account}

\begin{flashcard}[Definition]{Netzwerkzugriff auf Speicherkonten}
    \vspace*{\stretch{1}}
    \begin{itemize}
        \item Konnektivitätsmethode:
        \begin{itemize}
            \item öffentlich, alle Netzwerke
            \item öffentlich, ausgewählte Netzwerke\newline
            Nur ausgewählte virtuelle Netzwerke können auf den Speicher zugreifen
            \item privat
        \end{itemize}
        \item Netzwerkrouting
        \begin{itemize}
            \item Methode für Routing
            \item Microsoft-Routing
            \item Internet-Routing
        \end{itemize}

    \end{itemize}
    \vspace*{\stretch{1}}
\end{flashcard}

\subsubsectioncard{create and configure sotrage account}

\begin{flashcard}[Definition]{Azure Storage}
    \vspace*{\stretch{1}}
    Die Speicherdienste, die in einem Speicherkonto gespeichert werden
    \begin{itemize}
        \item Azure-Blob
        \item Azure-Files
        \item Azure-Warteschlangen
        \item Azure-Tabellen
    \end{itemize}
    Nicht dazu gehören weitere Speicherdienste
    \begin{itemize}
        \item Azure SQL
        \item Cosmos DB
    \end{itemize}
    \vspace*{\stretch{1}}
\end{flashcard}

\begin{flashcard}[Definition]{Speicherkonto}
    \vspace*{\stretch{1}}
    \begin{itemize}
        \item Azure Resourcen (in Resourcengruppe)
        \item kann mehrere Datendienste enthalten
        \item kombiniert Einstellungen, die für sämtliche enthaltenen Dienste gelten
    \end{itemize}
    \vspace*{\stretch{1}}
\end{flashcard}

\begin{flashcard}[Definition]{Speicherkonto-Typen}
    \vspace*{\stretch{1}}
    \begin{itemize}
        \item General Purpose v1\newline
            vorherige Version, unterstützt nicht alle Features
        \item General Purpose v2\newline
            sämtliche Features (inklusive Data Lake), empfohlen
        \item Blob\newline
            spezieller Speicher, der nur Blob unterstützt\newline
            bessere Leistung
    \end{itemize}
    v1 und v2 unterstützen Blob, Files, Queues, Tables
    \vspace*{\stretch{1}}
\end{flashcard}

\begin{flashcard}[Definition]{Erstellen eines Speicherkontos}
    \vspace*{\stretch{1}}
    \begin{itemize}
        \item Benötigt:
            \begin{itemize}
                \item Name. 3-24 Zeichen aus \texttt{[a-z0-9]}, weltweit eindeutig
                \item Bereitstellungsmodell: ARM (Azure Resource Manager), klassich\newline
                klassisch ist veraltet und wird schlecht unterstützt. Nicht verwenden.
                \item Kontoart: General Purpose v1/v2, Blob
            \end{itemize}
        \item Im Portal:
        \item CLI: \texttt{az storage account create --name name --resource-group grp\\ --sku Standard\_GRS --kind StorageV2}
    optional \texttt{--hierarchical-namespace true}
    \end{itemize}
    \vspace*{\stretch{1}}
\end{flashcard}

\begin{flashcard}[Definition]{Konfigurationsoptionen von Speicherkonten}
    \vspace*{\stretch{1}}
    \begin{itemize}
        \item Abonnement
        \item Standort
        \item Leistung: Standard, Premium
        \item Daten-Replikation: Datenkopien zum Schutz vor Hardwarefehlern und Katastrophen. LRS, ZRS, GRS, GZRS, RA-GZRS, ...
        \item Zugriffsebene: heiß, kalt, archiviert
        \item Sichere Übertragung: HTTPS zwangsweise
        \item IP-Filter: Viertuelle Netzwerke, oder IP-Adresse-Bereiche
        \item Hierarchischer Speicher
        \item vorläufiges Löschen (Blob, Files), Versionierung
    \end{itemize}
    \vspace*{\stretch{1}}
    $\Rightarrow$ jeder Unterschied erfordert eigenes Speicherkonto!
\end{flashcard}

\begin{flashcard}[Definition]{Eigenschaften von Azure File Storage}
    \vspace*{\stretch{1}}
    Standard:
    \begin{itemize}
        \item 100 TiB
        \item maximual 1000 IOPS
    \end{itemize}
    \vspace*{\stretch{1}}
\end{flashcard}

\subsubsectioncard{generate Shared Access Signature}

\begin{flashcard}[Definition]{SAS Token}
    \vspace*{\stretch{1}}
    Generierung:
    \begin{itemize}
        \item im Speicherkonto-Blade in Azure
        \item über Storage Explorer
        \item per command line\newline
            \texttt{az storage container generate-sas}
        \item als code (Signieren des Strings mit dem Zugriffsschlüssel)
    \end{itemize}
    \vspace*{\stretch{1}}
\end{flashcard}

\subsubsectioncard{implement Azure AD authentication for storage}

\begin{flashcard}[Definition]{Azure AD-Authorisierung für Speicher}
    \vspace*{\stretch{1}}
    \begin{itemize}
        \item empfohlener Zugriff (sicherer als SAS)
        \item für General Purpose Speicher und Blob für Blobs und Warteschlangen
        \item für Files: preview
        \item Table: not supported
    \end{itemize}
    \vspace*{\stretch{1}}
\end{flashcard}

\begin{flashcard}[Definition]{Zugriffskontrolle auf Speicher mit Azure AD}
    \vspace*{\stretch{1}}
    RBAC: verschiedene Rollen für Verwaltung und Datenzugriff
    \begin{itemize}
        \item Verwaltung:
            \begin{itemize}
                \item Besitzer, Mitwirkender, Speicherkontomitwirkender
            \end{itemize}
        \item Datenzugriff:
            \begin{itemize}
                \item Blobs: Mitwirkender an Blobdaten, Blobdatenleser
                \item Blobdatenbesitzer: Vollzugriff, POSIX-Zugriff auf Data Lake
                \item Blob-Delegierer: erstellt user SAS key
                \item Warteschlangen: Mitwirkender/Absender/Verarbeiter/Leser von Warteschlangendaten
            \end{itemize}
        \item Schlüssel:\newline
            Berechtigung \texttt{Microsoft.Storage/storageAccounts/listKeys/action} erlaubt Erstellung eines Shared Keys $\Rightarrow$ kann sich selbst Lesezugriff selbst verschaffen
    \end{itemize}
    \vspace*{\stretch{1}}
\end{flashcard}

\begin{flashcard}[Definition]{Umfang der Rechte}
    \vspace*{\stretch{1}}
    \begin{itemize}
        \item Container (geringste Berechtigung)
        \item Warteschlange (geringste Berechtigung)
        \item Speicherkonto
        \item Resourcengruppe
        \item Abonnement
        \item Managmentgruppe
    \end{itemize}
    \vspace*{\stretch{1}}
\end{flashcard}

\subsubsectioncard{install and use Azure Storage Explorer}

\begin{flashcard}[Definition]{Azure Storage Explorer}
    \vspace*{\stretch{1}}
    \begin{itemize}
        \item externes Tool für Windows, Linux, macOS, Azure Portal (als Preview)
        \item ermöglicht Ansicht/Bearbeitung von Speicherkonten, ähnlich Windows Explorer
        \item mehrere Speicherkonten und Abonnements können Verbunden werden
        \item unterstützt CosmosDB und Data Lake
        \item unterstützt lokale Speicher-Emulatoren
        \item hochladen, verschieben, runterladen
    \end{itemize}
    \vspace*{\stretch{1}}
\end{flashcard}

\begin{flashcard}[Definition]{Features Storage Explorer}
    \vspace*{\stretch{1}}
    \begin{itemize}
        \item Dateien/Blobs kopieren
        \item SAS-Generierung
        \item Snapshot-Erstellung
        \item Upload/Download Blobs oder Dateien
    \end{itemize}
    \vspace*{\stretch{1}}
\end{flashcard}


\begin{flashcard}[Definition]{Storage Explorer Verbinden}
    \vspace*{\stretch{1}}
    \begin{itemize}
        \item Azure-Account direkt verbinden\newline
        Zugriff auf Blobs, Files, Warteschlangen, Tabellen
        \item CosmosDB einbinden mit Verbindungszeichenfolge (Key aus Azure abrufen), Typ SQL oder Tabelle\newline
        Erstellen von Datenbanken, Sammlungen und Dokumenten. Editieren.
        \item Data Lake Gen2\newline
        Features im Storage Explorer wie Blobs.
    \end{itemize}
    \vspace*{\stretch{1}}
\end{flashcard}

\subsubsectioncard{manage access keys}

\begin{flashcard}[Definition]{Zugriffsschlüssel}
    \vspace*{\stretch{1}}
    \begin{itemize}
        \item müssen geheim gehalten werden
        \item entsprechen Benutzername und Kennwort\newline
            Speicherkontoname=Nutzer, Schlüssel=Kennwort
        \item Zwei Schlüssel ermöglichen Rotation
    \end{itemize}
    \vspace{1cm}
    Abruf von Zugriffsschlüsseln
    \begin{itemize}
        \item über Azure Portal (im Blade für das Speicherkonto)
        \item über Azure CLI\newline
            \texttt{az storage account key list}
    \end{itemize}
    \vspace*{\stretch{1}}
\end{flashcard}

\begin{flashcard}[Definition]{Zugriff auf ein Speicherkonto}
    \vspace*{\stretch{1}}
    Mit SAS-Token
    \begin{itemize}
        \item verschlüsselt mit Zugriffsschlüssel
        \item legen Zugriffsrechte fest:
        \begin{itemize}
            \item API-Endpunkt (Blob, File, Tabelle, Warteschlange)
            \item Rechte (Lesen, Schreiben, Auflisten, ...)
        \end{itemize}
    \end{itemize}

    Direkt mit Zugriffsschlüssel
    \begin{itemize}
        \item ermöglicht Vollzugriff
        \item Verbindungszeichenfolge:\newline
            \texttt{DefaultEndpointsProtocol=https;AccountName={\ldots};\\AccountKey={\ldots};EndpointSuffix=core.windows.net}
    \end{itemize}
    \vspace*{\stretch{1}}
\end{flashcard}

\begin{flashcard}[Definition]{Zugriffsschlüssel}
    \vspace*{\stretch{1}}
    \begin{itemize}
        \item zwei Schlüssel: primär und sekundäre
        \item ermöglicht key rotation
            \begin{enumerate}
            \item Primary -> Secondary
            \item Recreate Primary
            \item Secondary -> Primary
            \item Recreate Secondary
            \end{enumerate}
        \item Zugriffsschlüssel müssen geheim bleiben\newline
            im Unterschied zu SAS-Token
        \item Verwaltung in CLI:\newline
            \texttt{az storage account keys ...}
    \end{itemize}
    \vspace*{\stretch{1}}
\end{flashcard}

\begin{flashcard}[Definition]{Azure AD-Zugriff auf Speicherkonto}
    \vspace*{\stretch{1}}
    \begin{itemize}
        \item jeder Zugriff auf Speicherkonto ist über Azure AD authentifiziert
        \item Ausnahme: es kann öffentlicher Lesezugriff eingestellt werden (im Account)
        \item RBAC-Rollen: Betragender zum Speicherkonto
        \item Rollenverwaltung außerdem auch für Verwaltung, nicht nur Datenzugriff
        \item Rollen für:
            \begin{itemize}
                \item ganzes Speicherkonto
                \item einzelne Container
                \item Warteschlangen
            \end{itemize}
    \end{itemize}
    \vspace*{\stretch{1}}
\end{flashcard}

\subsubsectioncard{monitor Activity log by using Azure Monitor logs}

\begin{flashcard}[Definition]{Logging von Speicherkonten}
    \vspace*{\stretch{1}}
    \begin{itemize}
        \item aktiviere Logs als "Diagnoseeinstellungen 2.0": alle Aktivitäten
        \item Azure CLI + PowerShell
        \item gespeichert als \texttt{\$logs}-Container im Blob-Speicher
    \end{itemize}
    \vspace*{\stretch{1}}
\end{flashcard}

\begin{flashcard}[Definition]{Aktivitätsprotokoll-Warnungen}
    \vspace*{\stretch{1}}
    \begin{itemize}
        \item Arten:
            \begin{itemize}
                \item Vorgänge: werden bei Änderungen von Ressourcen ausgelöst
                \item Service-Health
            \end{itemize}
        \item Attribute:
            \begin{itemize}
                \item Kategorie: Verwaltung, Dienstintegrität, Autoskalierung, Richtlinie, Empfehlung
                \item Bereich: Abbonement/Ressourcengruppe/Ressource
                \item Ressourcengruppe: Speicherort der Regel
                \item Ressourcentyp:
                \item Vorgangsname
                \item Grad: 0 (Ausführlich) - 4 (Kritisch)
                \item Status: Gestartet, Fehlgeschlagen, Erfolgreich
                \item Initiiert von:
            \end{itemize}
    \end{itemize}
    \vspace*{\stretch{1}}
\end{flashcard}

\begin{flashcard}[Definition]{Aktivitätsprotokollregel erstellen}
    \vspace*{\stretch{1}}
    \begin{itemize}
        \item Signaltyp \texttt{Aktivitätsprotokoll}
        \item aus Liste der Warnungen auswählen
    \end{itemize}
    \vspace*{\stretch{1}}
\end{flashcard}

\subsubsectioncard{implement Azure storage replication}

\begin{flashcard}[Definition]{Speicher-Redundanz in Azure Speicherkonten}
    \vspace*{\stretch{1}}
    \begin{itemize}
        \item jeder Speicher in Azure ist immer redundant
        \item verschieden Stufen möglich
        \begin{itemize}
            \item LRS: Lokal redundanter Speicher: 3 Kopien in einem Datencenter\newline
                Schutz vor Hardwarefehlern (Rack + Disk)
            \item ZRS: Zonenredundanter Speicher: 3 Kopien Verfügbarkeitszonen\newline
                Schutz vor komplettem Datencenter-Ausfall
            \item GRS: Geo-redundanter Speicher: 6 Kopien in zwei Regionen (jeweils LRS)\newline
                Schutz vor Ausfall einer kompletten Region
            \item RA-GRS: Geo-redundanter Speicher mit Lesemöglichkeit in der zweiten Region\newline
            Angehängtes \texttt{-secondary} an den Namen in der URL
        \end{itemize}
    \end{itemize}
    \vspace*{\stretch{1}}
\end{flashcard}

\begin{flashcard}[Definition]{Speicher-Redundanz herstellen}
    \vspace*{\stretch{1}}
    \begin{itemize}
        \item bereits Angabe bei der Erstellung (Portal, CLI, \ldots)\newline
            Parameter \texttt{--sku}, z.\,B. \texttt{Standard\_RAGRS}
        \item Wechsel von LRS zu ZRS
    \end{itemize}
    \vspace*{\stretch{1}}
\end{flashcard}

\subsubsectioncard{implement Azure storage account failover}

\begin{flashcard}[Definition]{Speicherkonto-Failover}
    \vspace*{\stretch{1}}
    \begin{itemize}
        \item feste sekundäre Region für eine Region
        \item Daten werden (asynchron) in der zweiten Region gesichert
        \item delay wenige Minuten, aber ohne SLA
        \item falls primäre Region ausfällt, muss Failover durchgeführt werden
        \begin{itemize}
            \item Start über Portal oder CLI
            \item DNS-Einträge werden aktualisiert
            \item nach Failover ist das Speicherkonto LRS
            \item sobald Problem behoben ist, muss erneut auf (RA-)GRS gesetzt werden
        \end{itemize}
        \item Failover kann zu Datenverlust führen! (letztlich konsistent)
    \end{itemize}
    Statusabfrage: \texttt{az storage account show --name \ldots\\--query "[statusOfPrimary, statusOfSecondary]"}

    Durchführung: \texttt{az storage account failover --name \ldots}
    \vspace*{\stretch{1}}
\end{flashcard}

\begin{flashcard}[Definition]{Lesen von Geo-redundanten Daten}
    \vspace*{\stretch{1}}
    \begin{itemize}
        \item Leseanforderung bei mehreren Regionen mit Priorisierung:
        \begin{itemize}
            \item \texttt{PrimaryOnly} (standard)
            \item \texttt{PrimaryThenSecondary}
            \item \texttt{SecondaryOnly}
            \item \texttt{SecondaryThenPrimary}
        \end{itemize}
        \item Sicherungsmuster sollte von \texttt{PrimaryThenSecondary} auf \texttt{SecondaryOnly} wechseln, falls Probleme erkannt werden.
    \end{itemize}
    \vspace*{\stretch{1}}
\end{flashcard}
