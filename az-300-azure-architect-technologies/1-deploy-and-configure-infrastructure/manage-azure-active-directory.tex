\subsectioncard{Azure Active Directory verwalten}

\begin{flashcard}[Definition]{Azure Active Directory Lizenzen}
    \vspace*{\stretch{1}}
    \begin{itemize}
        \item Free: grundlegene Features
            \begin{itemize}
                \item Benutzer und Gruppenverwaltung
                \item lokale Synchronisierung
                \item Self-Service-Kennwortrücksetzung
                \item Single Sign On
            \end{itemize}
        \item Premium P1: lokale und Cloudbasierte Resourcen werden unterstützt, MFA, Kennwortrückschreiben (Azure Kennwörter werden ins lokale Active Directory rückübermittelt) (Andere Richtung ist in Free enthalten)
            \begin{itemize}
                \item automatische Gruppenverwaltung
                \item Identitätsverwltung: Identity Manager
                \item Self-Service-Kennwortrücksetzung auch für lokale Nutzer
            \end{itemize}
        \item Premium P2:
                unterstützt \emph{Identity Protection} und \emph{Privileged Identity Management} (Adminstratorverfolgung und -zugriff)
        \item nutzungsbasierte Zahlung: bestimmte Features
    \end{itemize}
    \vspace*{\stretch{1}}
\end{flashcard}

\begin{flashcard}[Definition]{Active Directory Enterprise Roaming}
    \vspace*{\stretch{1}}
    \begin{itemize}
        \item Synchronisierung von Nutzer-Einstellungen für mehrere Geräte (Windows)
        \item Aktivieren:\newline
            \texttt{Azure Active Directory | Devices | Enterprise State Roaming}
            \begin{itemize}
                \item alle Einstellungen synchronisieren
                \item Benutzerdefinierte
                \item keine Einstellungen synchronisieren
            \end{itemize}
    \end{itemize}
    \vspace*{\stretch{1}}
\end{flashcard}

\begin{flashcard}[Definition]{Persönliche Benachrichtigung}
    \vspace*{\stretch{1}}
    \begin{itemize}
        \item Im Portal:\newline
            \texttt{Azure Active Directory | Identity Governance | Nutzungsbedingungen }
    \end{itemize}
    \vspace*{\stretch{1}}
\end{flashcard}

\begin{flashcard}[Definition]{Privileged Identity Management}
    \vspace*{\stretch{1}}
    \begin{itemize}
        \item Erlaubt einmaliges Anmelden mit höherer Berechtigung
    \end{itemize}
    \vspace*{\stretch{1}}
\end{flashcard}

\subsubsectioncard{add custom domains}

\begin{flashcard}[Definition]{Benutzerdefinierte Domain}
    \vspace*{\stretch{1}}
    \begin{itemize}
        \item speichern in Azure Active Directory
        \item Standard-Domain-Name: \emph{domain.onmicrosoft.com}\newline
            $\Rightarrow$ kann nicht geändert werden
        \item ein benutzerdefinierter kann hinzugefügt werden
        \item Voraussetzungen:
            \begin{itemize}
                \item Eine Domain ist vorhanden
                \item Directory in Azure AD
            \end{itemize}
        \item \texttt{Custom domain names | + Add custom domain} im Active Directory-Blade\newline
            Domainname inklusive Top-Level-Domain
    \end{itemize}
    \vspace*{\stretch{1}}
\end{flashcard}

\begin{flashcard}[Definition]{Eine benutzerdefinierte Domain verifizieren}
    \vspace*{\stretch{1}}
    Standardmäßig erstellte benutzerdefinierte Domain ist nicht verifiziert
    \begin{itemize}
        \item Beim Registrar muss ein Record gespeichert werden
            \begin{itemize}
                \item \texttt{TXT}: Besitznachweis
                \item \texttt{MX}: Mail
            \end{itemize}
        \item Werte eintragen (von Azure zum Registrar kopieren):
            \begin{itemize}
                \item Alias/Hostname
                \item Destination
                \item TTL (Standard 3600, 1 Stunde)
            \end{itemize}
    \end{itemize}
    $\Rightarrow$ wenn alles eingetragen ist, kann verifiziert werden
    \vspace{1cm}
    ! Ein Domainname kann nur für ein AD-Verzeichnis genutzt werden\newline
    (aber mehrere Domains pro Verzeichnis)
    \vspace*{\stretch{1}}
\end{flashcard}

\subsubsectioncard{configure Azure AD Identity Protection}

\begin{flashcard}[Definition]{Identity Protection}
    \vspace*{\stretch{1}}
    \begin{itemize}
        \item automatische Erkennung von Risiken
        \item Warnung bei risikobehaftetem Zugriff
        \item exportieren von Daten, z.\,B. zum Erstellen von Berichten oder zur Analyse
        \item Risikorichtlinien mit Anweisungen für Bedrohungen
            \begin{itemize}
                \item Mehrfaktor-Authentifizierung erzwingen
            \end{itemize}
    \end{itemize}
    \vspace*{\stretch{1}}
\end{flashcard}

\subsubsectioncard{configure Azure AD Join}

\subsubsectioncard{configure self-service password reset}

\begin{flashcard}[Definition]{Self Service Kennwortrücksetzung}
    \vspace*{\stretch{1}}
    \begin{itemize}
        \item Benutzer können ihr Kennwort selbst ändern
            \begin{itemize}
                \item entweder, wenn sie eingeloggt sind und es ändern (immer)
                \item wenn sie keinen Zugriff mehr haben (nur wenn aktiviert, nicht in Free)
            \end{itemize}
        \item Azure verwaltet Authentifizierung und Sicherheit (z.\,B. Captcha)
            \begin{itemize}
                \item Mehrere sicherheitsmethoden: App, Code, Mail, Telefon, Sicherheitsfrage
                \item Mindestzahl an Methoden (1 oder 2)
                \item nicht alle für Administratorn verfügbar (und mindestens 2)!
            \end{itemize}
        \item Rückschreiben in lokales AD nur mit Azure AD Premium
    \end{itemize}
    \vspace*{\stretch{1}}
\end{flashcard}

\subsubsectioncard{implement conditional access policies}

\begin{flashcard}[Definition]{Legacy-Protokolle}
    \vspace*{\stretch{1}}
    Unterstützt von Azure Active Directory:
    \begin{itemize}
        \item alte Office-Versionen
        \item Mail-Protokolle: \texttt{IMAP}, \texttt{SMTP}, \texttt{POP3}
    \end{itemize}
    \vspace{1cm}
    trotz mehrstufiger Authentifizierung können Legacy-Protokolle noch genutzt werden\newline
    $\Rightarrow$ gefährliche logins
    \vspace*{\stretch{1}}
\end{flashcard}

\begin{flashcard}[Definition]{Bedingte Zugriffsrichtlinien}
    \vspace*{\stretch{1}}
    Nicht jeder Zugriff wird mit zweitem Faktor abgesichert

    \vspace{1cm}
    Abhängig von verschiedenen Faktoren mehr Sicherheit einfordern

    \vspace{1cm}
    Teile einer Richtlinie:
    \begin{itemize}
        \item Name
        \item Zuweisungen
        \item Zugriffs-Kontrolle
    \end{itemize}

    \vspace*{\stretch{1}}
\end{flashcard}

\begin{flashcard}[Definition]{Bedingte Zugriffsrichtlinien zuweisen}
    \vspace*{\stretch{1}}
    \begin{itemize}
        \item für Benutzer oder Gruppen
        \item mit Cloud-Apps und -Aktionen verbinden (zum Auslösen der Richtlinie)\newline
            $\Rightarrow$ einfordern zusätzlicher Aktionen, z.\,B. MFA
        \item weitere Bedingungen:\newline
            Plattform, Standort, \ldots
    \end{itemize}
    \vspace*{\stretch{1}}
\end{flashcard}

\begin{flashcard}[Definition]{Bedingungen für Zugriffsrichtlinie}
    \vspace*{\stretch{1}}
    \begin{itemize}
        \item Geräteplattform
        \item Standort
            \begin{itemize}
                \item Alle
                \item vertrauenswürdig
                \item ausgewählte Standorte (IP-Bereiche, benannt/Länder)
            \end{itemize}
        \item Apps
            \begin{itemize}
                \item Browser
                \item Apps
                    \begin{itemize}
                        \item Modern, mit Active Directory Authentication Library (ADAL)
                        \item Exchange ActiveSync
                        \item Andere
                    \end{itemize}
            \end{itemize}
        \item Gerätestatus
    \end{itemize}
    \vspace*{\stretch{1}}
\end{flashcard}

\begin{flashcard}[Definition]{Bedingte Zugriffsrichtlinien Zugriffs-Kontrolle}
    \vspace*{\stretch{1}}
    Legen fest, welche Kontrolle erforderlich ist
    \begin{itemize}
        \item blockieren
        \item verschiedene Methoden
        \item eine von mehreren, oder alle erzwingen
    \end{itemize}
    \vspace{1cm}
    Mögliche Zugriffsmethoden:
    \begin{itemize}
        \item mehrstufige Authentifizierung erforderlich
        \item Gerät muss konform sein
        \item Hybrid Azure AD-Gerät
        \item zugelassene Apps
        \item Schutz-Policy
    \end{itemize}

    \vspace*{\stretch{1}}
\end{flashcard}

\begin{flashcard}[Definition]{Baseline-Richtlinien}
    \vspace*{\stretch{1}}
    \begin{itemize}
        \item Standardwerte für Sicherheit
        \item mehrere zur Auswahl
            \begin{itemize}
                \item MFA für alle Nutzer verpflichtend
                \item MFA für Administratoren
                \item \emph{Alle} legacy logins verbieten. Für spezifischere Regeln, Conditional Access
                \item MFA für Nutzer, falls notwendig
                \item Schutz von Zugriff mit hohen Rechten: z.\,B. den Resource Manager nutzen
            \end{itemize}

    \end{itemize}
    \vspace*{\stretch{1}}
\end{flashcard}

\subsubsectioncard{manage multiple directories}

\subsubsectioncard{perform an access review}

\begin{flashcard}[Definition]{Azure AD Anmeldeprotokolle}
    \vspace*{\stretch{1}}
    \begin{itemize}
        \item Auflistung aller Anmeldungen, nach App und Benutzer
        \item Im Portal: \newline
            \texttt{Azure Active Directory | Anmeldungen}
        \item Filterbar nach Client-App, Zeitraum, Status, \ldots
    \end{itemize}
    \vspace*{\stretch{1}}
\end{flashcard}
