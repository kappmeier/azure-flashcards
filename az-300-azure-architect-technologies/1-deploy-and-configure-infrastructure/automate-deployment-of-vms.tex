\subsectioncard{Die Bereitstellung von VMs automatisieren}

\begin{flashcard}[Definition]{Deployment-Plattformen für Source Code}
    \vspace*{\stretch{1}}
    \begin{itemize}
        \item BitBucket
        \item GitHub
        \item Azure Repo
        \item Team Foundation Version Control
    \end{itemize}
    \vspace*{\stretch{1}}
\end{flashcard}

\subsubsectioncard{modify Azure Resource Manager template}

\begin{flashcard}[Definition]{Resource-Manager-Vorlage}
    \vspace*{\stretch{1}}
    \begin{itemize}
        \item JSON-Datei
        \item Vereinheitlichen VMs und Konfiguration (über Erweiterungen)
        \item können mit Variablen und Paramtern konfiguriert werden\newline
            $\Rightarrow$ benötigt Parameter-Datei
    \end{itemize}
    \vspace*{\stretch{1}}
\end{flashcard}

\begin{flashcard}[Definition]{Komponenten einer Resource-Manager-Vorlage}
    \vspace*{\stretch{1}}
    \begin{itemize}
        \item Parameter: vom Nutzer anzugebende Werte, ggf. mit Defaults\newline
            \texttt{"parameters"}
        \item Variablen: Werte, die mehrmals verwendet werden können\newline
            \texttt{"variables"}
        \item Funktionen: Mehrfach genutzte Funktionen (mit Parametern)\newline
            \texttt{"functions"}
        \item Ressourcen: Azure-Ressourcen, die bereitgestellt werden\newline
            \texttt{"resources"}
        \item Ausgaben: Generierte Informationen der Bereitstellung, z.\,B. IP-Adresse\newline
            \texttt{"outputs"}
    \end{itemize}
    \vspace*{\stretch{1}}
\end{flashcard}

\subsubsectioncard{configure Location of new VMs}

\subsubsectioncard{configure VHD template}

\begin{flashcard}[Definition]{VHD-Vorlagen}
    \vspace*{\stretch{1}}
    \begin{itemize}
        \item Virtuelle Festplatte, die in Azure gespeichert wird: hochverfügbar, managed, \ldots
        \item VHD Image: Festplatte, die ein Betreibssystem für VMs enthält
            \begin{itemize}
                \item Ubuntu und andere Linux-Varianten
                \item Windows Server, SQL
                \item zahllose Varianten
            \end{itemize}
        \item generalisiertes Image:\newline
            modifiziertes Image, ohne Nutzerinformationen
        \item spezialisiertes Image:\newline
            Kopie einer Live-VM, z.\,B. als Backup
    \end{itemize}
    \vspace*{\stretch{1}}
\end{flashcard}

\begin{flashcard}[Definition]{Unterstützte Formate}
    \vspace*{\stretch{1}}
    \begin{itemize}
        \item[!] nur VHD kann hochgeladen werden
        \item VHDX muss konvertiert werden
        \item Größe muss fest sein (kann konvertiert werden)
        \item Version 1 und 2 VMs
        \item maximale größe 2 T(i?)B für Generation 2
    \end{itemize}
    \vspace*{\stretch{1}}
\end{flashcard}

\begin{flashcard}[Definition]{Custom Image erstellen}
    \vspace*{\stretch{1}}
    \begin{enumerate}
        \item Zwei Möglichkeiten:
            \begin{itemize}
                \item VM starten mit existierendem Basisimage
                \item mit Hyper-V von Grund auf neu erstellen
            \end{itemize}
        \item In das System einloggen, ggf. konfigurieren
        \item Generalisieren im System vorbereiten
            \begin{itemize}
                \item Linux: waagent
                \item Windows: Sysprep
            \end{itemize}
        \item Generalisieren: \newline
                \texttt{az vm deallocate}
                \texttt{az vm generalize}
        \item Image erstellen:
            \begin{itemize}
                \item Portal: \texttt{Capture} im VM-Blade der generalisierten VM
                \item CLI: \texttt{az vm image create --source vm-name}
            \end{itemize}

    \end{enumerate}
    \vspace*{\stretch{1}}
\end{flashcard}

\subsubsectioncard{deploy from template}

\begin{flashcard}[Definition]{Ressourcen von einer Eine Resource-Manager-Vorlage bereitstellen}
    \vspace*{\stretch{1}}
    \begin{itemize}
        \item Benötigt:
            \begin{itemize}
                \item Resourcegruppe
                \item Vorlagen-Datei
                \item Parameter: \texttt{--parameters name=value name2=aValue}
            \end{itemize}
        \item \emph{Vor} dem Bereitstellen validieren:\newline
            CLI: \texttt{az deployment group validate}, benötigt auch Parameter
        \item Portal: \texttt{Neue Ressource} + Suchen nach \texttt{Template} und eigenes Template erstellen.
        \item CLI: \texttt{az deployment group create}
        \item Während der Bereitstellung: kann im Portal als Bereitstellung verfolgt werden
    \end{itemize}
    \vspace*{\stretch{1}}
\end{flashcard}

\begin{flashcard}[Definition]{Bereitstellungs-Modi}
    \vspace*{\stretch{1}}
    \begin{itemize}
        \item Komplett\newline
            Ressourcengruppe wird erstetzt mit der im Template konfigurierten Infrastruktur
        \item Inkrementell\newline
            Existierende Infrastruktur bleibt unverändert
    \end{itemize}
    \vspace*{\stretch{1}}
\end{flashcard}

\subsubsectioncard{save a deployment as an Azure Resource Manager template}

\begin{flashcard}[Definition]{Herunterladen einer ARM-Vorlage}
    \vspace*{\stretch{1}}
    \begin{itemize}
        \item Aus dem Portal:\newline
            \texttt{Vorlage Exportieren}
        \item Zip-Datei
        \item Enthält:
            \begin{itemize}
                \item template.json
                \item parameters.json
            \end{itemize}
    \end{itemize}
    \vspace*{\stretch{1}}
\end{flashcard}

\subsubsectioncard{deploy Windows and Linux VMs}

\begin{flashcard}[Definition]{VM mit Resource Manager Template definieren}
    \vspace*{\stretch{1}}
    Resource Manager Template:
    \begin{itemize}
        \item Resource vom Typ \texttt{Microsoft.Compute/virtualMachines}
        \item Eigenschaften:
            \begin{itemize}
                \item \texttt{hardwareProfile}: VM SKU
                \item \texttt{osProfile}: Username, Password, VM-Name
                \item \texttt{storageProfile}: OS disk, Daten-Disk
                \item \texttt{networkProfile}: die Netzwerkinterfaces
            \end{itemize}

    \end{itemize}
    \vspace*{\stretch{1}}
\end{flashcard}

\begin{flashcard}[Definition]{Spezifikation des Betriebssystems im ARM-Template}
    \vspace*{\stretch{1}}
    In den Eigenschaften für \texttt{storageProfile}:
    \begin{enumerate}
        \item \texttt{imageReference}
            \begin{itemize}
                \item \texttt{publisher} (Microsoft, oder RedHat, \ldots)
                \item \texttt{offer} (Betriebssystem, z.\,B. CentOS, WindowsServer, \ldots)
                \item \texttt{sku} (Version, z.\,B. 2016, oder 8.0)
                \item \texttt{version} (Bugfix-Release)
            \end{itemize}
        \item \texttt{osDisk}:\newline
            Setze auf \texttt{FromImage} um das unter 1 spezifizierte Image zu wählen
        \item \texttt{dataDisks}\newline
            Leer:
            \begin{itemize}
             \item \texttt{diskSizeGB}
             \item \texttt{lun}: fortlaufende Nummer
             \item \texttt{createOption}: \texttt{Empty}. (Für verwaltete Datenträger \texttt{Copy} oder \texttt{Attach})
            \end{itemize}
    \end{enumerate}
    \vspace*{\stretch{1}}
\end{flashcard}

\begin{flashcard}[Definition]{Windows als Betriebssystem}
    \vspace*{\stretch{1}}
    \begin{itemize}
        \item Publisher: MicrosoftWindowsServer
        \item Offer: WindowsServer
        \item SKU: 2016-Datacenter(-\ldots), 2019-Datacenter, 2012-Datacenter, \ldots
        \item Version: latest
    \end{itemize}
    \vspace*{\stretch{1}}
\end{flashcard}

\begin{flashcard}[Definition]{Linux als Betriebssystem}
    \vspace*{\stretch{1}}
    \begin{itemize}
        \item Publisher: OpenLogic, Debian, RedHat, SUSE, Canoncial
        \item Offer: CentOS, debian-10, RHEL, SLES, UbuntuServer
        \item SKU: 7.5, 10, 7-LVM, 15, 18.04-LTS
        \item version: latest
    \end{itemize}
    \vspace*{\stretch{1}}
\end{flashcard}

\begin{flashcard}[Definition]{Verschieben in andere Ressourcen}
    \vspace*{\stretch{1}}
    \begin{itemize}
        \item Portal:\newline
            Ressoruce auswählen, \texttt{Übersicht | Ressourcengruppe (Ändern)}, Zielgruppe auswählen
        \item CLI:\newline
            \texttt{az resource move --destination-group \ldots --ids \ldots}
        \item PowerShell:\newline
            \texttt{Move-AzResource\ -DestinationResourceGroupName "\ldots"\ -ResourceId <id>}
    \end{itemize}

    \vspace*{\stretch{1}}
\end{flashcard}
