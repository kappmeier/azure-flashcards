\subsectioncard{Einrichten und Verwalten von virtuellen Netzwerken}

\subsubsectioncard{configure private IP addressing}

\begin{flashcard}[Definition]{Private IP-Adressen}
    \vspace*{\stretch{1}}
    \begin{itemize}
        \item Dynamisch: DHCP-Lease basierend, können sich ändern
        \item Statisch: statisch während der Lebensdauer einer Resource
        \item Restriktionen: die ersten drei und die letztze IP-Adressen in jedem \emph{Subnetz}.
    \end{itemize}
    \vspace*{\stretch{1}}
\end{flashcard}

\subsubsectioncard{configure public IP addresses}

\begin{flashcard}[Definition]{Öffentliche IP-Adressen}
    \vspace*{\stretch{1}}
    \begin{itemize}
        \item Basic SKU:
            \begin{itemize}
                \item offen
                \item Netzwerksicherheitsgruppen empfohlen
                \item[!] Nicht für Verfügbarkeitszonen
            \end{itemize}
        \item Standard SKU:
            \begin{itemize}
                \item immer statisch
                \item Standardmäßig geschlossen
                \item Netzwerksicherheitsgruppen erforderlich
            \end{itemize}
        \item Load-Balancer muss zum SKU passen (Basic, Standard)
        \item IP-Bereiche je nach Region
    \end{itemize}
    \vspace*{\stretch{1}}
\end{flashcard}

\subsubsectioncard{create and configure network routes}

\begin{flashcard}[Definition]{Systemrouten}
    \vspace*{\stretch{1}}
    Typ des nächsten Hops für Systemrouten
    \begin{itemize}
        \item Virtuelles Netzwerk:
        \item Internet: \emph{0.0.0.0/0} leitet an das Internet weiter
        \item Keiner: Datenverkehr wird gelöscht
    \end{itemize}

    Weitere Systemrouten:
    \begin{itemize}
        \item VNET-Peering
        \item Dienstverkettung
        \item Gateway des virtuellen Netzwerks
        \item VNET-Dienstendpunkte
    \end{itemize}

    \vspace*{\stretch{1}}
\end{flashcard}

\begin{flashcard}[Definition]{Nächster Hop für Benutzerdefinierte Routen}
    \vspace*{\stretch{1}}
    \begin{itemize}
        \item Virtuelles Gerät: z.\,B. Firewall, typischerweise IP
        \item Gateway für virtuelle Netzwerke: z.\,B. VPN
        \item Virtuelles Netzwerk: zum Überschreiben der Systemroute
        \item Internet: leitet an das Internet weiter
        \item Keiner: Datenverkehr wird gelöscht
    \end{itemize}
    \vspace{1cm}
    Nächster Hop jeweils für Adress-Präfix (w.x.y.z/p)
    \vspace*{\stretch{1}}
\end{flashcard}

\begin{flashcard}[Definition]{Erstellen von Routen}
    \vspace*{\stretch{1}}
    \begin{enumerate}
        \item Routing Tabelle
        \item Route
        \item Zuordnen der Route an Subnetz
    \end{enumerate}
    \vspace*{\stretch{1}}
\end{flashcard}

\begin{flashcard}[Definition]{Route-Tabelle einrichten}
    \vspace*{\stretch{1}}
    \begin{itemize}
        \item Benötigt:
            \begin{itemize}
                \item Resourcegruppe
                \item Name
                \item disable-bgp-route-propagation / Gateway-Weiterleitung
            \end{itemize}
        \item CLI: \texttt{az network route-table create}
        \item Portal: neue Resource vom Typ Route Table
    \end{itemize}
    \vspace*{\stretch{1}}
\end{flashcard}

\begin{flashcard}[Definition]{Route erstellen}
    \vspace*{\stretch{1}}
    \begin{itemize}
        \item Benötigt:
            \begin{itemize}
                \item Resourcegruppe
                \item Name
                \item Route Table, zu dem die Route gehört
                \item address-prefix
                \item Nächster Hop: Typ und Adresse
            \end{itemize}
        \item CLI: \texttt{az network route-table route create}
        \item Portal: \texttt{Einstellungen | Routen | Hinzufügen} im Route-Tabellen-Blade
    \end{itemize}
    \vspace*{\stretch{1}}
\end{flashcard}

\begin{flashcard}[Definition]{Routentabelle einem Subnetz zuordnen}
    \vspace*{\stretch{1}}
    \begin{itemize}
        \item Benötigt:
            \begin{itemize}
                \item Identifikation: Subnetz + VNET + Resourcegruppe
                \item Routentabelle
            \end{itemize}
        \item CLI: \texttt{az network vnet subnet update}
        \item Portal: \texttt{Routingtabelle} im \texttt{Subnetz}-Blade ( \texttt{Einstellungen | Subnetz} im VNET-Blade) \newline
            oder \newline
            \texttt{Einstellungen | Subnetze | Subnetz zuordnen} im Routen-Tabellen-Blade
    \end{itemize}
    \vspace*{\stretch{1}}
\end{flashcard}

\subsubsectioncard{create and configure network interface}

\begin{flashcard}[Definition]{Erstellen von Netzwerkinterfaces}
    \vspace*{\stretch{1}}
    \begin{itemize}
        \item Standardmäßig hat jede VM ein Netzwerkinterfaces
        \item zusätzliche können erstellt werden
        \item je nach SKU können VMs mehrere haben
    \end{itemize}
    \vspace*{\stretch{1}}
\end{flashcard}

\begin{flashcard}[Definition]{Einstellungen für Netzwerkinterfaces}
    \vspace*{\stretch{1}}
    \begin{itemize}
        \item private IP
        \item öffentliche IP (optional)
    \end{itemize}
    \vspace*{\stretch{1}}
\end{flashcard}

\begin{flashcard}[Definition]{Nicht-Zugelassene IP-Adressen}
    \vspace*{\stretch{1}}
    \begin{itemize}
        \item Multicast:\newline
            224.0.0.0 bis 239.255.255.255
        \item Broadcast:\newline
            255.255.255.255\newline
            1-Bits für die CIDR-Maske, z.\,B. 10.20.0.255 für 10.20.0.0/24
        \item Loopback:\newline
            127.0.0.0
        \item Link-Local:\newline
            169.254.0.0 bis 169.254.255.255
        \item Azure-Internes DNS:\newline
            168.63.129.16
    \end{itemize}
    \vspace*{\stretch{1}}
\end{flashcard}

\subsubsectioncard{create and configure subnets}

\begin{flashcard}[Definition]{Subnetze}
    \vspace*{\stretch{1}}
    \begin{itemize}
        \item Teil eines virtuellen Netzwerks
    \end{itemize}
    \vspace*{\stretch{1}}
\end{flashcard}

\begin{flashcard}[Definition]{Subnetz erstellen}
    \vspace*{\stretch{1}}
    \begin{itemize}
        \item Benötigt:
        \begin{itemize}
            \item Name
            \item Adressbereich
        \end{itemize}
    \end{itemize}
    \vspace*{\stretch{1}}
\end{flashcard}

\begin{flashcard}[Definition]{Adressbereiche in Subnetzen}
    \vspace*{\stretch{1}}
    \begin{itemize}
        \item die ersten vier Adressen in jedem Subnetz sind reserviert
        \item[$\Rightarrow$] \texttt{x.y.z.4} ist die erste vergebbare Adresse
        \item[$\Rightarrow$] Ein /30-Netz macht keinen Sinn
        \item x.y.z.255 ist Broadcast-Adresse
    \end{itemize}
    \vspace*{\stretch{1}}
\end{flashcard}

\subsubsectioncard{create and configure virtual network}

\begin{flashcard}[Definition]{Virtuelles Netzwerk}
    \vspace*{\stretch{1}}
    \begin{itemize}
        \item Isolation von Resourcen
        \item Verbindung mit lokalen Computern (und anderen Netzen)
        \item externe Verbindung (zum Internet)
        \item Kommunikation zwischen Azure-Ressourcen
    \end{itemize}
    \vspace*{\stretch{1}}
\end{flashcard}

\begin{flashcard}[Definition]{Erstellen eines virtuellen Netzwerks}
    \vspace*{\stretch{1}}
    \begin{itemize}
        \item Benötigt:
        \begin{itemize}
            \item Ressourceengruppe + Abonnement, Location
            \item Name
            \item Adressraum (CIDR)
            \item Subnetz (oder neu anlegen)
            \item Eigenschaften: DDos-Schutz, Dienstendpunkte, Firewall
        \end{itemize}
        \item CLI: \texttt{az network vnet create}
        \item Portal: \texttt{Ressource erstellen | Netzwerk | virtuelles Netzwerk}
    \end{itemize}
    \vspace*{\stretch{1}}
\end{flashcard}

\subsubsectioncard{create and configure Network Security Groups and Application Security Groups}

\begin{flashcard}[Definition]{Netzwerksicherheitsgruppe}
    \vspace*{\stretch{1}}
    \begin{itemize}
        \item Teil eines virtuellen Netzwerks
        \item Filtern von Datenverkehr für virtuelles Netzwerk oder Subnetz
        \begin{itemize}
            \item IP-Adresse (Quelle oder Ziel)
            \item Port
            \item Protokoll
        \end{itemize}
    \end{itemize}
    \vspace*{\stretch{1}}
\end{flashcard}

\begin{flashcard}[Definition]{Erstellen einer Netzwerksicherheitsgruppe}
    \vspace*{\stretch{1}}
    \begin{itemize}
        \item Benötigt:
            \begin{itemize}
                \item Resourcengruppe: existierende Resourcengruppe
                \item Name: beliebiger Name
            \end{itemize}
        \item Portal: Ressource erstellen, Netzwerksicherheitsgruppe suchen
        \item CLI: \texttt{az network nsg create}
    \end{itemize}
    \vspace*{\stretch{1}}
\end{flashcard}

\begin{flashcard}[Definition]{Netzwerksicherheitsgruppe zuordnen}
    \vspace*{\stretch{1}}
    \begin{itemize}
        \item Benötigt:
            \begin{itemize}
                \item Virtuelles Netzwerk
                \item Subnetz
            \end{itemize}
        \item Portal: \texttt{Einstellungen | Subnetze | Zuordnen} im Blade der Netzwerksicherheitsgruppe
        \item CLI: \texttt{az network nsg ???}
    \end{itemize}
    \vspace*{\stretch{1}}
\end{flashcard}

\begin{flashcard}[Definition]{Konfigurieren von Netzwerksicherheitsgruppen}
    \vspace*{\stretch{1}}
    Konfiguration über Regeln:
    \begin{itemize}
        \item Regeln (>= 1) zum Zulassen oder Verweigern von Datenverkehr
        \item Priorität (100 bis 4069), kleiner ist besser (wird zuerst evaluiert)
        \item Quelle/Ziel: CIDR, Anwendungssicherheitsgruppe, Tag, oder \emph{Any}
        \item Ports
        \item Protokolle (TCP, UDP, \emph{Any}
        \item Ein- oder ausgehender Datenverkehr
        \item Aktion: Allow or Deny
    \end{itemize}
    Standardregeln: DenyAll, AllowLoadBalancer, AllowVNetInbound (vom gleichen Subnetz)
    \vspace*{\stretch{1}}
\end{flashcard}

\begin{flashcard}[Definition]{Regeln für Netzwerksicherheitsgruppe}
    \vspace*{\stretch{1}}
    \begin{itemize}
        \item Priorität 1000 bis 65000
        \item Port-Bereiche (einzeln, Kommagetrennt, Bereiche)
        \item Quellen+Zieladressen:
        \begin{itemize}
            \item IP-Adressen, z.\,B. \texttt{10.0.0.4}
            \item Anwendungssicherheitsgruppe: leerer Präfix, und asg definieren
            \item Nur Quelle: Virtuelles Netzwerk
            \item Nur Ziel: Service Tag, z.\,B. Endpunkte wie \emph{Storage}
        \end{itemize}
        \item CLI: \texttt{az network nsg rule create}
        \item Portal: Einstellungen im Netzwerksicherheitsgruppen-Blade:\newline
            \texttt{Eingangssicherheitsregeln}, \texttt{Ausgangssicherheitsregeln}
    \end{itemize}
    \vspace*{\stretch{1}}
\end{flashcard}

\begin{flashcard}[Definition]{Anwendungssicherheitsgruppen}
    \vspace*{\stretch{1}}
    \begin{itemize}
        \item Wird NIC zugeordnet
        \item Verbindung mit Netzwerksicherheitsgruppe über Regeln:\newline
            \texttt{--source-asgs ERP-DB-SERVERS-ASG} anstelle von \texttt{--source-address-prefixes}
        \item Allow/Deny funktioniert auch mit Dienstendpunkten, z.b. \texttt{Storage} \newline
            Dienstendpunkte werden als storage-account network-rule definiert
        \item \texttt{az network asg create}
    \end{itemize}
    \vspace*{\stretch{1}}
\end{flashcard}
