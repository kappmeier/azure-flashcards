\subsectioncard{Implementierung von Lösungen mit virtuellen Maschinen}

\subsubsectioncard{provision VMs}

\subsubsectioncard{create Azure Resource Manager templates}

\begin{flashcard}[Definition]{Azure Rsource Manager-Vorlagen-Format}
    \vspace*{\stretch{1}}
    JSON -- Java Script Object Notation
    \vspace*{\stretch{1}}
\end{flashcard}

\begin{flashcard}[Definition]{Ausdrücke in ARM-Templates}
    \vspace*{\stretch{1}}
    Zusätzlich zu JSON-Objekten (Arrays, Maps) gibt es Ausdrücke, die ausgewertet werden

    \vspace{1cm}
    Form: \texttt{[expression]}

    \vspace{1cm}
    In Ausdrücken kann auf objekte wie Ressourcen und Funktionen zugegriffen werden
    \vspace*{\stretch{1}}
\end{flashcard}


\subsubsectioncard{configure Azure Disk Encryption for VMs}

\begin{flashcard}[Definition]{Verschlüsselung von VMs}
    \vspace*{\stretch{1}}
    \begin{itemize}
        \item Standardmäßig Ausführugn
        \item benutzt Bitlocker (für Windows) oder DM-Crypt (Linux)
        \item Schlüssel in Azure KeyVault ($\rightarrow$ \texttt{key})\newline
            \emph{(in gleicher Region)}\newline
            [!] verwaltet von Azure
        \item Voraussetzung: bessere Maschine als A, mehr als 2 GiB Hauptspeicher, Generation 1 VM
    \end{itemize}
    \vspace*{\stretch{1}}
\end{flashcard}

\begin{flashcard}[Definition]{VM-Verschlüsselung einrichten}
    \vspace*{\stretch{1}}
    \begin{itemize}
        \item Zugang zu Azure Active Directory-Endpunkt
        \item Zugang zu Key Vault-Endpunkt
        \item Azure Storage-Endpunkt für das Speicherkonto mit VHD-Dateien
    \end{itemize}
    \vspace{1cm}
    $\Rightarrow$ überprüfen in \texttt{VM | Übersicht}
    \vspace*{\stretch{1}}
\end{flashcard}

\begin{flashcard}[Definition]{Azure Disk-Verschlüsselung aktivieren}
    \vspace*{\stretch{1}}
    \begin{itemize}
        \item CLI:\newline
            \texttt{az vm encryption enable -resource-group \ldots --name <vm-name>\\--disk-encryption-keyvault \ldots}
        \item PowerShell:\newline
            \texttt{Set-AzVMDiskEncryptionExtension -ResourceGroupName \ldots\\-VMName \ldots -DiskEncryptionKeyVaultUrl \$KeyVault.VaultUri\\-DiskEncryptionKeyVaultId \$KeyVault.ResourceId}
        \item Parameter: \texttt{--volume-type}: \texttt{DATA, OS, ALL}
    \end{itemize}
    \vspace*{\stretch{1}}
\end{flashcard}

\subsubsectioncard{implement Azure Backup for VMs}

\begin{flashcard}[Definition]{Azure Site Recovery}
    \vspace*{\stretch{1}}
    \begin{itemize}
        \item Tool zur Wiederherstellung von Daten
        \item Replikation zwischen primärem und sekundärem Standort
        \item unterstützt lokale (inkls. physische Maschinen) und Azure als Quell-Standort
        \item Failback (nicht zu physische Maschinen)
        \item Erstellt Momentaufnahmen und Wiederherstellungspunkte
    \end{itemize}
    \vspace*{\stretch{1}}
\end{flashcard}

\begin{flashcard}[Definition]{RTO}
    \vspace*{\stretch{1}}
    \begin{itemize}
        \item Recovery Time Objective
        \item Maximale Zeitspanne, die da Unternehmen nach einem Notfall überleben kann, bevor der normale Betrieb wiederhergestellt wurde
    \end{itemize}
    \vspace*{\stretch{1}}
\end{flashcard}

\begin{flashcard}[Definition]{RPO}
    \vspace*{\stretch{1}}
    \begin{itemize}
        \item Recovery Point Objective
        \item Maximaler Datenverlust, der im Notfall akzeptabel ist
    \end{itemize}
    \vspace*{\stretch{1}}
\end{flashcard}

\begin{flashcard}[Definition]{Failover}
    \vspace*{\stretch{1}}
    Übertragung der Ausführugn an einen sekundären Standort (z.\,B. weil der primäre nicht verfügbar ist)
    \vspace{1cm}
    Nach einem Failover wird der Schutz deaktiviert!
    \vspace*{\stretch{1}}
\end{flashcard}

\begin{flashcard}[Definition]{Failback}
    \vspace*{\stretch{1}}
    Rück-Übertragung der Ausführugn von einem sekundären Standort (z.\,B. weil der primäre wieder verfügbar ist)
    \vspace*{\stretch{1}}
\end{flashcard}

\begin{flashcard}[Definition]{Phasen der Replikation}
    \vspace*{\stretch{1}}
    \begin{enumerate}
        \item Failover
        \item erneutes Schützen (in der sekundären Region)
        \item Failback
        \item erneutes Schützen (in der primären Region)
    \end{enumerate}
    \vspace*{\stretch{1}}
\end{flashcard}

\begin{flashcard}[Definition]{Voraussetzungen für Site Recovery}
    \vspace*{\stretch{1}}
    \begin{itemize}
        \item Netzwerk für die Ziel-Ressourcen (Backup, replizierte VMs)
        \item Recovery-Services-Tresor
        \item Rechte:
            \begin{itemize}
                \item Mitwirkender für virtuelle Computer
                \item Site-Recovery-Mitwirkender
            \end{itemize}
        \item Speicherkonto außerhalb der Quellregion (in Zielregion)
    \end{itemize}
    \vspace*{\stretch{1}}
\end{flashcard}

\begin{flashcard}[Definition]{Replikation von lokalen VMs}
    \vspace*{\stretch{1}}
    Voraussetzungen:
    \begin{itemize}
        \item Konfigurationsserver (VMware, lokal)
        \item Konfigurationsserver muss auch existieren für Failback von \emph{physischen} Maschinen
    \end{itemize}
    Schritte:
    \begin{enumerate}
        \item Netzwerk einrichten
        \item Site-Recovery-Tresor erstellen
        \item Berechtigungen/Anmeldeinformationen
        \item Konfigurationsserver in vCenter über OVA
    \end{enumerate}

    \vspace*{\stretch{1}}
\end{flashcard}

\begin{flashcard}[Definition]{Replikation von Azure VMs}
    \vspace*{\stretch{1}}
    \begin{itemize}
        \item Netzwerk einrichten
        \item Site-Recovery-Tresor ertellen
        \item Azure-Mobilitätsdienst auf VMs installieren (Linux und Windows)\newline
            wird normalerweise von Site Recovery automatisch erledigt
    \end{itemize}
    \vspace*{\stretch{1}}
\end{flashcard}

\begin{flashcard}[Definition]{Failover vorbereiten}
    \vspace*{\stretch{1}}
    Vorbereitung: Ändern von Einstellungen
    \begin{itemize}
        \item Azure-Name
        \item Ressourcengruppe
        \item Ziel-SKU
        \item Verfügbarkeitsgruppe
        \item Verwaltete Disks
        \item Netzwerke: VNet + Subnetz, IP-Addresse
    \end{itemize}
    \ldots können alle angepasst werden!
    \vspace*{\stretch{1}}
\end{flashcard}

\begin{flashcard}[Definition]{Backup von virtuellen Maschinen}
    \vspace*{\stretch{1}}
    Azure Backup-Dienst
    \begin{itemize}
        \item erstellt und verwaltet Backups
        \item keine Infrastruktur notwendig
        \item erstellt Kopien zustandsbehafter Daten um alte Zustände wieder herzustellen
        \item Langzeitaufbewahrung (mehrere Jahre)
        \item sicherer Zugriff (RBAC, Verschlüsselung, \ldots)
        \item vorläufiges Löschen
        \item Sicherung autark im Azure-Netzwerk
    \end{itemize}
    \vspace*{\stretch{1}}
\end{flashcard}

\begin{flashcard}[Definition]{Szenarien von Azure Backup}
    \vspace*{\stretch{1}}
    \begin{itemize}
        \item Virtuelle Maschinen in Azure sichern
        \item lokale Maschinen sichern (MARS, DPM)
            \begin{itemize}
                \item physische server und Hyper-V und VMware
                \item Rückspielung nur auf virtuelle Maschinen
            \end{itemize}
        \item Dateifreigaben
        \item SQL Server
    \end{itemize}
    \vspace*{\stretch{1}}
\end{flashcard}

\begin{flashcard}[Definition]{Backup mit MARS}
    \vspace*{\stretch{1}}
    \begin{itemize}
        \item Agent-Software für Windows-Maschinen
        \item Für physikalische Windows-Maschinen und virtuelle Windows-Maschinen
        \item Sichert Verzeichnisse und Dateien, nicht komplette VMs
        \item kann zusammen mit DPM und MABS betrieben werden
    \end{itemize}
    \vspace*{\stretch{1}}
\end{flashcard}

\begin{flashcard}[Definition]{Backup mit DPM}
    \vspace*{\stretch{1}}
    Data Protection Manager
    \begin{itemize}
        \item Unterstützte Backups
            \begin{itemize}
                \item Anwendungen (SQL, \ldots)
                \item Dateien
                \item vollständige Systeme (Windows)
                \item Hyper-V VMs, (Windows + Linux)
            \end{itemize}
        \item Sicherungs-Ziele
            \begin{itemize}
                \item Platte (kurzzeitige Sicherungen)
                \item Azure (kurzzeitig, Langzeitaufbewahrung)
                \item Tape (Langzeitaufbewahrung)
            \end{itemize}
    \end{itemize}
    \vspace*{\stretch{1}}
\end{flashcard}

\begin{flashcard}[Definition]{Backup mit MABS}
    \vspace*{\stretch{1}}
    Azure Backup Server
    \begin{itemize}
        \item Agent-Software
        \item Unterstützte Backups
            \begin{itemize}
                \item Hyper-V oder VMware (Linux, vollständige Maschine)
                \item Hyper-V oder VMware (Windows: Maschinen, oder einzelne Dateien, Shares)
                \item Servers: Dateien, SQL, Exchange
            \end{itemize}
    \end{itemize}
    \vspace*{\stretch{1}}
\end{flashcard}


\begin{flashcard}[Definition]{VM Backup einrichten}
    \vspace*{\stretch{1}}
    \begin{itemize}
        \item Recovery Services-Tresor einrichten\newline
            enthält Sicherungen
        \item Azure Backup-Erweiterung auf der VM aktivieren\newline
            \emph{VMSnapshot} bzw. \emph{VMSnapshotLinux}
        \item Sicherungsrichtlinie erstellen. Konsistenzebenen für Backups:
            \begin{itemize}
             \item Anwendungskonsistent\newline
                Kopie der gesamten VM, vollstänidge Konsistenz aller Anwendungen
             \item Dateisystemkonsistent\newline
                Kopie aller Dateien, Anwendungen müssen ggf. bereinigt werden
             \item Absturzkonsistent\newline
                Falls während der Sicherung die VM heruntergefahren wird
            \end{itemize}

    \end{itemize}
    \vspace*{\stretch{1}}
\end{flashcard}

\begin{flashcard}[Definition]{Sicherung einer VM mit dem Poral}
    \vspace*{\stretch{1}}
    Einrichten:
    \begin{itemize}
        \item Benötigt:
            \begin{itemize}
                \item Resourcengruppe
                \item Recovery Services-Tresor
                \item Sicherungsrichtlinie
            \end{itemize}
        \item Im Portal:\newline
            \texttt{Vorgänge | Sicherung | Sicherung} im VM-Blade
        \item Tresor, Resourcengruppe und Richtlinie können neu erstellt werden
    \end{itemize}
    Durchführen:
    \begin{itemize}
        \item In \texttt{Vorgänge | Backup} im Tresor eine VM auswählen
        \item \texttt{Jetzt sichern}
    \end{itemize}

    \vspace*{\stretch{1}}
\end{flashcard}

\begin{flashcard}[Definition]{Sicherung einer VM mit Azure CLI}
    \vspace*{\stretch{1}}
    (Benötigt vorhandene Resourcnegruppe, Recovery-Services-Tresor, Policy
    \begin{itemize}
        \item Sicherung einrichten:\newline
            \texttt{az backup protection enable-for-vm --resource-group \ldots --vault-name \ldots --vm \ldots --policy-name \ldots}
        \item Sicherung durchführen:\newline
            \texttt{az backup protection backup-now --resource-group \ldots --vault-name \ldots --container-name \ldots --item-name \ldots --retain-until \ldots --backup-management-type \ldots}
    \end{itemize}
    \vspace*{\stretch{1}}
\end{flashcard}

\begin{flashcard}[Definition]{Wiederhertellen einer Sicherung}
    \vspace*{\stretch{1}}
    \begin{itemize}
        \item Erstellen einer VM: eine virtuelle Maschine in der gleichen Region als Kopie erstellen
        \item Datenträger wiederherstellen: erstellt einen Datenträger mit Daten aus dem Backup
        \item Vorhandene ersetzen: ersetzt einen Datenträger auf einer existierenden VM
        \item Regionsübergreifende Wiederherstellung: wiederherstellung in die gekoppelte Partner-Region
    \end{itemize}
    Ein Backup kann in wenigen Minuten wiederhergestellt werden!
    \vspace*{\stretch{1}}
\end{flashcard}
