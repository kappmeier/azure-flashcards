\subsectioncard{Erstellen und Konfigurieren von VMs für Windows und Linux}

\begin{flashcard}[Definition]{Hybrid Benefit}
    \vspace*{\stretch{1}}
    \begin{itemize}
        \item eigene Lizenzen (Windows, SQL)
        \item sparen 49\%
    \end{itemize}
    \vspace*{\stretch{1}}
\end{flashcard}

\subsubsectioncard{configure High Availability}

\begin{flashcard}[Definition]{Fehler- und Updatedomänen}
    \vspace*{\stretch{1}}
    \begin{itemize}
        \item Update-Domänen\newline
            Sichern zu, dass eine Verfügbarkeitsgruppe während Updates verfügbar bleibt
        \item Fehler-Domänen\newline
            Sichert zu, dass VMs einer Verfügbarkeitsgruppe bei Ausfall einer Zone verfügbar bleiben
        \item Im Portal:\newline
            \texttt{Eigenschaften | Hochverfügbarkeit}
    \end{itemize}
    \vspace*{\stretch{1}}
\end{flashcard}

\subsubsectioncard{configure Monitoring}

\begin{flashcard}[Definition]{Ressourcne-Auslastung einer VM überwachen}
    \vspace*{\stretch{1}}
    \begin{itemize}
        \item Azure Montior-Metriken
        \item Azure Monitoring Insights
    \end{itemize}
    \vspace*{\stretch{1}}
\end{flashcard}

\subsubsectioncard{configure Networking}

\subsubsectioncard{configure Storage}

\begin{flashcard}[Definition]{Disks for VMs}
    \vspace*{\stretch{1}}
    \begin{itemize}
        \item Benötigt:
            \begin{itemize}
                \item Resourcegruppe
                \item Location
                \item Name
                \item Betriebssystem-Image
                \item Admin-Username
                \item ggf. ssh keys
            \end{itemize}
        \item Portal: \texttt{Neue Ressource | Virtuelle Maschine}
        \item CLI: \texttt{az vm create}
        \item PowerShell: \texttt{New-AzVm}
    \end{itemize}
    NIC + Public IP + OS Disk werden automatisch erstellt
    \vspace*{\stretch{1}}
\end{flashcard}

\begin{flashcard}[Definition]{Disks for VMs}
    \vspace*{\stretch{1}}
    \begin{itemize}
        \item Betriebssystemdatenträger
            \begin{itemize}
                \item Ein pro VM, basierend auf Image
                \item maximal 2 TiB
            \end{itemize}

        \item Daten-Disks
            \begin{itemize}
                \item Ein oder mehrere pro VM
                \item Anzahl abhängig von VM Sku
                \item bis zu 32 TiB Größe
            \end{itemize}

        \item Temporärer Speicher\newline
            \begin{itemize}
                \item Ein pro VM
                \item geht bei Wartung verloren. Nicht auf einem Speicherkonto
            \end{itemize}
    \end{itemize}
    \vspace*{\stretch{1}}
\end{flashcard}

\begin{flashcard}[Definition]{Datenträger-Typen}
    \vspace*{\stretch{1}}
    Kurzlebig
    \begin{itemize}
        \item Lokal im Host
        \item sehr Schnell, auch bei der Initialisierung
        \item bei Ausfall sind Daten verloren
        \item gut für zustandslose Workloads (z.\,B. Microservice)
    \end{itemize}
    Verwaltet
    \begin{itemize}
        \item Skalierbar, hochverfügbar (Verfügbarkeitszonen),
        \item sind zwar Seitenblobs im Speicherkonto, aber im Hintergrund
        \item Sicherheit: RBAC, Verschlüsselung
    \end{itemize}
    Nicht-Verwaltet
    \begin{itemize}
        \item Nutzer müssen sich selbst um Speicherkonto kümmern
        \item Sicherheit pro Speicherkonto, nicht pro Disk
    \end{itemize}
    \vspace*{\stretch{1}}
\end{flashcard}

\begin{flashcard}[Definition]{Datenträger für VMs}
    \vspace*{\stretch{1}}
    \begin{itemize}
        \item beim Erstellen im Azure-Portal
        \item Bereich ``Datenträger''
        \item Auswahl, ob Datenträger verwaltet oder nicht verwaltet
    \end{itemize}
    \vspace*{\stretch{1}}
\end{flashcard}

\subsubsectioncard{configure Virtual Machine Size}

\begin{flashcard}[Definition]{VM-Größen}
    \vspace*{\stretch{1}}
    \begin{itemize}
        \item Viele SKUs:
            \begin{itemize}
                \item Allgemein
                \item Balanciert: Dsv, Dv3, DS, D
                \item Arbeitsspeicher: E, M, G, D, DS, Dv2
                \item Compute: F, Fs
                \item Speicher: L
                \item GPU: N,
                \item Leistung: H, A
            \end{itemize}
        \item Bestimmen Parameter wie Disk-Größe, anzahl NICs, Disks, \ldots
        \item \texttt{az vm list-sizes}
    \end{itemize}
    \vspace*{\stretch{1}}
\end{flashcard}

\begin{flashcard}[Definition]{VM-Größe ändern}
    \vspace*{\stretch{1}}
    \begin{itemize}
        \item Liste der verfügbaren:\newline
            \texttt{az vm list-vm-resize-options --resource-group \ldots --name \ldots}
        \item Nur im aktuellen Cluster verfügbare Größen können gewählt werden
        \item Portal: \texttt{Einstellungen | Größe | Größe ändern} im VM-Blade\newline
            (Verfügbare Größen werden angezeigt)
    \end{itemize}
    \vspace*{\stretch{1}}
\end{flashcard}

\subsubsectioncard{implement dedicated hosts}

\subsubsectioncard{deploy and configure scale sets}

\begin{flashcard}[Definition]{Skalierungsgruppe}
    \vspace*{\stretch{1}}
    \begin{itemize}
        \item Skalierung von Web-Services, Container-Workloads, Big-Data, \ldots
        \item horizontale Skalierung: dynamische Anzahl VMs
        \item vertikale Skalierung: dynamische VM-Größe
        \item alle VMs haben die gleiche Konfiguration
        \item maximal 1000 VMs
        \item bei Bedarf Lastenausgleich (z.\,B. über Ping)
        \item Skalierungsgruppe mit niedriger Priorität: günstiger
    \end{itemize}
    \vspace*{\stretch{1}}
\end{flashcard}

\begin{flashcard}[Definition]{Durchführung einer Skalierung}
    \vspace*{\stretch{1}}
    Geplante Skalierung
    \begin{itemize}
        \item geplante Bereitstellung von zusätzichen Maschinen
    \end{itemize}
    \vspace{1cm}
    Automatische Skalierung
    \begin{itemize}
        \item metrikbasierte Bereitstellung von Maschinen
        \item Schwellwertskalierung
    \end{itemize}

    \vspace*{\stretch{1}}
\end{flashcard}

\begin{flashcard}[Definition]{Bereitstellung einer Skalierungsgruppe}
    \vspace*{\stretch{1}}
    \begin{itemize}
        \item Benötigt:
            \begin{itemize}
                \item Abonnement und Resourcengruppe
                \item Name
                \item VM-Parameter, z.\,B. VM-Image, User, SSH, \ldots
                \item
            \end{itemize}
        \item CLI: \texttt{az vmss create}
    \end{itemize}
    zwei Instanzen und Lastenausgleich sind automatisch erstellt
    \vspace*{\stretch{1}}
\end{flashcard}

\begin{flashcard}[Definition]{Lastenausgleich mit Skalierungsgruppe}
    \vspace*{\stretch{1}}
    Integritätstest:
    \begin{itemize}
        \item Benötigt:
        \begin{itemize}
            \item Abonnement und Resourcengruppe
            \item Name
            \item Test-Spezifikation: Port, Protokoll, Pfad, \ldots
        \end{itemize}
        \item CLI: \texttt{az network lb probe create}
    \end{itemize}
    Lastenausgleich:
    \begin{itemize}
        \item Benötigt:
        \begin{itemize}
            \item Abonnement und Resourcengruppe
            \item Name
            \item Integritätstest-Referenz
            \item Scale-Set als Backend-Pool
            \item Front- und Backend-Spezifikation: Port, Potokoll, \ldots
        \end{itemize}
        \item CLI: \texttt{az network lb rule create}
    \end{itemize}
    \vspace*{\stretch{1}}
\end{flashcard}

\begin{flashcard}[Definition]{Manuelle Skalierung}
    \vspace*{\stretch{1}}
    \begin{itemize}
        \item Portal: \texttt{Einstellungen | Wird Skaliert | Manuelle Skalierung} im Skalierungsgruppen-Blade
        \item CLI: \texttt{az vmss scale --new-capacity \ldots}
    \end{itemize}
    \vspace*{\stretch{1}}
\end{flashcard}

\begin{flashcard}[Definition]{Automatische Skalierung}
    \vspace*{\stretch{1}}
    \begin{itemize}
        \item Zeitplan
            \begin{itemize}
                \item Portal: \texttt{Einstellungen | Wird Scaliert | Benutzerdefinierte Autoskalierung} im Skalierungsgruppen-Blade
                \item \texttt{"Auf eine bestimmte Anzahl von Instanzen skalieren"}
                \item mit Zeitplan
            \end{itemize}
        \item Metriken
            \item Portal: \texttt{Einstellungen | Wird Scaliert | Benutzerdefinierte Autoskalierung} im Skalierungsgruppen-Blade
            \item \texttt{"Basierend auf einer Metrik"}
            \item \texttt{"Regel hinufügen"}
            \begin{itemize}
                \item CPU-Auslastung
                \item Daten(ein|aus)fluss
                \item Datenträgervorgänge
                \item Warteschlangentiefe für Datenträger
            \end{itemize}
        \item Aggregierung von für \emph{alle} Instanzen
            \begin{itemize}
                \item Durchschnitt
                \item Minimum/Maximum
                \item Summe
                \item Letzte
                \item Anzahl
            \end{itemize}
        \item Aggregierungsdauer: mindestens 5 Minuten
        \item Operator: \emph{kleiner}, \emph{größer}, \ldots
        \item Abkühldauer: mindextens 5 Minuten. Währenddessen wird Regel nicht nochmal angewandt
        \item Regeln
            \begin{itemize}
                \item Erhöhung der Instanzzahl bei x\%
                \item Verringerung der Instanzzahl bey y\%
                \item für ein Regelpaar Schwellwerte \emph{nicht} auf einen Wert festlegen
            \end{itemize}

    \end{itemize}
    \vspace*{\stretch{1}}
\end{flashcard}

\begin{flashcard}[Definition]{Autoscale-Profil einrichten mit CLI}
    \vspace*{\stretch{1}}
    \begin{itemize}
        \item Benötigt Resourcegruppe und Skalierungsgruppe
        \item Standardwert, minimum, maximum von VMs
        \item \texttt{az monitor autoscale create\\--resource-group \ldots\\--resource <scale-set>\\--resource-type Microsoft.Compute/virtualMachineScaleSets\\--name autoscale --min-count \ldots --max-count \ldots --count \ldots}
    \end{itemize}
    \vspace*{\stretch{1}}
\end{flashcard}

\begin{flashcard}[Definition]{Autoscale-Regel einrichten mit CLI}
    \vspace*{\stretch{1}}
    \begin{itemize}
        \item Benötigt Resourcegruppe und Autoscale-Profil
        \item \texttt{az monitor autoscale rule create\\--resource-group \ldots\\--autoscale-name \ldots\\--condition <condition-string>\\--scale <out|in> \ldots}
    \end{itemize}
    \vspace*{\stretch{1}}
\end{flashcard}

\begin{flashcard}[Definition]{Aktualisierungen von VMs}
    \vspace*{\stretch{1}}
    Mit benutzerdefinierter Skripterweiterung
    \begin{itemize}
        \item genau so, wie bei VMs
        \item \texttt{az vmss extension set \ldots}
        \item nochmalige Anwendung zum Aktualisieren
        \item Upgraderichtlinie
            \begin{itemize}
                \item Automatisch: könnte für alle VMs gleichzeitig sein
                \item Parallel: vermeidet Dienstaufall
                \item Manuell: standard
            \end{itemize}
            Festlegen mit \texttt{upgrade-policy-mode}-Parameter bei Erstellung der Skalierungsgruppe
    \end{itemize}
    \vspace*{\stretch{1}}
\end{flashcard}
   
