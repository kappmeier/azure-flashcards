\subsectioncard{Analysieren von Resourcen-Nutzung und -Verbrauch}

\begin{flashcard}[Definition]{Services, die Autoscaling unterstützen}
    \vspace*{\stretch{1}}
    \begin{itemize}
        \item Skalierungsgruppen
        \item App Service (Plan) / Web Apps
        \item API Management
        \item Data Explorer
        \item Azure Cloud Services
        \item AKS
    \end{itemize}
    \vspace*{\stretch{1}}
\end{flashcard}

\begin{flashcard}[Definition]{Autoscaling-Muster}
    \vspace*{\stretch{1}}
    \begin{itemize}
        \item Off and on: Manuelle skalierung, also vom Nutzer im Portal\newline
            Wenn eine bestimmte gerade anfallende Last bedient werden soll
        \item Unverhersagbar nach CPU-Last: Matrikbasiert\newline
            Auslastung meistens stabil, aber unklar wann und wie groß eine zu erwartende Spitzenbelastung ist
        \item Kapaztiät erweitern: Manuell\newline
            Dauerhaft die Resourcen erhöhen, da die Nachfrage generell gestiegen ist
        \item Vorhersagbar: Skalierung nach Zeitplan\newline
            Wenn voraussehbar ist, dass eine Laststeigerung kommt (z.\,B. bestimmte Tage, Uhrzeiten, \ldots)
    \end{itemize}
    \vspace*{\stretch{1}}
\end{flashcard}
