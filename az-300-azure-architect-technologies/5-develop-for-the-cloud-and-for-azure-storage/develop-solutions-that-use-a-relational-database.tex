\subsectioncard{Entwickeln von Lösungen, die relationale Datenbanken umfassen}

\subsubsectioncard{provision and configure relational databases}

\begin{flashcard}[Definition]{Datenbank-Varianten}
    \vspace*{\stretch{1}}
    \begin{itemize}
        \item Single Database:\newline
            für kleine Services geeignet
        \item Verwlatet:\newline
            verwaltete, große Instanz
        \item elastische Pools:\newline
            mehrere Datenbanken, die einen Pool teilen
        \item virtuelle Maschine:\newline
            vollständige Kontrolle über Datenbank, Server, aber kostet
    \end{itemize}
    \vspace*{\stretch{1}}
\end{flashcard}


\begin{flashcard}[Definition]{Vorteile von Azure SQL-Datenbank}
    \vspace*{\stretch{1}}
    \begin{itemize}
        \item Verwaltet (PaaS): Hardware, Software und Betriebssysteme werden aktualisiert
        \item Kosten: abhängig von Instanzgröße, aber keine dedizierte Hardware
        \item dynamische Skalierbarkeit
        \item Sicherheit: Firewall standardmäßig aktiv
    \end{itemize}
    \vspace*{\stretch{1}}
\end{flashcard}

\begin{flashcard}[Definition]{Azure-SQL-Datenbank}
    \vspace*{\stretch{1}}
    \begin{itemize}
        \item Azure-SQL-Datenbank: logischer Server\newline
            Pro Server: Verwaltung von Anmeldung, Sicherheit, Firewall, \ldots für alle Datenbanken
        \item Datenbank: in einem logischen Server
    \end{itemize}
    \vspace*{\stretch{1}}
\end{flashcard}

\begin{flashcard}[Definition]{Typen von SQL-Datenbank}
    \vspace*{\stretch{1}}
    DTUs
    \begin{itemize}
        \item \emph{Database Transaction Unit}:\newline
            Kombination aus Rechenleistung, Speicher, Eingabe/Ausgabe
        \item vorkonfiguriert
        \item kann für kleine Anwendungen günstiger sein
    \end{itemize}
    vKern
    \begin{itemize}
        \item virtuelle Kerne (für den Datenbank-Server)
        \item Kerne und Speicher können unabhängig konfiguriert werden
    \end{itemize}

    \vspace*{\stretch{1}}
\end{flashcard}

\begin{flashcard}[Definition]{SQL-Server erstellen}
    \vspace*{\stretch{1}}
    \begin{itemize}
        \item Benötigt:
            \begin{itemize}
                \item Global eindeutiger Servername (URL)
                \item Administrator-Zugangsdaten
                \item Location
                \item Zugriff von Azure-Diensten?
            \end{itemize}
        \item Portal: \texttt{Resource erstellen | Datenbanken | SQL-Datenbank | Server | Neu erstellen }
        \item Azure CLI: \newline
            \texttt{az sql server create}
    \end{itemize}
    \vspace*{\stretch{1}}
\end{flashcard}

\begin{flashcard}[Definition]{SQL-Datenbank erstellen}
    \vspace*{\stretch{1}}
    \begin{itemize}
        \item Benötigt:
            \begin{itemize}
                \item Abonnement und Ressourcengruppe
                \item Datenbank-Name
                \item Server
                \item elastische Pools Ja/Nein?
                \item Leistung (Rechenkraft und Speicher)
            \end{itemize}
        \item Portal: \texttt{Resource erstellen | Datenbanken | SQL-Datenbank}
        \item Azure-CLI:\newline
            \texttt{az sql db create \ldots}
    \end{itemize}
    \vspace*{\stretch{1}}
\end{flashcard}

\begin{flashcard}[Definition]{Leistung für Datenbank einstellen}
    \vspace*{\stretch{1}}
    \begin{itemize}
        \item Portal: Beim Erstellen einer Datenbank ("Compute + Storage")
        \item vCore-Basiert
        \item Basic, Standard, Premium für DTUs
    \end{itemize}
    \vspace*{\stretch{1}}
\end{flashcard}

\begin{flashcard}[Definition]{Firewall für SQL-Datenbank konfigurieren}
    \vspace*{\stretch{1}}
    \begin{itemize}
        \item
        \item Portal: \texttt{Serverfirewall festlegen} im Datenbank-Blade
        \begin{itemize}
            \item \texttt{Client-IP-Adresse hinzufügen}: aktuelle Adresse hinzufügen
            \item[$\Rightarrow$] wird als Regel eingetragen
            \item Regeln: Name und IP-Bereich
            \item Virtuelle Netzwerke
        \end{itemize}
    \end{itemize}
    \vspace*{\stretch{1}}
\end{flashcard}

\subsubsectioncard{configure elastic pools for Azure SQL Database}

\begin{flashcard}[Definition]{Pools für elastische SQL-Datenbanken}
    \vspace*{\stretch{1}}
    \begin{itemize}
        \item gemeinsam genutzte Resourcen (eDTUs) für mehrere Datenbanken
        \item Datenbanken können (mit Beschränkung) Resourcen von dem Pool nutzen
        \item Anwendungsfälle: unterschiedlicher, unverhergesehener Ressourcnebedarf
        \item Architektur:
            \begin{itemize}
                \item gewisse Ressourcen für den gesamten Pool
                \item verschiedene Datenbanken werden dem Pool zugewiesen
                \item[$\Rightarrow$] teilen sich die Ressourcen
            \end{itemize}
        \item Größe: kombinierte Ressourcen (theoretisch) müssen 1,5 mal die Kapaztität für den Pool überschreiben. Mehr als 2 S3-Datenbanken oder 15 S0-Datenbanken
        \item Grenzen: 100 bzw. 500 Datenbanken pro Pool
    \end{itemize}
    \vspace*{\stretch{1}}
\end{flashcard}

\begin{flashcard}[Definition]{Elastischen Pool erstellen}
    \vspace*{\stretch{1}}
    \begin{itemize}
        \item Benötigt:
            \begin{itemize}
                \item Abonnement + Ressourcengruppe
                \item Pool-Name
                \item SQL-Server-Instanz
            \end{itemize}
        \item Azure-CLI:\newline
            \texttt{az sql elastic-pools create}
        \item Portal: \texttt{Elastischer SQL-Datenbankpool} suchen als Ressourcetyp
    \end{itemize}
    \vspace*{\stretch{1}}
\end{flashcard}

\begin{flashcard}[Definition]{Zuweisen eines Pools an eine Datenbank}
    \vspace*{\stretch{1}}
    \begin{itemize}
        \item Bei der Erstellung:\newline
            \texttt{az sql db create --elastic-pool-name \ldots}
        \item Im Portal: \newline
            \texttt{Einstellungen | Konfigurieren | Datenbanken | + Datenbanken hinzufügen} im Pool-Blade\newline
            (Datenbanken müssen bereits existieren)
        \item Im Portal: \newline
            \texttt{Übersicht | + Datenbank erstellen} für eine neue Datenbank direkt im Pool
    \end{itemize}
    \vspace*{\stretch{1}}
\end{flashcard}

\begin{flashcard}[Definition]{Elastische Pools für SQL-Datenbanken konfigurieren}
    \vspace*{\stretch{1}}
    \begin{itemize}
        \item DTU: drei Tarife, \texttt{Basic}, \texttt{Standard}, \texttt{Premium}
        \item vCore: Anzahl Kerne, Generation
        \item Portal: \texttt{Einstellungen | Konfiguration | Pooleinstellungen} im Pool-Blade
    \end{itemize}
    Datenbank:
    \begin{itemize}
        \item Grenzwerte pro Datenbank (min, max DTUs)
        \item Portal: \texttt{Einstellungen | Konfiguration | Pro Datenbank - Einstellungen} im Pool-Blade
    \end{itemize}
    \vspace*{\stretch{1}}
\end{flashcard}

\begin{flashcard}[Definition]{Datenbanken in Pool verwalten}
    \vspace*{\stretch{1}}
    \begin{itemize}
        \item CLI:\newline
            \texttt{az cli db create --resource-group \ldots\ --server \ldots\newline --name <db> --elastic-pool \ldots}
        \item PowerShell:\newline
            \texttt{Set-AzSqlDatabase}
    \end{itemize}
    \vspace*{\stretch{1}}
\end{flashcard}


\begin{flashcard}[Definition]{Zugriff auf SQL-Datenbank}
    \vspace*{\stretch{1}}
    \begin{itemize}
        \item Azure CLI: \texttt{az sql db \ldots}
        \item Verbindungszeichenfolge: \texttt{az sql db show-connection-string}
    \end{itemize}
    \vspace*{\stretch{1}}
\end{flashcard}

\subsubsectioncard{implement Azure SQL Database managed instances}

\subsubsectioncard{create, read, update, and delete data tables by using code}

\begin{flashcard}[Definition]{CRUD-Operationen}
    \vspace*{\stretch{1}}
    \begin{itemize}
        \item Erstellen\newline
            \texttt{CREATE TABLE \ldots} (Tabelle erstellen)\newline
            \texttt{INSERT INTO table (\ldots) VALUES (\ldots);} (Werte einfügen)
        \item Lesen\newline
            \texttt{SELECT x, y FROM table;}
        \item Update\newline
            \texttt{UPDATE table SET x=y WHERE a=b;}
        \item Löschen\newline
            \texttt{DELETE FROM table WHERE x=y;}
    \end{itemize}
    Login von CloudShell z.\,B. über \texttt{sqlcmd}
    \vspace*{\stretch{1}}
\end{flashcard}
