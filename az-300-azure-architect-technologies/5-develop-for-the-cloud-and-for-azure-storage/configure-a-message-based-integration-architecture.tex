\subsectioncard{Konfigurieren einer Nachrichtengestützten Architektur}

\begin{flashcard}[Definition]{SendGrid}
    \vspace*{\stretch{1}}
    \begin{itemize}
        \item Service zum Senden und Empfangen von Emails
        \item skalierbar
    \end{itemize}
    \vspace{1cm}
    Alternative zum Mailen: Action in Web App
    \vspace*{\stretch{1}}
\end{flashcard}

\begin{flashcard}[Definition]{Nachrichtenmodelle}
    \vspace*{\stretch{1}}
    \begin{itemize}
        \item Ziel: Kommunikation von Diensten
        \item lose Verbindung. Alle Kommunikation mit \emph{await} oder ähnlichen Konstrukten
        \item Nachrichten
            \begin{itemize}
                \item enthält komplette Rohdaten, nicht einen Verweis
                \item Sender erwartet, dass die Nachricht verarbeitet wird
            \end{itemize}
        \item Ereignisse
            \begin{itemize}
                \item Benachrichtigung über ein Ereignis (an oft viele Empfänger/Abonnenten)
                \item enthält oft einen Verweis auf Daten
                \item Sender hat keine Erwartung bezüglich Verarbeitung
                \item Ereignisse können zusammenhangslos sein, oder Teil einer Reihe
            \end{itemize}
    \end{itemize}
    \vspace*{\stretch{1}}
\end{flashcard}

\begin{flashcard}[Definition]{Notification Hub}
    \vspace*{\stretch{1}}
    \begin{itemize}
        \item Sended push-Benachrictigungen and alle Plattformen (iOS, Android, Windows, \ldots)
        \item Skalierbarer Service
        \item Quellen sowohl Azure als auch lokale Server
    \end{itemize}
    \vspace*{\stretch{1}}
\end{flashcard}

\begin{flashcard}[Definition]{Nachrichten-Zustellung in Azure}
    \vspace*{\stretch{1}}
    \begin{itemize}
        \item einfache Warteschlangen: Azure Queue Storage:\newline
            Teil eines Speicherkontos, keine Speicher-Limits, Fortschritt in der Schlange
        \item Azure Service Bus
            \begin{itemize}
                \item Warteschlangen:\newline
                    Warteschlange mit anderen Garantien als Azure Queue Storage
                \item Themen:\newline
                    Mehrere Abonnenten, die \emph{alle} die Nachricht verarbeiten müssen
            \end{itemize}
    \end{itemize}
    \vspace*{\stretch{1}}
\end{flashcard}

\begin{flashcard}[Definition]{Eigenschaften Azure Service-Bus}
    \vspace*{\stretch{1}}
    \begin{itemize}
        \item At-Most-Once-Zustellungsgarantie
        \item FIFO
        \item Transaktionen: Garantie, dass \emph{alle} Nachrichten einer Gruppe verarbeitet werden, \emph{oder keine}
        \item Unterstützt RBAC
        \item Nachrichtengröße von 64 KiB bis 256 KiB
        \item maximale Kapazität 80 G(i?)B
    \end{itemize}
    \vspace*{\stretch{1}}
\end{flashcard}

\begin{flashcard}[Definition]{Ereignis-Zustellung in Azure}
    \vspace*{\stretch{1}}
    \begin{itemize}
        \item Event Grid
        \item Event Hub
            \begin{itemize}
                \item Optimiert auf hohe Durchsätze, Sicherheit, Resilienz\newline
                    Big Data-Ereignisse: Unzählige
                \item ähnlich wie Event Grid, aber zusätzliche Features
            \end{itemize}
    \end{itemize}
    \vspace*{\stretch{1}}
\end{flashcard}

\begin{flashcard}[Definition]{Eigenschaften Azure Event Grid}
    \vspace*{\stretch{1}}
    \begin{itemize}
        \item Ereignisse: was ist passiert
        \item Maximale Ereignisgröße 64 KiB
        \item Ereignisquelle: Wo hat das Ereignis Stattgefunden.\newline
            Z.\,B. Storage, IoT Hub, \ldots
        \item Thema: Endpunkt, an den Ereignisse gesendet werden
        \item Abonnements: Weiterleitung von Ereignissen an Handler
        \item Handler: App/Dienst, der Ereignisse bearbeitet.\newline
            Z.\,B. Logik-App, Function, Webhook, \ldots
        \item Abrechnung pro Ereignis
    \end{itemize}
    \vspace*{\stretch{1}}
\end{flashcard}

\begin{flashcard}[Definition]{Event Grid erstellen}
    \vspace*{\stretch{1}}
    Automatisch von Azure-Fabric bereitgestellt.
    \vspace*{\stretch{1}}
\end{flashcard}

\begin{flashcard}[Definition]{Auf Ereignisse mit Event Grid reagieren}
    \vspace*{\stretch{1}}
    \begin{itemize}
        \item z.\,B. Logik-App
        \item als Trigger "Bei Auftreten eines Event-Grid-Ereignisses"
        \item[] oder
        \item "Bei Eintritt eines Ressourcenereignisses"
            \begin{itemize}
                \item Abonnement + Ressourcengruppe
                \item Ressourcentyp
                \item Ereignis-Typ
            \end{itemize}
    \end{itemize}
    \vspace*{\stretch{1}}
\end{flashcard}

\begin{flashcard}[Definition]{}
    \vspace*{\stretch{1}}
    \begin{itemize}
        \item
    \end{itemize}
    \vspace*{\stretch{1}}
\end{flashcard}

\begin{flashcard}[Definition]{Azure Relay-Service}
    \vspace*{\stretch{1}}
    \begin{itemize}
        \item Verfügbarmachen eines geschützten Dienstes (z.\,B. hinter Firewall) ohne Änderung der Netzwerkstruktur
        \item Azure Relay definiert HTTP-Endpunkt
        \item geschützter Dienst verbindet sich mit HTTP-Endpunkt (abgesichert mit Token)
        \item externe Clients verbinden sich mit HTTP-Endpunkt
        \item[$\Rightarrow$] Externer und geschützter Dienst können kommunizieren
    \end{itemize}
    \vspace*{\stretch{1}}
\end{flashcard}

\begin{flashcard}[Definition]{Erstellen von Azure Relay}
    \vspace*{\stretch{1}}
    \begin{itemize}
        \item benötigt:
            \begin{itemize}
                \item Abonnement + Ressourcengruppe
                \item Standort
                \item Name: global eindeutig, da Endpunkt
            \end{itemize}
        \item Im Portal: \texttt{Ressource erstellen | Relay}
        \item Für Anwendungen: \texttt{Freigegebene Zugriffsrichtlinien}:
        \begin{itemize}
            \item Primärschlüssel
            \item Verbindungszeichenfolge abrufen
        \end{itemize}
    \end{itemize}
    \vspace*{\stretch{1}}
\end{flashcard}

\begin{flashcard}[Definition]{Eigenschaften Azure Event-Hub}
    \vspace*{\stretch{1}}
    \begin{itemize}
        \item Partitionen:\newline
            Ereignispuffer zum Zwischenspeichern. Mindestens zwei, pro Puffer verschiedene Abonnements
        \item Capture:\newline
            Dauerhaftes Speichern von Ereignissen (Data Lake, Blob)
        \item Authentifizierung mit Token
        \item ermöglicht aggregierte Analyse der Ereignisse
        \item zuverlässig (Zwischenspeicher) und Resilienz
        \item millionen von Ereignissen mit geringer Latenz
    \end{itemize}
    \vspace*{\stretch{1}}
\end{flashcard}

\begin{flashcard}[Definition]{Event-Hub erstellen (Namespace)}
    \vspace*{\stretch{1}}
    \begin{itemize}
        \item Zunächst \emph{Namespace} erstellen, dann einzelne Event Hubs
        \item Benötigt:
            \begin{itemize}
                \item Abonnement + Ressourcengruppe
                \item Namespacename: global eindeutig
                \item Tarif: Basic, Standard, Dedicated
                \item Durchsatzeinheiten: kann nicht geändert werden, pro Namespace!
            \end{itemize}
        \item Im Portal: {+ Resource Erstellen | Event Hubs}
        \item CLI: \texttt{az eventhubs namespace create \ldots}
        \item zum Senden und Empfangen: Zugriffsrichtlinien mit Key und Verbindungszeichenfolge
    \end{itemize}
    \vspace*{\stretch{1}}
\end{flashcard}

\begin{flashcard}[Definition]{Event-Hub erstellen}
    \vspace*{\stretch{1}}
    \begin{itemize}
        \item wenn Namespace verfügbar ist
        \item Benötigt:
            \begin{itemize}
                \item
            \end{itemize}
        \item Im Portal: \texttt{Entitäten | Event Hubs | + Event Hub} im Event-Hubs-Blade des Namespaces
    \end{itemize}
    \vspace*{\stretch{1}}
\end{flashcard}

\begin{flashcard}[Definition]{Service Bus erstellen (Namespace)}
    \vspace*{\stretch{1}}
    \begin{itemize}
        \item Zunächst \emph{Namespace} erstellen, dann einezelne Topics und Warteschlangen
        \item Benötigt:
            \begin{itemize}
                \item Abonnement + Resourcengruppe
                \item Name: global eindeutig, da URL
                \item Tarif
                \item Standort
            \end{itemize}
        \item Im Portal: \texttt{+ Neue Ressource | Alle Dienste | Service Bus}
    \end{itemize}
    \vspace*{\stretch{1}}
\end{flashcard}

\begin{flashcard}[Definition]{Bus-Warteschlangen und Bus-Thema erstellen (Service Bus)}
    \vspace*{\stretch{1}}
    \begin{itemize}
        \item Benötigt:
            \begin{itemize}
                \item Name
                \item Größe (Gigabyte)
                \item Gültigkeitsdauer von Nachrichten
            \end{itemize}
        \item Bus-Warteschlange im Portal:\newline
            \texttt{+Warteschlange} im Namespace-Blade des Service Bus
        \item Bus-Thema im Portal:\newline
            \texttt{+Topic} im Namespace-Blade des Service Bus
        \item Topics können nur ab Standard-Tarif erstellt werden
        \item Anmelden mit Zugriffsrichtlinie: Schlüssel und Verbindungszeichenfolge
    \end{itemize}
    \vspace*{\stretch{1}}
\end{flashcard}

\begin{flashcard}[Definition]{}
    \vspace*{\stretch{1}}
    \begin{itemize}
        \item
    \end{itemize}
    \vspace*{\stretch{1}}
\end{flashcard}

\begin{flashcard}[Definition]{}
    \vspace*{\stretch{1}}
    \begin{itemize}
        \item
    \end{itemize}
    \vspace*{\stretch{1}}
\end{flashcard}

\begin{flashcard}[Definition]{}
    \vspace*{\stretch{1}}
    \begin{itemize}
        \item
    \end{itemize}
    \vspace*{\stretch{1}}
\end{flashcard}
