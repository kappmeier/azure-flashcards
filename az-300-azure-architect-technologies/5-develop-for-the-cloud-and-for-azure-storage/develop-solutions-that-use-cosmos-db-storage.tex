\subsectioncard{Entwicklung von Lösungen mit Azure Cosmos DB}

\subsubsectioncard{create, read, update, and delete data by using appropriate APIs}

\begin{flashcard}[Definition]{Cosmos DB}
    \vspace*{\stretch{1}}
    Account:
    \begin{itemize}
        \item Benötigt Cosmos DB-Konto (wie Speicherkonto)
        \item verschlüsselung: immer aktiviert (at rest)
    \end{itemize}
    mehrere Datenmodell-APIs verfügbar: SQL, Graphen, Tabellen
    \vspace*{\stretch{1}}
\end{flashcard}

\begin{flashcard}[Definition]{Cosmos DB-Struktur}
    \vspace*{\stretch{1}}
    \begin{enumerate}
        \item Instanz
        \item Datenbank
        \item Sammlung
        \item Dokument
    \end{enumerate}
    \vspace*{\stretch{1}}
\end{flashcard}

\begin{flashcard}[Definition]{Abrechnung}
    \vspace*{\stretch{1}}
    \begin{enumerate}
        \item Anforderungseinheit (RU): 1 GET-Anfrage für ein Dokument per id von 1 KiB
        \item Andere Anfragen (PUT, ...) teurer
        \item Preis steigt mit höheren Konsistenz-anforderungen (stark, begrenzte Veraltung)
        \item Für gleiche Anfragen ist der Preis deterministisch
        \item alle Container teilen sich die Anforderungseinheiten
    \end{enumerate}
    $\Rightarrow$ Anforderungseinheiten ermöglichen Skalierung und Anpassung an Last
    \vspace*{\stretch{1}}
\end{flashcard}

\begin{flashcard}{Azure Cosmos DB-Konto erstellen}
    \vspace*{\stretch{1}}
    \begin{itemize}
        \item Benötigt:
        \begin{itemize}
            \item Abonnement + Ressourcengruppe
            \item global eindeutiger Name
            \item API: Core (SQL), MongoDB, Cassandra, Tabelle, Gremlin
            \item Location
            \item Kontotyp
            \item Schreibvorgänge in mehreren Regionen?
        \end{itemize}
        \item Portal: \texttt{Neue Ressource | Erste Schritte | Azure Cosmos DB} (oder suchen)
        \item CLI:\newline
        \texttt{az cosmosdb create --name NAME --kind GlobalDocumentDB --resource-group grp}
    \end{itemize}
    \vspace*{\stretch{1}}
\end{flashcard}

\begin{flashcard}{Azure Cosmos DB}
    \vspace*{\stretch{1}}
    \begin{itemize}
        \item Benötigt:
        \begin{itemize}
            \item Cosmos DB-Kontoname + Resourcengruppe
            \item Datenbank-ID (Name), kann mehrere Container enthalten
            \item Durchsatz in RUs pro Sekunde
        \end{itemize}
        \item CLI:\newline
        \texttt{az cosmosdb sql database create}
    \end{itemize}
    $\Rightarrow$ Container muss per CLI extra erstellt werden
    \vspace*{\stretch{1}}
\end{flashcard}

\begin{flashcard}{Azure Cosmos DB Container}
    \vspace*{\stretch{1}}
    \begin{itemize}
        \item Benötigt:
        \begin{itemize}
            \item Cosmos DB-Kontoname + Resourcengruppe
            \item Datenbank-ID (Name), kann mehrere Container enthalten
            \item Container-ID (Name)
            \item Durchsatz in RUs pro Sekunde
        \end{itemize}
        \item Portal: \texttt{Daten-Explorer | Neuer Container} im Cosmos DB-Konto-Blade
        \item CLI:\newline
        \texttt{az cosmosdb sql container create}
    \end{itemize}
    \vspace*{\stretch{1}}
\end{flashcard}

\begin{flashcard}[Definition]{Daten-Explorer}
    \vspace*{\stretch{1}}
    \begin{itemize}
        \item Benutzeroberfläche um Cosmos DB-Daten zu verwalten
        \item Im Portal: \texttt{Daten-Explorer} im Blade der Cosmos DB
    \end{itemize}
    \vspace*{\stretch{1}}
\end{flashcard}

\begin{flashcard}[Definition]{Daten in Cosmos DB einfügen}
    \vspace*{\stretch{1}}
    \begin{itemize}
        \item Auswahl des Containers: \texttt{Konto | Container | Datenbank}
        \item Neues Element erstellen: \texttt{New Item}
        \item Dokument kann eingetragen Werden (als JSON)
    \end{itemize}
    \vspace*{\stretch{1}}
\end{flashcard}

\begin{flashcard}[Definition]{Daten Lesen}
    \vspace*{\stretch{1}}
    \begin{itemize}
        \item Eine SQL-Abfrage erstellen: Innerhalb einer Datenbank
        \item \texttt{FROM} kann einen Container oder eine Datenbank sein
        \item Zugriff auf Datenbank: \texttt{Container.Datenbank}
        \item Auch auf unter-elemente des JSON-Objekts kann so zugegriffen werden
    \end{itemize}
    \vspace*{\stretch{1}}
\end{flashcard}

\begin{flashcard}[Definition]{Gespeicherte Prozeduren}
    \vspace*{\stretch{1}}
    \begin{itemize}
        \item Sammlung von mehreren Operationen in einer Transaktion
        \item einzige Möglichkeit \emph{ACID}-Transaktionen zu garantieren
        \item im Container gespeichert
        \item JavaScript
    \end{itemize}
    \vspace*{\stretch{1}}
\end{flashcard}

\begin{flashcard}[Definition]{Verbinden}
    \vspace*{\stretch{1}}
    \begin{itemize}
        \item Verbindungs-Richtlinie: Liste von bevorzugten Regionen (geordnet)
        \item Verbindung herstellen
        \item benötigt den Schlüssel für die Verbindung
    \end{itemize}
    \vspace*{\stretch{1}}
\end{flashcard}

\subsubsectioncard{implement partitioning schemes}

\begin{flashcard}[Definition]{Partitionierung}
    \vspace*{\stretch{1}}
    \begin{itemize}
        \item horizontale Skalierung für Cosmos DB-Datenbanken
        \item \emph{Partitionsschlüssel}: Strategie für Erstellung von Partitionen \emph{pro Container}\newline
            \emph{Kann nicht geändert werden!}
        \item \emph{Hot Partition}: eine Partition, die deutlich mehr Anforderungen erhält, als andere\newline
            Schlüssel sollten sowohl eindeutig sein als auch zum Suchen benutzt werden (\emph{id} z.\,B.)
        \item Limit: 20 GiB pro Partitionsschlüssel\newline
            falls Schlüssel zu groß: Schlüssel kombinieren
        \item physische Partition: von Azure Cosmos DB verwaltet \newline
            Abhängig von Anforderungen der Leistung
    \end{itemize}
    \vspace*{\stretch{1}}
\end{flashcard}

\begin{flashcard}[Definition]{Partitionsschlüssel auswählen}
    \vspace*{\stretch{1}}
    \begin{itemize}
        \item viele Werte sind gut. Ermöglichen gute Skalierung (\emph{Hohe Kardinalität})
        \item am häufigsten in \texttt{WHERE} enthaltener Wert ist ein Kandidat
        \item die Anforderungseinheiten (RUs) sollten gleichmäßig auf alle Partitionen aufgeteilt sein
        \item Speicherplatz sollte gleichmäßig aufgeteilt werden
    \end{itemize}
    \vspace*{\stretch{1}}
\end{flashcard}

\subsubsectioncard{set the appropriate consistency level for operations}

\begin{flashcard}[Definition]{Cosmos DB Datenreplikation}
    \vspace*{\stretch{1}}
    \begin{itemize}
        \item für Hochverfügbarkeit
        \item mehrere Regionen können ausgewählt werden
        \item erlaubt Failover
        \item einzelne Schreib-Regionen: schreiben nur in einer Region
        \item multi Schreib-Regionen: teuer
        \item automatisches Failover: falls die Schreibregion ausfällt, automatisch eine andere Region zur Schreibregion promovieren
    \end{itemize}
    \vspace*{\stretch{1}}
\end{flashcard}

\begin{flashcard}[Definition]{Datenreplikation aktivieren}
    \vspace*{\stretch{1}}
    \begin{itemize}
        \item Portal: \texttt{Einstellungen | Daten global replizieren} im Cosmos DB-Konto-Blade
        \item Auswahl: \texttt{+ Region hinzufügen}
    \end{itemize}
    \vspace*{\stretch{1}}
\end{flashcard}

\begin{flashcard}[Definition]{Konsistenz}
    \vspace*{\stretch{1}}
    \begin{itemize}
        \item Wenn Daten repliziert werden, müssen sie konsistent gehalten werden
        \item Multimaster: regionen, die Schreibberechtigung haben (Aktiv/Aktiv)
        \begin{itemize}
            \item 99,999\% Lese/Schreibverfügbarkeit
            \item Unbegrenzte Skalierbarkeit
            \item Schreiblatenz < 10\,ms
        \end{itemize}
        \item Konfliktlösungen:
        \begin{itemize}
            \item Last-Writer-Wins (LWW): ganzzahlige Eigenschaft im Dokument, z.\,B. Zeitstempel
            \item Benutzerdefiniert: spezielle gespeicherte Prozedur, fallback Asynchron
            \item Asynchron: Konflikte werden ausgeschlossen und in einem Feed an Benutzer geleitet
        \end{itemize}
    \end{itemize}
    \vspace*{\stretch{1}}
\end{flashcard}

\begin{flashcard}[Definition]{Konsistenzmodelle}
    \vspace*{\stretch{1}}
    Von Konsistenz zu Inkonsistenz:
    \begin{enumerate}
        \item stark\newline
            Linearisiert. Garantiert neueste Version gelesen.
        \item Begrenzte veraltung\newline
            Präfixkonsistenz + Lesen höchstens hängt k zurück
        \item Sitzung\newline
            Präfixkonsistenz + Monotone schreib/lesevorgänge
        \item Präfixkonsistenz\newline
            Aktualisierungen sind Präfix aller Aktualisierungen ohne Lücken
        \item letztlich\newline
            Lesevorgänge in falscher Reihenfolge möglich
    \end{enumerate}
    \vspace*{\stretch{1}}
\end{flashcard}

\begin{flashcard}[Definition]{Starke Konsistenz}
    \vspace*{\stretch{1}}
    \begin{itemize}
        \item Linearisierbarkeit:\newline
            Lesevorgänge geben garantiert die neueste Version zurück
        \item Schreibvorgang wird erst sichtbar, nachdem er repliziert worden ist
        \item Lesevorgänge liefer immer neueste bestätigte Version zurück
        \item wenn mehrere Regionen genutzt werden, kein Vorteil durch geringe Latenz
        \item hohe Kosten (gleich wie \emph{begrenzte veraltung})
    \end{itemize}
    \vspace*{\stretch{1}}
\end{flashcard}

\begin{flashcard}[Definition]{Begrenzte veraltung}
    \vspace*{\stretch{1}}
    \begin{itemize}
        \item Ergebnis eines Lesevorgangs ist begrenzt alt
            \begin{itemize}
                \item entweder $K$ Versionen (Präfixlnge)
                \item oder Zeitintervall $t$
            \end{itemize}
            (wie starke Konsistenz, aber zeitlich verzögert)
        \item ideal für hohe Konsistenz und hohe Verfügbarkeit (99,99\%) und niedrige Latenz
        \item höhere Konsistenzgarantie als \emph{Sizung}, \emph{Präfixkonsistenz} und \emph{letztliche Konsistenz}
        \item unterstützen beliebige Anzahl von Regionen
        \item höhere Kosten, wie bei \emph{stark}er Konsistenz
    \end{itemize}
    \vspace*{\stretch{1}}
\end{flashcard}

\begin{flashcard}[Definition]{Sitzungskonsistenz}
    \vspace*{\stretch{1}}
    \begin{itemize}
        \item Beschränkt auf Clientsitzung (im Unterschied zu globaler Konsistenz)
        \item Geeignet für Szenarien mit Geräte- oder Benutzersitzung\newline
            Read Your Own Writes
        \item Lesen und Schreiben mit niedrigster Latenz
        \item unterstützen beliebige Anzahl von Regionen
        \item mittlere Kosten (bezüglich Anforderungseinheiten (RUs))
    \end{itemize}
    \vspace*{\stretch{1}}
\end{flashcard}

\begin{flashcard}[Definition]{Präfixkonsistenz}
    \vspace*{\stretch{1}}
    \begin{itemize}
        \item Konsistenz für clients in einer Region oder mit Single-Master-Konto. Multi-Region-Kontos: letztlich konsistent
        \item Keine Lücken oder Lesevorgänge außer der Reihenfolge
        \item unterstützen beliebige Anzahl von Regionen
    \end{itemize}
    \vspace*{\stretch{1}}
\end{flashcard}

\begin{flashcard}[Definition]{Letztliche Konsistenz}
    \vspace*{\stretch{1}}
    \begin{itemize}
        \item alle Replikate konvergieren zum gleichen Zustand
        \item es können sogar ältere Werte als bereits gelesene Werte gelesen werden
        \item niedrigste Latenz für Lese- und Schreibvorgänge (dafür auch schwächste Konsistenz)
        \item niedrigste Kosten (bezüglich Anforderungseinheiten (RUs))
    \end{itemize}
    \vspace*{\stretch{1}}
\end{flashcard}

\begin{flashcard}[Definition]{Konsistenz einstellen}
    \vspace*{\stretch{1}}
    \begin{itemize}
        \item Portal: \texttt{Einstellungen | Standardkonsistenz} im Cosmos DB-Konto-Blade
    \end{itemize}
    \vspace*{\stretch{1}}
\end{flashcard}
