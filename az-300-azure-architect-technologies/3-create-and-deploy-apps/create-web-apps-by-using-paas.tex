\subsectioncard{Erstellen von Webanwendungen und PaaS}

\subsubsectioncard{create an Azzure app service Web App}

\begin{flashcard}[Definition]{Was wird für eine Web App benötigt?}
    \vspace*{\stretch{1}}
    \begin{enumerate}
        \item App Service Plan
            \begin{itemize}
                \item App Service Plan: dauerhafte Bezahlung
                \item Consumption Plan: bezahlung nach Nutzen, maximale Laufzeit einer Funktion 10 Minuten
            \end{itemize}
        \item Web App
    \end{enumerate}
    \vspace*{\stretch{1}}
\end{flashcard}



\begin{flashcard}[Definition]{App Service Plan erstellen}
    \vspace*{\stretch{1}}
    \begin{itemize}
        \item Benötigt:
        \begin{itemize}
            \item Resourcegruppe + Location
            \item SKU
            \item Name
        \end{itemize}
        \item CLI: \texttt{az appservice plan create --resource-group \ldots\ --location \ldots --SKU \ldots\ --number-of-workers \ldots}
            \begin{itemize}
                \item \texttt{--is-linux} für Linux-Maschinen
            \end{itemize}
    \end{itemize}
    \vspace*{\stretch{1}}
\end{flashcard}

\begin{flashcard}[Definition]{App Service Plan-Tarife}
    \vspace*{\stretch{1}}
    Aufteilung basierend auf Rechen-Ressourcen:
    \begin{itemize}
        \item Freigegebene Ressourcen
            \begin{itemize}
                \item Free (F1)
                \item Shared (D1)
            \end{itemize}
        \item Dedizierte Ressourcen (1, 2, 4 CPUs)
            \begin{itemize}
                \item Basic (B1, B2, B3)
                \item Standard (S1, S2, S3)
                \item Premium
                \item Premium v2 (P1v2, P2v2, P3v3)
            \end{itemize}
        \item Isoliert (I1, I2, I3)
    \end{itemize}
    \vspace*{\stretch{1}}
\end{flashcard}

\begin{flashcard}[Definition]{App Service Plan-Features Free/Shared}
    \vspace*{\stretch{1}}
    \begin{itemize}
        \item nur für Entwicklung und Tests
        \item Ausführung auf Maschinen, die auch andere Azure-Nutzer verwenden
    \end{itemize}
    Eigenschaften:
    \begin{itemize}
        \item 10/100 Apps
        \item 1 G(i?)B Festplattenspeicher
        \item Ab \emph{Shared}: benutzerdefinierte Domäne
        \item Nur Free: kostenlos
    \end{itemize}
    \vspace*{\stretch{1}}
\end{flashcard}

\begin{flashcard}[Definition]{App Service Plan-Features Basic/Standard/Premium/Premium v2}
    \vspace*{\stretch{1}}
    \begin{itemize}
        \item Nur Apps desselben Plans können sich eine VM teilen
        \item bessere Stufe $\Rightarrow$ höhere Anzahl Instanzen für horizontale Skalierung verfügbar
    \end{itemize}
    Eigenschaften: (Über Pläne für freigegebene Resourcen hinaus)
    \begin{itemize}
        \item Zertifikate\vspace{-0.5mm}
        \item Unbegrenzte App-Anzahl\vspace{-0.5mm}
        \item Hybrid-Konnektivität\vspace{-0.5mm}
        \item SLA: 99,95 \%\vspace{-0.5mm}
        \item Skalierbarkeit: 3/10/30 Instanzen\vspace{-0.5mm}
        \item Ab \emph{Standard}:\vspace{-1mm}
            \begin{itemize}
                \item automatische Skalierung\vspace{-1mm}
                \item integration virtueller Netzwerke\vspace{-1mm}
                \item 5 Staging-Slots, dann 20\vspace{-1mm}
                \item Backups\vspace{-1mm}
            \end{itemize}
    \end{itemize}
    \vspace*{\stretch{1}}
\end{flashcard}

\begin{flashcard}[Definition]{App Service Plan-Features Isoliert}
    \vspace*{\stretch{1}}
    \begin{itemize}
        \item Netzwerkisolation: die dedizierten VMs werden in eigneen virtuellen Netzen ausgeführt\newline
            (zusätzlich zu Isolation für Berechnungen (wie dedizierte Ressourcen))
    \end{itemize}
    \vspace*{\stretch{1}}
\end{flashcard}

\begin{flashcard}[Definition]{Web App erstellen}
    \vspace*{\stretch{1}}
    \begin{itemize}
        \item Benötigt:
        \begin{itemize}
            \item Resourcegruppe
            \item Name
            \item App-Service-Plan
            \item deployment source (URL, Branch)
        \end{itemize}
        \item CLI: \texttt{az webapp create --name \ldots --resource-group}
    \end{itemize}
    \vspace*{\stretch{1}}
\end{flashcard}

\subsubsectioncard{create documentation for the API}

\subsubsectioncard{create an App Service Web App for Containers}

\begin{flashcard}[Definition]{Web App erstellen}
    \vspace*{\stretch{1}}
    \begin{itemize}
        \item Für Linux mit Docker-Image von DockerHub:
        \item CLI: \texttt{az webapp create --name \ldots\ --resource-group \ldots\ \\--deployment-container-image-name}
    \end{itemize}
    \vspace*{\stretch{1}}
\end{flashcard}

\subsubsectioncard{create an App Service background task by using WebJobs}

\begin{flashcard}[Definition]{WebJobs}
    \vspace*{\stretch{1}}
    \begin{itemize}
        \item Teil von Azure App Service
        \item Arten:
            \begin{itemize}
                \item Fortlaufend: werden in einer Schleife ausgeführt
                \begin{itemize}
                    \item Multi-Instance: Script auf jeder Instanz ausführen (Ja/Nein).\newline
                        (Nicht bei freigegebenen Resourcen (Free, Shared), da nicht horizontal skalierbar)
                \end{itemize}
                \item Ausgelöst: werden nach Zeitplan oder manuell ausgeführt
            \end{itemize}
    \end{itemize}
    \vspace*{\stretch{1}}
\end{flashcard}

\begin{flashcard}[Definition]{Vergleich Funktionen/WebJob}
    \vspace*{\stretch{1}}
    Features von Funktionen:
    \begin{itemize}
        \item automatische Skalierung
        \item Entwicklung im Browser
        \item Bezahlung nach Nutzung
        \item Integration in Logik-Apps
    \end{itemize}
    \vspace*{\stretch{1}}
\end{flashcard}


\subsubsectioncard{enable diagnostics logging}

\begin{flashcard}[Definition]{Diagnoseprotokollierung}
    \vspace*{\stretch{1}}
    \begin{itemize}
        \item automatisch verfügbares Logging zum Debuggen von Web Service-Anwendungen
        \item Aktivieren:\newline
            \texttt{App Service-Protokolle} im App-Blade
        \item Daten können im Speicher der Anwendung oder als Blob gespeichert werden
    \end{itemize}
    \vspace*{\stretch{1}}
\end{flashcard}

\begin{flashcard}[Definition]{Arten von Web Service-Logging}
    \vspace*{\stretch{1}}
    \begin{itemize}
        \item Anwendungsprotokollierung
        \item Webserverprotokollierung
        \item Detaillierte Fehlermeldungen
        \item Ablaufverfolgung fehlgeschlagener Anforderungen
        \item Bereitstellungsprotokollierung
    \end{itemize}
    \vspace*{\stretch{1}}
\end{flashcard}

\begin{flashcard}[Definition]{Web Service Anwendungsprotokollierung}
    \vspace*{\stretch{1}}
    \begin{itemize}
        \item Windows + Linux
        \item normale Anwendungslogs
        \item Webframework und Anwendungscode
        \item Levels Kritisch bis Trace
    \end{itemize}
    \vspace*{\stretch{1}}
\end{flashcard}

\begin{flashcard}[Definition]{Web Service Webserverprotokollierung}
    \vspace*{\stretch{1}}
    \begin{itemize}
        \item[!] nur Windows
        \item protokollierung von HTTP-Anforderungen
        \item geeignet für Metriken
    \end{itemize}
    \vspace*{\stretch{1}}
\end{flashcard}

\begin{flashcard}[Definition]{Detaillierte Web Service Fehlermeldungen}
    \vspace*{\stretch{1}}
    \begin{itemize}
        \item[!] nur Windows
        \item Kopie der Fehlerseiten
        \item ab 400 aufwärts
    \end{itemize}
    \vspace*{\stretch{1}}
\end{flashcard}

\begin{flashcard}[Definition]{Ablaufverfolgung fehlgeschlagener Web Service Anforderungen}
    \vspace*{\stretch{1}}
    \begin{itemize}
        \item[!] nur Windows
        \item detaillierte Verfolgung im Fehlerfall
        \item inklusive IIS
        \item enthält genaue Timings für jede Komponente
        \item praktisch, um Seitenperformance nachzuvollziehen
        \item genaue analyse eines speziellen HTTP-Fehlers
    \end{itemize}
    \vspace*{\stretch{1}}
\end{flashcard}

\begin{flashcard}[Definition]{Web Service Bereitstellungsprotokollierung}
    \vspace*{\stretch{1}}
    \begin{itemize}
        \item Windows + Linux
        \item automatische Protokollierung der Bereitstellung
    \end{itemize}
    \vspace*{\stretch{1}}
\end{flashcard}

\begin{flashcard}[Definition]{Backup}
    \vspace*{\stretch{1}}
    \begin{itemize}
        \item ein Backup kann erstellt und wieder eingespielt werden
        \item ab Standard-App Service Plan
    \end{itemize}
    \vspace*{\stretch{1}}
\end{flashcard}

\begin{flashcard}[Definition]{Snapshots}
    \vspace*{\stretch{1}}
    \begin{itemize}
        \item Snapshots ab Premium
        \item kann nur in die gleichen Slots zurückgespielt werden
    \end{itemize}
    \vspace*{\stretch{1}}
\end{flashcard}
