\subsectioncard{Design und Entwicklung von Apps in Containern}

\subsubsectioncard{configure diagnostic settings on resources}

\subsubsectioncard{create a container image by using a Dockerfile}

\begin{flashcard}[Definition]{Docker-Images mit Dockerfile}
    \vspace*{\stretch{1}}
    \begin{itemize}
        \item manuelle Anpassungen eines Docker-Images
        \item Datei namens \emph{Dockerfile}
        \item Teile eines Dockerfiles:
            \begin{itemize}
                \item \texttt{FROM}: Basisimage, das angepasst word
                \item \texttt{RUN}: beliebiger Befehl, der im Container ausgeführt wird
                \item \texttt{COPY}: kopiert Dateien vom Hostcomputer
                \item \texttt{EXPOSE}: öffnet einen Port
                \item \texttt{ENTRYPOINT}: was im Container ausgeführt wird
            \end{itemize}
        \item mit \texttt{docker build} wird ein neues Image gebaut
    \end{itemize}
    \vspace*{\stretch{1}}
\end{flashcard}

\subsubsectioncard{creat an Azure Kubernetes Service}

\begin{flashcard}[Definition]{Kobernetes-Komponenten}
    \vspace*{\stretch{1}}
    \begin{itemize}
        \item kube-apiserver\newline
            stellt Kubernetes-API bereit
        \item etcd\newline
            Key-Value-Speicher in Kubernetes mit Statusdaten
        \item kube-scheduler\newline
            weist Knoten zu
        \item kube-advisor\newline
            gives best practice hints
    \end{itemize}
    \vspace*{\stretch{1}}
\end{flashcard}

\begin{flashcard}[Definition]{Azure Kubernets-Cluster erstellen}
    \vspace*{\stretch{1}}
    \begin{itemize}
        \item Benötigt:
            \begin{itemize}
                \item Resourcegruppe + Abonnement
                \item Name
            \end{itemize}
        \item CLI:\newline
            \texttt{az aks create --resource-group \ldots\ --name \ldots}
    \end{itemize}
    \vspace*{\stretch{1}}
\end{flashcard}

\begin{flashcard}[Definition]{Eigenschaften AKS}
    \vspace*{\stretch{1}}
    \begin{itemize}
        \item Master von Azure bereitgestellt
        \item Knotenzahl konfigurierbar
        \item[!] Knotengröße nicht mehr änderbar
        \item HTTP-Routing
        \item SSH-Zugriff
    \end{itemize}
    \vspace*{\stretch{1}}
\end{flashcard}


\subsubsectioncard{publish an image to the Azure Container Registry}

\begin{flashcard}[Definition]{Azure Containerregistrierung}
    \vspace*{\stretch{1}}
    \begin{itemize}
        \item In Azure bereitgestellte Registrierung von Docker-Containern\newline
            Analog zu Docker Hub
        \item Benötigt:
            \begin{itemize}
                \item Name: global eindeutig
            \end{itemize}
        \item Erstellen:
            \begin{itemize}
                \item Portal: \texttt{+ Ressource erstellen | Container | Container Registry}
                \item CLI: \texttt{az acr create --name \ldots}
            \end{itemize}
        \item Administratorbenutzer für Zugriffsschlüssel
        \item[!] eine private Registrierunt, anonymer Zugriff ist nicht möglich
    \end{itemize}
    \vspace*{\stretch{1}}
\end{flashcard}

\begin{flashcard}[Definition]{Ein Image hochladen}
    \vspace*{\stretch{1}}
    \begin{itemize}
        \item \texttt{docker login myregistry.azurecr.io}\newline
            Credentials aus CLI oder Portal abrufen
        \item \texttt{docker tag myapp:v1 myregistry.azurecr.io/myapp:v1}
        \item \texttt{docker push myregistry.azurecr.io/myapp:v1}
    \end{itemize}
    \vspace*{\stretch{1}}
\end{flashcard}

\begin{flashcard}[Definition]{Ein Containerimage erstellen}
    \vspace*{\stretch{1}}
    \begin{itemize}
        \item Voraussetzung: \texttt{Dockerfile}
        \item \texttt{az acr build --registry \ldots --image name:version <path-to-dockerfile>}\newline
            Alternativ \texttt{--file} für die genaue Docker-Datei
        \item Das Image ist anschließend in der Registrierung gespeichert
    \end{itemize}
    \vspace*{\stretch{1}}
\end{flashcard}

\begin{flashcard}[Definition]{Container in der Registrierung}
    \vspace*{\stretch{1}}
    \begin{itemize}
        \item \texttt{az acr repository list --name registry-name}
    \end{itemize}
    \vspace*{\stretch{1}}
\end{flashcard}

\begin{flashcard}[Definition]{Container in der Registrierung replizieren}
    \vspace*{\stretch{1}}
    \begin{itemize}
        \item um Container in anderer Region zu speichern (zwecks schnellem Zugriff)
        \item \texttt{az acr replication create --registry \ldots --location <target>}
    \end{itemize}
    \vspace*{\stretch{1}}
\end{flashcard}

\subsubsectioncard{implement an application that runs on an Azure Container Instance}

\begin{flashcard}[Definition]{Container-Instanz starten}
    \vspace*{\stretch{1}}
    \begin{itemize}
        \item Benötigt:
            \begin{itemize}
                \item Abonnement + Ressourcengruppe
                \item Containername
                \item Quelle: Azure Containerregistrierung (oder Docker Hub, Schnellstart)
                \item Image und Tag
            \end{itemize}
        \item Portal: \texttt{+ Ressoruce Erstellen | Container | Container Instance}
    \end{itemize}
    \vspace*{\stretch{1}}
\end{flashcard}

\subsubsectioncard{manage container settings by using code}
