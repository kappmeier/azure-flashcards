% A flashcard like document containing content relevant to prepare for Microsoft AZ-300 Azure
% architect technologies exam. Supposed to be used by mobile/tablet phones, i.e. each page contains
%  single flash card.
\documentclass{scrartcl}

\usepackage[hidelinks]{hyperref}

% Set up XeLaTex font configuration
\usepackage{fontspec}
%\setmainfont{Latin Modern Sans}
\setmainfont{Myriad Pro}

\usepackage{polyglossia}
\enablehyphenation
\setdefaultlanguage[]{german}
\setotherlanguage[]{english}

% Size of flash cards
\usepackage[a6paper,landscape,margin=1cm]{geometry}

% Vertical alignment also of the title page
\usepackage{titling}
\renewcommand\maketitlehooka{\null\mbox{}\vfill}
\renewcommand\maketitlehookd{\vfill\null}

% Nothing to see on flash cards
\pagestyle{empty}
\thispagestyle{empty}

% Support showing the license as required by CC-BY-SA-4.0
\usepackage[type={CC},modifier={by-nc-sa},version={4.0}]{doclicense}

% Use meta data for creative commons in the produced pdf
\usepackage{xmpincl}
\includexmp{../CC-BY-SA-4.0}

% Flashcard environment similar to the flashcards package
% Similar to https://tex.stackexchange.com/a/89347/47921
\newenvironment{flashcard}[2][]{%
    #1
    \vfill
    \centerline{\Large{#2}}
    \vfill
\newpage
}
{\newpage}

% No intendation for multi line text on cards
\setlength{\parindent}{0cm}

% Use a single card for section
\newcommand{\sectioncard}[1]{
    \vspace*{\stretch{1}}
    \section{#1}
    \vspace*{\stretch{1}}
    \pagebreak
}

% Use a single card for sub section
\newcommand{\subsectioncard}[1]{
    \vspace*{\stretch{1}}
    \subsection{#1}
    \vspace*{\stretch{1}}
    \pagebreak
}

% Invisible sub section, not actually a card
\newcommand{\subsubsectioncard}[1]{
    \addcontentsline{toc}{subsubsection}{\protect\numberline{\thesubsubsection}#1}%
    \sectionmark{#1}
}

% Other packages required for content
\usepackage{booktabs}
\newcommand{\tabitem}{~~\llap{\textbullet}~~}
\newcommand{\tabitemindent}{~~~~}

\begin{document}

    \title{Microsoft Certified Azure Solutions Architect Expert Flashcards}
    \date{2020-09-27}
    \author{Jan-Philipp Kappmeier}

    \clearpage\maketitle
    \thispagestyle{empty}
    \pagebreak

    \sectioncard{Bereitstellung und Konfiguration von Infrastruktur}

    \subsectioncard{Analysieren von Resourcen-Nutzung und -Verbrauch}

\subsubsectioncard{configure diagnostic settings on resources}

\begin{flashcard}[Definition]{Diagnoseeinstellungen überprüfen}
    \vspace*{\stretch{1}}
    \begin{itemize}
        \item \texttt{Azure Monitor | Diagnoseeinstellungen}: zeigt für bis 5 Abonnements alle Resourcen
        \item \texttt{Diagnoseeinstellungen} für eine Ressourcengruppen zeigt Einstellungen für alle Ressourcen
         \item \texttt{Diagnoseeinstellungen} für eine einzelne Ressource
    \end{itemize}
    \vspace*{\stretch{1}}
\end{flashcard}

\begin{flashcard}[Definition]{Diagnoseeinstellungen}
    \vspace*{\stretch{1}}
    \begin{itemize}
        \item Anwendungslogs: Protokollierung der Anwendung
            \begin{itemize}
                \item Kritisch
                \item Verbose
            \end{itemize}
        \item Systemlogs: Systemmeldungen
        \item Absturzabbild: erstellt einen Speicherabbild, wenn eine Anwendung abstürzt
    \end{itemize}
    \vspace*{\stretch{1}}
\end{flashcard}

\subsubsectioncard{create baseline for resources}

\begin{flashcard}[Definition]{Dynamische Regeln}
    \vspace*{\stretch{1}}
    \begin{itemize}
        \item für eine Metrik-Regel typ \texttt{Dynamisch} auswählen
        \item von Azure wird ein dynamischer Schwellwert aus der Vergangenheit berechnet
        \item Regeln abhängig von diesem dynamischen Schwellwert ausgelöst
    \end{itemize}
    \vspace*{\stretch{1}}
\end{flashcard}

\subsubsectioncard{create and test alerts}

\begin{flashcard}[Definition]{Warnungstypen in Azure Monitor}
    \vspace*{\stretch{1}}
    \begin{itemize}
        \item Metrik-Warnungen:\newline
            basierend auf Schwellwerten von Metriken (CPU-Nutzung, \ldots)
        \item Activity Log-Warnungen:\newline
            basierend auf Zustandsänderungen von Ressourcen (löschen, \ldots)
        \item Log-Warnungen:\newline
            basierend auf Analyse von Log-Dateien (Fehlermeldungen, \ldots)
    \end{itemize}
    \vspace*{\stretch{1}}
\end{flashcard}

\begin{flashcard}[Definition]{Warnungsregel erstellen}
    \vspace*{\stretch{1}}
    Benötigte Einstellungen
    \begin{itemize}
        \item Ressource/Ziel-Ressource:\newline
            Die Ressource für die Regel (auch mehrere). Bestimmt auch den Signaltyp.
        \item Bedingung:\newline
            Typ des Signals (Metrik, Activity Log, Log) und angewendete Prüflogik
        \item Aktion:\newline
            Ausgeführte Tätigkeit (z.\,B. Email versenden) an eine Aktionsgruppe
        \item Details:\newline
            Textuelle Beschreibung und Schweregrad (0 bis 4)
    \end{itemize}
    \vspace{1cm}
    \vspace*{\stretch{1}}
\end{flashcard}

\begin{flashcard}[Definition]{Metrik-Warnung erstellen}
    \vspace*{\stretch{1}}
    \begin{itemize}
        \item CLI:\newline
            \texttt{az monitor metrics alert create \
            -name \ldots\
            --resource-group \ldots\\
            --scopes <vm>\\
            --condition "max percentage CPU > 80"\\
            --description \ldots\\
            --evaluation-frequency \ldots\ --window-size \ldots\\
            --severity <0-4>}
        \item Im Portal:\newline
            \texttt{Monitor | Warnungen | + Regel erstellen}
    \end{itemize}
    \vspace*{\stretch{1}}
\end{flashcard}

\subsubsectioncard{analyze alerts across subscription}

\begin{flashcard}[Definition]{Warnungszusammenfassung}
    \vspace*{\stretch{1}}
    \begin{itemize}
        \item \texttt{Montior | Warnungen}
        \item Liste der Warnungen z.\,B. nach Abonnements, Ressourcengruppen, Zeitbereich
        \item Zustände von Warnungen
            \begin{enumerate}
                \item Neu: initialer Zustand
                \item Bestätigt: tatsächlich ein Problem wurde festgestellt
                \item Geschlossen: wenn das Problem gelöst worden ist
            \end{enumerate}
    \end{itemize}
    \vspace*{\stretch{1}}
\end{flashcard}

\subsubsectioncard{analyze metrics across subscription}

\subsubsectioncard{create action group}

\begin{flashcard}[Definition]{Aktionsgruppe}
    \vspace*{\stretch{1}}
    \begin{itemize}
        \item Aktionen, die bei Auslösung einer Warnung ausgeführt werden
        \item mehrere Aktionen in einer Gruppe
        \item verfügbare Aktionen
            \begin{itemize}
                \item Nachricht senden (Email, SMS, Push, Sprachanruf)
                \item Azure-Funktion/Logik-App
                \item Webhook
                \item Ticket erstellen
                \item Runbook (zum Neustarten oder Skalieren von VMs)
            \end{itemize}
        \item Aktionsgruppen können mit mehreren Regeln verknüpft werden
    \end{itemize}
    \vspace*{\stretch{1}}
\end{flashcard}

\begin{flashcard}[Definition]{Aktionsgruppe erstellen}
    \vspace*{\stretch{1}}
    \begin{itemize}
        \item Im Portal:\newline
            \texttt{Metriken | + Neue Warnungsregel | AKTIONEN | Aktionsgruppe erstellen}
    \end{itemize}
    \vspace*{\stretch{1}}
\end{flashcard}

\subsubsectioncard{monitor for unused resources}

\begin{flashcard}[Definition]{Azure Advisor}
    \vspace*{\stretch{1}}
    integrierter Dienst in Azure mit Empfehlungen, z.\,B. zu Kosten
    \begin{itemize}
        \item nicht bereitgestellte ExpressRoute-Verbindungen
        \item erwerb von reservierten VM-Instanzen
        \item ungenutzte/überdimensionerte virtuelle Maschinen
    \end{itemize}
    Im Portal: \texttt{Verwaltung + Governance | Advisor | Kosten}
    oder \texttt{Kostenverwaltung + Abrechnung | Kostenverwaltung | Ratgeberempfehlungen }
    \vspace*{\stretch{1}}
\end{flashcard}

\subsubsectioncard{monitor spend}

\begin{flashcard}[Definition]{Azure Cost Management}
    \vspace*{\stretch{1}}

    \begin{itemize}
        \item Übersicht über Kosten
        \item Filtern und Gruppieren nach Service-Art, Resourcegruppe, Region, \ldots
        \item verschiedene Zeiträume, z.\,B. aktueller Monat (Rechnungszeitraum)
    \end{itemize}
    Im Portal: \texttt{Kostenverwaltung + Abrechnung | Kostenverwaltung | Kostenanalyse}
    \vspace*{\stretch{1}}
\end{flashcard}

\begin{flashcard}[Definition]{Budgetwarnungen}
    \vspace*{\stretch{1}}
    \begin{itemize}
        \item automatische Warnung sobald monatliches Budget erreicht ist
        \item für Kosten und Verbrauch
        \item Warnung im Portal bei Kostenwarnungen oder per Email
        \item auch Warnungen pro Abteilung
    \end{itemize}
    \vspace*{\stretch{1}}
\end{flashcard}

\subsubsectioncard{report on spend}

\begin{flashcard}[Definition]{Export von Kostenübersicht}
    \vspace*{\stretch{1}}
    \begin{itemize}
        \item Exporte nach: Abonnements, Resourcengruppen, Konten, Abteilungen, Registrierungen
        \item Zeitpläne: täglich aktueller Monat, wöchentlich, monatlich, benutzerdefiniert
        \item export in Speicherkonto
        \item Im Portal:\newline
            \texttt{Kostenanalyse | Einstellungen | Konfiguration | Exporte }
    \end{itemize}
    \vspace*{\stretch{1}}
\end{flashcard}

\begin{flashcard}[Definition]{Rechnung}
    \vspace*{\stretch{1}}
    Die (monatliche) Rechnung zeigt detaillierte Übersichten, welche Ressourcen welche Kosten verursachen.
    \vspace*{\stretch{1}}
\end{flashcard}

\subsubsectioncard{utilize Log Search query functions}

\subsubsectioncard{view Alerts in Azure Monitor logs}

\subsubsectioncard{visualize diagnostics data using Azure Monitor Workbooks}


    \subsectioncard{Erstellen und Konfigurieren von Speicherkonten}

\begin{flashcard}[Definition]{Datenklassifizierung}
    \vspace*{\stretch{1}}
    \begin{itemize}
        \item strukturiert\newline
        $\Rightarrow$ relationale Daten in Tabellen, Business-Daten, SQL
        \item teilweise strukturiert\newline
        $\Rightarrow$ getaggte Daten, unterschiedliche Klassifizierer, JSON
        \item unstrukturiert\newline
        $\Rightarrow$ Texte, Bilder, Videos, \ldots
    \end{itemize}
    \vspace*{\stretch{1}}
\end{flashcard}

\begin{flashcard}[Definition]{Anforderungen an Speicher ermitteln}
    \vspace*{\stretch{1}}
    \begin{itemize}
        \item Latenz
        \item Anzahl Operation pro Zeiteinheit
        \item Komplexität einzelner Anfragen
        \item notwendige Filtermöglichkeiten
    \end{itemize}
    \vspace*{\stretch{1}}
\end{flashcard}

\subsubsectioncard{configure network access to the storage account}

\begin{flashcard}[Definition]{Netzwerkzugriff auf Speicherkonten}
    \vspace*{\stretch{1}}
    \begin{itemize}
        \item Konnektivitätsmethode:
        \begin{itemize}
            \item öffentlich, alle Netzwerke
            \item öffentlich, ausgewählte Netzwerke\newline
            Nur ausgewählte virtuelle Netzwerke können auf den Speicher zugreifen
            \item privat
        \end{itemize}
        \item Netzwerkrouting
        \begin{itemize}
            \item Methode für Routing
            \item Microsoft-Routing
            \item Internet-Routing
        \end{itemize}

    \end{itemize}
    \vspace*{\stretch{1}}
\end{flashcard}

\subsubsectioncard{create and configure sotrage account}

\begin{flashcard}[Definition]{Azure Storage}
    \vspace*{\stretch{1}}
    Die Speicherdienste, die in einem Speicherkonto gespeichert werden
    \begin{itemize}
        \item Azure-Blob
        \item Azure-Files
        \item Azure-Warteschlangen
        \item Azure-Tabellen
    \end{itemize}
    Nicht dazu gehören weitere Speicherdienste
    \begin{itemize}
        \item Azure SQL
        \item Cosmos DB
    \end{itemize}
    \vspace*{\stretch{1}}
\end{flashcard}

\begin{flashcard}[Definition]{Speicherkonto}
    \vspace*{\stretch{1}}
    \begin{itemize}
        \item Azure Resourcen (in Resourcengruppe)
        \item kann mehrere Datendienste enthalten
        \item kombiniert Einstellungen, die für sämtliche enthaltenen Dienste gelten
    \end{itemize}
    \vspace*{\stretch{1}}
\end{flashcard}

\begin{flashcard}[Definition]{Speicherkonto-Typen}
    \vspace*{\stretch{1}}
    \begin{itemize}
        \item General Purpose v1\newline
            vorherige Version, unterstützt nicht alle Features
        \item General Purpose v2\newline
            sämtliche Features (inklusive Data Lake), empfohlen
        \item Blob\newline
            spezieller Speicher, der nur Blob unterstützt\newline
            bessere Leistung
    \end{itemize}
    v1 und v2 unterstützen Blob, Files, Queues, Tables
    \vspace*{\stretch{1}}
\end{flashcard}

\begin{flashcard}[Definition]{Erstellen eines Speicherkontos}
    \vspace*{\stretch{1}}
    \begin{itemize}
        \item Benötigt:
            \begin{itemize}
                \item Name. 3-24 Zeichen aus \texttt{[a-z0-9]}, weltweit eindeutig
                \item Bereitstellungsmodell: ARM (Azure Resource Manager), klassich\newline
                klassisch ist veraltet und wird schlecht unterstützt. Nicht verwenden.
                \item Kontoart: General Purpose v1/v2, Blob
            \end{itemize}
        \item Im Portal:
        \item CLI: \texttt{az storage account create --name name --resource-group grp\\ --sku Standard\_GRS --kind StorageV2}
    optional \texttt{--hierarchical-namespace true}
    \end{itemize}
    \vspace*{\stretch{1}}
\end{flashcard}

\begin{flashcard}[Definition]{Konfigurationsoptionen von Speicherkonten}
    \vspace*{\stretch{1}}
    \begin{itemize}
        \item Abonnement
        \item Standort
        \item Leistung: Standard, Premium
        \item Daten-Replikation: Datenkopien zum Schutz vor Hardwarefehlern und Katastrophen. LRS, ZRS, GRS, GZRS, RA-GZRS, ...
        \item Zugriffsebene: heiß, kalt, archiviert
        \item Sichere Übertragung: HTTPS zwangsweise
        \item IP-Filter: Viertuelle Netzwerke, oder IP-Adresse-Bereiche
        \item Hierarchischer Speicher
        \item vorläufiges Löschen (Blob, Files), Versionierung
    \end{itemize}
    \vspace*{\stretch{1}}
    $\Rightarrow$ jeder Unterschied erfordert eigenes Speicherkonto!
\end{flashcard}

\begin{flashcard}[Definition]{Eigenschaften von Azure File Storage}
    \vspace*{\stretch{1}}
    Standard:
    \begin{itemize}
        \item 100 TiB
        \item maximual 1000 IOPS
    \end{itemize}
    \vspace*{\stretch{1}}
\end{flashcard}

\subsubsectioncard{generate Shared Access Signature}

\begin{flashcard}[Definition]{SAS Token}
    \vspace*{\stretch{1}}
    Generierung:
    \begin{itemize}
        \item im Speicherkonto-Blade in Azure
        \item über Storage Explorer
        \item per command line\newline
            \texttt{az storage container generate-sas}
        \item als code (Signieren des Strings mit dem Zugriffsschlüssel)
    \end{itemize}
    \vspace*{\stretch{1}}
\end{flashcard}

\subsubsectioncard{implement Azure AD authentication for storage}

\begin{flashcard}[Definition]{Azure AD-Authorisierung für Speicher}
    \vspace*{\stretch{1}}
    \begin{itemize}
        \item empfohlener Zugriff (sicherer als SAS)
        \item für General Purpose Speicher und Blob für Blobs und Warteschlangen
        \item für Files: preview
        \item Table: not supported
    \end{itemize}
    \vspace*{\stretch{1}}
\end{flashcard}

\begin{flashcard}[Definition]{Zugriffskontrolle auf Speicher mit Azure AD}
    \vspace*{\stretch{1}}
    RBAC: verschiedene Rollen für Verwaltung und Datenzugriff
    \begin{itemize}
        \item Verwaltung:
            \begin{itemize}
                \item Besitzer, Mitwirkender, Speicherkontomitwirkender
            \end{itemize}
        \item Datenzugriff:
            \begin{itemize}
                \item Blobs: Mitwirkender an Blobdaten, Blobdatenleser
                \item Blobdatenbesitzer: Vollzugriff, POSIX-Zugriff auf Data Lake
                \item Blob-Delegierer: erstellt user SAS key
                \item Warteschlangen: Mitwirkender/Absender/Verarbeiter/Leser von Warteschlangendaten
            \end{itemize}
        \item Schlüssel:\newline
            Berechtigung \texttt{Microsoft.Storage/storageAccounts/listKeys/action} erlaubt Erstellung eines Shared Keys $\Rightarrow$ kann sich selbst Lesezugriff selbst verschaffen
    \end{itemize}
    \vspace*{\stretch{1}}
\end{flashcard}

\begin{flashcard}[Definition]{Umfang der Rechte}
    \vspace*{\stretch{1}}
    \begin{itemize}
        \item Container (geringste Berechtigung)
        \item Warteschlange (geringste Berechtigung)
        \item Speicherkonto
        \item Resourcengruppe
        \item Abonnement
        \item Managmentgruppe
    \end{itemize}
    \vspace*{\stretch{1}}
\end{flashcard}

\subsubsectioncard{install and use Azure Storage Explorer}

\begin{flashcard}[Definition]{Azure Storage Explorer}
    \vspace*{\stretch{1}}
    \begin{itemize}
        \item externes Tool für Windows, Linux, macOS, Azure Portal (als Preview)
        \item ermöglicht Ansicht/Bearbeitung von Speicherkonten, ähnlich Windows Explorer
        \item mehrere Speicherkonten und Abonnements können Verbunden werden
        \item unterstützt CosmosDB und Data Lake
        \item unterstützt lokale Speicher-Emulatoren
        \item hochladen, verschieben, runterladen
    \end{itemize}
    \vspace*{\stretch{1}}
\end{flashcard}

\begin{flashcard}[Definition]{Features Storage Explorer}
    \vspace*{\stretch{1}}
    \begin{itemize}
        \item Dateien/Blobs kopieren
        \item SAS-Generierung
        \item Snapshot-Erstellung
        \item Upload/Download Blobs oder Dateien
    \end{itemize}
    \vspace*{\stretch{1}}
\end{flashcard}


\begin{flashcard}[Definition]{Storage Explorer Verbinden}
    \vspace*{\stretch{1}}
    \begin{itemize}
        \item Azure-Account direkt verbinden\newline
        Zugriff auf Blobs, Files, Warteschlangen, Tabellen
        \item CosmosDB einbinden mit Verbindungszeichenfolge (Key aus Azure abrufen), Typ SQL oder Tabelle\newline
        Erstellen von Datenbanken, Sammlungen und Dokumenten. Editieren.
        \item Data Lake Gen2\newline
        Features im Storage Explorer wie Blobs.
    \end{itemize}
    \vspace*{\stretch{1}}
\end{flashcard}

\subsubsectioncard{manage access keys}

\begin{flashcard}[Definition]{Zugriffsschlüssel}
    \vspace*{\stretch{1}}
    \begin{itemize}
        \item müssen geheim gehalten werden
        \item entsprechen Benutzername und Kennwort\newline
            Speicherkontoname=Nutzer, Schlüssel=Kennwort
        \item Zwei Schlüssel ermöglichen Rotation
    \end{itemize}
    \vspace{1cm}
    Abruf von Zugriffsschlüsseln
    \begin{itemize}
        \item über Azure Portal (im Blade für das Speicherkonto)
        \item über Azure CLI\newline
            \texttt{az storage account key list}
    \end{itemize}
    \vspace*{\stretch{1}}
\end{flashcard}

\begin{flashcard}[Definition]{Zugriff auf ein Speicherkonto}
    \vspace*{\stretch{1}}
    Mit SAS-Token
    \begin{itemize}
        \item verschlüsselt mit Zugriffsschlüssel
        \item legen Zugriffsrechte fest:
        \begin{itemize}
            \item API-Endpunkt (Blob, File, Tabelle, Warteschlange)
            \item Rechte (Lesen, Schreiben, Auflisten, ...)
        \end{itemize}
    \end{itemize}

    Direkt mit Zugriffsschlüssel
    \begin{itemize}
        \item ermöglicht Vollzugriff
        \item Verbindungszeichenfolge:\newline
            \texttt{DefaultEndpointsProtocol=https;AccountName={\ldots};\\AccountKey={\ldots};EndpointSuffix=core.windows.net}
    \end{itemize}
    \vspace*{\stretch{1}}
\end{flashcard}

\begin{flashcard}[Definition]{Zugriffsschlüssel}
    \vspace*{\stretch{1}}
    \begin{itemize}
        \item zwei Schlüssel: primär und sekundäre
        \item ermöglicht key rotation
            \begin{enumerate}
            \item Primary -> Secondary
            \item Recreate Primary
            \item Secondary -> Primary
            \item Recreate Secondary
            \end{enumerate}
        \item Zugriffsschlüssel müssen geheim bleiben\newline
            im Unterschied zu SAS-Token
        \item Verwaltung in CLI:\newline
            \texttt{az storage account keys ...}
    \end{itemize}
    \vspace*{\stretch{1}}
\end{flashcard}

\begin{flashcard}[Definition]{Azure AD-Zugriff auf Speicherkonto}
    \vspace*{\stretch{1}}
    \begin{itemize}
        \item jeder Zugriff auf Speicherkonto ist über Azure AD authentifiziert
        \item Ausnahme: es kann öffentlicher Lesezugriff eingestellt werden (im Account)
        \item RBAC-Rollen: Betragender zum Speicherkonto
        \item Rollenverwaltung außerdem auch für Verwaltung, nicht nur Datenzugriff
        \item Rollen für:
            \begin{itemize}
                \item ganzes Speicherkonto
                \item einzelne Container
                \item Warteschlangen
            \end{itemize}
    \end{itemize}
    \vspace*{\stretch{1}}
\end{flashcard}

\subsubsectioncard{monitor Activity log by using Azure Monitor logs}

\begin{flashcard}[Definition]{Logging von Speicherkonten}
    \vspace*{\stretch{1}}
    \begin{itemize}
        \item aktiviere Logs als "Diagnoseeinstellungen 2.0": alle Aktivitäten
        \item Azure CLI + PowerShell
        \item gespeichert als \texttt{\$logs}-Container im Blob-Speicher
    \end{itemize}
    \vspace*{\stretch{1}}
\end{flashcard}

\begin{flashcard}[Definition]{Aktivitätsprotokoll-Warnungen}
    \vspace*{\stretch{1}}
    \begin{itemize}
        \item Arten:
            \begin{itemize}
                \item Vorgänge: werden bei Änderungen von Ressourcen ausgelöst
                \item Service-Health
            \end{itemize}
        \item Attribute:
            \begin{itemize}
                \item Kategorie: Verwaltung, Dienstintegrität, Autoskalierung, Richtlinie, Empfehlung
                \item Bereich: Abbonement/Ressourcengruppe/Ressource
                \item Ressourcengruppe: Speicherort der Regel
                \item Ressourcentyp:
                \item Vorgangsname
                \item Grad: 0 (Ausführlich) - 4 (Kritisch)
                \item Status: Gestartet, Fehlgeschlagen, Erfolgreich
                \item Initiiert von:
            \end{itemize}
    \end{itemize}
    \vspace*{\stretch{1}}
\end{flashcard}

\begin{flashcard}[Definition]{Aktivitätsprotokollregel erstellen}
    \vspace*{\stretch{1}}
    \begin{itemize}
        \item Signaltyp \texttt{Aktivitätsprotokoll}
        \item aus Liste der Warnungen auswählen
    \end{itemize}
    \vspace*{\stretch{1}}
\end{flashcard}

\subsubsectioncard{implement Azure storage replication}

\begin{flashcard}[Definition]{Speicher-Redundanz in Azure Speicherkonten}
    \vspace*{\stretch{1}}
    \begin{itemize}
        \item jeder Speicher in Azure ist immer redundant
        \item verschieden Stufen möglich
        \begin{itemize}
            \item LRS: Lokal redundanter Speicher: 3 Kopien in einem Datencenter\newline
                Schutz vor Hardwarefehlern (Rack + Disk)
            \item ZRS: Zonenredundanter Speicher: 3 Kopien Verfügbarkeitszonen\newline
                Schutz vor komplettem Datencenter-Ausfall
            \item GRS: Geo-redundanter Speicher: 6 Kopien in zwei Regionen (jeweils LRS)\newline
                Schutz vor Ausfall einer kompletten Region
            \item RA-GRS: Geo-redundanter Speicher mit Lesemöglichkeit in der zweiten Region\newline
            Angehängtes \texttt{-secondary} an den Namen in der URL
        \end{itemize}
    \end{itemize}
    \vspace*{\stretch{1}}
\end{flashcard}

\begin{flashcard}[Definition]{Speicher-Redundanz herstellen}
    \vspace*{\stretch{1}}
    \begin{itemize}
        \item bereits Angabe bei der Erstellung (Portal, CLI, \ldots)\newline
            Parameter \texttt{--sku}, z.\,B. \texttt{Standard\_RAGRS}
        \item Wechsel von LRS zu ZRS
    \end{itemize}
    \vspace*{\stretch{1}}
\end{flashcard}

\subsubsectioncard{implement Azure storage account failover}

\begin{flashcard}[Definition]{Speicherkonto-Failover}
    \vspace*{\stretch{1}}
    \begin{itemize}
        \item feste sekundäre Region für eine Region
        \item Daten werden (asynchron) in der zweiten Region gesichert
        \item delay wenige Minuten, aber ohne SLA
        \item falls primäre Region ausfällt, muss Failover durchgeführt werden
        \begin{itemize}
            \item Start über Portal oder CLI
            \item DNS-Einträge werden aktualisiert
            \item nach Failover ist das Speicherkonto LRS
            \item sobald Problem behoben ist, muss erneut auf (RA-)GRS gesetzt werden
        \end{itemize}
        \item Failover kann zu Datenverlust führen! (letztlich konsistent)
    \end{itemize}
    Statusabfrage: \texttt{az storage account show --name \ldots\\--query "[statusOfPrimary, statusOfSecondary]"}

    Durchführung: \texttt{az storage account failover --name \ldots}
    \vspace*{\stretch{1}}
\end{flashcard}

\begin{flashcard}[Definition]{Lesen von Geo-redundanten Daten}
    \vspace*{\stretch{1}}
    \begin{itemize}
        \item Leseanforderung bei mehreren Regionen mit Priorisierung:
        \begin{itemize}
            \item \texttt{PrimaryOnly} (standard)
            \item \texttt{PrimaryThenSecondary}
            \item \texttt{SecondaryOnly}
            \item \texttt{SecondaryThenPrimary}
        \end{itemize}
        \item Sicherungsmuster sollte von \texttt{PrimaryThenSecondary} auf \texttt{SecondaryOnly} wechseln, falls Probleme erkannt werden.
    \end{itemize}
    \vspace*{\stretch{1}}
\end{flashcard}


    \subsectioncard{Erstellen und Konfigurieren von VMs für Windows und Linux}

\begin{flashcard}[Definition]{Hybrid Benefit}
    \vspace*{\stretch{1}}
    \begin{itemize}
        \item eigene Lizenzen (Windows, SQL)
        \item sparen 49\%
    \end{itemize}
    \vspace*{\stretch{1}}
\end{flashcard}

\subsubsectioncard{configure High Availability}

\begin{flashcard}[Definition]{Fehler- und Updatedomänen}
    \vspace*{\stretch{1}}
    \begin{itemize}
        \item Update-Domänen\newline
            Sichern zu, dass eine Verfügbarkeitsgruppe während Updates verfügbar bleibt
        \item Fehler-Domänen\newline
            Sichert zu, dass VMs einer Verfügbarkeitsgruppe bei Ausfall einer Zone verfügbar bleiben
        \item Im Portal:\newline
            \texttt{Eigenschaften | Hochverfügbarkeit}
    \end{itemize}
    \vspace*{\stretch{1}}
\end{flashcard}

\begin{flashcard}[Definition]{Verfügbarkeit von Virtuellen Maschinen}
    \vspace*{\stretch{1}}
    \begin{tabular}{l|l}
        Setup       & Verfügbarkeit \\
        \hline
        VM in der \emph{alle} Datenträger verwaltete \texttt{Standard\_HDD} sind & 95\% \\
        VM in der \emph{alle} Datenträger verwaltete \texttt{Standard\_SSD} sind & 99,5\% \\
        VM in der \emph{alle} Datenträger \texttt{Premium\_SSD} oder \texttt{Ultra}
        sind & 99,9\% \\
        $\geq$ 2 VMs in einer \emph{Verfügbarkeitsgruppe} & 99,95\% für $\geq$ 1 VM  \\
        $\geq$ 2 VMs in \emph{verschiedenen} Verfügbarkeitszonen & 99,99\% für $\geq$ 1 VM
    \end{tabular}
    \vspace*{\stretch{1}}
\end{flashcard}

\subsubsectioncard{configure Monitoring}

\begin{flashcard}[Definition]{Ressourcne-Auslastung einer VM überwachen}
    \vspace*{\stretch{1}}
    \begin{itemize}
        \item Azure Montior-Metriken
        \item Azure Monitoring Insights
    \end{itemize}
    \vspace*{\stretch{1}}
\end{flashcard}

\subsubsectioncard{configure Networking}

\subsubsectioncard{configure Storage}

\begin{flashcard}[Definition]{Disks for VMs}
    \vspace*{\stretch{1}}
    \begin{itemize}
        \item Benötigt:
            \begin{itemize}
                \item Resourcegruppe
                \item Location
                \item Name
                \item Betriebssystem-Image
                \item Admin-Username
                \item ggf. ssh keys
            \end{itemize}
        \item Portal: \texttt{Neue Ressource | Virtuelle Maschine}
        \item CLI: \texttt{az vm create}
        \item PowerShell: \texttt{New-AzVm}
    \end{itemize}
    NIC + Public IP + OS Disk werden automatisch erstellt
    \vspace*{\stretch{1}}
\end{flashcard}

\begin{flashcard}[Definition]{Disks for VMs}
    \vspace*{\stretch{1}}
    \begin{itemize}
        \item Betriebssystemdatenträger
            \begin{itemize}
                \item Ein pro VM, basierend auf Image
                \item maximal 2 TiB
            \end{itemize}

        \item Daten-Disks
            \begin{itemize}
                \item Ein oder mehrere pro VM
                \item Anzahl abhängig von VM Sku
                \item bis zu 32 TiB Größe
            \end{itemize}

        \item Temporärer Speicher\newline
            \begin{itemize}
                \item Ein pro VM
                \item geht bei Wartung verloren. Nicht auf einem Speicherkonto
            \end{itemize}
    \end{itemize}
    \vspace*{\stretch{1}}
\end{flashcard}

\begin{flashcard}[Definition]{Datenträger-Typen}
    \vspace*{\stretch{1}}
    Kurzlebig
    \begin{itemize}
        \item Lokal im Host
        \item sehr Schnell, auch bei der Initialisierung
        \item bei Ausfall sind Daten verloren
        \item gut für zustandslose Workloads (z.\,B. Microservice)
    \end{itemize}
    Verwaltet
    \begin{itemize}
        \item Skalierbar, hochverfügbar (Verfügbarkeitszonen),
        \item sind zwar Seitenblobs im Speicherkonto, aber im Hintergrund
        \item Sicherheit: RBAC, Verschlüsselung
    \end{itemize}
    Nicht-Verwaltet
    \begin{itemize}
        \item Nutzer müssen sich selbst um Speicherkonto kümmern
        \item Sicherheit pro Speicherkonto, nicht pro Disk
    \end{itemize}
    \vspace*{\stretch{1}}
\end{flashcard}

\begin{flashcard}[Definition]{Datenträger für VMs}
    \vspace*{\stretch{1}}
    \begin{itemize}
        \item beim Erstellen im Azure-Portal
        \item Bereich ``Datenträger''
        \item Auswahl, ob Datenträger verwaltet oder nicht verwaltet
    \end{itemize}
    \vspace*{\stretch{1}}
\end{flashcard}

\subsubsectioncard{configure Virtual Machine Size}

\begin{flashcard}[Definition]{VM-Größen}
    \vspace*{\stretch{1}}
    \begin{itemize}
        \item Viele SKUs:
            \begin{itemize}
                \item Allgemein
                \item Balanciert: Dsv, Dv3, DS, D
                \item Arbeitsspeicher: E, M, G, D, DS, Dv2
                \item Compute: F, Fs
                \item Speicher: L
                \item GPU: N,
                \item Leistung: H, A
            \end{itemize}
        \item Bestimmen Parameter wie Disk-Größe, anzahl NICs, Disks, \ldots
        \item \texttt{az vm list-sizes}
    \end{itemize}
    \vspace*{\stretch{1}}
\end{flashcard}

\begin{flashcard}[Definition]{VM-Größe ändern}
    \vspace*{\stretch{1}}
    \begin{itemize}
        \item Liste der verfügbaren:\newline
            \texttt{az vm list-vm-resize-options --resource-group \ldots --name \ldots}
        \item Nur im aktuellen Cluster verfügbare Größen können gewählt werden
        \item Portal: \texttt{Einstellungen | Größe | Größe ändern} im VM-Blade\newline
            (Verfügbare Größen werden angezeigt)
    \end{itemize}
    \vspace*{\stretch{1}}
\end{flashcard}

\subsubsectioncard{implement dedicated hosts}

\subsubsectioncard{deploy and configure scale sets}

\begin{flashcard}[Definition]{Skalierungsgruppe}
    \vspace*{\stretch{1}}
    \begin{itemize}
        \item Skalierung von Web-Services, Container-Workloads, Big-Data, \ldots
        \item horizontale Skalierung: dynamische Anzahl VMs
        \item vertikale Skalierung: dynamische VM-Größe
        \item alle VMs haben die gleiche Konfiguration
        \item maximal 1000 VMs
        \item bei Bedarf Lastenausgleich (z.\,B. über Ping)
        \item Skalierungsgruppe mit niedriger Priorität: günstiger
    \end{itemize}
    \vspace*{\stretch{1}}
\end{flashcard}

\begin{flashcard}[Definition]{Durchführung einer Skalierung}
    \vspace*{\stretch{1}}
    Geplante Skalierung
    \begin{itemize}
        \item geplante Bereitstellung von zusätzichen Maschinen
    \end{itemize}
    \vspace{1cm}
    Automatische Skalierung
    \begin{itemize}
        \item metrikbasierte Bereitstellung von Maschinen
        \item Schwellwertskalierung
    \end{itemize}

    \vspace*{\stretch{1}}
\end{flashcard}

\begin{flashcard}[Definition]{Bereitstellung einer Skalierungsgruppe}
    \vspace*{\stretch{1}}
    \begin{itemize}
        \item Benötigt:
            \begin{itemize}
                \item Abonnement und Resourcengruppe
                \item Name
                \item VM-Parameter, z.\,B. VM-Image, User, SSH, \ldots
                \item
            \end{itemize}
        \item CLI: \texttt{az vmss create}
    \end{itemize}
    zwei Instanzen und Lastenausgleich sind automatisch erstellt
    \vspace*{\stretch{1}}
\end{flashcard}

\begin{flashcard}[Definition]{Lastenausgleich mit Skalierungsgruppe}
    \vspace*{\stretch{1}}
    Integritätstest:
    \begin{itemize}
        \item Benötigt:
        \begin{itemize}
            \item Abonnement und Resourcengruppe
            \item Name
            \item Test-Spezifikation: Port, Protokoll, Pfad, \ldots
        \end{itemize}
        \item CLI: \texttt{az network lb probe create}
    \end{itemize}
    Lastenausgleich:
    \begin{itemize}
        \item Benötigt:
        \begin{itemize}
            \item Abonnement und Resourcengruppe
            \item Name
            \item Integritätstest-Referenz
            \item Scale-Set als Backend-Pool
            \item Front- und Backend-Spezifikation: Port, Potokoll, \ldots
        \end{itemize}
        \item CLI: \texttt{az network lb rule create}
    \end{itemize}
    \vspace*{\stretch{1}}
\end{flashcard}

\begin{flashcard}[Definition]{Manuelle Skalierung}
    \vspace*{\stretch{1}}
    \begin{itemize}
        \item Portal: \texttt{Einstellungen | Wird Skaliert | Manuelle Skalierung} im Skalierungsgruppen-Blade
        \item CLI: \texttt{az vmss scale --new-capacity \ldots}
    \end{itemize}
    \vspace*{\stretch{1}}
\end{flashcard}

\begin{flashcard}[Definition]{Automatische Skalierung}
    \vspace*{\stretch{1}}
    \begin{itemize}
        \item Zeitplan
            \begin{itemize}
                \item Portal: \texttt{Einstellungen | Wird Scaliert | Benutzerdefinierte Autoskalierung} im Skalierungsgruppen-Blade
                \item \texttt{"Auf eine bestimmte Anzahl von Instanzen skalieren"}
                \item mit Zeitplan
            \end{itemize}
        \item Metriken
            \item Portal: \texttt{Einstellungen | Wird Scaliert | Benutzerdefinierte Autoskalierung} im Skalierungsgruppen-Blade
            \item \texttt{"Basierend auf einer Metrik"}
            \item \texttt{"Regel hinufügen"}
            \begin{itemize}
                \item CPU-Auslastung
                \item Daten(ein|aus)fluss
                \item Datenträgervorgänge
                \item Warteschlangentiefe für Datenträger
            \end{itemize}
        \item Aggregierung von für \emph{alle} Instanzen
            \begin{itemize}
                \item Durchschnitt
                \item Minimum/Maximum
                \item Summe
                \item Letzte
                \item Anzahl
            \end{itemize}
        \item Aggregierungsdauer: mindestens 5 Minuten
        \item Operator: \emph{kleiner}, \emph{größer}, \ldots
        \item Abkühldauer: mindextens 5 Minuten. Währenddessen wird Regel nicht nochmal angewandt
        \item Regeln
            \begin{itemize}
                \item Erhöhung der Instanzzahl bei x\%
                \item Verringerung der Instanzzahl bey y\%
                \item für ein Regelpaar Schwellwerte \emph{nicht} auf einen Wert festlegen
            \end{itemize}

    \end{itemize}
    \vspace*{\stretch{1}}
\end{flashcard}

\begin{flashcard}[Definition]{Autoscale-Profil einrichten mit CLI}
    \vspace*{\stretch{1}}
    \begin{itemize}
        \item Benötigt Resourcegruppe und Skalierungsgruppe
        \item Standardwert, minimum, maximum von VMs
        \item \texttt{az monitor autoscale create\\--resource-group \ldots\\--resource <scale-set>\\--resource-type Microsoft.Compute/virtualMachineScaleSets\\--name autoscale --min-count \ldots --max-count \ldots --count \ldots}
    \end{itemize}
    \vspace*{\stretch{1}}
\end{flashcard}

\begin{flashcard}[Definition]{Autoscale-Regel einrichten mit CLI}
    \vspace*{\stretch{1}}
    \begin{itemize}
        \item Benötigt Resourcegruppe und Autoscale-Profil
        \item \texttt{az monitor autoscale rule create\\--resource-group \ldots\\--autoscale-name \ldots\\--condition <condition-string>\\--scale <out|in> \ldots}
    \end{itemize}
    \vspace*{\stretch{1}}
\end{flashcard}

\begin{flashcard}[Definition]{Aktualisierungen von VMs}
    \vspace*{\stretch{1}}
    Mit benutzerdefinierter Skripterweiterung
    \begin{itemize}
        \item genau so, wie bei VMs
        \item \texttt{az vmss extension set \ldots}
        \item nochmalige Anwendung zum Aktualisieren
        \item Upgraderichtlinie
            \begin{itemize}
                \item Automatisch: könnte für alle VMs gleichzeitig sein
                \item Parallel: vermeidet Dienstaufall
                \item Manuell: standard
            \end{itemize}
            Festlegen mit \texttt{upgrade-policy-mode}-Parameter bei Erstellung der Skalierungsgruppe
    \end{itemize}
    \vspace*{\stretch{1}}
\end{flashcard}
   


    \subsectioncard{Die Bereitstellung von VMs automatisieren}

\begin{flashcard}[Definition]{Deployment-Plattformen für Source Code}
    \vspace*{\stretch{1}}
    \begin{itemize}
        \item BitBucket
        \item GitHub
        \item Azure Repo
        \item Team Foundation Version Control
    \end{itemize}
    \vspace*{\stretch{1}}
\end{flashcard}

\subsubsectioncard{modify Azure Resource Manager template}

\begin{flashcard}[Definition]{Resource-Manager-Vorlage}
    \vspace*{\stretch{1}}
    \begin{itemize}
        \item JSON-Datei
        \item Vereinheitlichen VMs und Konfiguration (über Erweiterungen)
        \item können mit Variablen und Paramtern konfiguriert werden\newline
            $\Rightarrow$ benötigt Parameter-Datei
    \end{itemize}
    \vspace*{\stretch{1}}
\end{flashcard}

\begin{flashcard}[Definition]{Komponenten einer Resource-Manager-Vorlage}
    \vspace*{\stretch{1}}
    \begin{itemize}
        \item Parameter: vom Nutzer anzugebende Werte, ggf. mit Defaults\newline
            \texttt{"parameters"}
        \item Variablen: Werte, die mehrmals verwendet werden können\newline
            \texttt{"variables"}
        \item Funktionen: Mehrfach genutzte Funktionen (mit Parametern)\newline
            \texttt{"functions"}
        \item Ressourcen: Azure-Ressourcen, die bereitgestellt werden\newline
            \texttt{"resources"}
        \item Ausgaben: Generierte Informationen der Bereitstellung, z.\,B. IP-Adresse\newline
            \texttt{"outputs"}
    \end{itemize}
    \vspace*{\stretch{1}}
\end{flashcard}

\subsubsectioncard{configure Location of new VMs}

\subsubsectioncard{configure VHD template}

\begin{flashcard}[Definition]{VHD-Vorlagen}
    \vspace*{\stretch{1}}
    \begin{itemize}
        \item Virtuelle Festplatte, die in Azure gespeichert wird: hochverfügbar, managed, \ldots
        \item VHD Image: Festplatte, die ein Betreibssystem für VMs enthält
            \begin{itemize}
                \item Ubuntu und andere Linux-Varianten
                \item Windows Server, SQL
                \item zahllose Varianten
            \end{itemize}
        \item generalisiertes Image:\newline
            modifiziertes Image, ohne Nutzerinformationen
        \item spezialisiertes Image:\newline
            Kopie einer Live-VM, z.\,B. als Backup
    \end{itemize}
    \vspace*{\stretch{1}}
\end{flashcard}

\begin{flashcard}[Definition]{Unterstützte Formate}
    \vspace*{\stretch{1}}
    \begin{itemize}
        \item[!] nur VHD kann hochgeladen werden
        \item VHDX muss konvertiert werden
        \item Größe muss fest sein (kann konvertiert werden)
        \item Version 1 und 2 VMs
        \item maximale größe 2 T(i?)B für Generation 2
    \end{itemize}
    \vspace*{\stretch{1}}
\end{flashcard}

\begin{flashcard}[Definition]{Custom Image erstellen}
    \vspace*{\stretch{1}}
    \begin{enumerate}
        \item Zwei Möglichkeiten:
            \begin{itemize}
                \item VM starten mit existierendem Basisimage
                \item mit Hyper-V von Grund auf neu erstellen
            \end{itemize}
        \item In das System einloggen, ggf. konfigurieren
        \item Generalisieren im System vorbereiten
            \begin{itemize}
                \item Linux: waagent
                \item Windows: Sysprep
            \end{itemize}
        \item Generalisieren: \newline
                \texttt{az vm deallocate}
                \texttt{az vm generalize}
        \item Image erstellen:
            \begin{itemize}
                \item Portal: \texttt{Capture} im VM-Blade der generalisierten VM
                \item CLI: \texttt{az vm image create --source vm-name}
            \end{itemize}

    \end{enumerate}
    \vspace*{\stretch{1}}
\end{flashcard}

\subsubsectioncard{deploy from template}

\begin{flashcard}[Definition]{Ressourcen von einer Eine Resource-Manager-Vorlage bereitstellen}
    \vspace*{\stretch{1}}
    \begin{itemize}
        \item Benötigt:
            \begin{itemize}
                \item Resourcegruppe
                \item Vorlagen-Datei
                \item Parameter: \texttt{--parameters name=value name2=aValue}
            \end{itemize}
        \item \emph{Vor} dem Bereitstellen validieren:\newline
            CLI: \texttt{az deployment group validate}, benötigt auch Parameter
        \item Portal: \texttt{Neue Ressource} + Suchen nach \texttt{Template} und eigenes Template erstellen.
        \item CLI: \texttt{az deployment group create}
        \item Während der Bereitstellung: kann im Portal als Bereitstellung verfolgt werden
    \end{itemize}
    \vspace*{\stretch{1}}
\end{flashcard}

\begin{flashcard}[Definition]{Bereitstellungs-Modi}
    \vspace*{\stretch{1}}
    \begin{itemize}
        \item Komplett\newline
            Ressourcengruppe wird erstetzt mit der im Template konfigurierten Infrastruktur
        \item Inkrementell\newline
            Existierende Infrastruktur bleibt unverändert
    \end{itemize}
    \vspace*{\stretch{1}}
\end{flashcard}

\subsubsectioncard{save a deployment as an Azure Resource Manager template}

\begin{flashcard}[Definition]{Herunterladen einer ARM-Vorlage}
    \vspace*{\stretch{1}}
    \begin{itemize}
        \item Aus dem Portal:\newline
            \texttt{Vorlage Exportieren}
        \item Zip-Datei
        \item Enthält:
            \begin{itemize}
                \item template.json
                \item parameters.json
            \end{itemize}
    \end{itemize}
    \vspace*{\stretch{1}}
\end{flashcard}

\subsubsectioncard{deploy Windows and Linux VMs}

\begin{flashcard}[Definition]{VM mit Resource Manager Template definieren}
    \vspace*{\stretch{1}}
    Resource Manager Template:
    \begin{itemize}
        \item Resource vom Typ \texttt{Microsoft.Compute/virtualMachines}
        \item Eigenschaften:
            \begin{itemize}
                \item \texttt{hardwareProfile}: VM SKU
                \item \texttt{osProfile}: Username, Password, VM-Name
                \item \texttt{storageProfile}: OS disk, Daten-Disk
                \item \texttt{networkProfile}: die Netzwerkinterfaces
            \end{itemize}

    \end{itemize}
    \vspace*{\stretch{1}}
\end{flashcard}

\begin{flashcard}[Definition]{Spezifikation des Betriebssystems im ARM-Template}
    \vspace*{\stretch{1}}
    In den Eigenschaften für \texttt{storageProfile}:
    \begin{enumerate}
        \item \texttt{imageReference}
            \begin{itemize}
                \item \texttt{publisher} (Microsoft, oder RedHat, \ldots)
                \item \texttt{offer} (Betriebssystem, z.\,B. CentOS, WindowsServer, \ldots)
                \item \texttt{sku} (Version, z.\,B. 2016, oder 8.0)
                \item \texttt{version} (Bugfix-Release)
            \end{itemize}
        \item \texttt{osDisk}:\newline
            Setze auf \texttt{FromImage} um das unter 1 spezifizierte Image zu wählen
        \item \texttt{dataDisks}\newline
            Leer:
            \begin{itemize}
             \item \texttt{diskSizeGB}
             \item \texttt{lun}: fortlaufende Nummer
             \item \texttt{createOption}: \texttt{Empty}. (Für verwaltete Datenträger \texttt{Copy} oder \texttt{Attach})
            \end{itemize}
    \end{enumerate}
    \vspace*{\stretch{1}}
\end{flashcard}

\begin{flashcard}[Definition]{Windows als Betriebssystem}
    \vspace*{\stretch{1}}
    \begin{itemize}
        \item Publisher: MicrosoftWindowsServer
        \item Offer: WindowsServer
        \item SKU: 2016-Datacenter(-\ldots), 2019-Datacenter, 2012-Datacenter, \ldots
        \item Version: latest
    \end{itemize}
    \vspace*{\stretch{1}}
\end{flashcard}

\begin{flashcard}[Definition]{Linux als Betriebssystem}
    \vspace*{\stretch{1}}
    \begin{itemize}
        \item Publisher: OpenLogic, Debian, RedHat, SUSE, Canoncial
        \item Offer: CentOS, debian-10, RHEL, SLES, UbuntuServer
        \item SKU: 7.5, 10, 7-LVM, 15, 18.04-LTS
        \item version: latest
    \end{itemize}
    \vspace*{\stretch{1}}
\end{flashcard}

\begin{flashcard}[Definition]{Verschieben in andere Ressourcen}
    \vspace*{\stretch{1}}
    \begin{itemize}
        \item Portal:\newline
            Ressoruce auswählen, \texttt{Übersicht | Ressourcengruppe (Ändern)}, Zielgruppe auswählen
        \item CLI:\newline
            \texttt{az resource move --destination-group \ldots --ids \ldots}
        \item PowerShell:\newline
            \texttt{Move-AzResource\ -DestinationResourceGroupName "\ldots"\ -ResourceId <id>}
    \end{itemize}

    \vspace*{\stretch{1}}
\end{flashcard}


    \subsectioncard{Verbindungen zwischen virtuellen Netzwerken}

\subsubsectioncard{create and configure Vnet peering}

\begin{flashcard}[Definition]{VNet-Peering}
    \vspace*{\stretch{1}}
    \begin{itemize}
        \item peering virtueller Netzwerke: VNETs in der gleichen Region
        \item peering globaler virtueller Netzwerke: VNETs in unterschiedlichen Regionen
        \item Peering ist \emph{nicht} transitiv
        \item überlappende Adressräume sind nicht möglich
        \item Limit: 500 peerings pro Netzwerk
    \end{itemize}
    \vspace*{\stretch{1}}
\end{flashcard}

\begin{flashcard}[Definition]{VNET Peering-Anwendungsfälle}
    \vspace*{\stretch{1}}
    \begin{itemize}
        \item einfachste Möglichkeit, Azure-Netzwerke zu verbinden\newline
            einfacher, als z.\,B. Verbindung mit lokalen Netzwerken
        \item über Peering und Gateways verbundene Netze \emph{bevorzugen} die Peering-Verbindung
        \item Geräte in VNETs als wenn sie im gleichen Netz wären
        \item Verbindung in Azure: schnell, nicht von Verbindung, Gateways, \ldots abhängig
    \end{itemize}
    \vspace*{\stretch{1}}
\end{flashcard}

\begin{flashcard}[Definition]{VNET Peering herstellen}
    \vspace*{\stretch{1}}
    Voraussetzungen:
    \begin{itemize}
        \item wenn in verschiedenen Abonnements, Rolle \emph{Netzwerkmitwirkender} notwendig
        \item beide Netzwerke müssen geegenseitig verbunden werden\newline
            sonst: Status ist \emph{Initiiert}
    \end{itemize}

    \begin{itemize}
        \item CLI: \texttt{az network vnet peering create}\newline
            CLI erfordert zwei Peerings zu erstellen
        \item Azure Portal: Im VNet-Blade \texttt{Einstellungen | Peerings | Hinzufügen}
        \begin{itemize}
            \item Name
            \item Resourcen-ID oder Abonnement + VNET
            \item VNET-Zugriff zulassen
        \end{itemize}
        Im Portal wird die Verbindung automatisch reziprok erstellt!
    \end{itemize}

    \vspace*{\stretch{1}}
\end{flashcard}

\begin{flashcard}[Definition]{VNET-Peering konfigurieren}
    \vspace*{\stretch{1}}
    Gatewaytransit
    \begin{itemize}
        \item erlaubt weiterleitung von Daten (also ähnlich wie Transitivität)
        \item Netzwerk mit Gateway: \emph{Gatewaytransit zulassen}\newline
            Transit-Verkehr darf das Gateway nutzen.
        \item anderes Netzwerk: \emph{Remotegateways benutzen}\newline
            Verkehr darf das Remote-Gateway benutzen
    \end{itemize}
    \vspace*{\stretch{1}}
\end{flashcard}

\subsubsectioncard{create and configure Vnet to Vnet connections}

\begin{flashcard}[Definition]{Transit über virtuelle Netzwerke}
    \vspace*{\stretch{1}}
    Eigenschaften einer Peer-Verbindung:
    \begin{itemize}
        \item weitergeleiteten Verkehr zulassen
        \item[$\Rightarrow$] erlaubt Verkehr aus anderen Netzwerken als dem Quell-Netzwerk weiterzuleiten
        \item ermöglicht Reihenschaltung von Netzwerken
    \end{itemize}
    \vspace*{\stretch{1}}
\end{flashcard}

\begin{flashcard}[Definition]{Globale VNet-Peerings}
    \vspace*{\stretch{1}}
    \begin{itemize}
        \item Resourcen in virtuellem netz können \emph{nicht} mit der Front-end-IP eines internen Load Balancers im Basic SKU kommunizieren
        \item Load Balancer kann generell eingeschränkt sein bei globalem Peering
    \end{itemize}
    \vspace*{\stretch{1}}
\end{flashcard}

\subsubsectioncard{verify virtual network connectivity}

\begin{flashcard}[Definition]{Verifizieren der Verbindung}
    \vspace*{\stretch{1}}
    \begin{itemize}
        \item CLI Liste der Peerings:\newline
            \texttt{az network vnet peering list --resource-group \ldots --vnet-name \ldots}
        \item Routen:\newline
            \texttt{az network nic show-effective-route-table --resource-group \ldots --name \ldots}
        \item Routen im Portal:\newline
            \texttt{VM | Netzwerkschnittstelle | Support + Problembehandlung | Effektive Routen}
        \item Ausgabe:
            \texttt{Default   Active   10.2.0.0/16       VNetPeering} falls verbunden
    \end{itemize}
    \vspace*{\stretch{1}}
\end{flashcard}

\subsubsectioncard{create virtual network gateway}


    \subsectioncard{Einrichten und Verwalten von virtuellen Netzwerken}

\subsubsectioncard{configure private IP addressing}

\begin{flashcard}[Definition]{Private IP-Adressen}
    \vspace*{\stretch{1}}
    \begin{itemize}
        \item Dynamisch: DHCP-Lease basierend, können sich ändern
        \item Statisch: statisch während der Lebensdauer einer Resource
        \item Restriktionen: die ersten drei und die letztze IP-Adressen in jedem \emph{Subnetz}.
    \end{itemize}
    \vspace*{\stretch{1}}
\end{flashcard}

\subsubsectioncard{configure public IP addresses}

\begin{flashcard}[Definition]{Öffentliche IP-Adressen}
    \vspace*{\stretch{1}}
    \begin{itemize}
        \item Basic SKU:
            \begin{itemize}
                \item offen
                \item Netzwerksicherheitsgruppen empfohlen
                \item[!] Nicht für Verfügbarkeitszonen
            \end{itemize}
        \item Standard SKU:
            \begin{itemize}
                \item immer statisch
                \item Standardmäßig geschlossen
                \item Netzwerksicherheitsgruppen erforderlich
            \end{itemize}
        \item Load-Balancer muss zum SKU passen (Basic, Standard)
        \item IP-Bereiche je nach Region
    \end{itemize}
    \vspace*{\stretch{1}}
\end{flashcard}

\subsubsectioncard{create and configure network routes}

\begin{flashcard}[Definition]{Systemrouten}
    \vspace*{\stretch{1}}
    Typ des nächsten Hops für Systemrouten
    \begin{itemize}
        \item Virtuelles Netzwerk:
        \item Internet: \emph{0.0.0.0/0} leitet an das Internet weiter
        \item Keiner: Datenverkehr wird gelöscht
    \end{itemize}

    Weitere Systemrouten:
    \begin{itemize}
        \item VNET-Peering
        \item Dienstverkettung
        \item Gateway des virtuellen Netzwerks
        \item VNET-Dienstendpunkte
    \end{itemize}

    \vspace*{\stretch{1}}
\end{flashcard}

\begin{flashcard}[Definition]{Nächster Hop für Benutzerdefinierte Routen}
    \vspace*{\stretch{1}}
    \begin{itemize}
        \item Virtuelles Gerät: z.\,B. Firewall, typischerweise IP
        \item Gateway für virtuelle Netzwerke: z.\,B. VPN
        \item Virtuelles Netzwerk: zum Überschreiben der Systemroute
        \item Internet: leitet an das Internet weiter
        \item Keiner: Datenverkehr wird gelöscht
    \end{itemize}
    \vspace{1cm}
    Nächster Hop jeweils für Adress-Präfix (w.x.y.z/p)
    \vspace*{\stretch{1}}
\end{flashcard}

\begin{flashcard}[Definition]{Erstellen von Routen}
    \vspace*{\stretch{1}}
    \begin{enumerate}
        \item Routing Tabelle
        \item Route
        \item Zuordnen der Route an Subnetz
    \end{enumerate}
    \vspace*{\stretch{1}}
\end{flashcard}

\begin{flashcard}[Definition]{Route-Tabelle einrichten}
    \vspace*{\stretch{1}}
    \begin{itemize}
        \item Benötigt:
            \begin{itemize}
                \item Resourcegruppe
                \item Name
                \item disable-bgp-route-propagation / Gateway-Weiterleitung
            \end{itemize}
        \item CLI: \texttt{az network route-table create}
        \item Portal: neue Resource vom Typ Route Table
    \end{itemize}
    \vspace*{\stretch{1}}
\end{flashcard}

\begin{flashcard}[Definition]{Route erstellen}
    \vspace*{\stretch{1}}
    \begin{itemize}
        \item Benötigt:
            \begin{itemize}
                \item Resourcegruppe
                \item Name
                \item Route Table, zu dem die Route gehört
                \item address-prefix
                \item Nächster Hop: Typ und Adresse
            \end{itemize}
        \item CLI: \texttt{az network route-table route create}
        \item Portal: \texttt{Einstellungen | Routen | Hinzufügen} im Route-Tabellen-Blade
    \end{itemize}
    \vspace*{\stretch{1}}
\end{flashcard}

\begin{flashcard}[Definition]{Routentabelle einem Subnetz zuordnen}
    \vspace*{\stretch{1}}
    \begin{itemize}
        \item Benötigt:
            \begin{itemize}
                \item Identifikation: Subnetz + VNET + Resourcegruppe
                \item Routentabelle
            \end{itemize}
        \item CLI: \texttt{az network vnet subnet update}
        \item Portal: \texttt{Routingtabelle} im \texttt{Subnetz}-Blade ( \texttt{Einstellungen | Subnetz} im VNET-Blade) \newline
            oder \newline
            \texttt{Einstellungen | Subnetze | Subnetz zuordnen} im Routen-Tabellen-Blade
    \end{itemize}
    \vspace*{\stretch{1}}
\end{flashcard}

\subsubsectioncard{create and configure network interface}

\begin{flashcard}[Definition]{Erstellen von Netzwerkinterfaces}
    \vspace*{\stretch{1}}
    \begin{itemize}
        \item Standardmäßig hat jede VM ein Netzwerkinterfaces
        \item zusätzliche können erstellt werden
        \item je nach SKU können VMs mehrere haben
    \end{itemize}
    \vspace*{\stretch{1}}
\end{flashcard}

\begin{flashcard}[Definition]{Einstellungen für Netzwerkinterfaces}
    \vspace*{\stretch{1}}
    \begin{itemize}
        \item private IP
        \item öffentliche IP (optional)
    \end{itemize}
    \vspace*{\stretch{1}}
\end{flashcard}

\begin{flashcard}[Definition]{Nicht-Zugelassene IP-Adressen}
    \vspace*{\stretch{1}}
    \begin{itemize}
        \item Multicast:\newline
            224.0.0.0 bis 239.255.255.255
        \item Broadcast:\newline
            255.255.255.255\newline
            1-Bits für die CIDR-Maske, z.\,B. 10.20.0.255 für 10.20.0.0/24
        \item Loopback:\newline
            127.0.0.0
        \item Link-Local:\newline
            169.254.0.0 bis 169.254.255.255
        \item Azure-Internes DNS:\newline
            168.63.129.16
    \end{itemize}
    \vspace*{\stretch{1}}
\end{flashcard}

\subsubsectioncard{create and configure subnets}

\begin{flashcard}[Definition]{Subnetze}
    \vspace*{\stretch{1}}
    \begin{itemize}
        \item Teil eines virtuellen Netzwerks
    \end{itemize}
    \vspace*{\stretch{1}}
\end{flashcard}

\begin{flashcard}[Definition]{Subnetz erstellen}
    \vspace*{\stretch{1}}
    \begin{itemize}
        \item Benötigt:
        \begin{itemize}
            \item Name
            \item Adressbereich
        \end{itemize}
    \end{itemize}
    \vspace*{\stretch{1}}
\end{flashcard}

\begin{flashcard}[Definition]{Adressbereiche in Subnetzen}
    \vspace*{\stretch{1}}
    \begin{itemize}
        \item die ersten vier Adressen in jedem Subnetz sind reserviert
        \item[$\Rightarrow$] \texttt{x.y.z.4} ist die erste vergebbare Adresse
        \item[$\Rightarrow$] Ein /30-Netz macht keinen Sinn
        \item x.y.z.255 ist Broadcast-Adresse
    \end{itemize}
    \vspace*{\stretch{1}}
\end{flashcard}

\subsubsectioncard{create and configure virtual network}

\begin{flashcard}[Definition]{Virtuelles Netzwerk}
    \vspace*{\stretch{1}}
    \begin{itemize}
        \item Isolation von Resourcen
        \item Verbindung mit lokalen Computern (und anderen Netzen)
        \item externe Verbindung (zum Internet)
        \item Kommunikation zwischen Azure-Ressourcen
    \end{itemize}
    \vspace*{\stretch{1}}
\end{flashcard}

\begin{flashcard}[Definition]{Erstellen eines virtuellen Netzwerks}
    \vspace*{\stretch{1}}
    \begin{itemize}
        \item Benötigt:
        \begin{itemize}
            \item Ressourceengruppe + Abonnement, Location
            \item Name
            \item Adressraum (CIDR)
            \item Subnetz (oder neu anlegen)
            \item Eigenschaften: DDos-Schutz, Dienstendpunkte, Firewall
        \end{itemize}
        \item CLI: \texttt{az network vnet create}
        \item Portal: \texttt{Ressource erstellen | Netzwerk | virtuelles Netzwerk}
    \end{itemize}
    \vspace*{\stretch{1}}
\end{flashcard}

\subsubsectioncard{create and configure Network Security Groups and Application Security Groups}

\begin{flashcard}[Definition]{Netzwerksicherheitsgruppe}
    \vspace*{\stretch{1}}
    \begin{itemize}
        \item Teil eines virtuellen Netzwerks
        \item Filtern von Datenverkehr für virtuelles Netzwerk oder Subnetz
        \begin{itemize}
            \item IP-Adresse (Quelle oder Ziel)
            \item Port
            \item Protokoll
        \end{itemize}
    \end{itemize}
    \vspace*{\stretch{1}}
\end{flashcard}

\begin{flashcard}[Definition]{Erstellen einer Netzwerksicherheitsgruppe}
    \vspace*{\stretch{1}}
    \begin{itemize}
        \item Benötigt:
            \begin{itemize}
                \item Resourcengruppe: existierende Resourcengruppe
                \item Name: beliebiger Name
            \end{itemize}
        \item Portal: Ressource erstellen, Netzwerksicherheitsgruppe suchen
        \item CLI: \texttt{az network nsg create}
    \end{itemize}
    \vspace*{\stretch{1}}
\end{flashcard}

\begin{flashcard}[Definition]{Netzwerksicherheitsgruppe zuordnen}
    \vspace*{\stretch{1}}
    \begin{itemize}
        \item Benötigt:
            \begin{itemize}
                \item Virtuelles Netzwerk
                \item Subnetz
            \end{itemize}
        \item Portal: \texttt{Einstellungen | Subnetze | Zuordnen} im Blade der Netzwerksicherheitsgruppe
        \item CLI: \texttt{az network nsg ???}
    \end{itemize}
    \vspace*{\stretch{1}}
\end{flashcard}

\begin{flashcard}[Definition]{Konfigurieren von Netzwerksicherheitsgruppen}
    \vspace*{\stretch{1}}
    Konfiguration über Regeln:
    \begin{itemize}
        \item Regeln (>= 1) zum Zulassen oder Verweigern von Datenverkehr
        \item Priorität (100 bis 4069), kleiner ist besser (wird zuerst evaluiert)
        \item Quelle/Ziel: CIDR, Anwendungssicherheitsgruppe, Tag, oder \emph{Any}
        \item Ports
        \item Protokolle (TCP, UDP, \emph{Any}
        \item Ein- oder ausgehender Datenverkehr
        \item Aktion: Allow or Deny
    \end{itemize}
    Standardregeln: DenyAll, AllowLoadBalancer, AllowVNetInbound (vom gleichen Subnetz)
    \vspace*{\stretch{1}}
\end{flashcard}

\begin{flashcard}[Definition]{Regeln für Netzwerksicherheitsgruppe}
    \vspace*{\stretch{1}}
    \begin{itemize}
        \item Priorität 1000 bis 65000
        \item Port-Bereiche (einzeln, Kommagetrennt, Bereiche)
        \item Quellen+Zieladressen:
        \begin{itemize}
            \item IP-Adressen, z.\,B. \texttt{10.0.0.4}
            \item Anwendungssicherheitsgruppe: leerer Präfix, und asg definieren
            \item Nur Quelle: Virtuelles Netzwerk
            \item Nur Ziel: Service Tag, z.\,B. Endpunkte wie \emph{Storage}
        \end{itemize}
        \item CLI: \texttt{az network nsg rule create}
        \item Portal: Einstellungen im Netzwerksicherheitsgruppen-Blade:\newline
            \texttt{Eingangssicherheitsregeln}, \texttt{Ausgangssicherheitsregeln}
    \end{itemize}
    \vspace*{\stretch{1}}
\end{flashcard}

\begin{flashcard}[Definition]{Anwendungssicherheitsgruppen}
    \vspace*{\stretch{1}}
    \begin{itemize}
        \item Wird NIC zugeordnet
        \item Verbindung mit Netzwerksicherheitsgruppe über Regeln:\newline
            \texttt{--source-asgs ERP-DB-SERVERS-ASG} anstelle von \texttt{--source-address-prefixes}
        \item Allow/Deny funktioniert auch mit Dienstendpunkten, z.b. \texttt{Storage} \newline
            Dienstendpunkte werden als storage-account network-rule definiert
        \item \texttt{az network asg create}
    \end{itemize}
    \vspace*{\stretch{1}}
\end{flashcard}


    \subsectioncard{Azure Active Directory verwalten}

\begin{flashcard}[Definition]{Azure Active Directory Lizenzen}
    \vspace*{\stretch{1}}
    \begin{itemize}
        \item Free: grundlegene Features
            \begin{itemize}
                \item Benutzer und Gruppenverwaltung
                \item lokale Synchronisierung
                \item Self-Service-Kennwortrücksetzung
                \item Single Sign On
            \end{itemize}
        \item Premium P1: lokale und Cloudbasierte Resourcen werden unterstützt, MFA, Kennwortrückschreiben (Azure Kennwörter werden ins lokale Active Directory rückübermittelt) (Andere Richtung ist in Free enthalten)
            \begin{itemize}
                \item automatische Gruppenverwaltung
                \item Identitätsverwltung: Identity Manager
                \item Self-Service-Kennwortrücksetzung auch für lokale Nutzer
            \end{itemize}
        \item Premium P2:
                unterstützt \emph{Identity Protection} und \emph{Privileged Identity Management} (Adminstratorverfolgung und -zugriff)
        \item nutzungsbasierte Zahlung: bestimmte Features
    \end{itemize}
    \vspace*{\stretch{1}}
\end{flashcard}

\begin{flashcard}[Definition]{Active Directory Enterprise Roaming}
    \vspace*{\stretch{1}}
    \begin{itemize}
        \item Synchronisierung von Nutzer-Einstellungen für mehrere Geräte (Windows)
        \item Aktivieren:\newline
            \texttt{Azure Active Directory | Devices | Enterprise State Roaming}
            \begin{itemize}
                \item alle Einstellungen synchronisieren
                \item Benutzerdefinierte
                \item keine Einstellungen synchronisieren
            \end{itemize}
    \end{itemize}
    \vspace*{\stretch{1}}
\end{flashcard}

\begin{flashcard}[Definition]{Persönliche Benachrichtigung}
    \vspace*{\stretch{1}}
    \begin{itemize}
        \item Im Portal:\newline
            \texttt{Azure Active Directory | Identity Governance | Nutzungsbedingungen }
    \end{itemize}
    \vspace*{\stretch{1}}
\end{flashcard}

\begin{flashcard}[Definition]{Privileged Identity Management}
    \vspace*{\stretch{1}}
    \begin{itemize}
        \item Erlaubt einmaliges Anmelden mit höherer Berechtigung
    \end{itemize}
    \vspace*{\stretch{1}}
\end{flashcard}

\subsubsectioncard{add custom domains}

\begin{flashcard}[Definition]{Benutzerdefinierte Domain}
    \vspace*{\stretch{1}}
    \begin{itemize}
        \item speichern in Azure Active Directory
        \item Standard-Domain-Name: \emph{domain.onmicrosoft.com}\newline
            $\Rightarrow$ kann nicht geändert werden
        \item ein benutzerdefinierter kann hinzugefügt werden
        \item Voraussetzungen:
            \begin{itemize}
                \item Eine Domain ist vorhanden
                \item Directory in Azure AD
            \end{itemize}
        \item \texttt{Custom domain names | + Add custom domain} im Active Directory-Blade\newline
            Domainname inklusive Top-Level-Domain
    \end{itemize}
    \vspace*{\stretch{1}}
\end{flashcard}

\begin{flashcard}[Definition]{Eine benutzerdefinierte Domain verifizieren}
    \vspace*{\stretch{1}}
    Standardmäßig erstellte benutzerdefinierte Domain ist nicht verifiziert
    \begin{itemize}
        \item Beim Registrar muss ein Record gespeichert werden
            \begin{itemize}
                \item \texttt{TXT}: Besitznachweis
                \item \texttt{MX}: Mail
            \end{itemize}
        \item Werte eintragen (von Azure zum Registrar kopieren):
            \begin{itemize}
                \item Alias/Hostname
                \item Destination
                \item TTL (Standard 3600, 1 Stunde)
            \end{itemize}
    \end{itemize}
    $\Rightarrow$ wenn alles eingetragen ist, kann verifiziert werden
    \vspace{1cm}
    ! Ein Domainname kann nur für ein AD-Verzeichnis genutzt werden\newline
    (aber mehrere Domains pro Verzeichnis)
    \vspace*{\stretch{1}}
\end{flashcard}

\subsubsectioncard{configure Azure AD Identity Protection}

\begin{flashcard}[Definition]{Identity Protection}
    \vspace*{\stretch{1}}
    \begin{itemize}
        \item automatische Erkennung von Risiken
        \item Warnung bei risikobehaftetem Zugriff
        \item exportieren von Daten, z.\,B. zum Erstellen von Berichten oder zur Analyse
        \item Risikorichtlinien mit Anweisungen für Bedrohungen
            \begin{itemize}
                \item Mehrfaktor-Authentifizierung erzwingen
            \end{itemize}
    \end{itemize}
    \vspace*{\stretch{1}}
\end{flashcard}

\subsubsectioncard{configure Azure AD Join}

\subsubsectioncard{configure self-service password reset}

\begin{flashcard}[Definition]{Self Service Kennwortrücksetzung}
    \vspace*{\stretch{1}}
    \begin{itemize}
        \item Benutzer können ihr Kennwort selbst ändern
            \begin{itemize}
                \item entweder, wenn sie eingeloggt sind und es ändern (immer)
                \item wenn sie keinen Zugriff mehr haben (nur wenn aktiviert, nicht in Free)
            \end{itemize}
        \item Azure verwaltet Authentifizierung und Sicherheit (z.\,B. Captcha)
            \begin{itemize}
                \item Mehrere sicherheitsmethoden: App, Code, Mail, Telefon, Sicherheitsfrage
                \item Mindestzahl an Methoden (1 oder 2)
                \item nicht alle für Administratorn verfügbar (und mindestens 2)!
            \end{itemize}
        \item Rückschreiben in lokales AD nur mit Azure AD Premium
    \end{itemize}
    \vspace*{\stretch{1}}
\end{flashcard}

\subsubsectioncard{implement conditional access policies}

\begin{flashcard}[Definition]{Legacy-Protokolle}
    \vspace*{\stretch{1}}
    Unterstützt von Azure Active Directory:
    \begin{itemize}
        \item alte Office-Versionen
        \item Mail-Protokolle: \texttt{IMAP}, \texttt{SMTP}, \texttt{POP3}
    \end{itemize}
    \vspace{1cm}
    trotz mehrstufiger Authentifizierung können Legacy-Protokolle noch genutzt werden\newline
    $\Rightarrow$ gefährliche logins
    \vspace*{\stretch{1}}
\end{flashcard}

\begin{flashcard}[Definition]{Bedingte Zugriffsrichtlinien}
    \vspace*{\stretch{1}}
    Nicht jeder Zugriff wird mit zweitem Faktor abgesichert

    \vspace{1cm}
    Abhängig von verschiedenen Faktoren mehr Sicherheit einfordern

    \vspace{1cm}
    Teile einer Richtlinie:
    \begin{itemize}
        \item Name
        \item Zuweisungen
        \item Zugriffs-Kontrolle
    \end{itemize}

    \vspace*{\stretch{1}}
\end{flashcard}

\begin{flashcard}[Definition]{Bedingte Zugriffsrichtlinien zuweisen}
    \vspace*{\stretch{1}}
    \begin{itemize}
        \item für Benutzer oder Gruppen
        \item mit Cloud-Apps und -Aktionen verbinden (zum Auslösen der Richtlinie)\newline
            $\Rightarrow$ einfordern zusätzlicher Aktionen, z.\,B. MFA
        \item weitere Bedingungen:\newline
            Plattform, Standort, \ldots
    \end{itemize}
    \vspace*{\stretch{1}}
\end{flashcard}

\begin{flashcard}[Definition]{Bedingungen für Zugriffsrichtlinie}
    \vspace*{\stretch{1}}
    \begin{itemize}
        \item Geräteplattform
        \item Standort
            \begin{itemize}
                \item Alle
                \item vertrauenswürdig
                \item ausgewählte Standorte (IP-Bereiche, benannt/Länder)
            \end{itemize}
        \item Apps
            \begin{itemize}
                \item Browser
                \item Apps
                    \begin{itemize}
                        \item Modern, mit Active Directory Authentication Library (ADAL)
                        \item Exchange ActiveSync
                        \item Andere
                    \end{itemize}
            \end{itemize}
        \item Gerätestatus
    \end{itemize}
    \vspace*{\stretch{1}}
\end{flashcard}

\begin{flashcard}[Definition]{Bedingte Zugriffsrichtlinien Zugriffs-Kontrolle}
    \vspace*{\stretch{1}}
    Legen fest, welche Kontrolle erforderlich ist
    \begin{itemize}
        \item blockieren
        \item verschiedene Methoden
        \item eine von mehreren, oder alle erzwingen
    \end{itemize}
    \vspace{1cm}
    Mögliche Zugriffsmethoden:
    \begin{itemize}
        \item mehrstufige Authentifizierung erforderlich
        \item Gerät muss konform sein
        \item Hybrid Azure AD-Gerät
        \item zugelassene Apps
        \item Schutz-Policy
    \end{itemize}

    \vspace*{\stretch{1}}
\end{flashcard}

\begin{flashcard}[Definition]{Baseline-Richtlinien}
    \vspace*{\stretch{1}}
    \begin{itemize}
        \item Standardwerte für Sicherheit
        \item mehrere zur Auswahl
            \begin{itemize}
                \item MFA für alle Nutzer verpflichtend
                \item MFA für Administratoren
                \item \emph{Alle} legacy logins verbieten. Für spezifischere Regeln, Conditional Access
                \item MFA für Nutzer, falls notwendig
                \item Schutz von Zugriff mit hohen Rechten: z.\,B. den Resource Manager nutzen
            \end{itemize}

    \end{itemize}
    \vspace*{\stretch{1}}
\end{flashcard}

\subsubsectioncard{manage multiple directories}

\subsubsectioncard{perform an access review}

\begin{flashcard}[Definition]{Azure AD Anmeldeprotokolle}
    \vspace*{\stretch{1}}
    \begin{itemize}
        \item Auflistung aller Anmeldungen, nach App und Benutzer
        \item Im Portal: \newline
            \texttt{Azure Active Directory | Anmeldungen}
        \item Filterbar nach Client-App, Zeitraum, Status, \ldots
    \end{itemize}
    \vspace*{\stretch{1}}
\end{flashcard}


    \subsectioncard{Einführen und verwalten von Hybrididentitäten}

\subsubsectioncard{install and configure Azure AD Connect}

\begin{flashcard}[Definition]{Azure AD Connect}
    \vspace*{\stretch{1}}
    \begin{itemize}
        \item zum Synchronisieren lokaler Benutzer mit Azure
        \item synchronisiert in beide Richtungen
        \item kostenlos in jedem Abonnement
        \item beidseitige Synchronisierung erfordert P2
    \end{itemize}
    \vspace*{\stretch{1}}
\end{flashcard}

\begin{flashcard}[Definition]{Hybride Identität}
    \vspace*{\stretch{1}}
    \begin{itemize}
        \item einzelne (Benutzer-)Identität für zwei Systeme
            \begin{itemize}
                \item Active Directory (für lokalen Zugriff)
                \item Azure Active Directory (für Web-Zugriff)
            \end{itemize}
        \item mehrere Authentifizierungsmethoden
        \item[$\Rightarrow$] erlauben Single sign on
    \end{itemize}
    \vspace*{\stretch{1}}
\end{flashcard}

\begin{flashcard}[Definition]{Benutzer-Authentifizierung für Hybrididentitäten}
    \vspace*{\stretch{1}}
    \begin{itemize}
        \item Azure AD-Kennwortsynchronisierung:\newline
            synchronisiert Benutzer + Passwort zwischen Azure AD und Active Directory\newline
            (Passwort-Hash-Synchronisierung (PHS)
        \item Azure AD-Passthrough:\newline
            Ein Agent leitet Authentifizierungsanfragen von Azure-AD an lokales Active Directory weiter\newline
            (PTA)
        \item Verbundauthentifizierung\newline
            Authentifizierung über lokalen AD FS-Server\newline
            (AD FS)
    \end{itemize}
    \vspace*{\stretch{1}}
\end{flashcard}

\subsubsectioncard{configure federation}

\begin{flashcard}[Definition]{Federated Authentifizierung}
    \vspace*{\stretch{1}}
    \begin{itemize}
        \item Authentifizierung wird von einer vertrauenswürdigen Organisation durchgeführt
        \item z.\,B. lokale Installation von \emph{Active Directory Federation Services (AD FS)}
        \item Authentifizierungssystem kann weitergehende Bedingungen erfüllen\newline
            z.\,B. Smartcard, einbindung Dritter, \ldots
    \end{itemize}
    \vspace*{\stretch{1}}
\end{flashcard}

\subsubsectioncard{configure single sign-on}

\begin{flashcard}[Definition]{Single Sign On}
    \vspace*{\stretch{1}}
    \begin{itemize}
        \item ein Login (eine Identität) für mehrere Services
        \item nur ein Passwort für mehrere Services
    \end{itemize}
    \vspace*{\stretch{1}}
\end{flashcard}

\begin{flashcard}[Definition]{Single Sign On-Varianten}
    \vspace*{\stretch{1}}
    Cloud-Anwendungen:
    \begin{itemize}
        \item OpenID/OAuth
        \item SAML
        \item Passwort-Basierend
        \item verknüpft
    \end{itemize}
    Lokale Anwendungen:
    \begin{itemize}
        \item Passwort-Basiert
        \item Integrated Windows Authentifizierung (IWA)
        \item Header-Basiert
        \item verknüpft
    \end{itemize}
    Deaktiviert: für Anwendungen, die inkompatibel sind
    \vspace*{\stretch{1}}
\end{flashcard}

\begin{flashcard}[Definition]{Voraussetzungen für Seamless Single Sign-On}
    \vspace*{\stretch{1}}
    \begin{itemize}
        \item Klienten müssen in der Domäne eingebunden sein (domain -joined)
        \item[!] nicht für Azure AD-eingebundene geräte
    \end{itemize}
    \vspace*{\stretch{1}}
\end{flashcard}

\subsubsectioncard{manage and troubleshoot Azure AD Connect}

\begin{flashcard}[Definition]{Es wird nicht synchronisiert}
    \vspace*{\stretch{1}}
    Das scheduling ist pausiert
    \begin{itemize}
        \item PowerShell:\newline
            \texttt{Set-ADSyncScheduler -SyncCycleEnabled \$true}
    \end{itemize}
    \vspace*{\stretch{1}}
\end{flashcard}


\subsubsectioncard{troubleshoot password sync and writeback}


    \subsectioncard{Implementierung von Lösungen mit virtuellen Maschinen}

\subsubsectioncard{provision VMs}

\subsubsectioncard{create Azure Resource Manager templates}

\begin{flashcard}[Definition]{Azure Rsource Manager-Vorlagen-Format}
    \vspace*{\stretch{1}}
    JSON -- Java Script Object Notation
    \vspace*{\stretch{1}}
\end{flashcard}

\begin{flashcard}[Definition]{Ausdrücke in ARM-Templates}
    \vspace*{\stretch{1}}
    Zusätzlich zu JSON-Objekten (Arrays, Maps) gibt es Ausdrücke, die ausgewertet werden

    \vspace{1cm}
    Form: \texttt{[expression]}

    \vspace{1cm}
    In Ausdrücken kann auf objekte wie Ressourcen und Funktionen zugegriffen werden
    \vspace*{\stretch{1}}
\end{flashcard}


\subsubsectioncard{configure Azure Disk Encryption for VMs}

\begin{flashcard}[Definition]{Verschlüsselung von VMs}
    \vspace*{\stretch{1}}
    \begin{itemize}
        \item Standardmäßig Ausführugn
        \item benutzt Bitlocker (für Windows) oder DM-Crypt (Linux)
        \item Schlüssel in Azure KeyVault ($\rightarrow$ \texttt{key})\newline
            \emph{(in gleicher Region)}\newline
            [!] verwaltet von Azure
        \item Voraussetzung: bessere Maschine als A, mehr als 2 GiB Hauptspeicher, Generation 1 VM
    \end{itemize}
    \vspace*{\stretch{1}}
\end{flashcard}

\begin{flashcard}[Definition]{VM-Verschlüsselung einrichten}
    \vspace*{\stretch{1}}
    \begin{itemize}
        \item Zugang zu Azure Active Directory-Endpunkt
        \item Zugang zu Key Vault-Endpunkt
        \item Azure Storage-Endpunkt für das Speicherkonto mit VHD-Dateien
    \end{itemize}
    \vspace{1cm}
    $\Rightarrow$ überprüfen in \texttt{VM | Übersicht}
    \vspace*{\stretch{1}}
\end{flashcard}

\begin{flashcard}[Definition]{Azure Disk-Verschlüsselung aktivieren}
    \vspace*{\stretch{1}}
    \begin{itemize}
        \item CLI:\newline
            \texttt{az vm encryption enable -resource-group \ldots --name <vm-name>\\--disk-encryption-keyvault \ldots}
        \item PowerShell:\newline
            \texttt{Set-AzVMDiskEncryptionExtension -ResourceGroupName \ldots\\-VMName \ldots -DiskEncryptionKeyVaultUrl \$KeyVault.VaultUri\\-DiskEncryptionKeyVaultId \$KeyVault.ResourceId}
        \item Parameter: \texttt{--volume-type}: \texttt{DATA, OS, ALL}
    \end{itemize}
    \vspace*{\stretch{1}}
\end{flashcard}

\subsubsectioncard{implement Azure Backup for VMs}

\begin{flashcard}[Definition]{Azure Site Recovery}
    \vspace*{\stretch{1}}
    \begin{itemize}
        \item Tool zur Wiederherstellung von Daten
        \item Replikation zwischen primärem und sekundärem Standort
        \item unterstützt lokale (inkls. physische Maschinen) und Azure als Quell-Standort
        \item Failback (nicht zu physische Maschinen)
        \item Erstellt Momentaufnahmen und Wiederherstellungspunkte
    \end{itemize}
    \vspace*{\stretch{1}}
\end{flashcard}

\begin{flashcard}[Definition]{RTO}
    \vspace*{\stretch{1}}
    \begin{itemize}
        \item Recovery Time Objective
        \item Maximale Zeitspanne, die da Unternehmen nach einem Notfall überleben kann, bevor der normale Betrieb wiederhergestellt wurde
    \end{itemize}
    \vspace*{\stretch{1}}
\end{flashcard}

\begin{flashcard}[Definition]{RPO}
    \vspace*{\stretch{1}}
    \begin{itemize}
        \item Recovery Point Objective
        \item Maximaler Datenverlust, der im Notfall akzeptabel ist
    \end{itemize}
    \vspace*{\stretch{1}}
\end{flashcard}

\begin{flashcard}[Definition]{Failover}
    \vspace*{\stretch{1}}
    Übertragung der Ausführugn an einen sekundären Standort (z.\,B. weil der primäre nicht verfügbar ist)
    \vspace{1cm}
    Nach einem Failover wird der Schutz deaktiviert!
    \vspace*{\stretch{1}}
\end{flashcard}

\begin{flashcard}[Definition]{Failback}
    \vspace*{\stretch{1}}
    Rück-Übertragung der Ausführugn von einem sekundären Standort (z.\,B. weil der primäre wieder verfügbar ist)
    \vspace*{\stretch{1}}
\end{flashcard}

\begin{flashcard}[Definition]{Phasen der Replikation}
    \vspace*{\stretch{1}}
    \begin{enumerate}
        \item Failover
        \item erneutes Schützen (in der sekundären Region)
        \item Failback
        \item erneutes Schützen (in der primären Region)
    \end{enumerate}
    \vspace*{\stretch{1}}
\end{flashcard}

\begin{flashcard}[Definition]{Voraussetzungen für Site Recovery}
    \vspace*{\stretch{1}}
    \begin{itemize}
        \item Netzwerk für die Ziel-Ressourcen (Backup, replizierte VMs)
        \item Recovery-Services-Tresor
        \item Rechte:
            \begin{itemize}
                \item Mitwirkender für virtuelle Computer
                \item Site-Recovery-Mitwirkender
            \end{itemize}
        \item Speicherkonto außerhalb der Quellregion (in Zielregion)
    \end{itemize}
    \vspace*{\stretch{1}}
\end{flashcard}

\begin{flashcard}[Definition]{Replikation von lokalen VMs}
    \vspace*{\stretch{1}}
    Voraussetzungen:
    \begin{itemize}
        \item Konfigurationsserver (VMware, lokal)
        \item Konfigurationsserver muss auch existieren für Failback von \emph{physischen} Maschinen
    \end{itemize}
    Schritte:
    \begin{enumerate}
        \item Netzwerk einrichten
        \item Site-Recovery-Tresor erstellen
        \item Berechtigungen/Anmeldeinformationen
        \item Konfigurationsserver in vCenter über OVA
    \end{enumerate}

    \vspace*{\stretch{1}}
\end{flashcard}

\begin{flashcard}[Definition]{Replikation von Azure VMs}
    \vspace*{\stretch{1}}
    \begin{itemize}
        \item Netzwerk einrichten
        \item Site-Recovery-Tresor ertellen
        \item Azure-Mobilitätsdienst auf VMs installieren (Linux und Windows)\newline
            wird normalerweise von Site Recovery automatisch erledigt
    \end{itemize}
    \vspace*{\stretch{1}}
\end{flashcard}

\begin{flashcard}[Definition]{Failover vorbereiten}
    \vspace*{\stretch{1}}
    Vorbereitung: Ändern von Einstellungen
    \begin{itemize}
        \item Azure-Name
        \item Ressourcengruppe
        \item Ziel-SKU
        \item Verfügbarkeitsgruppe
        \item Verwaltete Disks
        \item Netzwerke: VNet + Subnetz, IP-Addresse
    \end{itemize}
    \ldots können alle angepasst werden!
    \vspace*{\stretch{1}}
\end{flashcard}

\begin{flashcard}[Definition]{Backup von virtuellen Maschinen}
    \vspace*{\stretch{1}}
    Azure Backup-Dienst
    \begin{itemize}
        \item erstellt und verwaltet Backups
        \item keine Infrastruktur notwendig
        \item erstellt Kopien zustandsbehafter Daten um alte Zustände wieder herzustellen
        \item Langzeitaufbewahrung (mehrere Jahre)
        \item sicherer Zugriff (RBAC, Verschlüsselung, \ldots)
        \item vorläufiges Löschen
        \item Sicherung autark im Azure-Netzwerk
    \end{itemize}
    \vspace*{\stretch{1}}
\end{flashcard}

\begin{flashcard}[Definition]{Szenarien von Azure Backup}
    \vspace*{\stretch{1}}
    \begin{itemize}
        \item Virtuelle Maschinen in Azure sichern
        \item lokale Maschinen sichern (MARS, DPM)
            \begin{itemize}
                \item physische server und Hyper-V und VMware
                \item Rückspielung nur auf virtuelle Maschinen
            \end{itemize}
        \item Dateifreigaben
        \item SQL Server
    \end{itemize}
    \vspace*{\stretch{1}}
\end{flashcard}

\begin{flashcard}[Definition]{Backup mit MARS}
    \vspace*{\stretch{1}}
    \begin{itemize}
        \item Agent-Software für Windows-Maschinen
        \item Für physikalische Windows-Maschinen und virtuelle Windows-Maschinen
        \item Sichert Verzeichnisse und Dateien, nicht komplette VMs
        \item kann zusammen mit DPM und MABS betrieben werden
    \end{itemize}
    \vspace*{\stretch{1}}
\end{flashcard}

\begin{flashcard}[Definition]{Backup mit DPM}
    \vspace*{\stretch{1}}
    Data Protection Manager
    \begin{itemize}
        \item Unterstützte Backups
            \begin{itemize}
                \item Anwendungen (SQL, \ldots)
                \item Dateien
                \item vollständige Systeme (Windows)
                \item Hyper-V VMs, (Windows + Linux)
            \end{itemize}
        \item Sicherungs-Ziele
            \begin{itemize}
                \item Platte (kurzzeitige Sicherungen)
                \item Azure (kurzzeitig, Langzeitaufbewahrung)
                \item Tape (Langzeitaufbewahrung)
            \end{itemize}
    \end{itemize}
    \vspace*{\stretch{1}}
\end{flashcard}

\begin{flashcard}[Definition]{Backup mit MABS}
    \vspace*{\stretch{1}}
    Azure Backup Server
    \begin{itemize}
        \item Agent-Software
        \item Unterstützte Backups
            \begin{itemize}
                \item Hyper-V oder VMware (Linux, vollständige Maschine)
                \item Hyper-V oder VMware (Windows: Maschinen, oder einzelne Dateien, Shares)
                \item Servers: Dateien, SQL, Exchange
            \end{itemize}
    \end{itemize}
    \vspace*{\stretch{1}}
\end{flashcard}


\begin{flashcard}[Definition]{VM Backup einrichten}
    \vspace*{\stretch{1}}
    \begin{itemize}
        \item Recovery Services-Tresor einrichten\newline
            enthält Sicherungen
        \item Azure Backup-Erweiterung auf der VM aktivieren\newline
            \emph{VMSnapshot} bzw. \emph{VMSnapshotLinux}
        \item Sicherungsrichtlinie erstellen. Konsistenzebenen für Backups:
            \begin{itemize}
             \item Anwendungskonsistent\newline
                Kopie der gesamten VM, vollstänidge Konsistenz aller Anwendungen
             \item Dateisystemkonsistent\newline
                Kopie aller Dateien, Anwendungen müssen ggf. bereinigt werden
             \item Absturzkonsistent\newline
                Falls während der Sicherung die VM heruntergefahren wird
            \end{itemize}

    \end{itemize}
    \vspace*{\stretch{1}}
\end{flashcard}

\begin{flashcard}[Definition]{Sicherung einer VM mit dem Poral}
    \vspace*{\stretch{1}}
    Einrichten:
    \begin{itemize}
        \item Benötigt:
            \begin{itemize}
                \item Resourcengruppe
                \item Recovery Services-Tresor
                \item Sicherungsrichtlinie
            \end{itemize}
        \item Im Portal:\newline
            \texttt{Vorgänge | Sicherung | Sicherung} im VM-Blade
        \item Tresor, Resourcengruppe und Richtlinie können neu erstellt werden
    \end{itemize}
    Durchführen:
    \begin{itemize}
        \item In \texttt{Vorgänge | Backup} im Tresor eine VM auswählen
        \item \texttt{Jetzt sichern}
    \end{itemize}

    \vspace*{\stretch{1}}
\end{flashcard}

\begin{flashcard}[Definition]{Sicherung einer VM mit Azure CLI}
    \vspace*{\stretch{1}}
    (Benötigt vorhandene Resourcnegruppe, Recovery-Services-Tresor, Policy
    \begin{itemize}
        \item Sicherung einrichten:\newline
            \texttt{az backup protection enable-for-vm --resource-group \ldots --vault-name \ldots --vm \ldots --policy-name \ldots}
        \item Sicherung durchführen:\newline
            \texttt{az backup protection backup-now --resource-group \ldots --vault-name \ldots --container-name \ldots --item-name \ldots --retain-until \ldots --backup-management-type \ldots}
    \end{itemize}
    \vspace*{\stretch{1}}
\end{flashcard}

\begin{flashcard}[Definition]{Wiederhertellen einer Sicherung}
    \vspace*{\stretch{1}}
    \begin{itemize}
        \item Erstellen einer VM: eine virtuelle Maschine in der gleichen Region als Kopie erstellen
        \item Datenträger wiederherstellen: erstellt einen Datenträger mit Daten aus dem Backup
        \item Vorhandene ersetzen: ersetzt einen Datenträger auf einer existierenden VM
        \item Regionsübergreifende Wiederherstellung: wiederherstellung in die gekoppelte Partner-Region
    \end{itemize}
    Ein Backup kann in wenigen Minuten wiederhergestellt werden!
    \vspace*{\stretch{1}}
\end{flashcard}


    \sectioncard{Workloads und Sicherheit}

    \subsectioncard{Virtuelle Maschinen nach Azure migrieren}

\begin{flashcard}[Definition]{Azure Migrate}
    \vspace*{\stretch{1}}
    \begin{itemize}
        \item bereitgestellter Dienst
        \item analysiert lokale Systeme
        \item feststellen der Azure-Bereitschaft
        \item berechnet notwendige Dimensionierung
        \item kann Hyper-V, VMware-VMs und physische Server bewerten\newline
            phyisisch nur Windows!
        \item berechnet monatliche Kosten
    \end{itemize}
    \vspace*{\stretch{1}}
\end{flashcard}

\begin{flashcard}[Definition]{Durchführen einer Migration}
    \vspace*{\stretch{1}}
    Von virtuellen Maschinen:
    \begin{enumerate}
        \item Collectorapplience in vCenter Server installieren
        \item Abhängigkeits-Agenten auf lokalen VMs erstellen
        \item Bewertung für jeden Computer erstellen
        \item Bearbeiten der Bewertung je nach Anforderung
    \end{enumerate}
    \vspace*{\stretch{1}}
\end{flashcard}

\begin{flashcard}[Definition]{Migration im Portal}
    \vspace*{\stretch{1}}
    \begin{itemize}
        \item \emph{Azure Migrate: Servermigration} in Azure Migrate hinzufügen
        \item maximal 100 VMs können gleichzeitig migriert werden
    \end{itemize}
    \vspace*{\stretch{1}}
\end{flashcard}

\begin{flashcard}[Definition]{Datenbanken migrieren}
    \vspace*{\stretch{1}}
    \begin{itemize}
        \item Azure Database Migration Service
        \item Offline-Migration:\newline
            Server heruntergefahren
        \item Online-Migration\newline
            Migration während des laufenden Betriebs
    \end{itemize}
    \vspace*{\stretch{1}}
\end{flashcard}

\begin{flashcard}[Definition]{Durchführen einer Datenmigration}
    \vspace*{\stretch{1}}
    \begin{enumerate}
        \item Datenmigrations-Assistent auf dem lokalen Server installileren
        \item Azure VM für Datenbank erstellen
        \item Netzwerksicherheitsgruppe in Azure einrichten
        \item lokale Instanz vorbereiten: Firewall öffnen
        \item Anmeldeinformationen zu SQL-Server hinzufügen
        \item Datenbank in Azure bereitstellen
    \end{enumerate}
    \vspace*{\stretch{1}}
\end{flashcard}


    \subsectioncard{Konfigurieren von Serverless}

\begin{flashcard}[Definition]{Logic App}
    \vspace*{\stretch{1}}
    \begin{itemize}
        \item Kombination von vielen Diensten mit vorkonfigurierten Modulen
        \item Modellierung von Work-Flows in einem Design-First-Ansatz
        \item Erstellt im Logic-App-Designer im Portal
        \item Alternativ: JSON-Format
    \end{itemize}
    \vspace*{\stretch{1}}
\end{flashcard}

\begin{flashcard}[Definition]{Logic App-Komponenten}
    \vspace*{\stretch{1}}
    \begin{itemize}
        \item Trigger:\newline
            Auslösen einer Logic App
        \item Aktionen:\newline
            Aufruf eines Dienstes mit Daten: SQL, Machine Learning, \ldots
        \item Steureung:\newline
            Entscheidungen, Abfragen, Schleifen, \ldots
    \end{itemize}
    \vspace{1cm}
    Connector: Kombination von verknüpften Triggern und Aktionen
    \vspace*{\stretch{1}}
\end{flashcard}

\begin{flashcard}[Definition]{Trigger}
    \vspace*{\stretch{1}}
    Typen:
    \begin{itemize}
        \item Daten: Email, Tween, \ldots
            \begin{itemize}
                \item Abfragemodell: \emph{Häufigkeit (Einheit)}, \emph{Intervall (Anzahl)}
                \item Push-getrieben: webhooks
            \end{itemize}
            $\Rightarrow$ Abfragen verursachen Kosten!
        \item Serientrigger: zeitgesteuert
        \item manuelle Anforderung: webhooks, \ldots
    \end{itemize}
    Eigenschaften:
    \begin{itemize}
        \item Parameter: optional und notwendig
        \item Rückgabewerte: ein Objekt, oder mehrere ($\rightarrow$ Schleifen), teilen und parallel ausführen
    \end{itemize}

    \vspace*{\stretch{1}}
\end{flashcard}

\begin{flashcard}[Definition]{Aktionen}
    \vspace*{\stretch{1}}
    \begin{itemize}
        \item externe Dienste: Zugriff auf Dienste, z.\,B. mit Verbindungszeichenfolgen, Logins, \ldots
        \item Bearbeiten von Daten: z.\,B. Konkatenation, Array-Auswahl, \ldots
        \item Steuerung: Bedingungen, Schleifen, \ldots
    \end{itemize}
    \vspace{1cm}
    Eigenschaften:
    \begin{itemize}
        \item Parameter: Eingaben für die Aktion, statisch oder dynamisch
        \item Rückgabewerte:
    \end{itemize}
    \vspace*{\stretch{1}}
\end{flashcard}

\begin{flashcard}[Definition]{Logic App erstellen}
    \vspace*{\stretch{1}}
    Logic-App-Designer
    \begin{itemize}
        \item Benötigt:
            \begin{itemize}
                \item Name
                \item Abonnement + Ressourcengruppe
                \item Standort
            \end{itemize}
        \item im Portal: \texttt{+ Ressource erstellen | Logik-App}\newline
        \item Katalog von Konnectoren, Triggern, Aktionen
        \item \texttt{Logik-App-Designer | Vorlagen} im Logik-App-Blade, und entweder vorlage oder \texttt{Leere Logik-App}
        \item grafische Modellierung im Portal: \texttt{Logik-App-Designer} im Logik-App-Blade
    \end{itemize}
    \vspace*{\stretch{1}}
\end{flashcard}

\begin{flashcard}[Definition]{Logic-App-Designer}
    \vspace*{\stretch{1}}
    \begin{enumerate}
        \item Konnektor/Aktion suchen
        \item Parameter setzen
        \item neue Hinzufügen
    \end{enumerate}
    \vspace*{\stretch{1}}
\end{flashcard}

\begin{flashcard}[Definition]{}
    \vspace*{\stretch{1}}
    \begin{itemize}
        \item
    \end{itemize}
    \vspace*{\stretch{1}}
\end{flashcard}

\begin{flashcard}[Definition]{Azure Funktion}
    \vspace*{\stretch{1}}
    \begin{itemize}
        \item Code First
        \item Code im Portal oder in Quellcode-Verwaltung
        \item zustandslos (Alternative: Durable Functions)
        \item ereignisgesteuert: werden als Reaktion auf Trigger ausgeführt
        \item beschränkte Ausführungszeit: standard 5 Minuten, maximal 10
        \item bei häufiger Ausführung: VM kann günstiger sein
    \end{itemize}
    \vspace*{\stretch{1}}
\end{flashcard}

\begin{flashcard}[Definition]{App-Serviceplan}
    \vspace*{\stretch{1}}
    \begin{itemize}
        \item verbrauchsbasiert: serverless mit automatischer Skalierung, bezahlen nach Nutzung
        \item Azure App Service-Plan: Ausführung auf definierter VM, ermöglicht längere Funktionsausführung
    \end{itemize}
    \vspace*{\stretch{1}}
\end{flashcard}

\begin{flashcard}[Definition]{Funktions-App erstellen}
    \vspace*{\stretch{1}}
    Funktions-App: Container zum Verwalten von Funktionen
    \begin{itemize}
        \item Benötigt:
            \begin{itemize}
                \item Abonnement + Resourcengruppe
                \item Name (global eindeutig)
                \item Runtime/Sprache, Version
                \item Region
                \item Speicherkonto
                \item Betriebssystem
                \item Serviceplan
            \end{itemize}
        \item Im Portal: \texttt{+ Neue Ressource erstellen | Compute | Funktions-App}
    \end{itemize}
    \vspace*{\stretch{1}}
\end{flashcard}

\begin{flashcard}[Definition]{Konfigurieren einer Funktions-App}
    \vspace*{\stretch{1}}
    \begin{itemize}
        \item Ereignisse/Trigger zur Ausführung:
        \item Bindung
        \item Autorisierung
    \end{itemize}
    \vspace*{\stretch{1}}
\end{flashcard}

\begin{flashcard}[Definition]{Funktions-App-Trigger}
    \vspace*{\stretch{1}}
    \begin{itemize}
        \item Blob-Speicher: bei Änderungen an Blobs
        \item Azure Cosmos DB: Einfügung/Aktualisierung
        \item Event Grid: bei empfangenem Ereignis
        \item HTTP: bei hook
        \item Microsoft-Graph webhooks
        \item Queue Storage: bei Elementen in Warteschlangen, Element als Eingabe
        \item Service Bus: bei Nachrichten aus einer Service Bus-Schlange
        \item Timer: regelmäßig
    \end{itemize}
    \vspace*{\stretch{1}}
\end{flashcard}

\begin{flashcard}[Definition]{Funktions-App-Bindung}
    \vspace*{\stretch{1}}
    \begin{itemize}
        \item Kommunikation mit Diensten
        \item wird automatisch bereitgestellt
        \item Richtung
        \begin{itemize}
            \item Eingabebindung: zum Lesen von Daten
            \item Ausgabebindung: zum Ausgeben von Daten
        \end{itemize}
    \end{itemize}
    Beispiel:
    \begin{enumerate}
        \item \emph{Azure Queue Storage Trigger}: liest Daten aus Queue
        \item \emph{Azure Table Storage-Ausgabebindung}: schreibt Daten in Tabelle
    \end{enumerate}

    \vspace*{\stretch{1}}
\end{flashcard}

\begin{flashcard}[Definition]{Funktion App-Autorisierungsstufen}
    \vspace*{\stretch{1}}
    \begin{itemize}
        \item Funktion:\newline
            spezifischer API-Schlüssel
        \item Administrator\newline
            globaler Schlüssel
        \item anonym\newline
            keine Autorisierung erforderlich
    \end{itemize}
    Im Bereich \texttt{Entwickler | Funktionsschlüssel} im Funktions-Blade
    \vspace*{\stretch{1}}
\end{flashcard}

\begin{flashcard}[Definition]{Funktion erstellen}
    \vspace*{\stretch{1}}
    \begin{itemize}
        \item Im Portal:\newline
            \texttt{Function-App | Funktionen | + Hinzufügen}
        \item Vorlage auswählen
    \end{itemize}
    \vspace*{\stretch{1}}
\end{flashcard}



    \subsectioncard{Einrichten von Anwendungs-Load-Balancing}

\begin{flashcard}[Definition]{Lastenausgleichs-Lösungen in Azure}
    \vspace*{\stretch{1}}
    \begin{itemize}
        \item Application Gateway\newline
            Level 7, HTTP-Basierter Ausgleich
        \item Load Balancer\newline
            Hochperformanter Level 4-Load-Balancer (IP-Basiert)
        \item Traffic Manager\newline
            DNS lastenausgleich
    \end{itemize}
    \vspace*{\stretch{1}}
\end{flashcard}   

\subsubsectioncard{configure application gateway}

\begin{flashcard}[Definition]{Application Gateway}
    \vspace*{\stretch{1}}
    \begin{itemize}
        \item Weiterleitung abhängig von URL
        \item Anwendungsschichtrouting
        \item Ziele: VMs, Skalierungsgruppen, lokale Server\newline
            genaugenommen: nicht nur lokale Server sondern beliebige Endpunkte! (Externe URLs)
        \item Routing
            \begin{itemize}
                \item Pfadbasiert:\newline
                    abhängig von URL, z.\,B. \emph{/videos/}
                \item mehrere Webseiten
            \end{itemize}

    \end{itemize}
    \vspace*{\stretch{1}}
\end{flashcard}   

\begin{flashcard}[Definition]{Lastenausgleich im Application Gateway}
    \vspace*{\stretch{1}}
    \begin{itemize}
        \item Round robin
        \item möglich: Persistenz
        \item auf Anwendngsschicht (OSI 7)\newline
            $\Rightarrow$ im Unterschied zu Load Balancer (OSI 4, IP-Adressen)
        \item z.\,B. Ausgleich von Traffic zum Webserver
    \end{itemize}
    \vspace*{\stretch{1}}
\end{flashcard}

\begin{flashcard}[Definition]{Application Gateway-Komponenten}
    \vspace*{\stretch{1}}
    \begin{itemize}
        \item Front-End-IP-Adresse: öffentliche oder private Adresse
        \item Listener: empfängt Daten nach \emph{Protokoll,Port,Host,IP}
        \item Routing-Regeln: binden Listener an Back-End-Pools
        \item Back-End-Pools: mehrere Webserver, z.\,B. auch Skalierungsgruppen oder App Service App
        \item Web Application Firewall: optional, blockiert eingehende Anfragen vor Listener
        \item Integritätstest: tested Server im Back-End-Pool\newline
            Integritätstest für Port und Pool an den weitergeleitet wird
    \end{itemize}
    \vspace*{\stretch{1}}
\end{flashcard}

\begin{flashcard}[Definition]{Application Gateway-Listener}
    \vspace*{\stretch{1}}
    Basislistener
    \begin{itemize}
        \item beachtet nur URL
    \end{itemize}
    \vspace{1cm}
    Listener für mehrere Standorte
    \begin{itemize}
        \item nutzt auch Hostname
        \item für Unter-Domänen
    \end{itemize}
    \vspace*{\stretch{1}}
\end{flashcard}

\begin{flashcard}[Definition]{Application Gateway erstellen}
    \vspace*{\stretch{1}}
    Eigenes Subnetz im VNET für skalierung benötigt!
    \begin{itemize}
        \item Benötigt:
            \begin{itemize}
                \item Resourcegruppe + Name
                \item SKU: Standard, WAF, jeweils v1 und v2
                \item Kapazität
                \item VNET + Subnet
                \item Frontend-IP + Port
                \item HTTP-Einstellungen: Protokoll, Port, \ldots
            \end{itemize}
        \item CLI: \texttt{az network application-gateway create}
        \item Portal
    \end{itemize}
    \vspace*{\stretch{1}}
\end{flashcard}

\begin{flashcard}[Definition]{Application Gateway Leistungsstufen}
    \vspace*{\stretch{1}}
    \begin{itemize}
        \item Standard v1
            \begin{itemize}
             \item Web-Protokoll-Balancing
            \end{itemize}
        \item Standard v2 (empfohlen)
            \begin{itemize}
                \item Autoscaling
                \item Zonenredundanz (Verfügbarkeitszonen)
                \item Statische IPs
            \end{itemize}
        \item WAF v1: features von Web Application Firewall
        \item WAF v2: zusätzlich benutzerdefinierte Regeln
    \end{itemize}
    \vspace*{\stretch{1}}
\end{flashcard}

\begin{flashcard}[Definition]{Web Application Firewall}
    \vspace*{\stretch{1}}
    Managed Service, der eine HTTP Firewall darstellt
    \begin{itemize}
        \item Eingebundne in weitere Systeme:
            \begin{itemize}
                \item Application Gateway
                \item Front Door
                \item CDN
            \end{itemize}
        \item SQL injection
        \item Cross site scripting
        \item viele weitere Regeln
    \end{itemize}
    \vspace*{\stretch{1}}
\end{flashcard}

\begin{flashcard}[Definition]{Application Gateway Back-End-Pool erstellen}
    \vspace*{\stretch{1}}
    \begin{itemize}
        \item Benötigt:
            \begin{itemize}
                \item Gateway-Name
                \item Resourcegruppe + Name
                \item Server (als IPs)
            \end{itemize}
        \item CLI: \texttt{az network application-gateway address-pool create}
        \item Portal
    \end{itemize}
    \vspace*{\stretch{1}}
\end{flashcard}

\begin{flashcard}[Definition]{Application Gateway Frontend erstellen}
    \vspace*{\stretch{1}}
    \begin{itemize}
        \item Benötigt:
            \begin{itemize}
                \item Gateway-Name
                \item Resourcegruppe + Name
                \item Port
            \end{itemize}
        \item CLI: \texttt{az network application-gateway frontend-port create}
        \item Portal
    \end{itemize}
    \vspace*{\stretch{1}}
\end{flashcard}

\begin{flashcard}[Definition]{Application Gateway Listener erstellen}
    \vspace*{\stretch{1}}
    \begin{itemize}
        \item Benötigt:
            \begin{itemize}
                \item Gateway-Name
                \item Resourcegruppe
                \item Name
                \item Frontend-Port-Name
            \end{itemize}
        \item CLI: \texttt{az network application-gateway http-listener create}
        \item Portal
    \end{itemize}
    \vspace*{\stretch{1}}
\end{flashcard}

\begin{flashcard}[Definition]{Application Gateway Integritätstest}
    \vspace*{\stretch{1}}
    \begin{itemize}
        \item Ziel: testen, ob VMs im Backend erreichbar sind
        \item über HTTP-Anfragen
        \item mögliche Tests:
            \begin{itemize}
                \item HTTP-Result-Code:\newline
                    z.\,B. \texttt{200-399}
                \item return body:\newline
                    z.\,B. \texttt{Status: OK}
            \end{itemize}
    \end{itemize}
    \vspace*{\stretch{1}}
\end{flashcard}

\begin{flashcard}[Definition]{Application Gateway Integritätstest erstellen}
    \vspace*{\stretch{1}}
    \begin{itemize}
        \item Benötigt:
            \begin{itemize}
                \item Gateway-Name + Resourcegruppe
                \item Name
                \item Pfad
                \item Intervall, Threshold, Timeout
                \item Protokoll: \emph{HTTP}
            \end{itemize}
        \item CLI: \texttt{az network application-gateway probe create}
        \item Portal
    \end{itemize}
    \vspace*{\stretch{1}}
\end{flashcard}

\begin{flashcard}[Definition]{Application Gateway Regeln erstellen}
    \vspace*{\stretch{1}}
    Schritt 1: URL-Pfad-Zuordnung
    \begin{itemize}
        \item Benötigt:
            \begin{itemize}
                \item Gateway-Name + Resourcegruppe
                \item Name
                \item Pfade
                \item HTTP-Einstellungen
                \item zu benutzender Backend-Pool
            \end{itemize}
        \item CLI: \texttt{az network application-gateway url-path-map create}
        \item Portal:
    \end{itemize}
    \vspace*{\stretch{1}}
\end{flashcard}

\begin{flashcard}[Definition]{Application Gateway Regeln erstellen}
    \vspace*{\stretch{1}}
    Schritt 2: URL-Pfad-Zuordnuns-Regel erstellen
    \begin{itemize}
        \item Benötigt:
            \begin{itemize}
                \item Gateway-Name + Resourcegruppe
                \item Name
                \item Pfade
                \item HTTP-Einstellungen
                \item zu benutzender Backend-Pool
                \item zu benutzende Pfad-Map
            \end{itemize}
        \item CLI: \texttt{az network application-gateway url-path-map rule reate}
        \item Portal
    \end{itemize}
    \vspace*{\stretch{1}}
\end{flashcard}

\begin{flashcard}[Definition]{Application Gateway Regeln erstellen}
    \vspace*{\stretch{1}}
    Schritt 3: Routingregel mit der Pfadzuordnung erstellen
    \begin{itemize}
        \item Benötigt:
            \begin{itemize}
                \item Gateway-Name + Resourcegruppe
                \item Name
                \item Listener
                \item Regeltyp: \emph{PathBasedRouting}
                \item zu benutzender Backend-Pool
                \item zu benutzende Pfad-Map
            \end{itemize}
        \item CLI: \texttt{az network application-gateway rule reate}
        \item Portal
    \end{itemize}
    Achtung: standardmäßig wird eine Regel erstellt
    \vspace*{\stretch{1}}
\end{flashcard}

\begin{flashcard}[Definition]{Konfiguration von Application Gateway-Komponenten}
    \vspace*{\stretch{1}}
    \begin{itemize}
        \item Listener: \texttt{az network application-gateway http-listener}
        \item Regeln: \texttt{az network application-gateway rule}
        \item Front-End-IP: \texttt{az network application-gateway front-end-port}
        \item Back-End-Pools: \texttt{az network application-gateway address-pool}
        \item Integritätstest: \texttt{az network application-gateway http-settings}
    \end{itemize}
    \vspace*{\stretch{1}}
\end{flashcard}

\subsubsectioncard{configure Azure Front Door service}

\begin{flashcard}[Definition]{Front Door}
    \vspace*{\stretch{1}}
    \begin{itemize}
        \item Anwendungs-Auslieferungs-Netz, also level 7 Lastausgleich
        \item Beschleunigung
        \item Realzeit-Failover
        \item hochverfügbar und skalierbar, vollständig von Azure verwaltet
        \item unterstützte Protokolle: HTTP, HTTPS, HTTP/2 (nur Front End)
    \end{itemize}
    \vspace*{\stretch{1}}
\end{flashcard}

\begin{flashcard}[Definition]{Vergleich Front Door $\leftrightarrow$ Application Gateway}
    \vspace*{\stretch{1}}
    \begin{itemize}
        \item Front Door ist global\newline
            $\Rightarrow$ Ausgleich zwischen verschiedenen Clustern über Regionen
        \item Application Gateway ist lokal\newline
            $\Rightarrow$ Ausgleich zwischen VMs, Containern in einer Skalierungsgruppe
    \end{itemize}
    \vspace*{\stretch{1}}
\end{flashcard}

\begin{flashcard}[Definition]{Zusammenspiel mit Azure Load Balancer}
    \vspace*{\stretch{1}}
    \begin{itemize}
        \item Mögliche Verbindungen von/zu Front Door:
            \begin{itemize}
                \item eine öffentliche IP als Eingang
                \item öffentlicher DNS-Name als Ausgang
            \end{itemize}
        \item Azure Load Balancer kann intern eingesetzt werden
    \end{itemize}
    $\Rightarrow$ Load Balancer hinter Front Door ist möglich
    \vspace*{\stretch{1}}
\end{flashcard}

\begin{flashcard}[Definition]{Routing-Methoden}
    \vspace*{\stretch{1}}
    \begin{itemize}
        \item Latency: das am schnellsten zu erreichende Backend wird genutzt
        \item Priority: Priorisierung von Backends mit Backup-Backend
        \item Weighted: Gewichtung für die verschiedneen Backends
        \item Session Affinity: nachfolgende Anforderungen werden an das gleiche Backend geliefert
    \end{itemize}
    \vspace*{\stretch{1}}
\end{flashcard}

\begin{flashcard}[Definition]{Health-Probe}
    \vspace*{\stretch{1}}
    \begin{itemize}
        \item GET: ruft Information von URL ab
        \item HEAD: nur Header, kein Message Body
        \item Status: muss 200 sein
        \item Latency: zeit zwischen Senden der Probe und erhalt der Antwort
    \end{itemize}
    \vspace*{\stretch{1}}
\end{flashcard}

\begin{flashcard}[Definition]{Traffic-Weiterleitung}
    \vspace*{\stretch{1}}
    Front Door kann als Redirection-Service eingesetzt werden
    \begin{itemize}
        \item Untertsützt Forward/Redirect (301)
        \item Protokolle: HTTPS, HTTP, wie angefordert
    \end{itemize}
    \vspace*{\stretch{1}}
\end{flashcard}

\subsubsectioncard{configure Azure Traffic Manager}

\begin{flashcard}[Definition]{Azure Traffic Manager}
    \vspace*{\stretch{1}}
    \begin{itemize}
        \item DNS-basierendes Load Balancing
        \item unterstützt Hochverfügbarkeit, Resilienz, Reaktionszeit:
            \begin{itemize}
                \item automatisches Failover
                \item regionale Umleitung von Client-Anforderungen
            \end{itemize}
    \end{itemize}
    \vspace*{\stretch{1}}
\end{flashcard}

\begin{flashcard}[Definition]{Funktion von Traffic Manager}
    \vspace*{\stretch{1}}
    \begin{itemize}
        \item DNS-basiert, Traffic Manager hat keinen Zugriff auf Daten
        \item stellt passende IP-Adresse zur Verfügung (Klienten verbinden sich direkt dahin)\newline
            $\rightarrow$ \emph{Endpunkt}
        \item \emph{nicht}: Proxy, Gateway
    \end{itemize}
    \vspace{1cm}
    Endpunkte
    \begin{itemize}
        \item Azure: Dienste oder VMs in Azure
        \item Extern: \emph{externe} Dienste, per IP oder FQDN
        \item geschachtelt: kombination von Profilen für komplexe Anforderungen
    \end{itemize}
    \vspace*{\stretch{1}}
\end{flashcard}

\begin{flashcard}[Definition]{Traffic Manager Routing-Optionen}
    \vspace*{\stretch{1}}
    \begin{itemize}
        \item gewichtetes Routing
            \begin{itemize}
                \item gleichmäßig (ggf. gewichtete) Verteilung\newline
                    (von 1 - 1000)
                \item Auswahl zufällig anhand Gewichtung
            \end{itemize}
        \item leistungsbasiertes Routing
            \begin{itemize}
                \item möglich, wenn verschiedene geographische Standorte existierende
                \item Auswahl nach Internetlatenztabelle
            \end{itemize}
        \item geographisches Routing
            \begin{itemize}
                \item verbindung nach Geographieen (aber nicht Latenz)
                \item z.\,B. für Complience-Anforderungen
            \end{itemize}
    \end{itemize}
    \vspace*{\stretch{1}}
\end{flashcard}

\begin{flashcard}[Definition]{Traffic Manager Routing-Optionen 2}
    \vspace*{\stretch{1}}
    \begin{itemize}
        \item mehrere Endpunkte
            \begin{itemize}
                \item mehrere Endpunkte in einer DNS-Abfrage
                \item erhöht Verfügbarkeit und verringert Latenz im Fehlerfall
            \end{itemize}
        \item prioritätsbasiertes Routing
            \begin{itemize}
                \item Endpunkte sortiert nach Priorität
                \item es wird immer der mit der höchsten Priorität verwendet, der Verfügbar ist
            \end{itemize}
        \item Subnetz-Routing
            \begin{itemize}
                \item Endpunkte per IP-Adresse bestimmen
                \item Filtern z.\,  B. nach Nutzern, Anbietern, \ldots
            \end{itemize}
    \end{itemize}
    \vspace*{\stretch{1}}
\end{flashcard}

\begin{flashcard}[Definition]{Konfiguration von Traffic Manager}
    \vspace*{\stretch{1}}
    \begin{enumerate}
        \item Traffic Manager-Profil erstellen
        \item Endpunkt hinzufügen
    \end{enumerate}
    \vspace{1cm}
    Endpunkte können deaktiviert werden
    \vspace*{\stretch{1}}
\end{flashcard}

\begin{flashcard}[Definition]{Traffic Manager-Profil erstellen}
    \vspace*{\stretch{1}}
        \begin{itemize}
            \item Benötigte Parameter:
                \begin{itemize}
                    \item Name
                    \item eindeutiger DNS-Name\newline
                        DNS-Name: FQDN=name.trafficmanager.net
                    \item Abonnement
                    \item Resourcengruppe
                    \item Routingmethode: \emph{Performance}, \emph{Priority}
                    \item Standort
                \end{itemize}
            \item CLI: \texttt{az network traffic-manager profile create}\newline
                eindeutiger DNS-Name kann selbst vergeben werden
            \item Portal: Traffic Manager-Profil-Resource\newline
                Name ist gleichzeitig eindeutiger DNS-Name
        \end{itemize}
    \vspace*{\stretch{1}}
\end{flashcard}

\begin{flashcard}[Definition]{Traffic Manager Endpunkt erstellen}
    \vspace*{\stretch{1}}
    \begin{itemize}
        \item Benötigte Parameter:
            \begin{itemize}
                \item Resourcegruppe
                \item Profilname: das Traffic Manager-Profil
                \item Name
                \item Typ: Azure, Extern, geschachtelt
                \item Ziel-ID: z.\,B. IP-Adresse, Azure Dienst, \ldots
                \item Parameter für Routingmethode, z.\,B. Priorität, nichts für Performance
            \end{itemize}
        \item CLI: \texttt{az network traffic-manager endpoint create}
        \item Portal: \texttt{Einstellungen | Endpunkte | Hinzufügen} im Traffic-Manager-Blade
    \end{itemize}
    \vspace*{\stretch{1}}
\end{flashcard}


    \subsectioncard{Verbinden eines lokalen Netzwerks mit einem Azure VNET}

\subsubsectioncard{create and configure Azure VPN Gateway}

\begin{flashcard}[Definition]{VPN Gateways}
    \vspace*{\stretch{1}}
    Szenarios:
    \begin{itemize}
        \item Standord-zu-Standord (Site-to-Site) verbinden lokale Rechenzentren mit einem virtuellen Netzwerk in Azure
        \item Point-to-Site-Gateways verbinden ein einzelnes Gerät mit einem virtuellen Netzwerk in Azure
        \item Netzwerk-zu-Netzwerk-Verbindung zwischen zwei virtuellen Netzwerken in Azure
    \end{itemize}

    Maximal ein VPN-Gateway pro virtuellem Netzwerk

    \vspace*{\stretch{1}}
\end{flashcard}

\begin{flashcard}[Definition]{VPN Gateway Routing}
    \vspace*{\stretch{1}}
    Typen:
    \begin{itemize}
        \item richtlinienbasiert (statisches Routing)
        \item routenbasiert (dynamisches Routen)
    \end{itemize}

    \vspace{1cm}
    Wechsel nicht ohne Neu-Initialisierung möglich\newline
    $\Rightarrow$ IP-Adresse wechselt, bis 1 Stunde Einrichtungszeit

    \vspace*{\stretch{1}}
\end{flashcard}

\begin{flashcard}[Definition]{Richtlinienbasiertes VPN Gateway}
    \vspace*{\stretch{1}}
    \begin{itemize}
        \item nur IKEv1-Veschlüsselung
        \item statisches Routing über Routingtabelle und Adresspräfixen
        \item unterstützt gegebenfalls ältere lokale VPN-Hardware
    \end{itemize}
    \vspace*{\stretch{1}}
\end{flashcard}

\begin{flashcard}[Definition]{Routenbasiertes VPN Gateway}
    \vspace*{\stretch{1}}
    \begin{itemize}
        \item IKEv2
        \item Versand von Datenpaketen anhand von Netzwerkroutingtabellen
        \item $n$:$n$-Datenverkehrsselektoren
    \end{itemize}

    Scenarios:
    \begin{itemize}
        \item Netwerk-zu-Netzwerk-Verbindungen
        \item Point-to-Site
        \item mehrere Standorte
        \item parallele Nutzung mit ExpressRoute-Gateway
    \end{itemize}
    \vspace*{\stretch{1}}
\end{flashcard}

\begin{flashcard}[Definition]{Erstellen und Konfigurieren eines VPN Gateways}
    \vspace*{\stretch{1}}
    Benötigte Resourcen (in Azure!):
    \begin{itemize}
        \item Virtuelles Netzwerk\newline
            benötigt genügend Platz für zusätzliches Subnetzdes VPN Gateways
        \item Gateway Subnetz\newline
            groß genug, mindestens $/27$. Kann nicht für andere Zwecke verwendet werden
        \item öffentliche IP-Adresse: dynamisch, aber konstant solange VPN Gateway existiert
        \item Gateway im lokalen Netz (in Azure!)
        \item Gateway im virtuellen Netz (virtuelles Netzwerkgateway)
        \item Verbindung: von öffentlicher IP-Adresse mit IPv4-Adresse des lokalen Netzwerkgateways
    \end{itemize}

    Benötigte lokale Resourcen
    \begin{itemize}
        \item VPN-Gerät
        \item IPv4-Adresse (siehe lokales Netzwerkgateway)
    \end{itemize}
    \vspace*{\stretch{1}}
\end{flashcard}

\begin{flashcard}[Definition]{Konfiguration}
    \vspace*{\stretch{1}}
    \begin{itemize}
        \item Größenauswahl/SKU:
            \begin{itemize}
                \item Anzahl Tunnel, Durchsatz und BGP-Unterstützung
                \item wechsel von Basic in höherwertige SKUs erfordet erneute Bereitstellung
            \end{itemize}
        \item Hochverfügbarkeit:
            \begin{itemize}
                \item zwei VPN Gateways mit Active/Standby\newline
                    (automatisch aktiv)
                \item Active/Active (mit zwei öffentlichen IP-Adressen)\newline
                    (kann erweitert werden mit zweitem Lokalen VPN-Gerät)
                \item als Backup/Failover für ExpressRoute
                \item Zonenredundanz
            \end{itemize}
    \end{itemize}
    \vspace*{\stretch{1}}
\end{flashcard}

\begin{flashcard}[Definition]{Erstellen der Voraussetzungen}
    \vspace*{\stretch{1}}
    \begin{itemize}
        \item Erstellung eines VNETs:\newline
            \texttt{az network vnet create }
        \item Erstellen eines Subnets: (gateway subnet)\newline
            \texttt{az network vnet subnet create}
        \item Erstellen eines lokalen Gateways:\newline
            \texttt{az network local-gateway create}
        \item Erstellen einer öffentlichen IP:\newline
            \texttt{az network public-ip create}
    \end{itemize}

    Weiter: Erstellen eines VPN-Gateways
    \vspace*{\stretch{1}}
\end{flashcard}

\begin{flashcard}[Definition]{Erstellen eines VPN-Gateways}
    \vspace*{\stretch{1}}
    \begin{itemize}
        \item Erstellen eines VPN-Gateways: (langsam!)\newline
            \texttt{az network vnet-gateway create}
            \begin{itemize}
                \item \texttt{--public-ip-address PIP-VNG-HQ-Network}
                \item \texttt{--vnet HQ-Network}
                \item \texttt{--gateway-type Vpn}
                \item \texttt{--vpn-type RouteBased}
                \item \texttt{--sku VpnGw1}
            \end{itemize}
        \item Erstellen der VPN-Verbindung:\newline
            \texttt{az network vpn-connection create}
    \end{itemize}
    \vspace*{\stretch{1}}
\end{flashcard}

\subsubsectioncard{create and configure site to site VPN}

\subsubsectioncard{configure ExpressRoute}

\begin{flashcard}[Definition]{ExpressRoute-Voraussetzungen}
    \vspace*{\stretch{1}}
    \begin{itemize}
        \item Partner zum Einrichten der Verbindung
        \item Azure-Abonnement mit registriefung beim Partner
        \item Microsoft-Konto
        \item Optional: Office 365-Abonnement
        \item BGP-fähiger Router
        \item öffentliche IP-Adressen und NAT
        \item Weiterleitung des Datenverkehrs über Range /29 oder /30
    \end{itemize}
    \vspace*{\stretch{1}}
\end{flashcard}

\begin{flashcard}[Definition]{Express Route einrichten}
    \vspace*{\stretch{1}}
    Express Route ist eine direkte Verbindung! (Kein VPN)
    \begin{itemize}
        \item Verbindungstypen:
            \begin{itemize}
                \item Cloud Exchange: ISP, Server-Anbieter, etc verbinden
                \item Point-to-Point: lokales Rechenzenter verbinden
                \item Any-to-any: Einbinden ins WAN
            \end{itemize}
        \item Peering-Schema:
            \begin{itemize}
                \item Microsoft-Peering: für Office 365, \ldots
                \item privates Azure-Peering: für Azure services, vnets, \ldots
            \end{itemize}
    \end{itemize}
    \vspace*{\stretch{1}}
\end{flashcard}

\begin{flashcard}[Definition]{Express-Route-Anbieter}
    \vspace*{\stretch{1}}
    \begin{itemize}
        \item einer der verfügbaren Anbieter, mit Registrierung
        \item benötigte Informationen:
            \begin{itemize}
                \item Name
                \item Peeringstandorte: vom Anbieter unterstützte Orte
                \item Bandbreite
            \end{itemize}
        \item Abrufen der Informationen:
            \begin{itemize}
                \item CLI:\newline
                    \texttt{az network express-route list-service-providers}
                \item PowerShell:\newline
                    \texttt{Get-AzExpressRouteServiceProvider}
            \end{itemize}
    \end{itemize}
    \vspace*{\stretch{1}}
\end{flashcard}

\begin{flashcard}[Definition]{Express-Route im Portal erstellen}
    \vspace*{\stretch{1}}
    \begin{itemize}
        \item Resourcen-Definition
            \begin{itemize}
                \item Abonnement + Resourcengruppe
                \item Name: name ohne Leer- und Sonderzeichen
            \end{itemize}
        \item Anbieter
        \item SKU: Lokal, Standard, Premium: Premium für > 10 virtuelle Netzwerke
        \item Abrechnungsmodell: getaktet oder unbegrenzt\newline
            eingehende Datenübertragungen sind immer kostenlos
        \item Standort: Standord (in Azure) für die Leitung
    \end{itemize}
    Sobald die Leitung in Azure verfügbar ist, muss der \emph{Dienstschlüssel} an den Anbieter gesendet werden

    \vspace{1cm}
    Zum aktivieren (wenn alles erledigt ist), muss ein \emph{Gateway für virtuelle Netze} mit der Option \emph{ExpressRoute} erstellt werden
    \vspace*{\stretch{1}}
\end{flashcard}

\begin{flashcard}[Definition]{Privates ExpressRoute-Peering konfigurieren}
    \vspace*{\stretch{1}}
    \begin{itemize}
        \item Peer-ASN: private oder öffentliche autonome Systemnummer
        \item primäres Subnetz: ein /30-Subnetz, IP\#1 für router, IP\#2 für Microsoft-Router\newline
            Die anderen zwei?
        \item sekundäres Subnetz: ein /30-Subnetz, aus sicherheitsgründen
        \item VLAN-ID: VLAN in dem das Peering hergestellt wird
        \item Schlüssel: optionaler MD5-Hash für Codierung von Nachrichten
    \end{itemize}
    \vspace*{\stretch{1}}
\end{flashcard}

\begin{flashcard}[Definition]{Microsoft ExpressRoute-Peering konfigurieren}
    \vspace*{\stretch{1}}
    \begin{itemize}
        \item gleiche Daten wie für privates Peering: \newline
            (Peer-ASN, primäres Subnetz, sekundäres Subnetz, VLAN-ID, Schlüssel)
        \item Angekündigte öffentliche Präfixe: für BGP verwendete Adresspräfixe
        \item Kunden-ASN: optionale autonome Systemnummer
        \item Name: Registrierungsname für Kunden-ASN
    \end{itemize}
    \vspace*{\stretch{1}}
\end{flashcard}

\begin{flashcard}[Definition]{Vergleich VPN Gateway und ExpressRoute}
    \vspace*{\stretch{1}}
    \begin{tabular}{l|lll}
        Funktion       &  VPN Gateway               & ExpressRoute                    \\
        \hline
        Azure-Dienste  &  Azure-Dienste \& VMs      & Microsoft Cloud                 \\
        Bandbreite     &  10 GBit/s                 & 10 GBit/s, 100 GBit/S           \\
        Protokoll      &  SSTP, IPsec               & VLAN oder MPLS                  \\
        Routing        &  Statisch, dynamisch       & BGP (Border Gateway)            \\
        Resilienz      &  Aktiv-passiv, aktiv-aktiv & Aktiv-passiv, aktiv-aktiv       \\
        Use-Cases      & Prototyp, Dev              & Unternehmenskritsiche Workloads \\
        SLA            & 99,95\% -- 99,99\%         & 99,95\%                         \\
    \end{tabular}
    \vspace*{\stretch{1}}
\end{flashcard}

\subsubsectioncard{configure Virtual WAN}

\begin{flashcard}[Definition]{Virtuelles WAN}
    \vspace*{\stretch{1}}
    Azure-Dienst, der alle (viele) Netzwerkfunktionen in einer Oberfläche vereint
    \begin{itemize}
        \item virtuelle Netzwerke
        \item VPN Gateways
        \item Lastenausgleich
        \item ExpressRoute
        \item Site-2-Site
        \item Point-2-Site
    \end{itemize}
    \vspace*{\stretch{1}}
\end{flashcard}

\subsubsectioncard{verify on premises connectivity}

\subsubsectioncard{troubleshoot on premises connectivity with Azure}


    \subsectioncard{Mehrstufige Authentifikation einrichten}

\begin{flashcard}[Definition]{Mehrstufige Authentifikation}
    \vspace*{\stretch{1}}
    \begin{itemize}
        \item Mindestens zwei Methoden zum Login
            \begin{itemize}
                \item Passwort (immer)
                \item eine zweite Methode
            \end{itemize}
        \item Für Administratoren verpflichtend
        \item Kann aus, für einige Gruppen, für alle Mitglieder im Active Directory an sein
    \end{itemize}
    \vspace{1cm}
    Möglichkeit, mit Legacy-Login MFA zu umgehen!
    \vspace*{\stretch{1}}
\end{flashcard}

\subsubsectioncard{configure user accounts for MFA}

\begin{flashcard}[Definition]{Zeitraum zwischen Abfragen des zweiten Faktors}
    \vspace*{\stretch{1}}
    \begin{itemize}
        \item nicht jeder Authentifizierungs-Versuch muss mit zwei Methoden abgesichert sein
        \item Caching-Regel erlaubt, eine Periode, in der es nicht notwendig ist, anzugeben
    \end{itemize}
    \vspace*{\stretch{1}}
\end{flashcard}

\subsubsectioncard{configure fraud alerts}

\begin{flashcard}[Definition]{Betrugs-Benachrichtigung}
    \vspace*{\stretch{1}}
    Benutzer können betrügerische Anmeldeverusche melden
    \begin{itemize}
        \item Betrugsversuch:\newline
            falls eine MFA-Aufforderung ohne Anmeldeversuch eingeht
        \item Automatische Blockierung:\newline
            Sperrung von Benutzerkonten, die Betrug melden für 90 Tage oder \emph{bis ein Administrator die Sperre aufhebt}
        \item Code zur Meldung über Telefonanrufe (standard 0)
    \end{itemize}
    \vspace*{\stretch{1}}
\end{flashcard}

\subsubsectioncard{configure bypass methods}

\begin{flashcard}[Definition]{Organisations-Mitglieder}
    \vspace*{\stretch{1}}
    \begin{itemize}
        \item Nutzer innerhalb der Organisation können MFA überspringen
        \item benötigt einen Nachweis durch Active Directory Federation Services (AD FS)
        \item bestimmte Ip-Bereiche können übersprungen werden (Trusted IP)
    \end{itemize}
    \vspace*{\stretch{1}}
\end{flashcard}

\begin{flashcard}[Definition]{One-time bypass}
    \vspace*{\stretch{1}}
    \begin{itemize}
        \item Für eine einmalige Anmeldung wird der zweite Faktor deaktiviert
        \item Bypass ist zeitlich begrenzt für einige Sekunden
    \end{itemize}
    \vspace*{\stretch{1}}
\end{flashcard}

\subsubsectioncard{configure Trusted IPs}

\begin{flashcard}[Definition]{Trusted IPs}
    \vspace*{\stretch{1}}
    \begin{itemize}
        \item IP-Bereiche, für die keine mehrstufige Authentifizierung notwendig ist
        \item[!] für das intranet der Organisation\newline
            für generelle Standort-Freigate: Bedingter Zugriff
        \item Bis zu 50 CIDR-Bereiche
    \end{itemize}
    Zwei Möglichkeiten:
    \begin{itemize}
        \item Einstellung für mehrstufige Authentifizierung:\newline
            \texttt{Azure Active Directory | Benutzer | Multi-Factor Authentication | service Settings | Trusted IPs}
        \item Über Bedingten Zugriff:\newline
            \texttt{Azure Active Directory | Sicherheit | Conditional Access | Named Locations}
    \end{itemize}

    \vspace*{\stretch{1}}
\end{flashcard}

\subsubsectioncard{configure verification methods}

\begin{flashcard}[Definition]{Authentifizierungsmethoden}
    \vspace*{\stretch{1}}
    \begin{itemize}
        \item Passwort
        \item Textnachricht
        \item (Microsoft-)App
        \item Sicherheitsfragen
        \item FIDO
        \item Biometrie
    \end{itemize}
    \vspace{1cm}
    ! Sicherheitsfragen können nur in der Self-Service-Kennwortrücksetzung verwendet werden
    \vspace*{\stretch{1}}
\end{flashcard}



    \subsectioncard{Verwalten von rollenbasierter Zugrangsbeschränkung}

\subsubsectioncard{create a custom role}

\begin{flashcard}[Definition]{Zugriffsbeschränkung}
    \vspace*{\stretch{1}}
    \begin{itemize}
        \item Benötigt:
            \begin{itemize}
                \item Abonnement
                \item Name
                \item Liste der erlaubten+verbotenen Aktionen
                \item mögliche Anwendungsbereiche
            \end{itemize}
        \item Portal: Im \texttt{Subscription}-Blade
        \item CLI: \texttt{az role definition create}
        \item PowerShell: \texttt{New-AzRoleDefinition}
    \end{itemize}
    \vspace*{\stretch{1}}
\end{flashcard}

\subsubsectioncard{configure access to Azure resources by assigning roles}

\begin{flashcard}[Definition]{Zugriffsbeschränkung}
    \vspace*{\stretch{1}}
    \begin{itemize}
        \item einzelne Zugriffe können beschränkt werden
        \item Beschränkungen haben höheren Rang als RBAC-Rollen-Erlaubnisse
    \end{itemize}
    \vspace{1cm}
    z\,B. Mitarbeiter, der VMs nicht runterfahren darf
    \vspace*{\stretch{1}}
\end{flashcard}

\begin{flashcard}[Definition]{Besitzer-Rolle}
    \vspace*{\stretch{1}}
    \begin{itemize}
        \item hat vollständigen Zugriff auf Resourcen
        \item kann weitere Rollendefinitionen setzen
    \end{itemize}
    \vspace*{\stretch{1}}
\end{flashcard}

\begin{flashcard}[Definition]{Mitarbeiter-Rolle}
    \vspace*{\stretch{1}}
    \begin{itemize}
        \item hat vollständigen Zugriff auf Resourcen
    \end{itemize}
    $\Rightarrow$ kann nicht weitere Nutzer hinzufügen/entfernen
    \vspace*{\stretch{1}}
\end{flashcard}

\begin{flashcard}[Definition]{Leser-Rolle}
    \vspace*{\stretch{1}}
    \begin{itemize}
        \item hat Lesezugriff auf alle resourcen
        \item \emph{keine} Nutzerverwaltung möglich
        \item \emph{keine} Veränderungen von Resourcen möglich
    \end{itemize}
    \vspace*{\stretch{1}}
\end{flashcard}

\begin{flashcard}[Definition]{Nutzerzugriff-Adminstrator-Rolle}
    \vspace*{\stretch{1}}
    \begin{itemize}
        \item kann Rollen für Azure-Resourcen vergeben
        \item \emph{keine} Veränderungen von Resourcen mölgich
    \end{itemize}
    $\Rightarrow$ Besitzer = Mitarbeiter + Nutzerzugriff-Administrator
    \vspace*{\stretch{1}}
\end{flashcard}

\begin{flashcard}[Definition]{Backups}
    \vspace*{\stretch{1}}
    \begin{itemize}
        \item Sicherungsoperator: durchführen von Sicherungen, aber \emph{nicht löschen}, Tresorerstellung oder anderen Rechte zuweisen
        \item Sicherungsmitwirkender: komplette Verwaltung von Sicherungen, inklusive löschen, keine Tresorerstellung und Rechtezuweisung
    \end{itemize}
    \vspace*{\stretch{1}}
\end{flashcard}

\begin{flashcard}[Definition]{Site Recovery}
    \vspace*{\stretch{1}}
    \begin{itemize}
        \item Site-Recovery-Leser: Anzeige des Status
        \item Site-Recovery-Operator: durchführen von Failover und Failback
        \item Site-Recovery-Mitwirkender: verwaltung von Site Recovery, aber keine Tresorerstellung
    \end{itemize}
    \vspace*{\stretch{1}}
\end{flashcard}


\begin{flashcard}[Definition]{Administratoren}
    \vspace*{\stretch{1}}
    \begin{itemize}
        \item Password Administrator: darf Passwörter ändern (von normalen Nutzern)
        \item Nutzer-Administrator: Nutzer verwalten (erstellen, \ldots)
        \item Nutzerzugriff-Administrator: darf Rollen für Nutzer ändern
        \item Helpdesk-Administrator: kann verschiedene Service-Aufträge erledigen (auch Paswörter)
        \item Sicherheits-Adminsotrator: viele Rechte für Sicherheitsaspekte, aber keine Nutzerrechte
        \item Global Administrator: hat alle Rechte
    \end{itemize}
    \vspace*{\stretch{1}}
\end{flashcard}

\subsubsectioncard{configure management access to Azure}

\subsubsectioncard{troubleshoot RBAC}

\subsubsectioncard{implement Azure Policies}

\subsubsectioncard{assign RBAC roles}

\begin{flashcard}[Definition]{RBAC-Rollen zuweisen}
    \vspace*{\stretch{1}}
    \begin{itemize}
        \item Im Portal:\newline
            \item eine Resource auswählen
            \item \texttt{Zugriffssteuerung (IAM)}
            \item \texttt{+ Hinzufügen}
        \item CLI:\newline
            \texttt{az role assignment create --role \ldots\\--assignee <name|email-adress|object-id> --scope}
    \end{itemize}
    \vspace*{\stretch{1}}
\end{flashcard}

\begin{flashcard}[Definition]{RBAC-Rollen entfernen}
    \vspace*{\stretch{1}}
    \begin{itemize}
        \item Im Portal:\newline
            \item eine Resource auswählen
            \item \texttt{Zugriffssteuerung (IAM)}
        \item CLI:\newline
            \texttt{az role assignment delete --role \ldots\\--assignee <name|email-adress|object-id> --scope}
        \item PowerShell:\newline
            \texttt{Remove-AzRoleAssignment -SignInName \ldots\ -RoleDefinitionName \ldots\ -Scope \ldots}
    \end{itemize}
    \vspace*{\stretch{1}}
\end{flashcard}


    \sectioncard{Apps erstellen und bereitstellen}

    \subsectioncard{Erstellen von Webanwendungen und PaaS}

\subsubsectioncard{create an Azzure app service Web App}

\begin{flashcard}[Definition]{Was wird für eine Web App benötigt?}
    \vspace*{\stretch{1}}
    \begin{enumerate}
        \item App Service Plan
            \begin{itemize}
                \item App Service Plan: dauerhafte Bezahlung
                \item Consumption Plan: bezahlung nach Nutzen, maximale Laufzeit einer Funktion 10 Minuten
            \end{itemize}
        \item Web App
    \end{enumerate}
    \vspace*{\stretch{1}}
\end{flashcard}



\begin{flashcard}[Definition]{App Service Plan erstellen}
    \vspace*{\stretch{1}}
    \begin{itemize}
        \item Benötigt:
        \begin{itemize}
            \item Resourcegruppe + Location
            \item SKU
            \item Name
        \end{itemize}
        \item CLI: \texttt{az appservice plan create --resource-group \ldots\ --location \ldots --SKU \ldots\ --number-of-workers \ldots}
            \begin{itemize}
                \item \texttt{--is-linux} für Linux-Maschinen
            \end{itemize}
    \end{itemize}
    \vspace*{\stretch{1}}
\end{flashcard}

\begin{flashcard}[Definition]{App Service Plan-Tarife}
    \vspace*{\stretch{1}}
    Aufteilung basierend auf Rechen-Ressourcen:
    \begin{itemize}
        \item Freigegebene Ressourcen
            \begin{itemize}
                \item Free (F1)
                \item Shared (D1)
            \end{itemize}
        \item Dedizierte Ressourcen (1, 2, 4 CPUs)
            \begin{itemize}
                \item Basic (B1, B2, B3)
                \item Standard (S1, S2, S3)
                \item Premium
                \item Premium v2 (P1v2, P2v2, P3v3)
            \end{itemize}
        \item Isoliert (I1, I2, I3)
    \end{itemize}
    \vspace*{\stretch{1}}
\end{flashcard}

\begin{flashcard}[Definition]{App Service Plan-Features Free/Shared}
    \vspace*{\stretch{1}}
    \begin{itemize}
        \item nur für Entwicklung und Tests
        \item Ausführung auf Maschinen, die auch andere Azure-Nutzer verwenden
    \end{itemize}
    Eigenschaften:
    \begin{itemize}
        \item 10/100 Apps
        \item 1 G(i?)B Festplattenspeicher
        \item Ab \emph{Shared}: benutzerdefinierte Domäne
        \item Nur Free: kostenlos
    \end{itemize}
    \vspace*{\stretch{1}}
\end{flashcard}

\begin{flashcard}[Definition]{App Service Plan-Features Basic/Standard/Premium/Premium v2}
    \vspace*{\stretch{1}}
    \begin{itemize}
        \item Nur Apps desselben Plans können sich eine VM teilen
        \item bessere Stufe $\Rightarrow$ höhere Anzahl Instanzen für horizontale Skalierung verfügbar
    \end{itemize}
    Eigenschaften: (Über Pläne für freigegebene Resourcen hinaus)
    \begin{itemize}
        \item Zertifikate\vspace{-0.5mm}
        \item Unbegrenzte App-Anzahl\vspace{-0.5mm}
        \item Hybrid-Konnektivität\vspace{-0.5mm}
        \item SLA: 99,95 \%\vspace{-0.5mm}
        \item Skalierbarkeit: 3/10/30 Instanzen\vspace{-0.5mm}
        \item Ab \emph{Standard}:\vspace{-1mm}
            \begin{itemize}
                \item automatische Skalierung\vspace{-1mm}
                \item integration virtueller Netzwerke\vspace{-1mm}
                \item 5 Staging-Slots, dann 20\vspace{-1mm}
                \item Backups\vspace{-1mm}
            \end{itemize}
    \end{itemize}
    \vspace*{\stretch{1}}
\end{flashcard}

\begin{flashcard}[Definition]{App Service Plan-Features Isoliert}
    \vspace*{\stretch{1}}
    \begin{itemize}
        \item Netzwerkisolation: die dedizierten VMs werden in eigneen virtuellen Netzen ausgeführt\newline
            (zusätzlich zu Isolation für Berechnungen (wie dedizierte Ressourcen))
    \end{itemize}
    \vspace*{\stretch{1}}
\end{flashcard}

\begin{flashcard}[Definition]{Web App erstellen}
    \vspace*{\stretch{1}}
    \begin{itemize}
        \item Benötigt:
        \begin{itemize}
            \item Resourcegruppe
            \item Name
            \item App-Service-Plan
            \item deployment source (URL, Branch)
        \end{itemize}
        \item CLI: \texttt{az webapp create --name \ldots --resource-group}
    \end{itemize}
    \vspace*{\stretch{1}}
\end{flashcard}

\subsubsectioncard{create documentation for the API}

\subsubsectioncard{create an App Service Web App for Containers}

\begin{flashcard}[Definition]{Web App erstellen}
    \vspace*{\stretch{1}}
    \begin{itemize}
        \item Für Linux mit Docker-Image von DockerHub:
        \item CLI: \texttt{az webapp create --name \ldots\ --resource-group \ldots\ \\--deployment-container-image-name}
    \end{itemize}
    \vspace*{\stretch{1}}
\end{flashcard}

\subsubsectioncard{create an App Service background task by using WebJobs}

\begin{flashcard}[Definition]{WebJobs}
    \vspace*{\stretch{1}}
    \begin{itemize}
        \item Teil von Azure App Service
        \item Arten:
            \begin{itemize}
                \item Fortlaufend: werden in einer Schleife ausgeführt
                \begin{itemize}
                    \item Multi-Instance: Script auf jeder Instanz ausführen (Ja/Nein).\newline
                        (Nicht bei freigegebenen Resourcen (Free, Shared), da nicht horizontal skalierbar)
                \end{itemize}
                \item Ausgelöst: werden nach Zeitplan oder manuell ausgeführt
            \end{itemize}
    \end{itemize}
    \vspace*{\stretch{1}}
\end{flashcard}

\begin{flashcard}[Definition]{Vergleich Funktionen/WebJob}
    \vspace*{\stretch{1}}
    Features von Funktionen:
    \begin{itemize}
        \item automatische Skalierung
        \item Entwicklung im Browser
        \item Bezahlung nach Nutzung
        \item Integration in Logik-Apps
    \end{itemize}
    \vspace*{\stretch{1}}
\end{flashcard}


\subsubsectioncard{enable diagnostics logging}

\begin{flashcard}[Definition]{Diagnoseprotokollierung}
    \vspace*{\stretch{1}}
    \begin{itemize}
        \item automatisch verfügbares Logging zum Debuggen von Web Service-Anwendungen
        \item Aktivieren:\newline
            \texttt{App Service-Protokolle} im App-Blade
        \item Daten können im Speicher der Anwendung oder als Blob gespeichert werden
    \end{itemize}
    \vspace*{\stretch{1}}
\end{flashcard}

\begin{flashcard}[Definition]{Arten von Web Service-Logging}
    \vspace*{\stretch{1}}
    \begin{itemize}
        \item Anwendungsprotokollierung
        \item Webserverprotokollierung
        \item Detaillierte Fehlermeldungen
        \item Ablaufverfolgung fehlgeschlagener Anforderungen
        \item Bereitstellungsprotokollierung
    \end{itemize}
    \vspace*{\stretch{1}}
\end{flashcard}

\begin{flashcard}[Definition]{Web Service Anwendungsprotokollierung}
    \vspace*{\stretch{1}}
    \begin{itemize}
        \item Windows + Linux
        \item normale Anwendungslogs
        \item Webframework und Anwendungscode
        \item Levels Kritisch bis Trace
    \end{itemize}
    \vspace*{\stretch{1}}
\end{flashcard}

\begin{flashcard}[Definition]{Web Service Webserverprotokollierung}
    \vspace*{\stretch{1}}
    \begin{itemize}
        \item[!] nur Windows
        \item protokollierung von HTTP-Anforderungen
        \item geeignet für Metriken
    \end{itemize}
    \vspace*{\stretch{1}}
\end{flashcard}

\begin{flashcard}[Definition]{Detaillierte Web Service Fehlermeldungen}
    \vspace*{\stretch{1}}
    \begin{itemize}
        \item[!] nur Windows
        \item Kopie der Fehlerseiten
        \item ab 400 aufwärts
    \end{itemize}
    \vspace*{\stretch{1}}
\end{flashcard}

\begin{flashcard}[Definition]{Ablaufverfolgung fehlgeschlagener Web Service Anforderungen}
    \vspace*{\stretch{1}}
    \begin{itemize}
        \item[!] nur Windows
        \item detaillierte Verfolgung im Fehlerfall
        \item inklusive IIS
        \item enthält genaue Timings für jede Komponente
        \item praktisch, um Seitenperformance nachzuvollziehen
        \item genaue analyse eines speziellen HTTP-Fehlers
    \end{itemize}
    \vspace*{\stretch{1}}
\end{flashcard}

\begin{flashcard}[Definition]{Web Service Bereitstellungsprotokollierung}
    \vspace*{\stretch{1}}
    \begin{itemize}
        \item Windows + Linux
        \item automatische Protokollierung der Bereitstellung
    \end{itemize}
    \vspace*{\stretch{1}}
\end{flashcard}

\begin{flashcard}[Definition]{Backup}
    \vspace*{\stretch{1}}
    \begin{itemize}
        \item ein Backup kann erstellt und wieder eingespielt werden
        \item ab Standard-App Service Plan
    \end{itemize}
    \vspace*{\stretch{1}}
\end{flashcard}

\begin{flashcard}[Definition]{Snapshots}
    \vspace*{\stretch{1}}
    \begin{itemize}
        \item Snapshots ab Premium
        \item kann nur in die gleichen Slots zurückgespielt werden
    \end{itemize}
    \vspace*{\stretch{1}}
\end{flashcard}


    \subsectioncard{Design und Entwicklung von Apps in Containern}

\subsubsectioncard{configure diagnostic settings on resources}

\subsubsectioncard{create a container image by using a Dockerfile}

\begin{flashcard}[Definition]{Docker-Images mit Dockerfile}
    \vspace*{\stretch{1}}
    \begin{itemize}
        \item manuelle Anpassungen eines Docker-Images
        \item Datei namens \emph{Dockerfile}
        \item Teile eines Dockerfiles:
            \begin{itemize}
                \item \texttt{FROM}: Basisimage, das angepasst word
                \item \texttt{RUN}: beliebiger Befehl, der im Container ausgeführt wird
                \item \texttt{COPY}: kopiert Dateien vom Hostcomputer
                \item \texttt{EXPOSE}: öffnet einen Port
                \item \texttt{ENTRYPOINT}: was im Container ausgeführt wird
            \end{itemize}
        \item mit \texttt{docker build} wird ein neues Image gebaut
    \end{itemize}
    \vspace*{\stretch{1}}
\end{flashcard}

\subsubsectioncard{creat an Azure Kubernetes Service}

\begin{flashcard}[Definition]{Kobernetes-Komponenten}
    \vspace*{\stretch{1}}
    \begin{itemize}
        \item kube-apiserver\newline
            stellt Kubernetes-API bereit
        \item etcd\newline
            Key-Value-Speicher in Kubernetes mit Statusdaten
        \item kube-scheduler\newline
            weist Knoten zu
        \item kube-advisor\newline
            gives best practice hints
    \end{itemize}
    \vspace*{\stretch{1}}
\end{flashcard}

\begin{flashcard}[Definition]{Azure Kubernets-Cluster erstellen}
    \vspace*{\stretch{1}}
    \begin{itemize}
        \item Benötigt:
            \begin{itemize}
                \item Resourcegruppe + Abonnement
                \item Name
            \end{itemize}
        \item CLI:\newline
            \texttt{az aks create --resource-group \ldots\ --name \ldots}
    \end{itemize}
    \vspace*{\stretch{1}}
\end{flashcard}

\begin{flashcard}[Definition]{Eigenschaften AKS}
    \vspace*{\stretch{1}}
    \begin{itemize}
        \item Master von Azure bereitgestellt
        \item Knotenzahl konfigurierbar
        \item[!] Knotengröße nicht mehr änderbar
        \item HTTP-Routing
        \item SSH-Zugriff
    \end{itemize}
    \vspace*{\stretch{1}}
\end{flashcard}


\subsubsectioncard{publish an image to the Azure Container Registry}

\begin{flashcard}[Definition]{Azure Containerregistrierung}
    \vspace*{\stretch{1}}
    \begin{itemize}
        \item In Azure bereitgestellte Registrierung von Docker-Containern\newline
            Analog zu Docker Hub
        \item Benötigt:
            \begin{itemize}
                \item Name: global eindeutig
            \end{itemize}
        \item Erstellen:
            \begin{itemize}
                \item Portal: \texttt{+ Ressource erstellen | Container | Container Registry}
                \item CLI: \texttt{az acr create --name \ldots}
            \end{itemize}
        \item Administratorbenutzer für Zugriffsschlüssel
        \item[!] eine private Registrierunt, anonymer Zugriff ist nicht möglich
    \end{itemize}
    \vspace*{\stretch{1}}
\end{flashcard}

\begin{flashcard}[Definition]{Ein Image hochladen}
    \vspace*{\stretch{1}}
    \begin{itemize}
        \item \texttt{docker login myregistry.azurecr.io}\newline
            Credentials aus CLI oder Portal abrufen
        \item \texttt{docker tag myapp:v1 myregistry.azurecr.io/myapp:v1}
        \item \texttt{docker push myregistry.azurecr.io/myapp:v1}
    \end{itemize}
    \vspace*{\stretch{1}}
\end{flashcard}

\begin{flashcard}[Definition]{Ein Containerimage erstellen}
    \vspace*{\stretch{1}}
    \begin{itemize}
        \item Voraussetzung: \texttt{Dockerfile}
        \item \texttt{az acr build --registry \ldots --image name:version <path-to-dockerfile>}\newline
            Alternativ \texttt{--file} für die genaue Docker-Datei
        \item Das Image ist anschließend in der Registrierung gespeichert
    \end{itemize}
    \vspace*{\stretch{1}}
\end{flashcard}

\begin{flashcard}[Definition]{Container in der Registrierung}
    \vspace*{\stretch{1}}
    \begin{itemize}
        \item \texttt{az acr repository list --name registry-name}
    \end{itemize}
    \vspace*{\stretch{1}}
\end{flashcard}

\begin{flashcard}[Definition]{Container in der Registrierung replizieren}
    \vspace*{\stretch{1}}
    \begin{itemize}
        \item um Container in anderer Region zu speichern (zwecks schnellem Zugriff)
        \item \texttt{az acr replication create --registry \ldots --location <target>}
    \end{itemize}
    \vspace*{\stretch{1}}
\end{flashcard}

\subsubsectioncard{implement an application that runs on an Azure Container Instance}

\begin{flashcard}[Definition]{Container-Instanz starten}
    \vspace*{\stretch{1}}
    \begin{itemize}
        \item Benötigt:
            \begin{itemize}
                \item Abonnement + Ressourcengruppe
                \item Containername
                \item Quelle: Azure Containerregistrierung (oder Docker Hub, Schnellstart)
                \item Image und Tag
            \end{itemize}
        \item Portal: \texttt{+ Ressoruce Erstellen | Container | Container Instance}
    \end{itemize}
    \vspace*{\stretch{1}}
\end{flashcard}

\subsubsectioncard{manage container settings by using code}


    \sectioncard{Authentifikation und Sicherung von Daten}

    \subsectioncard{Authentifizierung einrichten}

\subsubsectioncard{implement authentication by using certificates, forms-based authentication, tokens, or Windows-integrated authentication}

\begin{flashcard}[Definition]{Authentifizierung}
    \vspace*{\stretch{1}}
    \begin{itemize}
        \item Feststellung der Identität einer/s
        \begin{itemize}
            \item Person
            \item Geräts
            \item Services
        \end{itemize}
        die/der zugriff auf eine Resource haben möchte.
    \end{itemize}

    Dabei muss die zu authentifizierende Seite gültige Zugriffsrechte nachweisen.
    \vspace*{\stretch{1}}
\end{flashcard}

\subsubsectioncard{implement multi-factor authentication by using Azure AD}

\subsubsectioncard{implement OAuth2 authentication}

\begin{flashcard}[Definition]{OAuth 2.0 Login}
    \vspace*{\stretch{1}}
    Anfrage eines OAuth 2.0-Tokens
    \begin{itemize}
        \item Authorisierungs-Code
        \item \texttt{client\_id}
    \end{itemize}
    \vspace*{\stretch{1}}
\end{flashcard}

\subsubsectioncard{implement Managed Identities for Azure resources Service Principal authentication}


    \subsectioncard{Implementieren von sicheren Lösungen}

\subsubsectioncard{encrypt and decrypt data at rest and in transit}

\begin{flashcard}[Definition]{Datenverschlüsselung}
    \vspace*{\stretch{1}}
    Ruhende Daten:
    \begin{itemize}
        \item symmetrische Verschlüsselung von gespeicherten Daten (Storage Account, \ldots)
        \item gestohlene Daten sind nutzlos
        \item Schlüssel wird in Key Vault gespeichert
        \item Azure-Verwaltete Schlüssel, oder Kundenschlüssel importieren
        \item Aktivieren nach Modell:
            \begin{itemize}
                \item SaaS: aktivieren in der Software
                \item Paas: Speicher (BLob, \ldots) kann verschlüsselt werden
                \item IaaS: Azure Disk Encryption für VHDs und VMs, und Speicher\newline
                    Standardmäßig verschlüsselt
            \end{itemize}
    \end{itemize}
    \vspace*{\stretch{1}}
\end{flashcard}

\begin{flashcard}[Definition]{Verschlüsselung Speicherkonto}
    \vspace*{\stretch{1}}
    \begin{itemize}
        \item Daten im Speicherkonto sind standardmäßig verschlüsselt
        \item Microsoft verwaltet Schlüssel standardmäßig (in Microsoft Key Store)
        \item Benutzer-Verwaltet: benötigt Azure Key Vault.\newline
            Nur Blob und Azure Files
        \item vom Benutzer bereitgestellt: eigener Key Store des nutzers\newline
            Nur Blob!
    \end{itemize}
    \vspace*{\stretch{1}}
\end{flashcard}

\begin{flashcard}[Definition]{Verschlüsselung in Transit}
    \vspace*{\stretch{1}}
    \begin{itemize}
        \item Zugriff auf Endpunkte nur verschlüsselt
        \item HTTPS-Protokoll
        \item aktivieren in Eigenschaften von Speicherkonto
    \end{itemize}
    \vspace*{\stretch{1}}
\end{flashcard}

\begin{flashcard}[Definition]{Transparent Data Encryption}
    \vspace*{\stretch{1}}
    \begin{itemize}
        \item Data at rest-Verschlüsselung von SQL-Datenbanken
        \item schützt vor Offline-Angriffen
        \item verschlüsselt die Datenbank, Backups, und Transaction logs
        \item[$\Rightarrow$] transparent für Anwendungen
        \item standardmäßig aktiviert
    \end{itemize}
    \vspace{1cm}
    Im Unterschied zu Always Encrypted kennt die Datenbank (bzw. der Azure Service) den Schlüssel (DEK)
    \vspace*{\stretch{1}}
\end{flashcard}

\subsubsectioncard{encrypt data with Always Encoded}

\begin{flashcard}[Definition]{Always Encrypted}
    \vspace*{\stretch{1}}
    \begin{itemize}
        \item Verschlüsselung in Transit
        \item Verschlüsselung von Daten in SQL-Datenbank
        \item Entschlüsselung der Daten erst in Client-Anwendung
        \item Durchführung:
            \begin{itemize}
             \item PowerShell/T-SQL (nicht vollständig unterstützt)
            \end{itemize}
    \end{itemize}
    \vspace*{\stretch{1}}
\end{flashcard}

\begin{flashcard}[Definition]{Always Encrypted und Windows Certificate Store}
    \vspace*{\stretch{1}}
    Speichert den Schlüssel auf der lokalen Maschine

    \vspace{1cm}
    $\Rightarrow$ nur in der Anwendung sind Daten verfügbar
    \vspace*{\stretch{1}}
\end{flashcard}

\subsubsectioncard{implement Azure Confidential Compute}

\begin{flashcard}[Definition]{Azure Confidential Computing}
    \vspace*{\stretch{1}}
    \begin{itemize}
        \item geschützte Verarbeitung von Daten während der Berechnung\newline
            (während Daten normalerweise nach Lesen vom Ruhezustand entschlüsselt sind)
        \item benötigt Trusted Execution Environment (TEE)
        \item Code wird in geschütztem Container ausgeführt
        \item benötigt eine VM mit Intel XGX
    \end{itemize}
    \vspace*{\stretch{1}}
\end{flashcard}

\subsubsectioncard{implement SSL/TLS communications}

\begin{flashcard}[Definition]{SSL/TLS-Verbindungen}
    \vspace*{\stretch{1}}
    \begin{itemize}
        \item Speicherkonto: in Einstellungen aktivieren
        \item Andere Endpunkte: aktivieren
        \item[$\Rightarrow$] muss erzwungen werden
        \item VMs: können Schlüsselpaar erstellen bei der Erstellung, oder eigenen Schlüssel
    \end{itemize}
    \vspace*{\stretch{1}}
\end{flashcard}

\subsubsectioncard{create, read, update, and delete keys, secrets, and certificates by using the KeyVault API}

\begin{flashcard}[Definition]{Key Vault}
    \vspace*{\stretch{1}}
    \begin{itemize}
        \item Speicher für verschlüsselte Daten
        \item Hardware (FIPS-2) oder Software
        \item Signierungen werden in sicherem Bereich durchgeführt, Schlüssel verlässt den Tresor nicht
        \item Verwaltung von Zertifikaten, inklusive Verlängerung
        \item Integration in Azure-Dienste: App Service, Virtuelle Maschinen, \ldots
    \end{itemize}
    \vspace*{\stretch{1}}
\end{flashcard}

\begin{flashcard}[Definition]{Key Vault-Zugriff}
    \vspace*{\stretch{1}}
    \begin{itemize}
        \item Geheimnis-URL:\newline
            \texttt{https://vault-name.vault.azure.net/secrets/secret-name}\newline
        \item Schlüssel URL: (POST)\newline
            \texttt{https://vault-name.vault.azure.net/keys/key-name}\newline
        \item Zertifikat-URL:\newline
            \texttt{https://vault-name.vault.azure.net/certificates/cert-name}\newline
        \item Versionen: können an die URL angehängt werden
    \end{itemize}
    Operationen:
    \begin{itemize}
        \item GET (abrufen), POST (erstellen), DELETE (löschen), PATCH (aktualisieren)
    \end{itemize}
    \vspace*{\stretch{1}}
\end{flashcard}


\begin{flashcard}[Definition]{Löschen und Wiederherstellen}
    \vspace*{\stretch{1}}
    \begin{itemize}
        \item wenn ein Wert gelöscht wird, ist er zunächst als gelöschtes Geheimnis weiterhin vorhanden
        \item das gelöschte Geheimnis kann wiederhergestellt, oder wirklich gelöscht werden
        \item Wiederherstellen:\newline
            \texttt{https://vault-name.vault.azure.net/deletedsecrets/secret-name/recover}\newline
            POST, da das alte Gemeimnis wieder hergestellt wird
        \item Löschen:\newline
            \texttt{https://vault-name.vault.azure.net/deletedsecrets/secret-name}\newline
            DELETE (löschen)
    \end{itemize}
    Analog: \texttt{deletedkeys}, \texttt{deletedcertificates}
    \vspace*{\stretch{1}}
\end{flashcard}


    \sectioncard{Entwickeln für Cloud und Azure Storage}

    \subsectioncard{Konfigurieren einer Nachrichtengestützten Architektur}

\begin{flashcard}[Definition]{SendGrid}
    \vspace*{\stretch{1}}
    \begin{itemize}
        \item Service zum Senden und Empfangen von Emails
        \item skalierbar
    \end{itemize}
    \vspace{1cm}
    Alternative zum Mailen: Action in Web App
    \vspace*{\stretch{1}}
\end{flashcard}

\begin{flashcard}[Definition]{Nachrichtenmodelle}
    \vspace*{\stretch{1}}
    \begin{itemize}
        \item Ziel: Kommunikation von Diensten
        \item lose Verbindung. Alle Kommunikation mit \emph{await} oder ähnlichen Konstrukten
        \item Nachrichten
            \begin{itemize}
                \item enthält komplette Rohdaten, nicht einen Verweis
                \item Sender erwartet, dass die Nachricht verarbeitet wird
            \end{itemize}
        \item Ereignisse
            \begin{itemize}
                \item Benachrichtigung über ein Ereignis (an oft viele Empfänger/Abonnenten)
                \item enthält oft einen Verweis auf Daten
                \item Sender hat keine Erwartung bezüglich Verarbeitung
                \item Ereignisse können zusammenhangslos sein, oder Teil einer Reihe
            \end{itemize}
    \end{itemize}
    \vspace*{\stretch{1}}
\end{flashcard}

\begin{flashcard}[Definition]{Notification Hub}
    \vspace*{\stretch{1}}
    \begin{itemize}
        \item Sended push-Benachrictigungen and alle Plattformen (iOS, Android, Windows, \ldots)
        \item Skalierbarer Service
        \item Quellen sowohl Azure als auch lokale Server
    \end{itemize}
    \vspace*{\stretch{1}}
\end{flashcard}

\begin{flashcard}[Definition]{Nachrichten-Zustellung in Azure}
    \vspace*{\stretch{1}}
    \begin{itemize}
        \item einfache Warteschlangen: Azure Queue Storage:\newline
            Teil eines Speicherkontos, keine Speicher-Limits, Fortschritt in der Schlange
        \item Azure Service Bus
            \begin{itemize}
                \item Warteschlangen:\newline
                    Warteschlange mit anderen Garantien als Azure Queue Storage
                \item Themen:\newline
                    Mehrere Abonnenten, die \emph{alle} die Nachricht verarbeiten müssen
            \end{itemize}
    \end{itemize}
    \vspace*{\stretch{1}}
\end{flashcard}

\begin{flashcard}[Definition]{Eigenschaften Azure Service-Bus}
    \vspace*{\stretch{1}}
    \begin{itemize}
        \item At-Most-Once-Zustellungsgarantie
        \item FIFO
        \item Transaktionen: Garantie, dass \emph{alle} Nachrichten einer Gruppe verarbeitet werden, \emph{oder keine}
        \item Unterstützt RBAC
        \item Nachrichtengröße von 64 KiB bis 256 KiB
        \item maximale Kapazität 80 G(i?)B
    \end{itemize}
    \vspace*{\stretch{1}}
\end{flashcard}

\begin{flashcard}[Definition]{Ereignis-Zustellung in Azure}
    \vspace*{\stretch{1}}
    \begin{itemize}
        \item Event Grid
        \item Event Hub
            \begin{itemize}
                \item Optimiert auf hohe Durchsätze, Sicherheit, Resilienz\newline
                    Big Data-Ereignisse: Unzählige
                \item ähnlich wie Event Grid, aber zusätzliche Features
            \end{itemize}
    \end{itemize}
    \vspace*{\stretch{1}}
\end{flashcard}

\begin{flashcard}[Definition]{Eigenschaften Azure Event Grid}
    \vspace*{\stretch{1}}
    \begin{itemize}
        \item Ereignisse: was ist passiert
        \item Maximale Ereignisgröße 64 KiB
        \item Ereignisquelle: Wo hat das Ereignis Stattgefunden.\newline
            Z.\,B. Storage, IoT Hub, \ldots
        \item Thema: Endpunkt, an den Ereignisse gesendet werden
        \item Abonnements: Weiterleitung von Ereignissen an Handler
        \item Handler: App/Dienst, der Ereignisse bearbeitet.\newline
            Z.\,B. Logik-App, Function, Webhook, \ldots
        \item Abrechnung pro Ereignis
    \end{itemize}
    \vspace*{\stretch{1}}
\end{flashcard}

\begin{flashcard}[Definition]{Event Grid erstellen}
    \vspace*{\stretch{1}}
    Automatisch von Azure-Fabric bereitgestellt.
    \vspace*{\stretch{1}}
\end{flashcard}

\begin{flashcard}[Definition]{Auf Ereignisse mit Event Grid reagieren}
    \vspace*{\stretch{1}}
    \begin{itemize}
        \item z.\,B. Logik-App
        \item als Trigger "Bei Auftreten eines Event-Grid-Ereignisses"
        \item[] oder
        \item "Bei Eintritt eines Ressourcenereignisses"
            \begin{itemize}
                \item Abonnement + Ressourcengruppe
                \item Ressourcentyp
                \item Ereignis-Typ
            \end{itemize}
    \end{itemize}
    \vspace*{\stretch{1}}
\end{flashcard}

\begin{flashcard}[Definition]{}
    \vspace*{\stretch{1}}
    \begin{itemize}
        \item
    \end{itemize}
    \vspace*{\stretch{1}}
\end{flashcard}

\begin{flashcard}[Definition]{Azure Relay-Service}
    \vspace*{\stretch{1}}
    \begin{itemize}
        \item Verfügbarmachen eines geschützten Dienstes (z.\,B. hinter Firewall) ohne Änderung der Netzwerkstruktur
        \item Azure Relay definiert HTTP-Endpunkt
        \item geschützter Dienst verbindet sich mit HTTP-Endpunkt (abgesichert mit Token)
        \item externe Clients verbinden sich mit HTTP-Endpunkt
        \item[$\Rightarrow$] Externer und geschützter Dienst können kommunizieren
    \end{itemize}
    \vspace*{\stretch{1}}
\end{flashcard}

\begin{flashcard}[Definition]{Erstellen von Azure Relay}
    \vspace*{\stretch{1}}
    \begin{itemize}
        \item benötigt:
            \begin{itemize}
                \item Abonnement + Ressourcengruppe
                \item Standort
                \item Name: global eindeutig, da Endpunkt
            \end{itemize}
        \item Im Portal: \texttt{Ressource erstellen | Relay}
        \item Für Anwendungen: \texttt{Freigegebene Zugriffsrichtlinien}:
        \begin{itemize}
            \item Primärschlüssel
            \item Verbindungszeichenfolge abrufen
        \end{itemize}
    \end{itemize}
    \vspace*{\stretch{1}}
\end{flashcard}

\begin{flashcard}[Definition]{Eigenschaften Azure Event-Hub}
    \vspace*{\stretch{1}}
    \begin{itemize}
        \item Partitionen:\newline
            Ereignispuffer zum Zwischenspeichern. Mindestens zwei, pro Puffer verschiedene Abonnements
        \item Capture:\newline
            Dauerhaftes Speichern von Ereignissen (Data Lake, Blob)
        \item Authentifizierung mit Token
        \item ermöglicht aggregierte Analyse der Ereignisse
        \item zuverlässig (Zwischenspeicher) und Resilienz
        \item millionen von Ereignissen mit geringer Latenz
    \end{itemize}
    \vspace*{\stretch{1}}
\end{flashcard}

\begin{flashcard}[Definition]{Event-Hub erstellen (Namespace)}
    \vspace*{\stretch{1}}
    \begin{itemize}
        \item Zunächst \emph{Namespace} erstellen, dann einzelne Event Hubs
        \item Benötigt:
            \begin{itemize}
                \item Abonnement + Ressourcengruppe
                \item Namespacename: global eindeutig
                \item Tarif: Basic, Standard, Dedicated
                \item Durchsatzeinheiten: kann nicht geändert werden, pro Namespace!
            \end{itemize}
        \item Im Portal: {+ Resource Erstellen | Event Hubs}
        \item CLI: \texttt{az eventhubs namespace create \ldots}
        \item zum Senden und Empfangen: Zugriffsrichtlinien mit Key und Verbindungszeichenfolge
    \end{itemize}
    \vspace*{\stretch{1}}
\end{flashcard}

\begin{flashcard}[Definition]{Event-Hub erstellen}
    \vspace*{\stretch{1}}
    \begin{itemize}
        \item wenn Namespace verfügbar ist
        \item Benötigt:
            \begin{itemize}
                \item
            \end{itemize}
        \item Im Portal: \texttt{Entitäten | Event Hubs | + Event Hub} im Event-Hubs-Blade des Namespaces
    \end{itemize}
    \vspace*{\stretch{1}}
\end{flashcard}

\begin{flashcard}[Definition]{Service Bus erstellen (Namespace)}
    \vspace*{\stretch{1}}
    \begin{itemize}
        \item Zunächst \emph{Namespace} erstellen, dann einezelne Topics und Warteschlangen
        \item Benötigt:
            \begin{itemize}
                \item Abonnement + Resourcengruppe
                \item Name: global eindeutig, da URL
                \item Tarif
                \item Standort
            \end{itemize}
        \item Im Portal: \texttt{+ Neue Ressource | Alle Dienste | Service Bus}
    \end{itemize}
    \vspace*{\stretch{1}}
\end{flashcard}

\begin{flashcard}[Definition]{Bus-Warteschlangen und Bus-Thema erstellen (Service Bus)}
    \vspace*{\stretch{1}}
    \begin{itemize}
        \item Benötigt:
            \begin{itemize}
                \item Name
                \item Größe (Gigabyte)
                \item Gültigkeitsdauer von Nachrichten
            \end{itemize}
        \item Bus-Warteschlange im Portal:\newline
            \texttt{+Warteschlange} im Namespace-Blade des Service Bus
        \item Bus-Thema im Portal:\newline
            \texttt{+Topic} im Namespace-Blade des Service Bus
        \item Topics können nur ab Standard-Tarif erstellt werden
        \item Anmelden mit Zugriffsrichtlinie: Schlüssel und Verbindungszeichenfolge
    \end{itemize}
    \vspace*{\stretch{1}}
\end{flashcard}

\begin{flashcard}[Definition]{}
    \vspace*{\stretch{1}}
    \begin{itemize}
        \item
    \end{itemize}
    \vspace*{\stretch{1}}
\end{flashcard}

\begin{flashcard}[Definition]{}
    \vspace*{\stretch{1}}
    \begin{itemize}
        \item
    \end{itemize}
    \vspace*{\stretch{1}}
\end{flashcard}

\begin{flashcard}[Definition]{}
    \vspace*{\stretch{1}}
    \begin{itemize}
        \item
    \end{itemize}
    \vspace*{\stretch{1}}
\end{flashcard}


    \subsectioncard{Analysieren von Resourcen-Nutzung und -Verbrauch}

\begin{flashcard}[Definition]{Services, die Autoscaling unterstützen}
    \vspace*{\stretch{1}}
    \begin{itemize}
        \item Skalierungsgruppen
        \item App Service (Plan) / Web Apps
        \item API Management
        \item Data Explorer
        \item Azure Cloud Services
        \item AKS
    \end{itemize}
    \vspace*{\stretch{1}}
\end{flashcard}

\begin{flashcard}[Definition]{Autoscaling-Muster}
    \vspace*{\stretch{1}}
    \begin{itemize}
        \item Off and on: Manuelle skalierung, also vom Nutzer im Portal\newline
            Wenn eine bestimmte gerade anfallende Last bedient werden soll
        \item Unverhersagbar nach CPU-Last: Matrikbasiert\newline
            Auslastung meistens stabil, aber unklar wann und wie groß eine zu erwartende Spitzenbelastung ist
        \item Kapaztiät erweitern: Manuell\newline
            Dauerhaft die Resourcen erhöhen, da die Nachfrage generell gestiegen ist
        \item Vorhersagbar: Skalierung nach Zeitplan\newline
            Wenn voraussehbar ist, dass eine Laststeigerung kommt (z.\,B. bestimmte Tage, Uhrzeiten, \ldots)
    \end{itemize}
    \vspace*{\stretch{1}}
\end{flashcard}


    \subsectioncard{Entwicklung von Lösungen mit Azure Cosmos DB}

\subsubsectioncard{create, read, update, and delete data by using appropriate APIs}

\begin{flashcard}[Definition]{Cosmos DB}
    \vspace*{\stretch{1}}
    Account:
    \begin{itemize}
        \item Benötigt Cosmos DB-Konto (wie Speicherkonto)
        \item verschlüsselung: immer aktiviert (at rest)
    \end{itemize}
    mehrere Datenmodell-APIs verfügbar: SQL, Graphen, Tabellen
    \vspace*{\stretch{1}}
\end{flashcard}

\begin{flashcard}[Definition]{Cosmos DB-Struktur}
    \vspace*{\stretch{1}}
    \begin{enumerate}
        \item Instanz
        \item Datenbank
        \item Sammlung
        \item Dokument
    \end{enumerate}
    \vspace*{\stretch{1}}
\end{flashcard}

\begin{flashcard}[Definition]{Abrechnung}
    \vspace*{\stretch{1}}
    \begin{enumerate}
        \item Anforderungseinheit (RU): 1 GET-Anfrage für ein Dokument per id von 1 KiB
        \item Andere Anfragen (PUT, ...) teurer
        \item Preis steigt mit höheren Konsistenz-anforderungen (stark, begrenzte Veraltung)
        \item Für gleiche Anfragen ist der Preis deterministisch
        \item alle Container teilen sich die Anforderungseinheiten
    \end{enumerate}
    $\Rightarrow$ Anforderungseinheiten ermöglichen Skalierung und Anpassung an Last
    \vspace*{\stretch{1}}
\end{flashcard}

\begin{flashcard}{Azure Cosmos DB-Konto erstellen}
    \vspace*{\stretch{1}}
    \begin{itemize}
        \item Benötigt:
        \begin{itemize}
            \item Abonnement + Ressourcengruppe
            \item global eindeutiger Name
            \item API: Core (SQL), MongoDB, Cassandra, Tabelle, Gremlin
            \item Location
            \item Kontotyp
            \item Schreibvorgänge in mehreren Regionen?
        \end{itemize}
        \item Portal: \texttt{Neue Ressource | Erste Schritte | Azure Cosmos DB} (oder suchen)
        \item CLI:\newline
        \texttt{az cosmosdb create --name NAME --kind GlobalDocumentDB --resource-group grp}
    \end{itemize}
    \vspace*{\stretch{1}}
\end{flashcard}

\begin{flashcard}{Azure Cosmos DB}
    \vspace*{\stretch{1}}
    \begin{itemize}
        \item Benötigt:
        \begin{itemize}
            \item Cosmos DB-Kontoname + Resourcengruppe
            \item Datenbank-ID (Name), kann mehrere Container enthalten
            \item Durchsatz in RUs pro Sekunde
        \end{itemize}
        \item CLI:\newline
        \texttt{az cosmosdb sql database create}
    \end{itemize}
    $\Rightarrow$ Container muss per CLI extra erstellt werden
    \vspace*{\stretch{1}}
\end{flashcard}

\begin{flashcard}{Azure Cosmos DB Container}
    \vspace*{\stretch{1}}
    \begin{itemize}
        \item Benötigt:
        \begin{itemize}
            \item Cosmos DB-Kontoname + Resourcengruppe
            \item Datenbank-ID (Name), kann mehrere Container enthalten
            \item Container-ID (Name)
            \item Durchsatz in RUs pro Sekunde
        \end{itemize}
        \item Portal: \texttt{Daten-Explorer | Neuer Container} im Cosmos DB-Konto-Blade
        \item CLI:\newline
        \texttt{az cosmosdb sql container create}
    \end{itemize}
    \vspace*{\stretch{1}}
\end{flashcard}

\begin{flashcard}[Definition]{Daten-Explorer}
    \vspace*{\stretch{1}}
    \begin{itemize}
        \item Benutzeroberfläche um Cosmos DB-Daten zu verwalten
        \item Im Portal: \texttt{Daten-Explorer} im Blade der Cosmos DB
    \end{itemize}
    \vspace*{\stretch{1}}
\end{flashcard}

\begin{flashcard}[Definition]{Daten in Cosmos DB einfügen}
    \vspace*{\stretch{1}}
    \begin{itemize}
        \item Auswahl des Containers: \texttt{Konto | Container | Datenbank}
        \item Neues Element erstellen: \texttt{New Item}
        \item Dokument kann eingetragen Werden (als JSON)
    \end{itemize}
    \vspace*{\stretch{1}}
\end{flashcard}

\begin{flashcard}[Definition]{Daten Lesen}
    \vspace*{\stretch{1}}
    \begin{itemize}
        \item Eine SQL-Abfrage erstellen: Innerhalb einer Datenbank
        \item \texttt{FROM} kann einen Container oder eine Datenbank sein
        \item Zugriff auf Datenbank: \texttt{Container.Datenbank}
        \item Auch auf unter-elemente des JSON-Objekts kann so zugegriffen werden
    \end{itemize}
    \vspace*{\stretch{1}}
\end{flashcard}

\begin{flashcard}[Definition]{Gespeicherte Prozeduren}
    \vspace*{\stretch{1}}
    \begin{itemize}
        \item Sammlung von mehreren Operationen in einer Transaktion
        \item einzige Möglichkeit \emph{ACID}-Transaktionen zu garantieren
        \item im Container gespeichert
        \item JavaScript
    \end{itemize}
    \vspace*{\stretch{1}}
\end{flashcard}

\begin{flashcard}[Definition]{Verbinden}
    \vspace*{\stretch{1}}
    \begin{itemize}
        \item Verbindungs-Richtlinie: Liste von bevorzugten Regionen (geordnet)
        \item Verbindung herstellen
        \item benötigt den Schlüssel für die Verbindung
    \end{itemize}
    \vspace*{\stretch{1}}
\end{flashcard}

\subsubsectioncard{implement partitioning schemes}

\begin{flashcard}[Definition]{Partitionierung}
    \vspace*{\stretch{1}}
    \begin{itemize}
        \item horizontale Skalierung für Cosmos DB-Datenbanken
        \item \emph{Partitionsschlüssel}: Strategie für Erstellung von Partitionen \emph{pro Container}\newline
            \emph{Kann nicht geändert werden!}
        \item \emph{Hot Partition}: eine Partition, die deutlich mehr Anforderungen erhält, als andere\newline
            Schlüssel sollten sowohl eindeutig sein als auch zum Suchen benutzt werden (\emph{id} z.\,B.)
        \item Limit: 20 GiB pro Partitionsschlüssel\newline
            falls Schlüssel zu groß: Schlüssel kombinieren
        \item physische Partition: von Azure Cosmos DB verwaltet \newline
            Abhängig von Anforderungen der Leistung
    \end{itemize}
    \vspace*{\stretch{1}}
\end{flashcard}

\begin{flashcard}[Definition]{Partitionsschlüssel auswählen}
    \vspace*{\stretch{1}}
    \begin{itemize}
        \item viele Werte sind gut. Ermöglichen gute Skalierung (\emph{Hohe Kardinalität})
        \item am häufigsten in \texttt{WHERE} enthaltener Wert ist ein Kandidat
        \item die Anforderungseinheiten (RUs) sollten gleichmäßig auf alle Partitionen aufgeteilt sein
        \item Speicherplatz sollte gleichmäßig aufgeteilt werden
    \end{itemize}
    \vspace*{\stretch{1}}
\end{flashcard}

\subsubsectioncard{set the appropriate consistency level for operations}

\begin{flashcard}[Definition]{Cosmos DB Datenreplikation}
    \vspace*{\stretch{1}}
    \begin{itemize}
        \item für Hochverfügbarkeit
        \item mehrere Regionen können ausgewählt werden
        \item erlaubt Failover
        \item einzelne Schreib-Regionen: schreiben nur in einer Region
        \item multi Schreib-Regionen: teuer
        \item automatisches Failover: falls die Schreibregion ausfällt, automatisch eine andere Region zur Schreibregion promovieren
    \end{itemize}
    \vspace*{\stretch{1}}
\end{flashcard}

\begin{flashcard}[Definition]{Datenreplikation aktivieren}
    \vspace*{\stretch{1}}
    \begin{itemize}
        \item Portal: \texttt{Einstellungen | Daten global replizieren} im Cosmos DB-Konto-Blade
        \item Auswahl: \texttt{+ Region hinzufügen}
    \end{itemize}
    \vspace*{\stretch{1}}
\end{flashcard}

\begin{flashcard}[Definition]{Konsistenz}
    \vspace*{\stretch{1}}
    \begin{itemize}
        \item Wenn Daten repliziert werden, müssen sie konsistent gehalten werden
        \item Multimaster: regionen, die Schreibberechtigung haben (Aktiv/Aktiv)
        \begin{itemize}
            \item 99,999\% Lese/Schreibverfügbarkeit
            \item Unbegrenzte Skalierbarkeit
            \item Schreiblatenz < 10\,ms
        \end{itemize}
        \item Konfliktlösungen:
        \begin{itemize}
            \item Last-Writer-Wins (LWW): ganzzahlige Eigenschaft im Dokument, z.\,B. Zeitstempel
            \item Benutzerdefiniert: spezielle gespeicherte Prozedur, fallback Asynchron
            \item Asynchron: Konflikte werden ausgeschlossen und in einem Feed an Benutzer geleitet
        \end{itemize}
    \end{itemize}
    \vspace*{\stretch{1}}
\end{flashcard}

\begin{flashcard}[Definition]{Konsistenzmodelle}
    \vspace*{\stretch{1}}
    Von Konsistenz zu Inkonsistenz:
    \begin{enumerate}
        \item stark\newline
            Linearisiert. Garantiert neueste Version gelesen.
        \item Begrenzte veraltung\newline
            Präfixkonsistenz + Lesen höchstens hängt k zurück
        \item Sitzung\newline
            Präfixkonsistenz + Monotone schreib/lesevorgänge
        \item Präfixkonsistenz\newline
            Aktualisierungen sind Präfix aller Aktualisierungen ohne Lücken
        \item letztlich\newline
            Lesevorgänge in falscher Reihenfolge möglich
    \end{enumerate}
    \vspace*{\stretch{1}}
\end{flashcard}

\begin{flashcard}[Definition]{Starke Konsistenz}
    \vspace*{\stretch{1}}
    \begin{itemize}
        \item Linearisierbarkeit:\newline
            Lesevorgänge geben garantiert die neueste Version zurück
        \item Schreibvorgang wird erst sichtbar, nachdem er repliziert worden ist
        \item Lesevorgänge liefer immer neueste bestätigte Version zurück
        \item wenn mehrere Regionen genutzt werden, kein Vorteil durch geringe Latenz
        \item hohe Kosten (gleich wie \emph{begrenzte veraltung})
    \end{itemize}
    \vspace*{\stretch{1}}
\end{flashcard}

\begin{flashcard}[Definition]{Begrenzte veraltung}
    \vspace*{\stretch{1}}
    \begin{itemize}
        \item Ergebnis eines Lesevorgangs ist begrenzt alt
            \begin{itemize}
                \item entweder $K$ Versionen (Präfixlnge)
                \item oder Zeitintervall $t$
            \end{itemize}
            (wie starke Konsistenz, aber zeitlich verzögert)
        \item ideal für hohe Konsistenz und hohe Verfügbarkeit (99,99\%) und niedrige Latenz
        \item höhere Konsistenzgarantie als \emph{Sizung}, \emph{Präfixkonsistenz} und \emph{letztliche Konsistenz}
        \item unterstützen beliebige Anzahl von Regionen
        \item höhere Kosten, wie bei \emph{stark}er Konsistenz
    \end{itemize}
    \vspace*{\stretch{1}}
\end{flashcard}

\begin{flashcard}[Definition]{Sitzungskonsistenz}
    \vspace*{\stretch{1}}
    \begin{itemize}
        \item Beschränkt auf Clientsitzung (im Unterschied zu globaler Konsistenz)
        \item Geeignet für Szenarien mit Geräte- oder Benutzersitzung\newline
            Read Your Own Writes
        \item Lesen und Schreiben mit niedrigster Latenz
        \item unterstützen beliebige Anzahl von Regionen
        \item mittlere Kosten (bezüglich Anforderungseinheiten (RUs))
    \end{itemize}
    \vspace*{\stretch{1}}
\end{flashcard}

\begin{flashcard}[Definition]{Präfixkonsistenz}
    \vspace*{\stretch{1}}
    \begin{itemize}
        \item Konsistenz für clients in einer Region oder mit Single-Master-Konto. Multi-Region-Kontos: letztlich konsistent
        \item Keine Lücken oder Lesevorgänge außer der Reihenfolge
        \item unterstützen beliebige Anzahl von Regionen
    \end{itemize}
    \vspace*{\stretch{1}}
\end{flashcard}

\begin{flashcard}[Definition]{Letztliche Konsistenz}
    \vspace*{\stretch{1}}
    \begin{itemize}
        \item alle Replikate konvergieren zum gleichen Zustand
        \item es können sogar ältere Werte als bereits gelesene Werte gelesen werden
        \item niedrigste Latenz für Lese- und Schreibvorgänge (dafür auch schwächste Konsistenz)
        \item niedrigste Kosten (bezüglich Anforderungseinheiten (RUs))
    \end{itemize}
    \vspace*{\stretch{1}}
\end{flashcard}

\begin{flashcard}[Definition]{Konsistenz einstellen}
    \vspace*{\stretch{1}}
    \begin{itemize}
        \item Portal: \texttt{Einstellungen | Standardkonsistenz} im Cosmos DB-Konto-Blade
    \end{itemize}
    \vspace*{\stretch{1}}
\end{flashcard}


    \subsectioncard{Entwickeln von Lösungen, die relationale Datenbanken umfassen}

\subsubsectioncard{provision and configure relational databases}

\begin{flashcard}[Definition]{Datenbank-Varianten}
    \vspace*{\stretch{1}}
    \begin{itemize}
        \item Single Database:\newline
            für kleine Services geeignet
        \item Verwlatet:\newline
            verwaltete, große Instanz
        \item elastische Pools:\newline
            mehrere Datenbanken, die einen Pool teilen
        \item virtuelle Maschine:\newline
            vollständige Kontrolle über Datenbank, Server, aber kostet
    \end{itemize}
    \vspace*{\stretch{1}}
\end{flashcard}


\begin{flashcard}[Definition]{Vorteile von Azure SQL-Datenbank}
    \vspace*{\stretch{1}}
    \begin{itemize}
        \item Verwaltet (PaaS): Hardware, Software und Betriebssysteme werden aktualisiert
        \item Kosten: abhängig von Instanzgröße, aber keine dedizierte Hardware
        \item dynamische Skalierbarkeit
        \item Sicherheit: Firewall standardmäßig aktiv
    \end{itemize}
    \vspace*{\stretch{1}}
\end{flashcard}

\begin{flashcard}[Definition]{Azure-SQL-Datenbank}
    \vspace*{\stretch{1}}
    \begin{itemize}
        \item Azure-SQL-Datenbank: logischer Server\newline
            Pro Server: Verwaltung von Anmeldung, Sicherheit, Firewall, \ldots für alle Datenbanken
        \item Datenbank: in einem logischen Server
    \end{itemize}
    \vspace*{\stretch{1}}
\end{flashcard}

\begin{flashcard}[Definition]{Typen von SQL-Datenbank}
    \vspace*{\stretch{1}}
    DTUs
    \begin{itemize}
        \item \emph{Database Transaction Unit}:\newline
            Kombination aus Rechenleistung, Speicher, Eingabe/Ausgabe
        \item vorkonfiguriert
        \item kann für kleine Anwendungen günstiger sein
    \end{itemize}
    vKern
    \begin{itemize}
        \item virtuelle Kerne (für den Datenbank-Server)
        \item Kerne und Speicher können unabhängig konfiguriert werden
    \end{itemize}

    \vspace*{\stretch{1}}
\end{flashcard}

\begin{flashcard}[Definition]{SQL-Server erstellen}
    \vspace*{\stretch{1}}
    \begin{itemize}
        \item Benötigt:
            \begin{itemize}
                \item Global eindeutiger Servername (URL)
                \item Administrator-Zugangsdaten
                \item Location
                \item Zugriff von Azure-Diensten?
            \end{itemize}
        \item Portal: \texttt{Resource erstellen | Datenbanken | SQL-Datenbank | Server | Neu erstellen }
        \item Azure CLI: \newline
            \texttt{az sql server create}
    \end{itemize}
    \vspace*{\stretch{1}}
\end{flashcard}

\begin{flashcard}[Definition]{SQL-Datenbank erstellen}
    \vspace*{\stretch{1}}
    \begin{itemize}
        \item Benötigt:
            \begin{itemize}
                \item Abonnement und Ressourcengruppe
                \item Datenbank-Name
                \item Server
                \item elastische Pools Ja/Nein?
                \item Leistung (Rechenkraft und Speicher)
            \end{itemize}
        \item Portal: \texttt{Resource erstellen | Datenbanken | SQL-Datenbank}
        \item Azure-CLI:\newline
            \texttt{az sql db create \ldots}
    \end{itemize}
    \vspace*{\stretch{1}}
\end{flashcard}

\begin{flashcard}[Definition]{Leistung für Datenbank einstellen}
    \vspace*{\stretch{1}}
    \begin{itemize}
        \item Portal: Beim Erstellen einer Datenbank ("Compute + Storage")
        \item vCore-Basiert
        \item Basic, Standard, Premium für DTUs
    \end{itemize}
    \vspace*{\stretch{1}}
\end{flashcard}

\begin{flashcard}[Definition]{Firewall für SQL-Datenbank konfigurieren}
    \vspace*{\stretch{1}}
    \begin{itemize}
        \item
        \item Portal: \texttt{Serverfirewall festlegen} im Datenbank-Blade
        \begin{itemize}
            \item \texttt{Client-IP-Adresse hinzufügen}: aktuelle Adresse hinzufügen
            \item[$\Rightarrow$] wird als Regel eingetragen
            \item Regeln: Name und IP-Bereich
            \item Virtuelle Netzwerke
        \end{itemize}
    \end{itemize}
    \vspace*{\stretch{1}}
\end{flashcard}

\subsubsectioncard{configure elastic pools for Azure SQL Database}

\begin{flashcard}[Definition]{Pools für elastische SQL-Datenbanken}
    \vspace*{\stretch{1}}
    \begin{itemize}
        \item gemeinsam genutzte Resourcen (eDTUs) für mehrere Datenbanken
        \item Datenbanken können (mit Beschränkung) Resourcen von dem Pool nutzen
        \item Anwendungsfälle: unterschiedlicher, unverhergesehener Ressourcnebedarf
        \item Architektur:
            \begin{itemize}
                \item gewisse Ressourcen für den gesamten Pool
                \item verschiedene Datenbanken werden dem Pool zugewiesen
                \item[$\Rightarrow$] teilen sich die Ressourcen
            \end{itemize}
        \item Größe: kombinierte Ressourcen (theoretisch) müssen 1,5 mal die Kapaztität für den Pool überschreiben. Mehr als 2 S3-Datenbanken oder 15 S0-Datenbanken
        \item Grenzen: 100 bzw. 500 Datenbanken pro Pool
    \end{itemize}
    \vspace*{\stretch{1}}
\end{flashcard}

\begin{flashcard}[Definition]{Elastischen Pool erstellen}
    \vspace*{\stretch{1}}
    \begin{itemize}
        \item Benötigt:
            \begin{itemize}
                \item Abonnement + Ressourcengruppe
                \item Pool-Name
                \item SQL-Server-Instanz
            \end{itemize}
        \item Azure-CLI:\newline
            \texttt{az sql elastic-pools create}
        \item Portal: \texttt{Elastischer SQL-Datenbankpool} suchen als Ressourcetyp
    \end{itemize}
    \vspace*{\stretch{1}}
\end{flashcard}

\begin{flashcard}[Definition]{Zuweisen eines Pools an eine Datenbank}
    \vspace*{\stretch{1}}
    \begin{itemize}
        \item Bei der Erstellung:\newline
            \texttt{az sql db create --elastic-pool-name \ldots}
        \item Im Portal: \newline
            \texttt{Einstellungen | Konfigurieren | Datenbanken | + Datenbanken hinzufügen} im Pool-Blade\newline
            (Datenbanken müssen bereits existieren)
        \item Im Portal: \newline
            \texttt{Übersicht | + Datenbank erstellen} für eine neue Datenbank direkt im Pool
    \end{itemize}
    \vspace*{\stretch{1}}
\end{flashcard}

\begin{flashcard}[Definition]{Elastische Pools für SQL-Datenbanken konfigurieren}
    \vspace*{\stretch{1}}
    \begin{itemize}
        \item DTU: drei Tarife, \texttt{Basic}, \texttt{Standard}, \texttt{Premium}
        \item vCore: Anzahl Kerne, Generation
        \item Portal: \texttt{Einstellungen | Konfiguration | Pooleinstellungen} im Pool-Blade
    \end{itemize}
    Datenbank:
    \begin{itemize}
        \item Grenzwerte pro Datenbank (min, max DTUs)
        \item Portal: \texttt{Einstellungen | Konfiguration | Pro Datenbank - Einstellungen} im Pool-Blade
    \end{itemize}
    \vspace*{\stretch{1}}
\end{flashcard}

\begin{flashcard}[Definition]{Datenbanken in Pool verwalten}
    \vspace*{\stretch{1}}
    \begin{itemize}
        \item CLI:\newline
            \texttt{az cli db create --resource-group \ldots\ --server \ldots\newline --name <db> --elastic-pool \ldots}
        \item PowerShell:\newline
            \texttt{Set-AzSqlDatabase}
    \end{itemize}
    \vspace*{\stretch{1}}
\end{flashcard}


\begin{flashcard}[Definition]{Zugriff auf SQL-Datenbank}
    \vspace*{\stretch{1}}
    \begin{itemize}
        \item Azure CLI: \texttt{az sql db \ldots}
        \item Verbindungszeichenfolge: \texttt{az sql db show-connection-string}
    \end{itemize}
    \vspace*{\stretch{1}}
\end{flashcard}

\subsubsectioncard{implement Azure SQL Database managed instances}

\subsubsectioncard{create, read, update, and delete data tables by using code}

\begin{flashcard}[Definition]{CRUD-Operationen}
    \vspace*{\stretch{1}}
    \begin{itemize}
        \item Erstellen\newline
            \texttt{CREATE TABLE \ldots} (Tabelle erstellen)\newline
            \texttt{INSERT INTO table (\ldots) VALUES (\ldots);} (Werte einfügen)
        \item Lesen\newline
            \texttt{SELECT x, y FROM table;}
        \item Update\newline
            \texttt{UPDATE table SET x=y WHERE a=b;}
        \item Löschen\newline
            \texttt{DELETE FROM table WHERE x=y;}
    \end{itemize}
    Login von CloudShell z.\,B. über \texttt{sqlcmd}
    \vspace*{\stretch{1}}
\end{flashcard}


    \vspace*{\stretch{1}}
    \doclicenseThis
    \vspace*{\stretch{1}}
    \pagebreak
\end{document}
