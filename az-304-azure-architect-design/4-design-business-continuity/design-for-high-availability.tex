\subsectioncard{Auswahl eines Speicherkontos}

\subsubsectioncard{recommend a recovery solution for Azure hybrid and on-premises workloads that meets recovery objectives (RTO, RLO, RPO)}

\begin{flashcard}[Definition]{Geschäftskontinuiät}
  \vspace*{\stretch{1}}
  \begin{itemize}
    \item BCDR - Business Continuity and Disaster Recovery
  \end{itemize}
  \vspace*{\stretch{1}}
\end{flashcard}

\begin{flashcard}[Definition]{RTO}
  \vspace*{\stretch{1}}
  \begin{itemize}
    \item Recovery Time Objective
    \item Maximale Zeitspanne, die da Unternehmen nach einem Notfall überleben kann, bevor der normale Betrieb wiederhergestellt wurde
  \end{itemize}
  \vspace*{\stretch{1}}
\end{flashcard}

\begin{flashcard}[Definition]{RPO}
  \vspace*{\stretch{1}}
  \begin{itemize}
    \item Recovery Point Objective
    \item Maximaler Datenverlust, der im Notfall akzeptabel ist
  \end{itemize}
  \vspace*{\stretch{1}}
\end{flashcard}

\begin{flashcard}[Definition]{RLO}
  \vspace*{\stretch{1}}
  \begin{itemize}
    \item Recovery Level objective
    \item Level/Ebene, die widerhergestellt werden müssen
    \item z.\,B. die ganze Umgebung, eine Anwendung, eine einzelne Datei
  \end{itemize}
  \vspace*{\stretch{1}}
\end{flashcard}

\begin{flashcard}[Definition]{Azure Site Recovery}
  \vspace*{\stretch{1}}
  \begin{itemize}
    \item Backup für
      \begin{itemize}
        \item verschiedene VMs: Azure, VMware, lokale Server und VMs, Hyper-V
        \item verschiedene Workloads: Apps, \ldots
      \end{itemize}
    \item Datenresilienz durch Speicherung in Azure Storage
    \item automatisches Failover
  \end{itemize}
  \vspace*{\stretch{1}}
\end{flashcard}

\begin{flashcard}[Definition]{Eigenschaften von Sicherungen}
  \vspace*{\stretch{1}}
  \begin{itemize}
    \item kontinuierliche Replikation für Azure, VMware, phyisische Server (RPO)
    \item 30 Sekunden maximal für Hyper-V (RPO)
    \item Disks können gesichert und ausgeschlossen werden
    \item Konsistenz von Anwendungssicherungen (mit Skripten)
    \item Absturzkonsistenz: alle 5 Minuten (kein Speicher-Backup)
  \end{itemize}
  \vspace*{\stretch{1}}
\end{flashcard}

\begin{flashcard}[Definition]{VMware Site Recovery}
  \vspace*{\stretch{1}}
  \begin{itemize}
    \item initialie Replikation und Deltas
    \item Sicherung auf verwaltete Datenträger
    \item Installation des Agenten auf physischen und virtuellen Computern
    \item Konto im vCenter
    \item viele Windows- und Linux-Versionen unterstützt
    \item Anwendungsbackup unterstützt für viele verbreitete Anwendungen
  \end{itemize}
  \vspace*{\stretch{1}}
\end{flashcard}

\begin{flashcard}[Definition]{Hyper-V Site Recovery}
  \vspace*{\stretch{1}}
  \begin{itemize}
    \item Sicherung mit oder ohne VMM (Virtual Machine Manager)
    \item Sicherung nicht nur nach Azure sondern auch in weitere Datencenter (nur VMM)\newline
      eingeschränkte Betriebssystemauswahl
    \item alle virtuellen Datenträger werden gesichert
    \item Sicherung beginnt mit Hyper-V-Snapshot
    \item automatisches und manuelles Failover und Failback
  \end{itemize}
  \vspace*{\stretch{1}}
\end{flashcard}

\begin{flashcard}[Definition]{Azure Backup}
  \vspace*{\stretch{1}}
  \begin{itemize}
    \item einfacher Service für Datensicherung von Speicher
    \item lokale und Cloud-Daten\newline
      $\Rightarrow$ keine Notwendigkeit von lokalen Backup-Lösungen
    \item automatische Verwaltung von Speicherpunkten
    \item keine Begrenzung der Datenmenge beim Transfer
    \item verschlüsselt (in Transit und abgelegt)
    \item alle Azure-Speichertypen werden unterstützt: Redundanz, Replikation, \ldots
    \item Test auf (teilweise Schutz vor) Verschlüsselungstrojaner
  \end{itemize}
  \vspace*{\stretch{1}}
\end{flashcard}

\begin{flashcard}[Definition]{Unterstützte Daten von Azure Backup}
  \vspace*{\stretch{1}}
  \begin{itemize}
    \item lokale Daten (mit MARS-Agent)
    \item Daten von VM (Dateien, System, \ldots) (mit MARS-Agent)
    \item verwaltete Platten
    \item Azure Speicher: Files, Blobs
    \item Datenbanken: SQL, PostgreSQL, SAP HANA
  \end{itemize}
  \vspace*{\stretch{1}}
\end{flashcard}

\subsubsectioncard{design an Azure Site Recovery solution}

\begin{flashcard}[Definition]{Azure Site Recovery}
  \vspace*{\stretch{1}}
  \begin{itemize}
    \item Tool zur Wiederherstellung von Daten
    \item Replikation zwischen primärem und sekundärem Standort
    \item unterstützt lokale (inkls. physische Maschinen) und Azure als Quell-Standort
    \item Failback (nicht zu physische Maschinen)
    \item erstellt Momentaufnahmen und Wiederherstellungspunkte
  \end{itemize}
  \vspace*{\stretch{1}}
\end{flashcard}

\begin{flashcard}[Definition]{Voraussetzungen in Azure}
  \vspace*{\stretch{1}}
  \begin{itemize}
    \item Berechtigung für Resourcen (Subscription, VM, Speicher, \ldots)
    \item Recovery Services Vault
    \item Netzwerk, Speicher für die gesicherten VMs
    \item[!] Vault, Speicher und Netzwerk müssen in der gleichen Region sein
  \end{itemize}
  \vspace*{\stretch{1}}
\end{flashcard}

\begin{flashcard}[Definition]{Azure Backup Server}
  \vspace*{\stretch{1}}
  \begin{itemize}
    \item MABS
    \item Software zur Verwaltung von lokalen Backups
    \item MARS muss auf den zu sichernden Servern installiert sein
    \item MABS auf dediziertem Server zur Verwaltung von Endpunkten und Sicherungen
    \item verbunden mit Recovery Services Vault
  \end{itemize}
  \vspace*{\stretch{1}}
\end{flashcard}

\subsubsectioncard{recommend a solution for recovery in different regions}

\begin{flashcard}[Definition]{Sicherung in sekundäre Region}
  \vspace*{\stretch{1}}
  \begin{itemize}
    \item Azure VMs können einfach in die sekundäre Region repliziert werden
    \item Im Portal: Notfallwiederherstellung aktivieren
    \item Auswählen einer Region (viele sind möglich)
    \item einfache
  \end{itemize}
  \vspace*{\stretch{1}}
\end{flashcard}

\begin{flashcard}[Definition]{Replikation in Verfügbarkeitszone}
  \vspace*{\stretch{1}}
  \begin{itemize}
    \item wenn VM in Verfügbarkeitszone bereitgestellt worden ist
    \item kann in andere Zone repliziert werden
  \end{itemize}
  \vspace*{\stretch{1}}
\end{flashcard}

\begin{flashcard}[Definition]{Datenssicherung in Regionen}
  \vspace*{\stretch{1}}
  \begin{itemize}
    \item Azure Speicherkonten unterstützen eine sekundäre Region\newline
    $\Rightarrow$ Sicherung mit letztlicher Konsistenz
    \item Geo-Redundanter Speicher (auch zusammen mit Zonenredundanz)
    \item RA-GRS: erlaubt Lesen des gesicherten Speichers (auch ohne Failover)
    \item nach Failover wird die sekundäre Region die primäre
    \item nicht möglich für Premium-Speicherkonten
  \end{itemize}
  $\Rightarrow$ Wenn Beschränkungen zu stark, möglicherweise manuell mit Tools
  \vspace*{\stretch{1}}
\end{flashcard}

\begin{flashcard}[Definition]{Failover für Datenredundanz}
  \vspace*{\stretch{1}}
  \begin{itemize}
    \item möglicherweise Datenverlust durch asynchrone Speicherung\newline
      $\Rightarrow$ Vorsicht bei Failback!    
    \item Anwendungen müssen Failover unterstützen
    \item Schlüsseltechnologie für Hochverfügbarkeit
    \item archivierte Blobs müssen erst wieder Rehydriert werden
    \item hierarchischer Namespace (Datalake Gen 2) ist noch nicht unterstützt
    \item kein Failover bei aktiviertem WORM
    \item möglicherweise automatisch von Microsoft initiiert (im Notfall!)
  \end{itemize}
  \vspace*{\stretch{1}}
\end{flashcard}

\subsubsectioncard{recommend a solution for geo-redundancy of workloads}

\begin{flashcard}[Definition]{Zugriff auf redundante Daten}
  \vspace*{\stretch{1}}
  \begin{itemize}
    \item RA-GRS, RA-GZRS: Lesezugriff ständig möglich, möglicherweise verzögert\newline
    $\Rightarrow$ testen der letztlichen Konsistenz mit \emph{Last Sync Time}
    \item Lese-Kopien in Azure Files werden nicht unterstützt
    \item automatisches Fallback falls Timeout
    \item Entscheidung: kompletter Failover (Table, Queue, Blob) oder einzeln je nach Service
    \item Anwendungen: einfach, Lese-Anforderungen an sekundäre Region weiterzuleiten, aber nicht Schreib-Anforderungen
  \end{itemize}
  \vspace*{\stretch{1}}
\end{flashcard}

\begin{flashcard}[Definition]{Automatischer Endpunktwechsel}
  \vspace*{\stretch{1}}
  \begin{itemize}
    \item im Failover-Fall muss der Endpunkt gewechselt werden
    \item ideal: transparent für Anwendungen
    \item unterstützt von Traffic Manager
    \item DNS-basierter Load-Balancer mit Endpunkt-Monitoring
    \item unterstüzt auch lokale Endpunkte\newline
    (und damit auch Failover in die Cloud)
  \end{itemize}
  \vspace*{\stretch{1}}
\end{flashcard}

\subsubsectioncard{recommend a solution for Azure Backup management}

\begin{flashcard}[Definition]{Monitoring und Verwaltung von Azure Recovery}
  \vspace*{\stretch{1}}
  \begin{itemize}
    \item Overview-Dashboard im Azure Portal
    \item Alarme (Kritisch, oder nur Warnungen)
    \item laufende und abgeschlossene Backups
    \item gesicherte Daten (Mengen!)
    \item auch Warnungen über VMs mit problemen (\emph{vor} dem Backup)
    \item automatische Benachrichtigungen möglich
  \end{itemize}
  \vspace*{\stretch{1}}
\end{flashcard}

\begin{flashcard}[Definition]{Sicherungsvorüberprüfung}
  \vspace*{\stretch{1}}
  \begin{itemize}
    \item Prüfen der Konfiguration von virtuellen Computern
    \item Zustände:
      \begin{itemize}
        \item Erfolgreich: vermutlich erfolgreiche Sicherungen
        \item Warnungen: möglicherweise Probleme, mit Lösungen
        \item Kritisch: Sicherung \emph{wird} fehlschlagen, erforderliche Schritte zur Problemlösung
      \end{itemize}
  \end{itemize}
  \vspace*{\stretch{1}}
\end{flashcard}

\begin{flashcard}[Definition]{Azure Backup-Konfiguration}
  \vspace*{\stretch{1}}
  \begin{itemize}
    \item lokale VMs
      \begin{itemize}
        \item Windows mit MARS
        \item Data Protection Manager (DPM) oder MABS
      \end{itemize}
    \item Azure VMs
      \begin{itemize}
        \item alle VMs können direkt gesichert werden (mit Agent als VM Erweiterung)
        \item einzelne Verzeichnisse/Dateien: mit MARS
        \item auch mit MABS
      \end{itemize}
    \item Sicherungstypen: Voll, Inkrementell (differentiell wird nicht unterstützt)
  \end{itemize}
  \vspace*{\stretch{1}}
\end{flashcard}

\begin{flashcard}[Definition]{Backup-Richtlinien}
  \vspace*{\stretch{1}}
  \begin{itemize}
    \item eine pro Recovery Vault\newline
      $\Rightarrow$ wird für viele VMs (oder andere Ressourcen) genutzt
    \item Zeitplan: taglich, wöchentlich
    \item Aufbewahrung: taglich, wöchentlich, monatlich, jährlich
  \end{itemize}
  \vspace*{\stretch{1}}
\end{flashcard}

\subsubsectioncard{design a solution for data archiving and retention}

\begin{flashcard}[Definition]{Lebenszyklus von Daten}
  \vspace*{\stretch{1}}
  \begin{itemize}
    \item Daten haben unterschiedliche Lebenszyklen (manchmal gesetzlich, manchmal ganz praktisch)
    \item Zugriffshäufigkeit ändert sich (alte Daten werden möglicherweise nicht mehr zugegriffen)
  \end{itemize}
  \vspace*{\stretch{1}}
\end{flashcard}

\begin{flashcard}[Definition]{Lebenszyklusverwaltung in Azure}
  \vspace*{\stretch{1}}
  \begin{itemize}
    \item regelbasierte Änderung der Speicherart für Blobs in Azure (\emph{cool}, \emph{hot})
    \item überführung von Blobs in kältere Speicherebenen (bis zu \emph{archive})
      \begin{itemize}
        \item Snapshots
        \item alte Versionen
        \item aber auch: aktuelle Versionen, wenn selten (nicht) genutzt
      \end{itemize}
     \item Löschen von Blobs am Ende des Lebenszyklus
     \item pro Container, oder Blob-Präfix, oder Blob-Index-Tags
     \item Änderung von Blobs nur einmal am Tag
  \end{itemize}
  \vspace*{\stretch{1}}
\end{flashcard}

\begin{flashcard}[Definition]{Regeln zum Lebenszyklus}
  \vspace*{\stretch{1}}
  \begin{itemize}
    \item abhängig vom Alter eines Blobs (nach Tagen)
    \item abhängig vom letzten Zugriff (z.\,B. nach mehreren Tagen)
    \item abhängig von Index-Tags
    \item abhängig von Aktualität (z.\,B. gelöschte oder geänderte Versionen)
  \end{itemize}
  \vspace*{\stretch{1}}
\end{flashcard}
