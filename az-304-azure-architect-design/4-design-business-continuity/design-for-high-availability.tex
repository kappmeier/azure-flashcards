\subsectioncard{Designen für Hochverfügbarkeit}

\subsubsectioncard{recommend a solution for application and workload redundancy, including compute, database, and storage}

\begin{flashcard}[]{Regions}
  \vspace*{\stretch{1}}
  \begin{itemize}
    \item mehrere Geographien können Azure-Services enthalten
    \item innerhalb einer Geographie kann es mehrere Regionen geben
    \item Regionen sind weit voneinander entfernt um for regionalen und größeren Katastrophen zu schützen
    \item einige Azure-Services erlauben Replikation in mehrere Regionen\newline
      $\Rightarrow$ Partner-Region
  \end{itemize}
  \vspace*{\stretch{1}}
\end{flashcard}

\begin{flashcard}[]{Verfügbarkeitszonen}
  \vspace*{\stretch{1}}
  \begin{itemize}
    \item Innerhalb einer region physikalisch getrennte Orte
    \item jede Verfügbarkeitszone kann ein oder mehrere Rechenzentren enthalten
    \item Verfügbarkeitszonen sind über schnelle Verbindungen miteinander verbunden
    \item Anwendungen können zwischen Verfügbarkeitszonen wechslen, falls ein Rechenzentrum ausfällt
  \end{itemize}
  $\Rightarrow$ schutz vor kleineren Szenarios, wie ausfälle eines Rechenzentrums
  \vspace*{\stretch{1}}
\end{flashcard}

\begin{flashcard}[]{Verfügbarkeitsgruppe}
  \vspace*{\stretch{1}}
  \begin{itemize}
    \item mehrere VMs können in eine Verfügbarkeitsgruppe bereitgestellt werden
    \item unterstützt Update-Domänen
      \begin{itemize}
        \item Neustart der VMs der gleichen Domäne \emph{zeitgleich}
      \end{itemize}
    \item unterstützt Fehler-Domänen
      \begin{itemize}
         \item gemeinsam genutzter Speicher
         \item gemeinsam genutzte Netzwerke
         \item gemeinsam genutzte Stromversorgung
      \end{itemize}
    \item[!] VMs müssen bei der Erstellung hinzugefügt werden
  \end{itemize}
  \vspace*{\stretch{1}}
\end{flashcard}

\begin{flashcard}[]{}
  \vspace*{\stretch{1}}
  \begin{itemize}
    \item 
  \end{itemize}
  \vspace*{\stretch{1}}
\end{flashcard}

\begin{flashcard}[]{Speicher-Redundanz}
  \vspace*{\stretch{1}}
  \begin{itemize}
    \item innerhalb einer Region
      \begin{itemize}
        \item standardmäßig LRS: bereits drei Replikationen innerhalb \emph{eines} Rechenzentrums
        \item Zonen-Redundanter Speicher: Replikation in drei Verfügbarkeitszonen\newline
          $\Rightarrow$ also verschiedene Rechenzentren
      \end{itemize}
    \item in mehreren Regionen
      \begin{itemize}
        \item zusätzlich wird in eine Kopie in der zweiten Region erstellt
        \item asynchrone Kopie in die zweite Region, innerhalb dann synchron\newline
          RPO: 15 Minuten, aber nicht garantiert
        \item geo-redundanter Speicher: LRS in der primären Region, LRS sekundär
        \item geo-zonen-redundanter Speicher: ZRS primär, LRS sekundär
        \item Daten in der sekundären Region \emph{nicht} für Lesen (und natürlich Schreiben) verfügbar
      \end{itemize}
    \item ebenfalls: zusätzlich Lese-Zugriff in sekundärer Region möglich
  \end{itemize}
  \vspace*{\stretch{1}}
\end{flashcard}

\begin{flashcard}[]{Lesezugriff auf sekundären Speicher}
  \vspace*{\stretch{1}}
  \begin{itemize}
    \item mit RA-SKUs kann aus der sekundären Region gelesen werden
    \item regulär ohne Failover
    \item[!] wegen Asynchronität und RPO möglicherweise verzögert\newline
      $\Rightarrow$ abfragen von \texttt{Last Sync Time}
    \item Zugriff über \texttt{-secondary}-URL
  \end{itemize}
  \vspace*{\stretch{1}}
\end{flashcard}

\begin{flashcard}[]{Hochverfügbarkeit von SQL-Datenbanken}
  \vspace*{\stretch{1}}
  \begin{itemize}
    \item Ziel: 99,99\% Uptime ohne Probleme durch Wartung, \ldots
    \item automatische Updates durch Azure
    \item commitete Änderungen bleiben erhalten
  \end{itemize}
  \vspace*{\stretch{1}}
\end{flashcard}

\begin{flashcard}[]{Azure-Modelle für hochverfügbare Datenbanken}
  \vspace*{\stretch{1}}
  \begin{itemize}
    \item Standard
      \begin{itemize}
        \item hochverfügbarer Speicher (Blob)
        \item Trennung von Prozess (zustandslos!) und Daten\newline
          $\Rightarrow$ erlaubt Absturz des Datenbank-Prozesses ohne Datenverlust
        \item teilweise schlechtere Performance während Wartung
        \item günstiger Preis
        \item Servicelevels: Basic, Standard, General Purpose
        \item demnächst: Zonenredundanz (Speicher ZRS, failover VMs in anderen Zonen)
      \end{itemize}
    \item premium
      \begin{itemize}
        \item Datenbank-Cluster (optional in Verfügbarkeitszonen ohne Aufpreis)
        \item auch während Wartung bleibt die Mehrheit des Clusters verfügbar
        \item teuer, für kritische Anwendungen
      \end{itemize}
  \end{itemize}
  \vspace*{\stretch{1}}
\end{flashcard}

\begin{flashcard}[]{Kontinuität bei SQL-Datenbanken}
  \vspace*{\stretch{1}}
  \begin{itemize}
    \item Sicherungen: komplett (1 Woche), differentiell (12 Stunden), Transaktionslogs (10 Minuten)
    \item temporäre Tabellen erlauben einzelne Einträge wiederherzustellen
    \item gelöschte komplette Datenbanken können wiederhergestellt werden (solagne der Server da ist)
    \item Geo-Replikation mit Failover schützt vor Rechenzentrumsausfall\newline
      auch (Auto-Failover-Gruppe)
  \end{itemize}
  \vspace*{\stretch{1}}
\end{flashcard}

\subsubsectioncard{recommend a solution for autoscaling}

\begin{flashcard}[]{Skalierungsgruppen}
  \vspace*{\stretch{1}}
  \begin{itemize}
    \item automatische Vergrößerung oder Verkleinerung (Anzahl VMs)\newline
      $\Rightarrow$ Reaktion auf schwankende Nachfrage, reduzierte Kosten
    \item eine Gruppe VMs mit Load Balancer
    \item daher hohe Verfügbarkeit, auch Verfügbarkeitsgruppen möglich
    \item einfache, zentrale Verwaltung mehrerer VMs
    \item bis zu 600 (verwaltetes Image)/1000 (angebotene Images) VMs in einer Gruppe
  \end{itemize}
  \vspace*{\stretch{1}}
\end{flashcard}

\begin{flashcard}[]{Verwaltung von Skalierungsgruppen}
  \vspace*{\stretch{1}}
  \begin{itemize}
    \item konsistente Konfiguration, gleiche VM-Einstellungen, Betriemssystem-Image, \ldots
    \item Azure Load Balancer für OSI Level 4 und Application Gateway für Levl 7 Lastverteilung
  \end{itemize}
  \vspace*{\stretch{1}}
\end{flashcard}

\begin{flashcard}[]{Automatische Skalierung}
  \vspace*{\stretch{1}}
  \begin{itemize}
    \item automatische Anpassung von Ressourcen an schwankenden Bedarf
    \item vertikale Skalierung: (hoch/runter skalieren)\newline
      Kapazität verändern, z.\,B. VM size
    \item horizontale Skalierung: (in/out skalieren)\newline
      Anzahl Ressourcen erhöhen
    \item Nachteile vertikale Skalierung: mögliche Auszeiten
  \end{itemize}
  \vspace*{\stretch{1}}
\end{flashcard}

\begin{flashcard}[]{Services mit automatischer Skalierung}
  \vspace*{\stretch{1}}
  Regelbasiert:
  \begin{itemize}
    \item Skalierungsgruppen
    \item Service Fabric
    \item App Service
    \item Cloud Services
  \end{itemize}
  Vollautomatisch:
  \begin{itemize}
    \item Functions
  \end{itemize}
  \vspace*{\stretch{1}}
\end{flashcard}

\begin{flashcard}[]{Automatische Skalierung}
  \vspace*{\stretch{1}}
  Allgemein mit Azure Monitor
  \begin{itemize}
    \item Zeitplan (Wochentage/Wochenende)
    \item nach CPU-Auslastung
    \item Größe der Aufträge in Warteschlangen
    \item \ldots andere messbare Größen
    \item auch benutzerdefinierte Metriken (Über Application Insights)
  \end{itemize}
  $\Rightarrow$ Auch Kombinationen möglich
  \begin{itemize}
    \item Fluktuation (abwechselndes Hoch- und Runterskalieren)
    \item Minimum und Maximum müssen festgelegt sein
    \item[!] manuelle Grenzüberschreitung gilt immer nur temporär
  \end{itemize}
  
  \vspace*{\stretch{1}}
\end{flashcard}

\begin{flashcard}[]{Skalierung von Datenbanken}
  \vspace*{\stretch{1}}
  \begin{itemize}
    \item Skalierung möglich während des Betriebs (z.\,B. DTUs, vCores (vertikale Skalierung))
    \item[!] Transaktionen können fehlschlagen
    \item Lese-Skalierung: zusätzliche Nur-Lesen-Replikationen
    \item sharding: Aufteilen der Datenbank auf mehrere Server (horizontale Skalierung)
  \end{itemize}
  \vspace*{\stretch{1}}
\end{flashcard}

\subsubsectioncard{identify resources that require high availability}

\begin{flashcard}[]{Hochverfügbarkeit}
  \vspace*{\stretch{1}}
  \begin{itemize}
    \item verfügbarkeit (als Prozent), SLA, z.\,B. 99 recommend a solution for recovery in different regions.99\%
    \item durchschnittliche Zeit bis zur Wiederherstellung (MTTR)
    \item Vergleich mit SLAs von Azure
  \end{itemize}
  \vspace*{\stretch{1}}
\end{flashcard}

\begin{flashcard}[]{Datenspeicher überprüfen}
  \vspace*{\stretch{1}}
  \begin{itemize}
    \item überprüfung jedes Speichers, ob erforderliche Verfügbarkeit vorhanden ist
    \item falls nicht: möglicherweise skalieren, SKU wechseln, \ldots
    \item Service Wechseln (z.\,B. verwaltete Datenbank)
    \item Backups in anderen Regionen erstellen
    \item Anwendungen auf Failover vorbereiten
  \end{itemize}
  \vspace*{\stretch{1}}
\end{flashcard}

\begin{flashcard}[]{Berechnung}
  \vspace*{\stretch{1}}
  \begin{itemize}
    \item Analyse von VMs, Instanzen, etc
    \item Backups erstellen
    \item Skalierungsgruppen, Fehler-Domänen, Update-Domänen einführen
    \item Verfügbarkeitszonen
    \item Anwendungen auf kurze Timeouts (z.\,B. bei Skalierung) vorbereiten
  \end{itemize}
  \vspace*{\stretch{1}}
\end{flashcard}

\begin{flashcard}[]{Key Vault-Verfügbarkeit}
  \vspace*{\stretch{1}}
  \begin{itemize}
    \item[!] Schlüsseltresore müssen verfügbar sein, um Anwendungen, die darauf zugreifen, zu unterstützen
    \item Key-Vault-Daten werden automatisch in der Partner-Region gesichert (Redundanz)
    \item wenn eine Region nicht mehr verfügbar ist, wird automatisch ein Failover durchgeführt
    \item Wartezeit wenige Minuten\newline
      $\Rightarrow$ während des Failovers ist der Schlüsseltresor im Nur-Lesen-Modus
  \end{itemize}
  \vspace*{\stretch{1}}
\end{flashcard}

\subsubsectioncard{identify storage types for high availability}

\begin{flashcard}[]{Hochverfügbarkeit von Speicherkonten}
  \vspace*{\stretch{1}}
  \begin{itemize}
    \item generell haben Speicherkonten hohe Verfügbarkeit
    \item Verfügbarkeitszonen
    \item kann verbessert werden durch Geo-redundante Speicherung\newline
      $\Rightarrow$ automatischer Failover (general purpose and Blob storage, \emph{nicht} für hierarchischen Speicher)
    \item GRS/GZRS als Backup vs. RA-GRS/RA-GZRS mit zusätzlichem zur-Lesen-Speicher
    \item[!] sämtliche Maßnahmen werden nur von Blob-Speicher erfüllt\newline
      $\Rightarrow$ Files, Tables, Queues nur eingeschränkt. Ebenso Premium-Blob und Konten mit WORM-Richtlinien
    \item verwaltete Datenträger
  \end{itemize}
  \vspace*{\stretch{1}}
\end{flashcard}
