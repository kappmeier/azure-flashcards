\subsectioncard{Designen von Anwendungsarchitekturen}

\subsubsectioncard{recommend microservices architecture including Event Grid, Event Hubs, Service Bus, Azure Queue Storage, Logic Apps, Azure Functions, Service Fabric, AKS, Azure App Configuration and webhooks}

\begin{flashcard}[]{Microservice-Architektur}
  \vspace*{\stretch{1}}
  \begin{itemize}
    \item kleine, unabhängigige Dienste
    \item nur lose Abhängigkeiten
    \item unabhängige Bereitstellung
    \item verantwortlich für eigene Daten
    \item Kommunikation über API
  \end{itemize}
  \vspace*{\stretch{1}}
\end{flashcard}

\begin{flashcard}[]{Umgebung einer Mikroservice-Architektur}
  \vspace*{\stretch{1}}
  \begin{itemize}
    \item Orchestrierungsdienst:
      \begin{itemize}
        \item Verwaltung
        \item Lastausgleich
        \item Fehlerbehandlung
      \end{itemize}
    \item API-Gateway:\newline
      \begin{itemize}
        \item Verbindung nach Außen (zu Kunden)
        \item Trennung von Diensten und Kunden
        \item SSL-Terminierung
        \item Unterstützung für nicht-HTTP-Protokolle (z\,B. AMQP)
      \end{itemize}
  \end{itemize}
  \vspace*{\stretch{1}}
\end{flashcard}

\begin{flashcard}[]{Vorteile von Mikroservices}
  \vspace*{\stretch{1}}
  \begin{itemize}
    \item agile Entwicklung (da geringe Abhängigkeit)
    \item kleine Teams, DevOps
    \item kleine Codebase und Technologiemix
    \item Isolierung im Fehlerfall
    \item Isolierung der Anwendungsdaten
    \item Skalierbarkeit
  \end{itemize}
  \vspace*{\stretch{1}}
\end{flashcard}

\begin{flashcard}[]{Probleme bei Mikroservices}
  \vspace*{\stretch{1}}
  \begin{itemize}
    \item Komplexität: einzelne Dienste gering, aber Zusammenwirken groß!
    \item Entwicklung gemeinsamer Komponenten und Testen
    \item Netzwerkbelastung durch hohe API-Auslastung
    \item Bereitstellung und Verwaltung (DevOps!)
    \item Isolierung und Hoheit der Daten erfordert Sicherstellen der Datenintegrität
  \end{itemize}
  \vspace*{\stretch{1}}
\end{flashcard}

\begin{flashcard}[]{Event Grid}
  \vspace*{\stretch{1}}
  \begin{itemize}
    \item Dienst für ereignisbsierte Anwendungen
    \item viele Azure-Dienste als Quellen
    \item Ereignis-Handler oder WebHooks
  \end{itemize}
  \vspace*{\stretch{1}}
\end{flashcard}

\begin{flashcard}[]{Ablauf der Ereignsbehandlung}
  \vspace*{\stretch{1}}
  \begin{enumerate}
    \item ein \emph{Ereignis} tritt an einer \emph{Quelle} auf
    \item das Ereignis wird an einem \emph{Topic} veröffentlicht
    \item über \emph{Ereignis-Abonnements} wird das Ereignis an \emph{Handler} weitergeleitet
  \end{enumerate}
  \vspace*{\stretch{1}}
\end{flashcard}

\begin{flashcard}[]{Features von Event Grid}
  \vspace*{\stretch{1}}
  \begin{itemize}
    \item einfaches Aufsetzen (inklusive graphischer Oberfläche)
    \item vielzahl von Filtermöglichkeiten um die korrekten Handler zu erreichen
    \item viele Ziele für ein Event
    \item Cloud-Vorteile: Verfügbarkeit, hoher Durchsatz
    \item zahllose Vordefinierte Ereignisse, erweiterbar durch eigene für Anwendungen
  \end{itemize}
  \vspace*{\stretch{1}}
\end{flashcard}

\begin{flashcard}[]{Szenarien für Event Grid}
  \vspace*{\stretch{1}}
  \begin{itemize}
    \item Serverlose Architekturen:\newline
      Auslösen von Funktionen, Datenspeichern, \ldots
    \item Automatisierung\newline
      Auslösen von Bereitstellungen, Aktualisierungen, operationelle Aufgaben
    \item Anwendungsintegration:\newline
      vorhandene Daten bearbeiten (z.\,B. Funktionen, Logik-Apps, \ldots) oder Daten von Anwendungen bereitstellen
  \end{itemize}
  \vspace*{\stretch{1}}
\end{flashcard}

\begin{flashcard}[]{Event Hub}
  \vspace*{\stretch{1}}
  \begin{itemize}
    \item Dienst zur Verarbeitung von großen Mengen an Ereignissen
    \item Kafka in Azure (unterstützt auch Kafka-Prokoll)
    \item verfügbar über DNS-Netzwerk-Endpunkte (inklusive Filter, \ldots)
  \end{itemize}
  \vspace*{\stretch{1}}
\end{flashcard}

\begin{flashcard}[]{Event-Hub-Ereignisse}
  \vspace*{\stretch{1}}
  \begin{itemize}
    \item veröffentlicht über HTTPS, AMQP, oder Kafka
    \item hohe Performance, aber aufwendiger mit AMQP
    \item jedes Ereignis wird mit \emph{Partition-Key} veröffentlicht (alternativ: Round-Robin)\newline
      $\Rightarrow$ zusammengehörige Ereignisse kommen in die gleiche Partition
    \item alternativ: Publisher-Richtlinien
    \item Speicherdauer: 1 tag bis 7 Tage (90 für Premium)
    \item Archivieren: Integration in Azure Synapse und Speicherkonten (mit \emph{Event Hub Capture})\newline
      $\Rightarrow$ Ereignisströme sollen nciht als Datensenke oder Langzeitspeicher genutzt werden!
  \end{itemize}
  \vspace*{\stretch{1}}
\end{flashcard}

\begin{flashcard}[]{Event-Hub-Partitionen}
  \vspace*{\stretch{1}}
  \begin{itemize}
    \item Sequenzen von Ereignissen werden in Partitionen sortiert
    \item Durchsatz für geordnete Events nur pro Partition garantiert
    \item Anzahl Ereignisse kann zu hoch sein für Anwendungen\newline
      $\Rightarrow$ einfache Skalierung auf mehrere Prozesse, die einzelne Partitionen bearbeiten
    \item Anzahl der Partitionen wird festgelegt und ist \emph{nicht Änderbar}
    \item Kosten unabhängig von der Anzahl der Partitionen
  \end{itemize}
  \vspace*{\stretch{1}}
\end{flashcard}

\begin{flashcard}[]{Event-Hub-Ereignisse konsumieren}
  \vspace*{\stretch{1}}
  \begin{itemize}
    \item Konsumption durch Consumer-Gruppen
    \item jede Gruppe hat eigene Offsets und kann in eigener Geschwindigkeit konsumieren
    \item mehrere (bis 5) Leser pro Consumer-Gruppe möglich
  \end{itemize}
  \vspace*{\stretch{1}}
\end{flashcard}


\begin{flashcard}[]{Service Bus}
  \vspace*{\stretch{1}}
  \begin{itemize}
    \item Nachrichten-Broker mit Warteschlangen
    \item Abruf und Einreihen der Nachrichten in Topics/Namespaces
  \end{itemize}
  \vspace*{\stretch{1}}
\end{flashcard}

\begin{flashcard}[]{Szenarien für Service Bus}
  \vspace*{\stretch{1}}
  \begin{itemize}
    \item Nachrichtenversand
    \item Entkopplung von Anwendungen (sicherer Versand)
    \item Lastausgleich (z.\,B. bei Anwendungen mit vielen Knoten)
    \item \texttt{1:n}-Beziehungen über Veröffentlichungen und Abonnements
    \item Koordination von Arbeitsschritten mit Transaktionen\newline
      (blockieren weiter Schritte durch belassen der Nachricht in der Warteschlange)
  \end{itemize}
  \vspace*{\stretch{1}}
\end{flashcard}

\begin{flashcard}[]{Hintergrund von Service Bus}
  \vspace*{\stretch{1}}
  \begin{itemize}
    \item Nachrichten sortiert nach Zeitstempel in Warteschlange
    \item gesichert über Verfügbarkeitszonen
    \item[!] Nachrichten werden nur auf Nachfrage abgerufen
    \item optional Topics zum für mehrere Abnehmer
    \item Verwaltung mehrerer Warteschlangen und Topics in Namensbereichen
  \end{itemize}
  \vspace*{\stretch{1}}
\end{flashcard}

\begin{flashcard}[]{Features von Service Bus}
  \vspace*{\stretch{1}}
  \begin{itemize}
    \item FIFO-Unterstützung durch Sitzungen
    \item Forwading durch verbinden mehrerer Warteschlangen
    \item terminierte/zurückgehaltene Nachrichten
    \item Senke für Nachrichten ohne Empfänger
    \item Zurückhalten von Nachrichten auf Empfängerwunsch
    \item fortgeschrittene Filtern
    \item automatisches Löschen und Erkennung von Duplikaten
    \item failover in andere Regionen
  \end{itemize}
  \vspace*{\stretch{1}}
\end{flashcard}

\begin{flashcard}[]{Azure Warteschlangen}
  \vspace*{\stretch{1}}
  \begin{itemize}
    \item teil von Speicherkonten
    \item millionen von Elementen, aber nur 64 KiB
    \item Lebensdauer von Nachrichten beliebig
    \item 
  \end{itemize}
  \vspace*{\stretch{1}}
\end{flashcard}

\begin{flashcard}[]{Logik-App in Anwendungen}
  \vspace*{\stretch{1}}
  \begin{itemize}
    \item können von Wehooks aufgerufen werden (gut für Integration in Mikroservices, alternativ auch bei anderen Bedingungen)
    \item über Konnektoren können andere Dienste aufgerufen werden
      $\Rightarrow$ verbindet unterschiedliche Services
    \item Workflow: eine Reihe von Schritten
    \item Design ohne Code, kann von Business oder Product Management gemacht werden
  \end{itemize}
  \vspace*{\stretch{1}}
\end{flashcard}

\begin{flashcard}[]{Azure Functions als Microservice}
  \vspace*{\stretch{1}}
  \begin{itemize}
    \item jede Function kan als Microservice angesehen werden
    \item Aufruf über HTTP(S)/API
    \item automatische Skalierung, serverlos
  \end{itemize}
  \vspace*{\stretch{1}}
\end{flashcard}

\begin{flashcard}[]{Runbooks mit WebHook}
  \vspace*{\stretch{1}}
  \begin{itemize}
    \item kann als Element einer Mikroservice-Architektur eingesetzt werden
    \item startet ein Runbook in Azure Automation über eine HTTP(S)-Anfrage
      $\Rightarrow$ erlaubt Zugriff auf Runbooks ohne Azure-Automation-API zu implementieren
    \item ieingabe von \texttt{WebHookData} als JSON im HTTP-Request
    \item Zugriff über URL mit enthaltendem Token (kein Login notwendig)\newline
      $\Rightarrow$ nicht für kritische Operationen!
  \end{itemize}
  \vspace*{\stretch{1}}
\end{flashcard}

\subsubsectioncard{recommend an orchestration solution for deployment and maintenance of applications including ARM templates, Azure Automation, Azure Pipelines, Logic Apps, or Azure Functions}

\begin{flashcard}[]{}
  \vspace*{\stretch{1}}
  \begin{itemize}
    \item
  \end{itemize}
  \vspace*{\stretch{1}}
\end{flashcard}

\begin{flashcard}[]{}
  \vspace*{\stretch{1}}
  \begin{itemize}
    \item
  \end{itemize}
  \vspace*{\stretch{1}}
\end{flashcard}

\begin{flashcard}[]{}
  \vspace*{\stretch{1}}
  \begin{itemize}
    \item
  \end{itemize}
  \vspace*{\stretch{1}}
\end{flashcard}

\begin{flashcard}[]{}
  \vspace*{\stretch{1}}
  \begin{itemize}
    \item
  \end{itemize}
  \vspace*{\stretch{1}}
\end{flashcard}

\begin{flashcard}[]{}
  \vspace*{\stretch{1}}
  \begin{itemize}
    \item
  \end{itemize}
  \vspace*{\stretch{1}}
\end{flashcard}

\begin{flashcard}[]{}
  \vspace*{\stretch{1}}
  \begin{itemize}
    \item
  \end{itemize}
  \vspace*{\stretch{1}}
\end{flashcard}


\subsubsectioncard{recommend a solution for API integrations}

\begin{flashcard}[]{}
  \vspace*{\stretch{1}}
  \begin{itemize}
    \item
  \end{itemize}
  \vspace*{\stretch{1}}
\end{flashcard}


\begin{flashcard}[]{API-Management}
  \vspace*{\stretch{1}}
  \begin{itemize}
    \item
  \end{itemize}
  \vspace*{\stretch{1}}
\end{flashcard}

\begin{flashcard}[]{API-Management in virtuellen Netzwerken}
  \vspace*{\stretch{1}}
  \begin{itemize}
    \item API-Management kann in einem virtuellen Netzwerk bereitgestellt werden\newline
      $\Rightarrow$ um Backend-Dienste im Netzwerk zu verbinden
    \item VMs im selben virtuellen Netzwerk können über API-Management-Instanz erreicht werden
    \item Modi:
      \begin{itemize}
        \item extern: API-Management-Endpunkte (Entwickler-Portal, API-Gateway, \ldots) können vom Internet erreicht werden
        \item intern: Endpunkte (API-Gateway, Entwickler-Portal, Direct management, Git) können nur aus dem virtuellen Netz erreicht werden
      \end{itemize}
  \end{itemize}
  \vspace*{\stretch{1}}
\end{flashcard}

\begin{flashcard}[]{}
  \vspace*{\stretch{1}}
  \begin{itemize}
    \item
  \end{itemize}
  \vspace*{\stretch{1}}
\end{flashcard}
\begin{flashcard}[]{}
  \vspace*{\stretch{1}}
  \begin{itemize}
    \item
  \end{itemize}
  \vspace*{\stretch{1}}
\end{flashcard}
\begin{flashcard}[]{}
  \vspace*{\stretch{1}}
  \begin{itemize}
    \item
  \end{itemize}
  \vspace*{\stretch{1}}
\end{flashcard}
