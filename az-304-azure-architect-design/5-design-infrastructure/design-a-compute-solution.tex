\subsectioncard{Designen von Prozessierungs-Lösungen}

\subsubsectioncard{recommend a solution for compute provisioning}

\begin{flashcard}[]{Azure Rechendienste}
  \vspace*{\stretch{1}}
  \begin{itemize}
    \item Virtuelle Maschinen (VMs)
    \item Azure Batch
    \item App Service
    \item Azure Functions
    \item Azure Container Instances (ACI)
    \item Service Fabric
    \item Azure Spring Cloud
    \item Red Hat Openshift
    \item Azure Kubernetes Service (AKS)
  \end{itemize}
  \vspace*{\stretch{1}}
\end{flashcard}

\begin{flashcard}[]{Container-basierte Dienste}
  \vspace*{\stretch{1}}
  \begin{itemize}
    \item Azure Container Instances (ACI)
    \item Service Fabric
    \item Azure Spring Cloud
    \item Red Hat Openshift
    \item Azure Kubernetes Service (AKS)
  \end{itemize}
  \vspace*{\stretch{1}}
\end{flashcard}

\begin{flashcard}[]{Migration von Prozessen}
  \vspace*{\stretch{1}}
  \begin{itemize}
    \item Lift \& Shift
      \begin{itemize}
        \item ohne neues Design von anwendungen
        \item nutzt PaaS und IaaS
        \item geeignete Dienste: VMs und App Service
      \end{itemize}
    \item Cloud-Optimiert
      \begin{itemize}
        \item Refactoring von Anwendungen um von Cloud-Diensten zu profitieren
        \item neue Anwendungen
      \end{itemize}
  \end{itemize}
  \vspace*{\stretch{1}}
\end{flashcard}

\begin{flashcard}[]{Zustandsspeicher}
  \vspace*{\stretch{1}}  
  \begin{itemize}
    \item Alle Compute-Lösungen können zustandslos betrieben werden
    \item zusätzlich zustandsbasiert sind:
      \begin{itemize}
        \item VMs
        \item Service Fabric
        \item AKS
      \end{itemize}
  \end{itemize}
  \vspace*{\stretch{1}}
\end{flashcard}

\begin{flashcard}[]{Web-Hosting}
  \vspace*{\stretch{1}}  
  \begin{itemize}
    \item Eingebaut:
      \begin{itemize}
        \item App Service
        \item Spring Cloud
        \item AKS
      \end{itemize}
    \item manuell verfügbar:
      \begin{itemize}
        \item VMs
        \item Service Fabric
        \item AKS
        \item ACI
      \end{itemize}
  \end{itemize}
  \vspace*{\stretch{1}}
\end{flashcard}

\begin{flashcard}[]{Skalierbarkeit von Rechendienste}
  \vspace*{\stretch{1}}  
  \begin{itemize}
    \item Manuell:
      \begin{itemize}
        \item VMs (auch teilautomatisiert über Skalierungsgruppen)
      \end{itemize}
    \item unterstützt:
      \begin{itemize}
        \item App Service
        \item Spring Cloud
        \item AKS
        \item Azure Batch
      \end{itemize}
    \item serverlos:
      \begin{itemize}
        \item Azure Functions
      \end{itemize}
  \end{itemize}
  \vspace*{\stretch{1}}
\end{flashcard}

\begin{flashcard}[]{Unterstützte Architekturen}
  \vspace*{\stretch{1}}
  \begin{itemize}
    \item Mikroservices:\newline
      Spring Cloud, Service Fabric, Functions, AKS, ACI
    \item Allgemein/hierarchische Ebenen:\newline
      VMs, App Service, Azure Batch, ACI
    \item Ereignisbasiert:\newline
      Service Fabric, Functions, AKS
    \item Sonstiges:\newline
      Spring Cloud, ACI, App Service
  \end{itemize}
  \vspace*{\stretch{1}}
\end{flashcard}

\subsubsectioncard{determine appropriate compute technologies, including virtual machines, App Services, Service Fabric, Azure Functions, Azure Virtual Desktop, Batch, HPC and containers}

\begin{flashcard}[]{App Service}
  \vspace*{\stretch{1}}
  \begin{itemize}
    \item Hosting von Web-Anwendungen, REST-APIs, \ldots
    \item viele Programmiersprachen verfügbar
    \item profitieren von Azure-Services (Logging, Sicherheit, DevOps, \ldots)
    \item Zahlung nach genutzten Ressourcen gemäß \emph{App Service Plan}
  \end{itemize}
  \vspace*{\stretch{1}}
\end{flashcard}

\begin{flashcard}[]{Vorteile App Service}
  \vspace*{\stretch{1}}
  \begin{itemize}
    \item viele Umgebungen nativ unterstützt (Sprachen, Betriebssysteme)
    \item komplett verwaltete Systeme, kein Bedarf für Betriebssystem- oder Framework-Updates
    \item Unterstützung von Containern
    \item automatische Skalierung
    \item Integration in DevOps (dev, test-Deployments), Enterprise-Umgebungen, \ldots
    \item automatische Sicherheit, Richtlinieneinhaltung, \ldots
  \end{itemize}
  \vspace*{\stretch{1}}
\end{flashcard}

\begin{flashcard}[]{App Service Environment}
  \vspace*{\stretch{1}}
  \begin{itemize}
    \item dedizierte Umgebung
    \item nicht geteilt mit anderen Anwendern
    \item höhere Unterstützung von Regeln, Sicherheit, \ldots
    \item Multi-Tier-Anwendungen in der gleichen Umgebung
    \item unterstützung von Zonen-Redundanz
  \end{itemize}
  \vspace*{\stretch{1}}
\end{flashcard}

\begin{flashcard}[]{App Service Plan}
  \vspace*{\stretch{1}}
  \begin{itemize}
    \item beschreiben Ressourcen, die für App Service (und Functions) genutzt werden können
    \item mehrere Apps in den gleichen Ressourcen
    \item Eigenschaften: Betriebssystem, Region, VMs, Preisstufe (definiert Features)
  \end{itemize}
  \vspace*{\stretch{1}}
\end{flashcard}

\begin{flashcard}[]{App Service Plan Preisstufen}
  \vspace*{\stretch{1}}
  \begin{itemize}
    \item gemeinsam genutzt: Free, Shared\newline
      Anwendungen teilen sich Server mit anderen Kunden
    \item dediziert: Basic, Standard, PremiumV\texttt{x}\newline
      dedizierte VMs (Isolation der Rechenkapazität)
    \item isoliert: IsolatedV\texttt{x}\newline
      zusätzlich dediziertes Netzwerk (schnellste und weiteste Skalierung)
  \end{itemize}
  \vspace*{\stretch{1}}
\end{flashcard}

\begin{flashcard}[]{Azure Batch}
  \vspace*{\stretch{1}}
  \begin{itemize}
    \item Lösung für paralleles Rrechnen großer Aufträge
    \item BatchJobs
    \item automatische Skalierung
    \item unterstützung für automatisches VM-Management
    \item Bezahlung nur der genutzten Ressourcen für die Berechnung
  \end{itemize}
  \vspace*{\stretch{1}}
\end{flashcard}

\begin{flashcard}[]{Jobs für Azure Batch}
  \vspace*{\stretch{1}}
  \begin{itemize}
    \item leicht und stark parallelisierbare Jobs\newline
      $\Rightarrow$ Aufträge gut aufteilbar
    \item Aufträge sollen unabhängig voneinander arbeiten\newline
      $\Rightarrow$ keine Interprozess-Kommunikation, aber natürlich gemeinsam genutzte Daten
    \item Beispiele:
      \begin{itemize}
        \item Simulation
        \item Grafikrendering \& Medienbearbeitung
        \item Daten-Prozessierung
      \end{itemize}
  \end{itemize}
  \vspace*{\stretch{1}}
\end{flashcard}

\begin{flashcard}[]{Azure Batch für verbundene Aufträge}
  \vspace*{\stretch{1}}
  \begin{itemize}
    \item Verbindung von Batch-Aufträgen mit Nachrichten-Versand\newline
      z.\,B. Microsoft MPI
    \item Training von KI-Modellen, physikalische Modellierungen
  \end{itemize}
  \vspace*{\stretch{1}}
\end{flashcard}

\begin{flashcard}[]{Einbindung in Azure-Abläufe}
  \vspace*{\stretch{1}}
  \begin{itemize}
    \item Unterstützung für Rechenjobs in Data Factory
    \item Tools für Rendering
  \end{itemize}
  \vspace*{\stretch{1}}
\end{flashcard}

\begin{flashcard}[]{Azure Container Instances}
  \vspace*{\stretch{1}}
  \begin{itemize}
    \item einfacher möglichkeit Container in Azure auszuführen
    \item keine Verwaltung virtueller Maschinen notwendig
    \item für isolierte Aufgaben\newline
      $\Rightarrow$ insbesondere keine automatische Erkennung von anderen Containern
    \item Unterstützung von Linux und Windows Docker containern von Docker oder Azure Container Registries
  \end{itemize}
  \vspace*{\stretch{1}}
\end{flashcard}

\begin{flashcard}[]{Container-Eigenschaften}
  \vspace*{\stretch{1}}
  \begin{itemize}
    \item öffentlich erreichbar mit DNS-Eintrag
    \item verfügbare Größen: maximal 15 GiB, 4 Cores
    \item Unterstützung von virtuellen Netzwerken in vielen Regionen
    \item Mounting von Azure Files
  \end{itemize}
  \vspace*{\stretch{1}}
\end{flashcard}

\begin{flashcard}[]{Azure Functions}
  \vspace*{\stretch{1}}
  \begin{itemize}
    \item serverlose Ausführung von Code\newline
      $\Rightarrow$ keine Serververwaltung
    \item mit wenig Code viel kostengünstig erreichen
    \item automatische Skalierung
  \end{itemize}
  \vspace*{\stretch{1}}
\end{flashcard}

\begin{flashcard}[]{Anwendungsszenarien von Functions}
  \vspace*{\stretch{1}}
  \begin{itemize}
    \item Reagieren auf Speicheränderungen (z.\,B. Blob-Upload)
    \item APIs (über HTTP-Trigger)
    \item Tasks (über terminierte Ausführungen)
    \item Verkettung von Funktionen für Workflows
    \item Reagieren auf Ereignisse in Warteschlangen, Hubs, \ldots
    \item jede Art von On-Demand-Ausführung mit Ereignissen: IoT, Datenbankänderungen, \ldots
  \end{itemize}
  \vspace*{\stretch{1}}
\end{flashcard}

\begin{flashcard}[]{Service Fabric}
  \vspace*{\stretch{1}}
  \begin{itemize}
    \item verteilte Plattform für Microservices
    \item Container-basiert (Container-Orchestrator)
    \item leicht skalierbar
    \item unterstützung für zustandslose und zustandsbehaftete Services
    \item Unterstützung für DevOps and kontinuierliche Bereitstellung
    \item Java (nur Linux) und .NET
  \end{itemize}
  \vspace*{\stretch{1}}
\end{flashcard}

\begin{flashcard}[]{Virtuelle Maschinen}
  \vspace*{\stretch{1}}
  \begin{itemize}
    \item komplett konfigurierbar
    \item unterstützen zahllose Betriebssysteme (Windows, Linux, MacOS) in vielen Ausprägungen
    \item Einbindung ins Netzwerk
    \item skalierbar über Skalierungsgruppen (Semi-Automatisch)
    \item unterstützen Lift \& Shift-Szenarien
  \end{itemize}
  \vspace*{\stretch{1}}
\end{flashcard}

\begin{flashcard}[]{Virtueller Desktop}
  \vspace*{\stretch{1}}
  \begin{itemize}
    \item vollständiges Windows Server, 7 und 10, 11 in der Cloud
    \item oder Microsoft 365-Anwendungen und andere Anwendungen
    \item unterstützung von Multi-Anwender-Systemen
    \item erlaubt Migration von existierenden RDS und Windows Server-Systemen
  \end{itemize}
  \vspace*{\stretch{1}}
\end{flashcard}

\begin{flashcard}[]{Eigenschaften virtueller Desktops}
  \vspace*{\stretch{1}}
  \begin{itemize}
    \item kostengünstig skalierbar mit Pools
    \item eigene Images
    \item persistente Desktops
    \item Zugriffsrichtlinien und Sicherheitsfunktionen
  \end{itemize}
  \vspace*{\stretch{1}}
\end{flashcard}

\subsubsectioncard{recommend a solution for containers}

\begin{flashcard}[]{Container in Azure}
  \vspace*{\stretch{1}}
  \begin{itemize}
    \item ACI
      \begin{itemize}
        \item verwaltetes Kubernetes-Cluster
        \item nur benutzte Knoten müssen gezahlt werden
        \item von Azure bereitgestellte Sicherheit, Logging, Monitoring
      \end{itemize}
    \item Kubernetes
      \begin{itemize}
        \item einfach und schnell einzelne Container bereitstellen
        \item einfacher als VMs zu bedienen
        \item nicht ganz so flexibel wie AKS
      \end{itemize}
  \end{itemize}
  \vspace*{\stretch{1}}
\end{flashcard}

\begin{flashcard}[]{Eigenschaften AKS}
  \vspace*{\stretch{1}}
  \begin{itemize}
    \item Identität \& Sicherheit: Kubernetes RBAC, integration in Azure AD
    \item Logging in Log Analytics, Service Container Health
    \item jede beliebige VM-Größen (+ alle VM-Features)
    \item automatische Skalierung von Knoten und Pods
    \item Unterstützung von GPU-Unterstützung in Knoten
    \item vollständige Integration in Azure-Netzwerke
  \end{itemize}
  \vspace*{\stretch{1}}
\end{flashcard}

\subsubsectioncard{recommend a solution for automating compute management}

\begin{flashcard}[]{Azure Automation}
  \vspace*{\stretch{1}}
  \begin{itemize}
    \item Bereiche für Automatisierung:
      \begin{itemize}
        \item Bereitstellung von Infrastruktur/Ressourcen
        \item Reagieren auf Anforderungen (Warnungen, \ldots)
        \item Orchestrierung mehrerer Produkte
      \end{itemize}
  \end{itemize}
  \vspace*{\stretch{1}}
\end{flashcard}

\begin{flashcard}[]{Prozess-Automatisierung}
  \vspace*{\stretch{1}}
  \begin{itemize}
    \item automatisierung einfacher Prozesse
    \item reduziert Fehleranfäligkeit
    \item Unterstützung in Azure Runbooks\newline
      \begin{itemize}
        \item graphisch
        \item PowerShell
        \item Python
      \end{itemize}
    \item werden direkt auf VMs in der Umgebung ausgeführt
    \item ausgeführt durch Webhooks
  \end{itemize}
  \vspace*{\stretch{1}}
\end{flashcard}

\begin{flashcard}[]{Konfigurationsverwaltung}
  \vspace*{\stretch{1}}
  \begin{itemize}
    \item Nachverfolgung von Änderungen
      \begin{itemize}
        \item Alarmierung im Fall ungewünschter/unerwarteter Änderungen
        \item unterstützt viele Dämonen, Services, Registrierungen, Konfigurationsdateien, \ldots
      \end{itemize}
    \item Verwaltung des gewünschten Status (DSC)
      \begin{itemize}
        \item Statusverwaltung über Server in Azure
      \end{itemize}
  \end{itemize}
  \vspace*{\stretch{1}}
\end{flashcard}

\begin{flashcard}[]{Updateverwaltung}
  \vspace*{\stretch{1}}
  \begin{itemize}
    \item automatische Aktualisierung von VMs
    \item Reaktion auf Anforderungsänderungen (in der Cloud und lokal)
    \item geplante Bereitstellungen im Wartungsbereich
  \end{itemize}
  \vspace*{\stretch{1}}
\end{flashcard}

\begin{flashcard}[]{Automatisierungs-Features}
  \vspace*{\stretch{1}}
  \begin{itemize}
    \item heterogene Systeme (Linux, Windwos, Cloud und lokal (über \emph{Arc}))
    \item Aufgaben (VM runterfahren, monatlich, \ldots)
    \item Bereitstellung von Ressourcen (inklusive Integration in Tools wie Jenkins und DevOps)
    \item Ressourcen-Management (jede Art)
    \item Test-unterstützung
    \item Nachverfolgung von Änderung, Inventarerstellung
    \item Integration in andere Services:\newline
      Arc, Monitor und Alerts, Policy, Site Recovery
    \item Aufruf über Azure Services:\newline
      Logik-Apps, Power Apps, Event Grid, Power Automate
  \end{itemize}
  \vspace*{\stretch{1}}
\end{flashcard}

\begin{flashcard}[]{Logik-apps}
  \vspace*{\stretch{1}}
  \begin{itemize}
    \item automatisierte Workflows
    \item hervorragende Integration in Anwendungen, Daten, Services und Cloud-Systeme
    \item einfach skalierbar (serverlos)
    \item Verbindung von Legacy, Drittanbieter- und eigenen Anwendungen
    \item grafischer Designer (zur Benutzung von (Produkt-)Management)
    \item automatisierte Deployments über ARM-Vorlagen
    \item manuelle Schritte im Workflow (z.\,B. über Email oder Slack-Integration)
  \end{itemize}
  \vspace*{\stretch{1}}
\end{flashcard}
