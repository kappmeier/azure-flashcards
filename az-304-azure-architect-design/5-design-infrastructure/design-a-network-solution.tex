\subsectioncard{Designen von Netzwerk-Lösungen}

\subsubsectioncard{recommend a network architecture (hub and spoke, Virtual WAN)}

\begin{flashcard}[]{Hub-And-Spoke-Netzwerktopologie}
  \vspace*{\stretch{1}}
  \begin{itemize}
    \item ein virtuelles Hub-Netzwerk fungiert als zentrale Verbindung
    \item viele virtuelle Spoke-Netzwerke sind mit dem Hub verbunden
    \item auch lokale Netzwerke sind mit dem Hub verbunden
    \item Peering zwischen Hub und anderen Netzwerken
  \end{itemize}
  $\Rightarrow$ isolation von Workloads in Spoke-Netzwerken
  \vspace*{\stretch{1}}
\end{flashcard}

\begin{flashcard}[]{Infrastruktur im Hub}
  \vspace*{\stretch{1}}
  \begin{itemize}
    \item Gateway um lokale Netzwerke zu verbinden
    \item Sicherheitseinrichtungen
      \begin{itemize}
        \item Bastions-Hosts
        \item Firwall
        \item DMZ
      \end{itemize}
  \end{itemize}
  \vspace*{\stretch{1}}
\end{flashcard}

\begin{flashcard}[]{Szenarien für Hub-and-Spoke}
  \vspace*{\stretch{1}}
  \begin{itemize}
    \item getrennte Umgebungen (in Spoke-Netzen), die gemeinssame Dienste nutzen (im Hub-Netzwerk)
    \item keine Verbindung zwischen Services notwendig (in Spoke-Netzen)
    \item zentral verwaltete Sicherheitsmaßnahmen (Firewall, Hub als DMZ)
  \end{itemize}
  \vspace*{\stretch{1}}
\end{flashcard}

\begin{flashcard}[]{Beispielarchitektur für Hub-and-Spoke}
  \vspace*{\stretch{1}}
  \begin{itemize}
    \item ein virtuelles Hub-Netzwerk
    \item mehrere virtuelle Spoke-Netzwerke
    \item Peeringverbindungen zwischen Spoke- und Hub-Netzwerken
    \item Bastionhost (im Hub-Netzwerk)
    \item Azure Firewall als verwalteter Dienst im Hub-Netz (in eigenem Subnetz)
    \item VPN-Gateway für virtuelles Netzwerk oder ExpressRoute-Gateway\newline
      zur Anbindung lokaler Netze
    \item VPN-Gerät zur Verbindung ins lokale Netz
  \end{itemize}
  \vspace*{\stretch{1}}
\end{flashcard}

\begin{flashcard}[]{Eigenschaften der Hub-and-Spoke-Architektur}
  \vspace*{\stretch{1}}
  \begin{itemize}
    \item Peering zwischen Hub- und Spoke-Netzwerken ist \emph{nicht} transitiv
    \item wenn erforderlich, Peering zwischen Spoke-Netzwerken
    \item[!] Grenzwert für Peeringverbindungen muss beachtet werden\newline
      $\Rightarrow$ möglicherwiese benutzerdefinierte Routen
  \end{itemize}
  \vspace*{\stretch{1}}
\end{flashcard}

\begin{flashcard}[]{Virtuelles WAN}
  \vspace*{\stretch{1}}
  \begin{itemize}
    \item verwalteter Azure-Dienst
    \item vollständig Microsoft-Verwaltet
    \item Netwerk spannt mehrere Regionen
  \end{itemize}
  \vspace*{\stretch{1}}
\end{flashcard}

\begin{flashcard}[]{Hub-and-Spoke mit Virtuellem WAN}
  \vspace*{\stretch{1}}
  \begin{itemize}
    \item Einteilung in Branches
    \item Verbindung zwischen Branches und von Branch zu Azure
    \item mehrere lokale private Netzwerke (Büros, Standorte, Endgeräte)
    \item Verbindungen über VPN und ExpressRoute
    \item Endgeräte und Standorte mit VPN-to-Site
  \end{itemize}
  \vspace*{\stretch{1}}
\end{flashcard}

\begin{flashcard}[]{Gründe für Einsatz eines virtuellen WAN}
  \vspace*{\stretch{1}}
  \begin{itemize}
    \item bessere Skalierung bei hohen Durchsätzen
    \item private Verbindungen möglich
    \item ExpressRoute-VPN-Verbindung
    \item bessere Überwachung mit Monitor für Metriken und Health-Status
    \item Mesh zwischen Spoke-Netzwerken
  \end{itemize}
  \vspace*{\stretch{1}}
\end{flashcard}

\subsubsectioncard{recommend a solution for network addressing and name resolution}

\begin{flashcard}[]{Name resolution in Azure}
  \vspace*{\stretch{1}}
  \begin{itemize}
    \item Azure DNS private Zonen
    \item Azure Namensauflösung
    \item eigener DNS-Server\newline
      Weiterleitung an Azure DNS-Server
  \end{itemize}
  \vspace*{\stretch{1}}
\end{flashcard}

\begin{flashcard}[]{Szenarien Azure DNS private Zonen}
  \vspace*{\stretch{1}}
  \begin{itemize}
    \item Auflösung von VMs im selben oder anderen virtuellen Netzwerk
    \item Reverse-DNS für interne IPs
  \end{itemize}
  \vspace*{\stretch{1}}
\end{flashcard}

\begin{flashcard}[]{Szenarien eigener DNS-Server}
  \vspace*{\stretch{1}}
  \begin{itemize}
    \item Auflösung von VMs in verschiedenen virtuellen Netzwerken
    \item Auflösung von App Service VMs in virtuellen Netzwerken
      \begin{itemize}
        \item im eigenen virtuellen Netz
        \item zu anderen virtuellen Netzen
        \item mit Integration in virtuelle Netzwerke
      \end{itemize}
    \item Auflösung von lokalen VMs aus Azure
    \item Azure-Hostnamen aus dem lokalen Netz\newline
      (weiterleitung an Azure)
  \end{itemize}
  \vspace*{\stretch{1}}
\end{flashcard}

\begin{flashcard}[]{Szenarien von Azure bereitgestellter Namensauflösung}
  \vspace*{\stretch{1}}
  \begin{itemize}
    \item Auflösung von VMs im gleichen virtuellen Netzwerk
    \item Reverse-DNS für interne IPs
  \end{itemize}
  \vspace*{\stretch{1}}
\end{flashcard}

\begin{flashcard}[]{Azure Namensauflösung}
  \vspace*{\stretch{1}}
  \begin{itemize}
    \item sehr grundlegender Service, dafür einfach zu nutzen
    \item DNS-Zonen und DNS-Einträge von Azure verwaltet\newline
      $\Rightarrow$ keine manuelle Verwaltung und fester Name!
    \item hochverfügbar (da von Azure verwaltet)
    \item Arbeitet gut mit eigenen DNS-Servern zusammen
    \item Auflösung in \emph{gleichen} Services und Netzwerken ohne FQDN
    \item
  \end{itemize}
  \vspace*{\stretch{1}}
\end{flashcard}

\begin{flashcard}[]{Eigener DNS-Server}
  \vspace*{\stretch{1}}
  \begin{itemize}
    \item zuständig für eine Domäne
    \item weiterleitung an Azure\newline
      $\Rightarrow$ ermöglicht Auflösung von unterschiedlichen virtuellen Netzen
    \item Azure verteilt interne DNS-Namen, die in diesem Fall nicht genutzt werden
    \item mehrere DNS-Server pro virtuellem Netzwerk möglich
    \item[!] DNS dürfen \emph{nicht} in der VM konfiguriert werden
  \end{itemize}
  \vspace*{\stretch{1}}
\end{flashcard}

\begin{flashcard}[]{Azure Private DNS}
  \vspace*{\stretch{1}}
  \begin{itemize}
    \item DNS-Domänen, die nicht im öffentlichen Internet erreichbar sind
    \item keine benutzerdefinierten DNS-Lösungen notwendig
    \item unterstützt alle Einträge (A, AAA, CNAME, MX, TXT, PTR, SOA, SRV)
    \item automatische Verwaltung von VM-Hostnamen
    \item gemeinsame Nutzung in mehreren virtuellen Netzwerken\newline
      $\Rightarrow$ Auflösung zwischen diesen Netzen
    \item unterstützt Split-Horizon-DNS
    \item VMs erhalten automatisch A-Eintrag
  \end{itemize}
  \vspace*{\stretch{1}}
\end{flashcard}

\begin{flashcard}[]{Benutzerdefinierte Domains}
  \vspace*{\stretch{1}}
  \begin{itemize}
    \item Azure-DNS kann mit benutzerdefinierten Apps genutzt werden
    \item die Domain muss an Azure-DNS delegiert werden
    \item Services können über CNAME-Eintrag mit der Domain verbunden werden
      \begin{itemize}
        \item App Service
        \item Function App
        \item Blob storace
        \item CDN
      \end{itemize}
    \item kann auch mit öffentlicher IP verbunden werden
  \end{itemize}
  \vspace*{\stretch{1}}
\end{flashcard}

\begin{flashcard}[]{Azure-DNS}
  \vspace*{\stretch{1}}
  \begin{itemize}
    \item Voraussetzung: eigene Domain
    \item benötigt Ressourcen-Gruppe, ist aber global
    \item Verbinden mit Nameservern ist notwendig (wird von Azure bereitgestellt)
    \item im Registrar der Domäne muss auf Azure DNS umgesteltl werden
  \end{itemize}
  \vspace*{\stretch{1}}
\end{flashcard}

\begin{flashcard}[]{Privates Azure-DNS}
  \vspace*{\stretch{1}}
  \begin{itemize}
    \item Besitz der Domän nicht notwendig
    \item verbunden mit einem virtuellen Netzwerk
    \item Verbindung über Virtual network link
  \end{itemize}
  \vspace*{\stretch{1}}
\end{flashcard}

\subsectioncard{recommend a solution for network provisioning}

\begin{flashcard}[]{Virtuelles Netzwerk}
  \vspace*{\stretch{1}}
  \begin{itemize}
    \item ein Netzwerk in Azure
    \item ermöglicht Ressourcen sichere und isolierte Kommunikation
    \item ermöglicht Cloud-Aspekte wie Sicherheit und Skalierbarkeit
  \end{itemize}
  \vspace*{\stretch{1}}
\end{flashcard}

\begin{flashcard}[]{Features von virtuellen Netzwerken}
  \vspace*{\stretch{1}}
  \begin{itemize}
    \item Kommunikation mit dem Internet/draußen
    \item Kommunikation innerhalb des Netzwerkes
    \item Kommunikation zwischen virtuellen Netzwerken
    \item Kommunikation mit lokalen Netzwerken
    \item Filter (z.\,B. durch Firewall)
    \item benutzerdefinierte Routen
  \end{itemize}
  \vspace*{\stretch{1}}
\end{flashcard}

\begin{flashcard}[]{Kommunikationswege in virtuellen Netzwerken}
  \vspace*{\stretch{1}}
  \begin{itemize}
    \item Ressourcen innerhalb eines Netzwerks (automatisch)
    \item durch virtuellen Dienst-Endpunkt (erweiterung des Netzwerks auf einen Service)
    \item Peering (zu anderen virtuellen Netzen)
    \item VPN mit lokalen Netzen (Point-to-site, Site-to-Site, ExpressRoute)
    \item Private Link: Verbindung zu einem spezifischen Dienst über ein virtuelles Netzwerk
  \end{itemize}
  \vspace*{\stretch{1}}
\end{flashcard}

\begin{flashcard}[]{Sicherheit in virtuellen Netzwerken}
  \vspace*{\stretch{1}}
  \begin{itemize}
    \item Netzwerksicherheitsgruppe
      \begin{itemize}
        \item Filter für ein- und ausgehende Verbinungen
        \item abhängig von Ports, Protokoll, IPs, Diensten, \ldots
      \end{itemize}
    \item virtuelle Geräte
      \begin{itemize}
        \item Firewall
        \item WAN
        \item \ldots
      \end{itemize}
  \end{itemize}
  \vspace*{\stretch{1}}
\end{flashcard}

\begin{flashcard}[]{Netzwerkrouting}
  \vspace*{\stretch{1}}
  \begin{itemize}
    \item Standard-Routing von Azure (zwischen Subnetzen, lokalen Netzen, Internet)
    \item benutzerdefinierte Routing-Tabellen
    \item BGP/Border-gateway-protocol: Verbinungen von lokalen Netzen über VPN/ExpressRoute
  \end{itemize}
  \vspace*{\stretch{1}}
\end{flashcard}

\begin{flashcard}[]{Dienstendpunkte}
  \vspace*{\stretch{1}}
  \begin{itemize}
    \item Filtern von ausgehendem Verkehr zu Diensten
    \item Endpunkt-Richtlinien
    \item Filter, z.\,B. für spezifischen Speicherkonten
    \item Service-Tags, z.\,B. zum Filtern nach Regionen
  \end{itemize}
  \vspace*{\stretch{1}}
\end{flashcard}

\begin{flashcard}[]{Szenarien}
  \vspace*{\stretch{1}}
  \begin{itemize}
    \item Filtern von Verkehr zwischen gepeerten Netzwerken
    \item Filtern von Internet-Verkehr
    \item Filtern von Verkehr aus lokalen Netzwerken
  \end{itemize}
  \vspace*{\stretch{1}}
\end{flashcard}

\begin{flashcard}[]{Vorteile von Dienstendpunkten}
  \vspace*{\stretch{1}}
  \begin{itemize}
    \item Sicherheit für Azure-Ressourcen:\newline
      Dienste in einem Netzwerk sind von außen nicht mehr erreichbar
    \item Effizienz:\newline
      von Azure optimierte Route durch das Microsoft-Backbone
    \item einfaches Aufsetzen ohne Konfiguration von IP-Adressen, NAT, Gateways, \ldots
  \end{itemize}
  \vspace*{\stretch{1}}
\end{flashcard}

\subsubsectioncard{recommend a solution for network security including Private Link, firewalls, gateways, network segmentation (perimeter networks/DMZs/NVAs}

\begin{flashcard}[]{Netzwerksicherheitsgruppe}
  \vspace*{\stretch{1}}
  \begin{itemize}
    \item Sammlung von Sicherheitsregeln zum Filtern
    \item für eingehenden und ausgehenden Netzwerkverkehr
    \item verfügbar für mehrere Arten von Azure-Ressourcen
    \item werden auf privater IP-Ebene (Azure) ausgewertet
  \end{itemize}
  \vspace*{\stretch{1}}
\end{flashcard}

\begin{flashcard}[]{Sicherheitsregel}
  \vspace*{\stretch{1}}
  \begin{itemize}
    \item Priorität: 100 -- 4096, werden in der Reihenfolge bearbeitet \emph{bis eine Regel zutrifft!}
    \item Quelle/Ziel: IP-Adress/CIDR, Service Tag, Anwendungssicherheitsgruppe, \ldots
    \item Port: einzelne oder mehrere Ports
    \item Protokoll: TCP, UDP, ICMP, ESP, AH, alle
    \item Aktion: zulassen oder verbieten
  \end{itemize}
  \vspace*{\stretch{1}}
\end{flashcard}

\begin{flashcard}[]{Anwendungssicherheitsgruppe}
  \vspace*{\stretch{1}}
  \begin{itemize}
    \item Sicherheit als Teil einer Anwendung
    \item VMs einer Anwendung können gruppiert werden
    \item wiederverwendbare Sicherheitsregeln
    \item[!] Azure verwaltet die tatsächlichen IPs
    \item VMs (Netzwerkinterfaces) können teil mehrerer Anwendungssicherheitsgruppen sein
  \end{itemize}
  \vspace*{\stretch{1}}
\end{flashcard}

\begin{flashcard}[]{Sicherere Verbindung ins Netz}
  \vspace*{\stretch{1}}
  \begin{itemize}
    \item lokale Netzwerke zu Azure:
      \begin{itemize}
        \item VPN-Gateway (IPsec)
        \item ExpressRoute
      \end{itemize}
    \item Application Gateway (OSI-Level 7-Load-Balancer) enthält WAF
    \item ExpressRoute 
  \end{itemize}
  \vspace*{\stretch{1}}
\end{flashcard}

\begin{flashcard}[]{Azure Firewall}
  \vspace*{\stretch{1}}
  \begin{itemize}
    \item Cloud-native Lösung für Firewall\newline
      $\Rightarrow$ optimiert für hohe Verfügbarkeit und Skalierbarkeit
    \item vollständig zustandsbehaftete Firewall
    \item Untersuchung von Ost-West und Nord-Süd-Datenverkehr
    \item SKUs: Standard \& Premium (mehr Features)
    \item in virtuellen Netzwerken und Virtual WAN
    \item verwaltet über Azure Firewall Manager
  \end{itemize}
  \vspace*{\stretch{1}}
\end{flashcard}

\begin{flashcard}[]{Features Azure Firewall}
  \vspace*{\stretch{1}}
  \begin{itemize}
    \item OSI Level 3 bis Level 7-Filter
    \item intelligente Bedrohungsfilter (mit Microsoft Cyber Security)
  \end{itemize}
  Premium-Features:
  \begin{itemize}
    \item TLS-Inspektion
    \item IDPS (Signaturfilter basierend auf Patterns)
    \item URL-Filterung
    \item Web-Kategorien (soziale Medien, Glücksspiel, \ldots)
  \end{itemize}
  \vspace*{\stretch{1}}
\end{flashcard}

\begin{flashcard}[]{Virtuelle Netzwerkgeräte (NVA)}
  \vspace*{\stretch{1}}
  \begin{itemize}
    \item filtert Verkehr zwischen Netzwerk-Segmenten mit unterschielichem Sicherheitslevel\newline
      z.\,B. Internet, DMZ, \ldots
    \item Anwendungsfälle
      \begin{itemize}
        \item erkennen/verhindern von Datenabfluss (ausgehender Verkehr)
        \item erkennen/verhindern von Angriffen (eingehender Verkehr)
        \item verhindern von Ausbreitung nach einem Angriff innerhalb des Netzwerks (zwischen virtuellen Netzwerken)
        \item Filter für Netzwerkverkehr zwischen lokalen/virtuellen Netzwerken unterschiedlicher Sicherheitsstufe
      \end{itemize}
    \item Hochverfügbarkeit ist Schlüsselkriterium! Ohne Netzwerkverbindung klappt kein Service
  \end{itemize}
  \vspace*{\stretch{1}}
\end{flashcard}

\begin{flashcard}[]{Biespiele von NVAs}
  \vspace*{\stretch{1}}
  \begin{itemize}
    \item Firewalls
    \item OSI-Level 4-(reverse)-Proxies
    \item VPN-Endpunkte (IPsec)
    \item WAF (Web Application Firewall)
    \item Internet-Proxy-Server
    \item OSI-Level 7-Lastausgleich
    \item \ldots
  \end{itemize}
  \vspace*{\stretch{1}}
\end{flashcard}

\begin{flashcard}[]{Architekturen für hochverfügbare Applikationen}
  \vspace*{\stretch{1}}
  (In Hub-Spoke-Netzwerken)
  \begin{itemize}
    \item Azure Load Balancer:\newline
      zwei Load-Balancer, einer öffentlich, einer privat
    \item Azure Route Server:\newline
      RouteServer bestimmt Routen und teilt den NVAs mit, Traffic über BGP-Protokoll
    \item Gateway Load Balancer:\newline
      NVA wird transparent in die Route eingebunden, so dass vom Gateway über das NVA geleitet wird
    \item Changing PIP/UDR:\newline
      Design ohne Redundanz, modifizierte Routen wenn ein Problem auftritt
  \end{itemize}
  \vspace*{\stretch{1}}
\end{flashcard}

\begin{flashcard}[]{DMZ oder Perimeter-Netzwerk}
  \vspace*{\stretch{1}}
  \begin{itemize}
    \item zwischen zwei Netzwerkbereichen (lokal $\leftrightarrow$ Cloud, öffentlich $\leftrightarrow$ intern)
    \item eingehender und ausgehender Verkehr läuft über eine Firewall
    \item Zugriff auf die Infrastruktur sollte nur eingeschränkt möglich sein (RBAC)
    \item alle Nutzerdaten müssen durch die Firewall gehen
  \end{itemize}
  \vspace*{\stretch{1}}
\end{flashcard}

\begin{flashcard}[]{Anwendungsfälle}
  \vspace*{\stretch{1}}
  \begin{itemize}
    \item Anwendungen, die lokal und in Azure laufen
    \item Analyse des ausgehenden Verkehrs, z.\,B. für Audits
  \end{itemize}
  \vspace*{\stretch{1}}
\end{flashcard}

\subsubsectioncard{recommend a solution for network connectivity to the Internet, on-premises networks, and other Azure virtual networks}

\begin{flashcard}[]{Gateways in Azure}
  \vspace*{\stretch{1}}
  \begin{itemize}
    \item VMs in einem Subnetz (Gateway-Subnets)
    \item stellen Routing-Tabellen bereit
    \item Verwaltet über Azure (\emph{nicht} über die VMs)
    \item Typen
      \begin{itemize}
        \item VPN-Gateway
        \item ExpressRoute
      \end{itemize}
    \item je eins pro virtellem Netzwerk
  \end{itemize}
  \vspace*{\stretch{1}}
\end{flashcard}

\begin{flashcard}[]{VPN-Gateway}
  \vspace*{\stretch{1}}
  \begin{itemize}
    \item Verbindung von Azure Netzwerken mit lokalen Netzen über das öffentliche Internet (sicher)
    \item alternativ: Nutzung des Microsoft-Netzwerks (anstelle des öffentlichen Netzes)
    \item maximal ein pro virtuellem Netz
    \item \emph{aber}: mehrere Verbindungen für ein Gateway möglich\newline
      $\Rightarrow$ geteilte Kapazität
  \end{itemize}
  \vspace*{\stretch{1}}
\end{flashcard}

\begin{flashcard}[]{VPN-Gateway-Verbindungstypen}
  \vspace*{\stretch{1}}
  \begin{itemize}
    \item VNet-zu-VNet: zwei VPN-Gateways mit IPsec/IKE
    \item Site-zu-Site: zu lokalem Netz mit IPsec/IKE-Tunnel zu lokalem VPN-Gerät
    \item Point-zuSite: verbindung mit entferntem Ort/Gerät über OpenVPN, IKEv2, SSTP
  \end{itemize}
  \vspace*{\stretch{1}}
\end{flashcard}

\begin{flashcard}[]{Point-zu-Site-Verbindungen}
  \vspace*{\stretch{1}}
  Verbindet Benutzer mit Azure:
  \begin{itemize}
    \item mit Azure-Cloud-Diensten
    \item mit virtuellen Maschinen
    \item Protokolle: SSTP, OpenVPN, IPsec
    \item Verbindung über dynamische Routen
    \item aktiv-passiv
  \end{itemize}
  \vspace*{\stretch{1}}
\end{flashcard}

\begin{flashcard}[]{Site-zu-Site}
  \vspace*{\stretch{1}}
  Verbindet lokale Netze (kleine Rechenzentren, Entwicklungsabteilung, \ldots) mit Azure
  \begin{itemize}
    \item mit Azure-Cloud-Diensten
    \item mit virtuellen Maschinen
    \item IPsec
    \item Verbindung über danamische Routen und statisch mit Richtlinien
    \item aktiv-passiv oder aktiv-aktiv
  \end{itemize}
  \vspace*{\stretch{1}}
\end{flashcard}

\begin{flashcard}[]{ExpressRoute}
  \vspace*{\stretch{1}}
  \begin{itemize}
    \item Erweiterung des lokalen Netzwerks in die Cloud
    \item Verbindung zu Services mit öffentlichen IPs über Microsoft-Peering
    \item Verbindungen gehen über das Azure-Netzwerk, \emph{nicht} das öffentliche Internet
  \end{itemize}
  \vspace*{\stretch{1}}
\end{flashcard}

\begin{flashcard}[]{ExpressRoute-Eigenschaften}
  \vspace*{\stretch{1}}
  \begin{itemize}
    \item OSI-Level 3-Verbindungen
    \item Verbindung mit Microsoft-Cloud in allen Regionen
    \item dynamische Routen mit BGP
    \item hohe Redundanz
  \end{itemize}
  \vspace*{\stretch{1}}
\end{flashcard}

\begin{flashcard}[]{ExpressRoute-Verbindungen}
  \vspace*{\stretch{1}}
  Verbindet kritische Infrastruktur mit Azure
  \begin{itemize}
    \item mit virtuellen Netzwerken (Azure-Dienste, VMs, \ldots)
    \item mit Enterprise-Diensten (Office, PowerBI, Azure AD, Azure DevOps, \ldots)
    \item \emph{nicht} mit CDN, Front Door, Traffic Manager, Logic Apps
    \item Protokoll: BGP
    \item aktiv-aktiv
  \end{itemize}
  \vspace*{\stretch{1}}
\end{flashcard}

\subsubsectioncard{recommend a solution for automating network management}

\begin{flashcard}[]{Automated virtual network}
  \vspace*{\stretch{1}}
  \begin{itemize}
    \item ARM Manager-Vorlagen
    \item Provisioning unterstützt auch Aure CLI und Powershell
    \item Voralgen von Microsoft verfügbar (z.\,B. Hub-Spoke)
  \end{itemize}
  \vspace*{\stretch{1}}
\end{flashcard}

\subsubsectioncard{recommend a solution for load balancing and traffic routing}

\begin{flashcard}[]{Dimensionen von Azure-Load-Balancern}
  \vspace*{\stretch{1}}
  \begin{itemize}
    \item Protokoll
      \begin{itemize}
        \item HTTP/HTTPS (OSI-Level 7)
        \item andere Protokolle
      \end{itemize}
    \item Region
      \begin{itemize}
        \item Global: Nutzer in mehrere Regionen\newline
          $\Rightarrow$ Ausgleich der Azure-Endpunkte
        \item Regional
      \end{itemize}
  \end{itemize}
  \vspace*{\stretch{1}}
\end{flashcard}

\begin{flashcard}[]{Lastausgleichsservices in Azure}
  \vspace*{\stretch{1}}
  \begin{itemize}
    \item Front Door: Globaler HTTP(S)-Ausgleich
    \item Traffic Manager: Globaler DNS-basierter Lastausgleich
    \item Application Gateway: lokaler Level-7-Lastausgleich
    \item Azure Load Balancer: lokaler Level-4-Lastausgleich
  \end{itemize}
  \vspace*{\stretch{1}}
\end{flashcard}

\begin{flashcard}[]{Lastausgleichs-Szenarien}
  \vspace*{\stretch{1}}
  \begin{itemize}
    \item Öffentliche globale nicht-HTTP-Anwendungen: Traffic Manager + Load Balancer
    \item lokale nicht-HTTP-Anwendungen: Load Balancer
    \item lokale Web-Anwendungen (nicht öffnetlich, oder ohne Performance-Unterstützung): Application Gateway
    \item globale, öffentliche Web-Anwendungen: Front Door\newline
      wenn \emph{nicht} PaaS: zusätzlich Load Balancer oder Application Gateway\newline
      (PaaS-Dienste enthalten die Lastverteilung)
  \end{itemize}
  \vspace*{\stretch{1}}
\end{flashcard}

\begin{flashcard}[]{Front Door}
  \vspace*{\stretch{1}}
  \begin{itemize}
    \item globaler Verbindungspunkt zum Microsoft Edge-Netzwerk
    \item OSI-Level-7-Lastausgleich für Web-Anwendungen
    \item für sichere, skalierbare Webanwendungen
    \item automatisches Failover
  \end{itemize}
  \vspace*{\stretch{1}}
\end{flashcard}

\begin{flashcard}[]{Features/Vorteile Front Door}
  \vspace*{\stretch{1}}
  \begin{itemize}
    \item hohe Performance (split TCP, anycast)
    \item effizienter Failover mit Health-Probes
    \item URL-basiertes Routing, URL-Redirect, URL-Rewrite
    \item SSL offloading
    \item integrierte WAF
  \end{itemize}
  \vspace*{\stretch{1}}
\end{flashcard}

\begin{flashcard}[]{Traffic Manager}
  \vspace*{\stretch{1}}
  \begin{itemize}
    \item DNS-Basierter Load Balancer
    \item Verteilung des Verkehrs über mehrere Regionen
    \item etwas langsamer im Failover (durch DNS-Basierung)
    \item Resilient gegenüber Ausfällen einer ganzen Azure-Region
  \end{itemize}
  \vspace*{\stretch{1}}
\end{flashcard}

\begin{flashcard}[]{Features Traffic Manager}
  \vspace*{\stretch{1}}
  \begin{itemize}
    \item hohe Verfügbarkeit
    \item geringe Latenz (Weiterleitung an nächsten Endpunkt)
    \item Wartung ohne Down-Time
    \item hybride Anwendungen (lokal-cloud, Burst-to-Cloud, failover-to-cloud)
  \end{itemize}
  \vspace*{\stretch{1}}
\end{flashcard}

\begin{flashcard}[]{Application Gateway}
  \vspace*{\stretch{1}}
  \begin{itemize}
    \item HTTP(s)-Lastausgleich (basierend auf URL, nicht IP)\newline
      $\Rightarrow$ unterstützt URL-basiertes Routing und Redirection, Rewrite
    \item möglich hinter globalem Laustausgleich einzusetzen
  \end{itemize}
  \vspace*{\stretch{1}}
\end{flashcard}

\begin{flashcard}[]{Features Application Gatway}
  \vspace*{\stretch{1}}
  \begin{itemize}
    \item SSL-Terminierung
    \item automatische Skalierung
    \item Zonen-Redundanz
    \item eingebaute WAF
  \end{itemize}
  \vspace*{\stretch{1}}
\end{flashcard}

\begin{flashcard}[]{Azure Load Balancer}
  \vspace*{\stretch{1}}
  \begin{itemize}
    \item OSI-Level-4
    \item öffentlich oder privat: z.\,B. in \textit{n}-Tier-Umgebungen\newline
      vor Datenbanken oder anderen internen Services
    \item unterstützt Zonenredundanz
    \item Health probes zum Überwachen
    \item mehrere IPs und Ports
    \item jede art von TCP und UDP-Datenverkehr
    \item Sicherheit: benötigt Netzwerksicherheitsgruppe, andernfalls\newline
      \texttt{kein Datenverkehr möglich}
  \end{itemize}
  \vspace*{\stretch{1}}
\end{flashcard}
