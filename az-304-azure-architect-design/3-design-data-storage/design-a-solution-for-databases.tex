\subsectioncard{Design einer Lösung für Datanbanken}

\subsubsectioncard{select an appropriate data platform based on requirements}

\begin{flashcard}[Definition]{Auswahl einer Datenplattform}
  \vspace*{\stretch{1}}
  \begin{itemize}
    \item zahllose Datenplattformen verfügbar (auch in Azure)
    \item Art der Datenstrukturierung
    \item unterstützte Operationen auf den Datenbanken
    \item Kosten
    \item Wartbarkeit
    \item möglicherweise sogar Multi-Model-Datenbanken
    \item Features müssen zu den Anforderungen passen
  \end{itemize}
  \vspace*{\stretch{1}}
\end{flashcard}

\begin{flashcard}[Definition]{Datensysteme in Azure}
  \vspace*{\stretch{1}}
  \begin{itemize}
    \item Relationale Datenbanken (RDBMS) (meistens SQL)
    \item Schlüssel/Wert-Speicher
    \item Dokument-Datenbanken
    \item Graph-Datenbanken
    \item Datenanalyse (für parallelisierung, Big Data, \ldots)
    \item Spaltenfamilien
    \item Datenbanken für Suche
    \item Zeitreihen
    \item Objektspeicher
    \item Dateien/Dateisystem
  \end{itemize}
  \vspace*{\stretch{1}}
\end{flashcard}

\begin{flashcard}[Definition]{Relationale Datenbanken in Azure}
  \vspace*{\stretch{1}}
  \begin{itemize}
    \item klassische 2-dimensionaler Tabellen
    \item verfügbare Richtungen
      \begin{itemize}
        \item Azure SQL Database (Microsoft SQL Server)
        \item Azure Database for MySQL, PostreSQL, MariaDB
      \end{itemize}
    \item konsistent, ACID, operationen in Transaktionen
    \item keine horizontale Skalieung
    \item normalisierte Daten erforderlich
    \item Daten passen in Schema (vorab)
  \end{itemize}
  \vspace*{\stretch{1}}
\end{flashcard}

\begin{flashcard}[Definition]{Schlüssel/Wert-Speicher}
  \vspace*{\stretch{1}}
  \begin{itemize}
    \item verfügbare Services:
      \begin{itemize}
        \item Azure Cosmos DB (Tabellen und SQL APIs)
        \item Azure Cache für Redis
        \item Azure Table Speicher
      \end{itemize}
    \item jeder Eintrag hat eindeutigen Schlüssel
    \item Operationen auf Wert-Basis sind Atomar
    \item optimiert für Schlüssel-Lookups
    \item komplexe Operationen (Join, Locks, \ldots) nicht unterstützt
    \item keine Relation zwischen einzelnen Einträgen
    \item ein Schema wird nicht erzwungen
  \end{itemize}
  \vspace*{\stretch{1}}
\end{flashcard}

\begin{flashcard}[Definition]{Dokument-Datenbanken}
  \vspace*{\stretch{1}}
  \begin{itemize}
    \item in azure als Azure Cosmos DB (SQL API)
    \item Sammlung von \emph{Dokumenten}, von einfach bis Komplex (z.\,B. JSON)
    \item Dokumente können Objekte im Code gut repräsentieren
    \item denormalisierte Daten
    \item verschiedene Schemata pro Dokumenttyp, auch optionale Felder\newline
      $\Rightarrow$ halb-strukturierte Daten
  \end{itemize}
  \vspace*{\stretch{1}}
\end{flashcard}

\begin{flashcard}[Definition]{Graph-Datenbanken}
  \vspace*{\stretch{1}}
  \begin{itemize}
    \item verfügbare Services in Azure
      \begin{itemize}
        \item Azure Cosmos DB (Gremlin API)
        \item SQL Server
      \end{itemize}
    \item erlauben effiziente Anfragen an Graphen
    \item Objekte: Knoten, Kanten (zwischen zwei Knoten), Eigenschaften, Richtung
    \item Ermöglicht Abfragen über Abhängigkeiten
    \item im Unterschied zu relationalen Daten sind Relationen Teil des Datenmodells
  \end{itemize}
  \vspace*{\stretch{1}}
\end{flashcard}

\begin{flashcard}[Definition]{Datenanalyse}
  \vspace*{\stretch{1}}
  \begin{itemize}
    \item verfügbare Azure-Services
      \begin{itemize}
        \item Azure Synapse Analytics
        \item Azure Data Lake
        \item Azure Data Explorer
        \item Azure Analyisis Services
        \item HDInsight
        \item Azure Databricks
      \end{itemize}
    \item Dateiformate wie parquet, ORC
    \item denormalisierte Daten, z.\,B. Stern- oder Schneeflockenartig
    \item unterstützt massive parallelisierung
  \end{itemize}
  \vspace*{\stretch{1}}
\end{flashcard}

\begin{flashcard}[Definition]{Spaltenfamilien}
  \vspace*{\stretch{1}}
  \begin{itemize}
    \item Unterstützung in Azure:
      \begin{itemize}
         \item Azure Cosmus DB (Cassandra API)
         \item HBase (teil von HDInsight)
      \end{itemize}
    \item Daten in Spalten und Zeilen \emph{ähnlich} wie relationale Datenbanken
    \item Spalten zu Familien zusammengefasst, Operationen oft nur auf Familien\newline
    dafür aber atomar
    \item Daten können \emph{dünn} sein, also optional
    \item Zugriff auf Zeilen über einen Schlüssel, der indizierbar ist
  \end{itemize}
  \vspace*{\stretch{1}}
\end{flashcard}

\begin{flashcard}[Definition]{Datenbanken für Suche}
  \vspace*{\stretch{1}}
  \begin{itemize}
    \item Azure Search
    \item Unterstützung für Suchanfragen, exact oder unscharf
    \item basiert auf Index großer Datenmengen (mehrdimensional, Freitext)
    \item Freitext- und komplexe Anfragen
    \item Unstrukturierter Text, oder teilstrukturiert, mit Referenzen
  \end{itemize}
  \vspace*{\stretch{1}}
\end{flashcard}

\begin{flashcard}[Definition]{Zeitreihendatenbanken}
  \vspace*{\stretch{1}}
  \begin{itemize}
    \item Azure Time Series Insights
    \item fast nur Schreiboperationen, Löschen nur in Blöcken
    \item kleine Einträge, aber viele
    \item Schlüssel sind Zeitpunkte, meistens sortiert
  \end{itemize}
  \vspace*{\stretch{1}}
\end{flashcard}

\begin{flashcard}[Definition]{Objektspeicher}
  \vspace*{\stretch{1}}
  \begin{itemize}
    \item verfügbare Services in Azure
      \begin{itemize}
        \item Azure Blob Speicher
        \item Azure Data Lake Storage Gen2
      \end{itemize}
      \item große (binäre) Objekte identifiziert als Schlüssel
      \item z.\,B. Filme, PDFs, \ldots
  \end{itemize}
  \vspace*{\stretch{1}}
\end{flashcard}

\begin{flashcard}[Definition]{Dateien/Dateisystem}
  \vspace*{\stretch{1}}
  \begin{itemize}
    \item Azure Files
    \item Standardzugriff auf objekte in Verzeichnishierarchie
    \item ermöglicht Einbindung in Legacy-Systeme\newline
    (Zugriff mit standard I/O-Bibliotheken)
    \item gemeinsamer Zugriff auf vielen Systemen (NFS/SMB)
  \end{itemize}
  \vspace*{\stretch{1}}
\end{flashcard}

\begin{flashcard}[Definition]{Azure SQL Database}
  \vspace*{\stretch{1}}
  \begin{itemize}
    \item PAAS-Bereitstellung von Microsoft SQL Server
    \item einzelne Datenbanken, oder Datenbankenpools
    \item skalierbare Performance
    \item Bezahlmodelle: Kern-Basiert, DTU, Serverless
    \item SLA: 99,99\%, automatische Backups, Replikation, Redundanz, \ldots
  \end{itemize}
  \vspace*{\stretch{1}}
\end{flashcard}

\begin{flashcard}[Definition]{Azure SQL}
  \vspace*{\stretch{1}}
  \begin{itemize}
    \item verschiedene Klassen von SQL-Datenbanken
      \begin{itemize}
        \item Azure SQL Database: DBaaS als PaaS, SQL Server
        \item Azure SQL Managed Instance: PaaS, verschiedene Engines
          \item SQL Server auf einer VM: IaaS, VM ist verwaltet
      \end{itemize}
      \item SQL auf einer VM ist \emph{selbstverwaltet} (Updates, \ldots)
  \end{itemize}
  \vspace*{\stretch{1}}
\end{flashcard}

\subsubsectioncard{recommend database service tier sizing}

\begin{flashcard}[Definition]{Azure SQL Service Tiers}
  \vspace*{\stretch{1}}
  \begin{itemize}
    \item Verfügbare Service Tiers für Kerne
      \begin{itemize}
        \item Allgemein/Standard
        \item Geschäftskritisch/Premium: hohe Transaktionsrate, low-latency I/O,
          \newline isolierte Replikationen
        \item Hyperscale: automatische Skalierung von Speicher und Rechenleistung\newline
          (nur serverlos)
      \end{itemize}
    \item Service Tiers für DTU
      \begin{itemize}
        \item Basic
        \item Standard
        \item Premium
      \end{itemize}
  \end{itemize}
  \vspace*{\stretch{1}}
\end{flashcard}

\begin{flashcard}[Definition]{Allgemeines SQL Service Tier}
  \vspace*{\stretch{1}}
  \begin{itemize}
    \item Cloud-Architektur mit guter Verfügbarkeit im Falle von Ausfällen
    \item Speicher und Prozessierung laufen separiert
      \begin{itemize}
        \item Prozesse: stateless, können auf andere Hardware verschoben werden
        \item Speicher: stateful Datenbank-Dateien in Azure Blob Storage
      \end{itemize}
    \item Prozesse laufen auf redundanten Knoten
    \item gut für generelle Aufgaben, latenz 5-10 ms
    \item Standard-SLA 99,99\%
  \end{itemize}
  \vspace*{\stretch{1}}
\end{flashcard}

\begin{flashcard}[Definition]{Geschäftskritisches SQL Service Tier}
  \vspace*{\stretch{1}}
  \begin{itemize}
    \item optimiert für hohen Durchsatz, geringe Fehlerwahrscheinlihckeit, Resilienz, Updates, \ldots
    \item Datenbanken laufen auf Cluster und Replikas
    \item unterschiedliche Schreib- und Lese-Knoten
    \item Hochveffügbarkeit durch Verfügbarkeitsgruppen
    \item Daten werden auf lokalen Datenspeichern gespeichert
    \item kostenlose Skalierung für Leseoperationen
    \item latenz 1-2 ms, Datenschutz
    \item Premium-SLA 99,995\%
  \end{itemize}
  \vspace*{\stretch{1}}
\end{flashcard}

\begin{flashcard}[Definition]{Hyperscale}
  \vspace*{\stretch{1}}
  \begin{itemize}
    \item nur für Azure SQL Database
    \item nur vCore-Bezahlmodell
    \item 100 TiB Datenbankgröße
    \item sofortige BAckups, schnelle Dateneinspielung
    \item schnelle Skalierung (vertikal und horizontal)
    \item Lese-Replizierunge, Caches
  \end{itemize}
  \vspace*{\stretch{1}}
\end{flashcard}

\begin{flashcard}[Definition]{Auswertung/Effizienz}
  \vspace*{\stretch{1}}
  \begin{itemize}
    \item Query Performance Insight
    \begin{itemize}
      \item Analyse für einzelne und gepoolte Datenbanken
      \item erkennt die teuerste und am längsten laufenden Queries
    \end{itemize}
  \end{itemize}
  \vspace*{\stretch{1}}
\end{flashcard}

 Query Performance Insight

\subsubsectioncard{recommend a solution for database scalability}

\begin{flashcard}[Definition]{SQL-Skalierbarkeit}
  \vspace*{\stretch{1}}
  \begin{itemize}
    \item automatische Skalierung und minimale Downtime durch PaaS
    \item dynamische Anpassung an wachsende (schrumpfende) Anforderungen
    \item Skalierung in beiden Bezahlmodellen
      \begin{itemize}
        \item DTU-basiert\newline
          Operationen, Durchsatz
        \item Kern-basiert\newline
          anzahl Kerne
      \end{itemize}
    \item dynamisch können Datenbanken in einen Pool verschoben werden\newline
      (oder extrahiert)
  \end{itemize}
  \vspace*{\stretch{1}}
\end{flashcard}

\begin{flashcard}[Definition]{Arten der Skalierung}
  \vspace*{\stretch{1}}
  \begin{itemize}
    \item hoch und runter skalieren
      \begin{itemize}
        \item Prozess startet neu!
        \item möglicherweise verschoben auf andere virtuelle Maschine
        \item Verschiebung ist ein online-Prozess ohne downtime\newline
          (Transaktionen werden möglicherweise abgebrochen)
        \item Skalierung während eines langen Prozesses sollte nicht durchgeführt werden!
        \item benötigt Wiederholungs-Logik in Clientanwendung
      \end{itemize}
    \item Lese-Skalierung: Nur-Lesen-Replika der Datenbank um aufwändige\newline
      Leseanfragen zu beantworten (z.\,B. Reporting)
    \item Sharding: Aufteilen der Daten auf mehrere Datenbanken\newline
      (nicht automatisch)
  \end{itemize}
  \vspace*{\stretch{1}}
\end{flashcard}

\subsubsectioncard{recommend a solution for encrypting data at rest, data in transmission, and data in use}

\begin{flashcard}[Definition]{Verschlüsselung}
  \vspace*{\stretch{1}}
  \begin{itemize}
    \item oft von Azure direkt bereitgestellt
    \item möglich eigene Schlüssel zu verwenden
    \item Schlüssel werden im Key Vault gespeichert\newline
      auch ohne Möglichkeit von Microsoft darauf zuzugreifen
    \item Unterstützung für Schlüsselverschlüsselungsschlüssel
  \end{itemize}
  \vspace*{\stretch{1}}
\end{flashcard}

\begin{flashcard}[Definition]{Verschlüsselung ruhender Daten}
  \vspace*{\stretch{1}}
  \begin{itemize}
    \item oftmals erforderlich wegen Richtlinien
    \item Verschlüsselung mit einem symmetrischen Schlüssel
    \item transparent verschlüsseln beim Schreiben und entschlüsseln beim Lesen
    \item falls Zugriff auf eine Festplatte besteht, können die Daten nicht gelesen werden
    \item Schlüssel sollten im Key Vault gespeichert werden
    \item Verschlüsselung mit Umschlagverschlüsselung:\newline
      ermöglicht leichtes Austauschen des Schlüsselverschlüsselungsschlüssels ohne Daten neu zu verschlüsseln
    \item Kundenschlüssel möglich
  \end{itemize}
  \vspace*{\stretch{1}}
\end{flashcard}

\begin{flashcard}[Definition]{Unterstützung für Verschlüsselung ruhender Daten}
  \vspace*{\stretch{1}}
  \begin{itemize}
    \item Azure Disk Encryption
    \item Azure Storage (Blob, Queue, Tabele, Files)
    \item Azure SQL-Datenbank
    \item verwaltete Datenträger
  \end{itemize}
  \vspace*{\stretch{1}}
\end{flashcard}

\begin{flashcard}[Definition]{Verschlüsselung von Daten während der Übertragung}
  \vspace*{\stretch{1}}
  Verschlüsselung beim Übertragen zwischen verschiedenen Speicherorten
  \begin{itemize}
    \item Schutz z.\,B. vor Man-in-the-middle-Attacken
    \item Verschlüsselung im Rechenzentrum (\emph{MACsec}), in der Netzwerkhardware
    \item TLS-Versclüsselung bei allen Anfragen von Kunden
    \item Datenzugriff: HTTPS über web, SMB 3.0
    \item Zugriff auf Virtuelle Computer: RDP, SSH
    \item Zugriff auf Azure durch ein VPN (VPN Gateway, Point-to-Site-VPN, Site-to-Site)
  \end{itemize}
  \vspace*{\stretch{1}}
\end{flashcard}

\begin{flashcard}[Definition]{Transparente Verschlüsselung}
  \vspace*{\stretch{1}}
  \begin{itemize}
    \item Transparente Datenverschlüsselung von SQL-Datenbanken und Synapse
    \item Daten sind Verschlüsselt im ruhenden Speicher \emph{und} in Speicherseiten
    \item Verarbeitung der unverschlüsselten Daten nur im SQL-Prozess
  \end{itemize}
  \vspace*{\stretch{1}}
\end{flashcard}

\begin{flashcard}[Definition]{Verstecken von Daten}
  \vspace*{\stretch{1}}
  Data Masking
  \begin{itemize}
    \item dynamisches Verstecken von sensitiven Daten
    \item richtlinienbasiert
    \item versteckt Daten in Antworten/Queries
    \item Anwendbare Dienste:
      \begin{itemize}
        \item Azure SQL Database
        \item Azure SQL Managed Instance
        \item Azure Synapse Analytics
      \end{itemize}
    \item Richtlinie:
      \begin{itemize}
        \item Benutzer: SQL-Nutzer/Azure AD-Entitäten mit Vollzugriff
        \item Maskierungsregeln und Funktionen, z.\,B.\newline
          Kreditkarten, Emails, Standardwerte
      \end{itemize}
  \end{itemize}
  \vspace*{\stretch{1}}
\end{flashcard}

