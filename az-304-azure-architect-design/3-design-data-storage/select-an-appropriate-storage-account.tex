\subsectioncard{Auswahl eines Speicherkontos}

\subsubsectioncard{choose between storage tiers}

\begin{flashcard}[Definition]{Zugriffsebenen in Azure}
  \vspace*{\stretch{1}}
  \begin{itemize}
    \item Ebene bestimmt Kosten und Geschwindigkeit (online/offline)
    \item verfügbare Zugriffsebenen
      \begin{itemize}
        \item Heiß
        \item Kalt
        \item Archiv
      \end{itemize}
  \end{itemize}
  \vspace*{\stretch{1}}
\end{flashcard}

\begin{flashcard}[Definition]{Online-Zugriff}
  \vspace*{\stretch{1}}
  \begin{itemize}
    \item heiße und kalte Zugriffsebenen
    \item schneller (direkter) Datenzugriff
    \item Kosten pro Zugriff (API-Aufruf) und für Speicher
    \item kalt: höhere Kosten für Zugriff, geringere für Daten, Mindestspeicherzeit 30 Tage, geringere Availability
  \end{itemize}
  \vspace*{\stretch{1}}
\end{flashcard}
 
\begin{flashcard}[Definition]{Offline-Zugriff}
  \vspace*{\stretch{1}}
  \begin{itemize}
    \item Archiv-Zugriffsebene
    \item Daten müssen bereitgestellt werden, bevor Zugriff möglich ist (nur lesen!)\newline
      \emph{rehydrierung}
    \item Mindestspeicherzeit: 180 Tage
  \end{itemize}
  \vspace*{\stretch{1}}
\end{flashcard}
 
\begin{flashcard}[Definition]{Wechsel der Zugriffsebene}
  \vspace*{\stretch{1}}
  \begin{itemize}
    \item Standardebene für neue Blobs (vererbt)
    \item pro Blob kann Zugriffsebene gesetzt werden
    \item Zugriffsebenen können mit Richtlinien wechseln (Lebenszyklus-Management)
  \end{itemize}
  \vspace*{\stretch{1}}
\end{flashcard}
 
\subsubsectioncard{recommend a storage access solution}

\begin{flashcard}[Definition]{Azure Speicherdienste}
  \vspace*{\stretch{1}}
  \begin{itemize}
    \item Typen:
      \begin{itemize}
        \item Blob (auch Data Lake Gen2): Objektspeicher
        \item Files: Dateisystem, SMB/NFS mountbar
        \item Tabellen
        \item Queue
        \item Disks: Festplatten, die auf Blobs liegen
      \end{itemize}
    \item Redundanz: Zonen, Geo-Redundanz
    \item Verfügbarkeit: hochverfügbar
    \item Skalierbarkeit
  \end{itemize}
  \vspace*{\stretch{1}}
\end{flashcard}

\subsubsectioncard{recommend storage management tools}

\begin{flashcard}[Definition]{Datenfluss}
  \vspace*{\stretch{1}}
  \begin{itemize}
    \item 
  \end{itemize}
  \vspace*{\stretch{1}}
\end{flashcard}
