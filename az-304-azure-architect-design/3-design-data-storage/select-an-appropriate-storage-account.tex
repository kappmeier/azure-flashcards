\subsectioncard{Auswahl eines Speicherkontos}

\subsubsectioncard{choose between storage tiers}

\begin{flashcard}[Definition]{Zugriffsebenen in Azure}
  \vspace*{\stretch{1}}
  \begin{itemize}
    \item Ebene bestimmt Kosten und Geschwindigkeit (online/offline)
    \item verfügbare Zugriffsebenen
      \begin{itemize}
        \item Heiß
        \item Kalt
        \item Archiv
      \end{itemize}
  \end{itemize}
  \vspace*{\stretch{1}}
\end{flashcard}

\begin{flashcard}[Definition]{Online-Zugriff}
  \vspace*{\stretch{1}}
  \begin{itemize}
    \item heiße und kalte Zugriffsebenen
    \item schneller (direkter) Datenzugriff
    \item Kosten pro Zugriff (API-Aufruf) und für Speicher
    \item kalt: höhere Kosten für Zugriff, geringere für Daten, Mindestspeicherzeit 30 Tage, geringere Availability
  \end{itemize}
  \vspace*{\stretch{1}}
\end{flashcard}
 
\begin{flashcard}[Definition]{Offline-Zugriff}
  \vspace*{\stretch{1}}
  \begin{itemize}
    \item Archiv-Zugriffsebene
    \item Daten müssen bereitgestellt werden, bevor Zugriff möglich ist (nur lesen!)\newline
      \emph{rehydrierung}
    \item Mindestspeicherzeit: 180 Tage
  \end{itemize}
  \vspace*{\stretch{1}}
\end{flashcard}
 
\begin{flashcard}[Definition]{Wechsel der Zugriffsebene}
  \vspace*{\stretch{1}}
  \begin{itemize}
    \item Standardebene für neue Blobs (vererbt)
    \item pro Blob kann Zugriffsebene gesetzt werden
    \item Zugriffsebenen können mit Richtlinien wechseln (Lebenszyklus-Management)
  \end{itemize}
  \vspace*{\stretch{1}}
\end{flashcard}
 
\subsubsectioncard{recommend a storage access solution}

\begin{flashcard}[Definition]{Azure Speicherdienste}
  \vspace*{\stretch{1}}
  \begin{itemize}
    \item Typen:
      \begin{itemize}
        \item Blob (auch Data Lake Gen2): Objektspeicher
        \item Files: Dateisystem, SMB/NFS mountbar
        \item Tabellen: NoSQL
        \item Queue
        \item Disks: Festplatten, die auf Blobs liegen
      \end{itemize}
    \item Redundanz: Zonen, Geo-Redundanz
    \item Verfügbarkeit: hochverfügbar
    \item Skalierbarkeit
    \item automatisch verwaltet
  \end{itemize}
  \vspace*{\stretch{1}}
\end{flashcard}

\begin{flashcard}[Definition]{Blobspeicher}
  \vspace*{\stretch{1}}
  \begin{itemize}
    \item Daten-Streams und wahlfreier Zugriff von überall
    \item Unterstützung für Data Lakes
    \item riesige Mengen unstrukturierter Daten
  \end{itemize}
  \vspace*{\stretch{1}}
\end{flashcard}

\begin{flashcard}[Definition]{Files-Speicher}
  \vspace*{\stretch{1}}
  \begin{itemize}
    \item geeignet für Lift-and-Shift-Szenarien
    \item erweitert lokalen NFS-Speicher
    \item Zugriff von vielen virtuellen Maschinen
    \item einbinden über NFS oder SMB
    \item keine Unterstützung für Azure AD-Integration!
  \end{itemize}
  \vspace*{\stretch{1}}
\end{flashcard}

\begin{flashcard}[Definition]{Queue-Speicher}
  \vspace*{\stretch{1}}
  \begin{itemize}
    \item asynchroner Nachrichtenversand\newline
    (ähnlich wie Service Bus-Schlangen)
    \item unbegrenzte Nachrichten von maximal 64 KiB
  \end{itemize}
  \vspace*{\stretch{1}}
\end{flashcard}

\begin{flashcard}[Definition]{Tabellenspeicher}
  \vspace*{\stretch{1}}
  \begin{itemize}
    \item flexible Daten, z.\,B. für Web-Anwendungen\newline
    (ähnlich wie Cosmos DB Tabellen-API)
  \end{itemize}
  \vspace*{\stretch{1}}
\end{flashcard}

\begin{flashcard}[Definition]{Disk-Speicher}
  \vspace*{\stretch{1}}
  \begin{itemize}
    \item Lift-and-Shift-Szenarien mit Festplatten
    \item Daten, die von einer VM und von außerhalb zugegriffen werden
    \item verwaltete virtuelle Festplatten
  \end{itemize}
  \vspace*{\stretch{1}}
\end{flashcard}


\begin{flashcard}[Definition]{Speicherkontotypen}
  \vspace*{\stretch{1}}
  \begin{itemize}
    \item Allgemein v2:\newline
      Standardspeicher, der alle Zugriffsebenen und Redundanzen unterstützt
    \item Premium-Blockblobs\newline
      Optimiert für Block- und Append-Blobs. Keine Zugriffsebenen und nur LRS/ZRS-Redundanz. Schnellster Zugriff.
    \item Premium-Files\newline
      Optimiert für effiziente Fileshares. Nur LRS/ZRS-Redundanz. Schnellster Zugriff.
    \item Premium-Blockblobs\newline
      Nur Page-Blobs, LRS-Redundanz, keine Zugriffsebenen.
  \end{itemize}
  \vspace*{\stretch{1}}
\end{flashcard}

\begin{flashcard}[Definition]{Sicherer Zugriff auf Speicher}
  \vspace*{\stretch{1}}
  \begin{itemize}
    \item Azure AD-Integration und RBAC\newline
      Lese- und Schreibzugriff auf Daten, Zugriff auf Schlüssel
    \item Einbindung von Azure Files in virtuelle Maschinen
    \item SAS-Token erlauben Zugriff mit Einschränkungen
    \item Zugriff mit gemeinsamen Schlüssel möglich, aber möglicherweise Sicherheitsrisiko:\newline
      benötigt Zugriff auf den Account-Schlüssel (zum signieren)
    \item anonymer Zugriff: öffentliche Blobs/Dateien
    \item HTTPS: im idealfall, HTTP sollte nicht verwendet werden
  \end{itemize}
  \vspace*{\stretch{1}}
\end{flashcard}

\begin{flashcard}[Definition]{Active Directory-Zuang zu Blobs}
  \vspace*{\stretch{1}}
  \begin{itemize}
    \item Verschiedene Level: Container, Speicherkonto,\newline
      und die übliche Hierarchie darüber
    \item Rollen: Daten-Rollen und Verwaltungsrollen
      \begin{itemize}
        \item Storage Blob Data Owner: Besitzer und POSIX-Rollen verwalten
        \item Storage Blob Data Contributor: voller Datenzugriff
        \item Storage Blob Data Reader: Lese-Zugriff
        \item Storage Blob Delegator: kann Benutzerdelegierungsschlüssel erstellen\newline
        (mit eigenen Rechten)
      \end{itemize}
    \item \texttt{Microsoft.Storage/storageAccounts/listkeys/action} für Vollzugriff, z.\,B. im Portal
    \item Reader-Rolle ist auch nötig (auch für Datenzugriff!)
  \end{itemize}
  \vspace*{\stretch{1}}
\end{flashcard}

\begin{flashcard}[Definition]{Speicherkontoschlüssel}
  \vspace*{\stretch{1}}
  \begin{itemize}
    \item zwei Schlüssel
    \item erlauben Rotieren
    \item werden idealerweise in einem Key Vault gespeichert
    \item automatische Erinnerung für Ablaufdatum
  \end{itemize}
  \vspace*{\stretch{1}}
\end{flashcard}

\begin{flashcard}[Definition]{Unveränderlicher Zugriff}
  \vspace*{\stretch{1}}
  \begin{itemize}
    \item Blobs können unveränderlich gemacht werden
    \item nur noch Lese-Operationen möglich, kein Löschen oder Überschreiben
    \item Konfiguration über Richtlinien
      \begin{itemize}
        \item zeitliche Richtlinie: gilt für bestimmte Zeit (Jahre)
        \item gesperrt/ungesperrt: wenn gesperrt, kann die Richtlinie\newline
          (z.\,B. Geltungsdauer) nicht mehr verändert werden
        \item juristische Richtlinie: gilt bis die Richtlinie gelöscht ist
      \end{itemize}
    \item Definiert auf Container-Ebene (Container-Richtlinien), oder für eine Blobversion
  \end{itemize}
  \vspace*{\stretch{1}}
\end{flashcard}


\subsubsectioncard{recommend storage management tools}

\begin{flashcard}[Definition]{Tools mit Zugriff}
  \vspace*{\stretch{1}}
  \begin{itemize}
    \item Azure Portal (mit eingebautem Storage Explorer)
    \item Azure Storage Explorer
    \item Visual Studio Cloud Explorer
    \item \texttt{azcopy}
    \item Azure CLI
    \item \ldots Powershell
  \end{itemize}
  \vspace*{\stretch{1}}
\end{flashcard}
