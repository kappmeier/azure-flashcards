\subsectioncard{Designen von Awendungssicherheit}

\subsubsectioncard{recommend a solution that includes Key Vault}

\begin{flashcard}[Definition]{Key Vault}
  \vspace*{\stretch{1}}
  \begin{itemize}
    \item Verwaltung geheimer Daten:
    \begin{itemize}
      \item Geheimnisse: Passwörter, API-Keys, \ldots
      \item Keys: Erstellung und Speicherung für Verschlüsselung
      \item Zertifikate: TLS, Generierung und Speichern
    \end{itemize}
    \item Sicherheitsstufen: Standard (Software-Schlüssel) und Premium (Hardware-Sicherheitsmodul)
  \end{itemize}
  \vspace*{\stretch{1}}
\end{flashcard}

\begin{flashcard}[Definition]{Key Vault Anwendungsfälle und Szenarien}
  \vspace*{\stretch{1}}
  \begin{itemize}
    \item verhindert versehentliches Leaken\newline
      Zugriff über URI, keine Speichrung im Code
    \item Authentifizierung und Authorisierung bevor Zugriff gestattet wird\newline
      offiziell hat sogar Microsoft keinen Zugriff \ldots
    \item einfache Handhabe: Automatisierung, Lebensdauer, Replikation
    \item einfache Einbindung in andere Azure-Services
    \item Versionierung: Versionen haben UUIDs
  \end{itemize}
  \vspace*{\stretch{1}}
\end{flashcard}

\begin{flashcard}[Definition]{Sicherheitsfeatures Key Vault}
  \vspace*{\stretch{1}}
  \begin{itemize}
    \item Zugriffsbeschränkung über Netzwerk-Firewall\newline
      (IP-Bereiche (CIDR), virtuelle Subnetze, orivate Endpunkte)
    \item Zugriff über HTTPS (TLS)
    \item RBAC-Authentifizierung für Management und optional Daten
    \item optional Datenzugriff über Richtlinien
    \item Unterstützung für bedingten Zugriff und PIM
    \item Logging über Azure Monitor, Ereignisse über Event Grid
    \item Multi-Regional: entweder Partner-Regionen oder ZRS.\newline
      Automatisches Failover wenn möglich. (Nur read-only während failover)
  \end{itemize}
  \vspace*{\stretch{1}}
\end{flashcard}

\begin{flashcard}[Definition]{Datentypen im Key Vault}
  \vspace*{\stretch{1}}
  \begin{itemize}
    \item kryptografische Schlüssel: verschiedene Algorithmen
    \item Geheimnisse
    \item Zertifikate: automatische Erneuerung
    \item Speicherkontoschlüssel: automatische Rotation
  \end{itemize}
  \vspace*{\stretch{1}}
\end{flashcard}

\subsubsectioncard{recommend a solution that includes managed identities}

\begin{flashcard}[Definition]{Verwaltete Identitäten}
  \vspace*{\stretch{1}}
  \begin{itemize}
    \item von Azure verwaltete Identität um Ressourcenzugriff zu erhalten
    \item Vorteil: keine Verwaltung von Zugangsdaten
    \item können überall verwendet werden, wo Azure AD-Authentifizierung möglich ist\newline
      $\Rightarrow$ unzählige Dienste
  \end{itemize}
  \vspace*{\stretch{1}}
\end{flashcard}

\begin{flashcard}[Definition]{Verwaltete Identitäten nutzen}
  \vspace*{\stretch{1}}
  \begin{itemize}
    \item Arten von Identitäten
    \begin{itemize}
      \item Systemseitig zugewiesen\newline
        Automatisch in Azure AD erstellte Identität für eine Ressource in Azure
      \item Nutzerseitig zugewiesen\newline
        Selbst erstellte Identität (als Azure Resource); kann mehreren Ressourcen in Azure zugewiesen werden
    \end{itemize}
    \item Identitäten können Zielressourcen zugewiesen werden (mit RBAC-Rollen)
    \item gängige Operationen wie Anmeldeprotokolle, erstellen, aktivieren, \ldots
  \end{itemize}
  \vspace*{\stretch{1}}
\end{flashcard}

\subsubsectioncard{recommend a solution for integrating applications into Azure AD}


\begin{flashcard}[Definition]{Webanwendung in Azure AD registrieren}
  \vspace*{\stretch{1}}
  \begin{itemize}
    \item benötigt Azure B2C-Mandant
    \item App-Registrierung erstellen, in Azure CLI, Powershell, Portal, Terraform, \ldots
    \item benötigt eine Weiterleitungs-URI, an die das Token nach der Authentifizierung gesendet wird
    \item benötigt Admin-Zustimmung zur Benutzung von Open ID
    \item ein Passwort für die Anwendung muss erstellt werden
  \end{itemize}
  \vspace*{\stretch{1}}
\end{flashcard}
