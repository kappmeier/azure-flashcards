\subsectioncard{Designen von Authorisierung}

\subsubsectioncard{choose an authorization approach}

\begin{flashcard}[Definition]{Authorisierung}
  \vspace*{\stretch{1}}
  \begin{itemize}
    \item Geewährung von Zugriff auf Ressourcen an eine \emph{bereits authentifizierte} Entität\newline
    (normalerweise Person, aber auch Service Principal, \ldots)
    \item Präzise Spezifikation welche Handlung mti der Ressource vorgenommen werden darf
    \item Oauth 2 für Microsoft/Azure AD
  \end{itemize}
  \vspace*{\stretch{1}}
\end{flashcard}

\begin{flashcard}[Definition]{Authentifizierung ist Authorisierung}
  \vspace*{\stretch{1}}
  \begin{itemize}
    \item einfachste Art der authentifizierung
    \item Zugangsentscheidung nur Abhängig davon, dass die Entität Authentifiziert worden ist
  \end{itemize}
  \vspace*{\stretch{1}}
\end{flashcard}

\begin{flashcard}[Definition]{Authorisierungs-Ansätze}
  \vspace*{\stretch{1}}
  \begin{itemize}
    \item Authentifizierung ist Authorisierung
    \item Zugriffskontrolllisten\newline
      z.\,B. für Schlüsseltresore
    \item RBAC\newline
      z.\,B. rollenbasierte Zugriffssteuerung für Anwendungen, Azure RBAC oder Azure AD RBAC
    \item Attributbasierte Zugriffskontrolle\newline
      z.\,B dynamische Gruppen, Azzure RBAC mit Bedingungen
  \end{itemize}
  \vspace*{\stretch{1}}
\end{flashcard}

\begin{flashcard}[Definition]{RBAC}
  \vspace*{\stretch{1}}
  \begin{itemize}
    \item Rollenbasierte Zugriffssteuerung
    \item häufig genutzte Methode für Authorisierung von Anwendungen
    \item Rollen beschreiben genau die Art der zugelassenen Handlungen
    \item Zugriff wird dann für Rollen (\emph{nicht Entitäten}) gewährt
    \item Administratoren können Entitäten bestimmten Rollen zuweisen,\newline
      der Zugriff leitet sich dann von der Rolle ab
  \end{itemize}
  \vspace*{\stretch{1}}
\end{flashcard}

\begin{flashcard}[Definition]{Attributbasierte Zugriffskontrolle}
  \vspace*{\stretch{1}}
  \begin{itemize}
    \item ABAC, präziser als RBAC und erlaubt dynamischen Zugriff
    \item Zugriff über Attribute der zugreifenden Entität
    \item Festlegung durch Regeln über Attribute der zugreifenden Entität \emph{und} der Umgebung
  \end{itemize}
  \vspace*{\stretch{1}}
\end{flashcard}

\begin{flashcard}[Definition]{Zugriffssteuerungslisten}
  \vspace*{\stretch{1}}
  \begin{itemize}
    \item ACL
    \item Verwaltung von expliziten Listen von Zugriffsberechtigten Entitäten
    \item mehr Kontrolle als als nur über Authorisierung
    \item Aufwändig bei wachsender Anzahl an Entitäten
  \end{itemize}
  \vspace*{\stretch{1}}
\end{flashcard}

\subsubsectioncard{recommend a hierarchical structure that includes management groups, subscriptions and
resource groups}

\begin{flashcard}[Definition]{Verwaltungsebenen}
  \vspace*{\stretch{1}}
  \begin{itemize}
    \item Hierarchische Anordnung von Ebenen
    \item niedrigere Ebenen erben Einstellungen von höheren Ebenen
    \item Verwaltungseinstellungen werden auf ein Element einer Ebene angewendet
  \end{itemize}
  \vspace*{\stretch{1}}    
\end{flashcard}

\begin{flashcard}[Definition]{Die 4 Verwaltungsebenen}
  \vspace*{\stretch{1}}
  \begin{itemize}
    \item Verwaltungsgruppen
    \item Abonnements
    \item Ressourcnegruppen
    \item Ressourcen
  \end{itemize}
  \vspace*{\stretch{1}}
\end{flashcard}

\begin{flashcard}[Definition]{Verwaltungsgruppen}
  \vspace*{\stretch{1}}
  \begin{itemize}
    \item können mehrere Abonnements und \emph{auch Verwaltungsgruppen} enthalten\newline
      (Oberste und 6 weitere Ebenen)
    \item nur Verwaltungseinstellungen möglich, z.\,B. Rechtevergabe
    \item übergeordnet existiert noch der Mandant selbst / die Stammverwaltungsgruppe
    \item ermöglicht effiziente Verwaltung bei zahlreichen Abonnements\newline
      Richtlinien, Complience, \ldots
  \end{itemize}
  \vspace*{\stretch{1}}
\end{flashcard}

\begin{flashcard}[Definition]{Abonnements}
  \vspace*{\stretch{1}}
  \begin{itemize}
    \item können, müssen aber nicht in einer Verwaltungsgruppe sein
    \item oberste Ebene zum Verwalten von Ressourcen
    \item derzeit noch oberste Ebene auf der benutzerdefinierte Rollen zugewiesen werden können
  \end{itemize}
  \vspace*{\stretch{1}}
\end{flashcard}

\begin{flashcard}[Definition]{Ressourcengruppen}
  \vspace*{\stretch{1}}
  \begin{itemize}
    \item Container für einzelne Ressourcen
    \item sind in einem Abonnement enthalten\newline
      (maximal 980)
    \item verwaltet von Azure Resource Manager
  \end{itemize}
  \vspace*{\stretch{1}}
\end{flashcard}

\begin{flashcard}[Definition]{Ressourcen}
  \vspace*{\stretch{1}}
  \begin{itemize}
    \item einzelne Ressourcen\newline
      (800 pro Ressourcentyp!)
    \item sind in einer Ressourcengruppe
  \end{itemize}
  \vspace*{\stretch{1}}
\end{flashcard}

\subsubsectioncard{recommend an access management solution including RBAC policies, access reviews, role assignments, Privileged Identity Management (PIM), Azure AD Identity Protection, Just In Time (JIT) access}

\begin{flashcard}[Definition]{RBAC-Richtlinien}
  \vspace*{\stretch{1}}
  \begin{itemize}
    \item Authorisierung zum Zugriff auf Ressourcen in Azure
    \item Rollen werden Nutzern, Identitäten, Gruppen, oder Service Principals zugewiesen\newline
      \emph{Rollenzuweisung}
    \item Zugriff auf verschiedenen Ebenen (Verwaltungsebene -> Ressourcen)
    \item einige Rollen haben zusätzliche Bedingungen
    \item Berechtigungen mehrerer Zuweisungen werden vereinigt
    \item zwei Ebenen:
      \begin{itemize}
        \item Aktionen
        \item Daten
      \end{itemize}
  \end{itemize}
  \vspace*{\stretch{1}}
\end{flashcard}

\begin{flashcard}[Definition]{Ablehnungszuweisungen}
  \vspace*{\stretch{1}}
  \begin{itemize}
    \item blockieren Aktionen für Nutzern
    \item haben Vorrang vor Rollenzuweisungen
    \item erlauben es für Benutzer Aktionen einzuschränken,\newline
      auch wenn diese eigentlich die Rechte hätten
  \end{itemize}
  \vspace*{\stretch{1}}
\end{flashcard}

\begin{flashcard}[Definition]{Active Directory-Zugriffsüberprüfungen}
  \vspace*{\stretch{1}}
  Access Review
  \begin{itemize}
    \item regelmäßige Überprüfung von Gruppenmitgliedschaften für Anwendungen und Rollen
    \item z.\,B. automatische Zuweisungen für neue Mitarbeiter, oder Entfernung bei Wechsel
    \item weit reichende Zugriffe werden entfernt
  \end{itemize}
  \vspace*{\stretch{1}}
\end{flashcard}

\begin{flashcard}[Definition]{Szenarien von Zugriffsüberprüfungen}
  \vspace*{\stretch{1}}
  \begin{itemize}
    \item viele Nutzer mit Privilegien
    \item weiternutzung von Gruppen zu anderen Zwecken
    \item Zugriff auf kritische Daten
    \item Verwalten von Ausnahmen
    \item externe Nutzer, Drittanbieter, Self-Service
  \end{itemize}
  \vspace*{\stretch{1}}
\end{flashcard}

\begin{flashcard}[Definition]{Identity Protection}
  \vspace*{\stretch{1}}
  \begin{itemize}
    \item automatisierte Behandlung von Risiken
    \item erkennt Risiken und kann in gewissem Umfang automatisch behandeln
    \item Untersuchung von Risiken
    \item Exportieren der gesammelten Risikodaten
  \end{itemize}
  Verwendet die gleichen Signale, wie für automatisierten bedingten Zugriff
  \vspace*{\stretch{1}}
\end{flashcard}

\begin{flashcard}[Definition]{Erkannte Risiken}
  \vspace*{\stretch{1}}
  Abgeleitet aus beobachteten Daten von Microsoft:
  \begin{itemize}
    \item anonyme IP-Adressen
    \item ungewöhnliche Ortswechsel
    \item Schadsoftwareeinsatz
    \item ungewöhnliche Anmeldungen, komprommitierte Accounts
    \item Kennwortspray
    \item \ldots
  \end{itemize}
  \vspace*{\stretch{1}}
\end{flashcard}

\begin{flashcard}[Definition]{Einsatz von Identity Protection}
  \vspace*{\stretch{1}}
  Abgeleitet aus beobachteten Daten von Microsoft:
  \begin{itemize}
    \item Risikoeinstufung in niedrig, mittel, hoch
    \item exportieren der Daten zur Analyse mit anderen Tools
    \item benötigt globale Rechte (Admin, Sicherheitsadmin, Sicherheitsoperator, \ldots)
    \item enthalten in AZ Premium P2-Lizenz
    \item \ldots
  \end{itemize}
  \vspace*{\stretch{1}}
\end{flashcard}

\begin{flashcard}[Definition]{Just in Time Access (JIT)}
  \vspace*{\stretch{1}}
  \begin{itemize}
    \item begrenzter Zugang zu VMs
    \item enthält Auditing während des Logins
    \item Kann für einzelne Ports konfiguriert werden
  \end{itemize}
  \vspace*{\stretch{1}}
\end{flashcard}

\begin{flashcard}[Definition]{Privileged Identity Management (PIM)}
  \vspace*{\stretch{1}}
  \begin{itemize}
    \item möglichst wenige Nutzer bekommen Zugriff auf wichtige Ressourcen
    \item andere Nutzer können Just-in-Time privilegierten Zugriff auf Azure und Azure AD bekommen
    \item benötigt Premium P2-Lizenz
  \end{itemize}
  \vspace*{\stretch{1}}
\end{flashcard}

\begin{flashcard}[]{Audit mit PIM}
  \vspace*{\stretch{1}}
  \begin{itemize}
    \item Aktionen abhängig von Administrator-Zuweisungen
    \item Alarme für Benutzer
    \item Review von Administrator-Rollenzuweisungen
  \end{itemize}
  \vspace*{\stretch{1}}
\end{flashcard}

\begin{flashcard}[Definition]{Dynamische Rollenzuweisung mit PIM}
  \vspace*{\stretch{1}}
  \begin{itemize}
    \item Just-in-Time-Zugriff, zeitgebunden
    \item Genehmigung
    \item mehrstufige Authentifizierung
    \item Anforderung einer Begründung für die erhöhung der Privilegien
    \item Benachrichtigungen, Zugriffsüberpr und Überwachung
  \end{itemize}
  Mit der Überwachung kann verhindert werden, dass der letzte aktive Administrator entfernt wird
  \vspace*{\stretch{1}}
\end{flashcard}

