\subsectioncard{Designen von Aufsicht}

\subsubsectioncard{recommend a strategy for tagging}

\begin{flashcard}[Definition]{Tags}
  \vspace*{\stretch{1}}
  \begin{itemize}
    \item Key-Value Text-Paare die allen ARM-Ressourcen zugewiesen werden können
    \item Einteilung von Ressourcen in Gruppen zur Klassifizierung:\newline
      Bereich, Stage, Umgebungen, Costcenter, \ldots
    \item Tags erlauben Gruppierung unabhängig von Abonnements, Verwaltungsgruppen, oder anderen Beschränkungen
    \item werden nicht vererbt (z.\,B. von Ressourcengruppe zu Ressorucen)\newline
      lässt sich aber mit Richtlinien bewirken
    \item Grenzwerte: 50 Tags pro Ressource, 512 Zeichen für den Namen, 256 für den Wert\newline
      (Speicherkonten nur 128 für den Namen)
  \end{itemize}
  $\Rightarrow$ nützlich zur Umsetzung einer Government-Strategie
  \vspace*{\stretch{1}}
\end{flashcard}

\begin{flashcard}[Definition]{Mögliche Anwendungsfälle für Tags}
  \vspace*{\stretch{1}}
  \begin{itemize}
    \item Ressourcenverwaltung: Umgebung, Anwendungsfälle, Besitzer
    \item Kosten: übersicht über Kostencenter, Budgets
    \item Operationelles: markieren von kritischen Ressourcen, SLAs, \ldots
    \item Sicherheit: Klassifizierung von kritischen Daten
    \item Aufsicht: garantieren von regulatorischen Bedingungen
    \item Automatisierung: Ziele von Operationen auf Grundlage von Ressourcentags
  \end{itemize}
  \vspace*{\stretch{1}}
\end{flashcard}

\begin{flashcard}[Definition]{Vergeben von Tags}
  \vspace*{\stretch{1}}
  \begin{itemize}
    \item Allgemeine Tag-Berechtigung: \texttt{Microsoft.Resources/tags}\newline
      Erlaubt \emph{alle} Ressorucen zu taggen 
    \item Schreibzugriff auf eine Ressource, z.\,B. \emph{Mitwirkender}
  \end{itemize}
  \vspace*{\stretch{1}}
\end{flashcard}

\subsubsectioncard{recommend a strategy for using Azure Policy}

\begin{flashcard}[Definition]{Azure Policy}
  \vspace*{\stretch{1}}
  \begin{itemize}
    \item Service, der Richtlinien und Regeln verwaltet
    \item Richtlinien werden für Ressourcen vergeben nach\newline
      Verwaltungsgruppe, Abonnement, Ressourcengruppe
  \end{itemize}
  $\Rightarrow$: arbeiten mit vielen Services/Bereichen zusammen
  \vspace*{\stretch{1}}
\end{flashcard}

\begin{flashcard}[Definition]{Durchsetzen von Regeln}
  \vspace*{\stretch{1}}
  \begin{itemize}
    \item Resourcen, die gegen Regeln verstoßen können nicht erstellt werden
    \item einige Regeln können automatisch durchgesetzt werden
    \begin{itemize}
      \item Tags z.B.
    \end{itemize}
    \item Übersicht über Regelverletzung
  \end{itemize}
  \vspace*{\stretch{1}}
\end{flashcard}

\begin{flashcard}[Definition]{Bereiche der Regelung}
  \vspace*{\stretch{1}}
  \begin{itemize}
    \item erlaubte VM SKU
    \item zulässige Regionen für Bereitstellungen
    \item Authentifikation: 2-Faktor-Authentifikation
    \item CORS (zulässige adressen für Web-Services)
    \item Systemupdates erzwingen (Security Center)
  \end{itemize}
  \vspace*{\stretch{1}}
\end{flashcard}

\begin{flashcard}[Definition]{Initiativen}
  \vspace*{\stretch{1}}
  \begin{itemize}
    \item Sammlung mehrerer Policies
    \item Ziel: Compliance für ein Ziel herstellen (z.B. Sicherheitszertifikat, ISO 27k1, \ldots)
    \item Zuweisen wie Policy an Gruppen
  \end{itemize}
  \vspace*{\stretch{1}}
\end{flashcard}

\subsubsectioncard{recommend a solution for using Azure Blueprints}

\begin{flashcard}[Definition]{Azure Blueprint}
  \vspace*{\stretch{1}}
  \begin{itemize}
    \item Verwaltet Richtlinien und Bereitstellungen über mehrere Abonnements hinweg
    \item Folgendes kann in Blueprints definiert werden:
      \begin{itemize}
        \item Rollenzuweisungen (RBAC)
        \item Richtlinien
        \item Ressourcengruppen
        \item Vorlagen für Ressourcen-Manager (ARM)
      \end{itemize}
  \end{itemize}
  \vspace*{\stretch{1}}
\end{flashcard}

\begin{flashcard}[Definition]{Anwendungszenarien für Blueprints}
  \vspace*{\stretch{1}}
  \begin{itemize}
    \item Skalierung von Richtlinien über eine ganze Organisation (mit vielen Abonnements) hinweg\newline
      zentrale Verwaltung!
  \end{itemize}
  \vspace*{\stretch{1}}
\end{flashcard}

\begin{flashcard}[Definition]{Blueprints anwenden}
  \vspace*{\stretch{1}}
  \begin{enumerate}
    \item Blueprint erstellen und zuweisen (z.\,B. Verwaltungsgruppe)
    \item[!] zunächst ist ein Blueprint im Status ``Entwurf''. Muss veröffentlicht werden, bevor Zuweisung möglich ist
    \item Ressourcen werden erstellt (z.\,B. Policy, Ressourcegruppen, \ldots): \emph{Artefakt}
    \item Analyse der erstellten Ressourcen möglich
  \end{enumerate}
  Parameter können pro Gruppe (Zuweisung) festgelegt werden
  \vspace*{\stretch{1}}
\end{flashcard}

\subsubsectioncard{recommend a solution that leverages Azure Resource Graph}

\begin{flashcard}[Definition]{Resource Graph}
  \vspace*{\stretch{1}}
  \begin{enumerate}
    \item Service, der alle Azure Resourcen durchsucht
    \item erlaubt Query-Anfragen, um Resourcen aufzulisten
    \item Visualisierung der Ergebnisse
  \end{enumerate}
  \vspace*{\stretch{1}}
\end{flashcard}

\begin{flashcard}[Definition]{Anwendungen für Resource Graph}
  \vspace*{\stretch{1}}
  \begin{enumerate}
    \item filtern von Ressourcen, die Bedingungen erfüllen
    \item z.\,B. VMs, die Sicherheitsfunktionen nicht erfüllen
    \item Einhaltung von Regionen, Namen, \ldots
    \item zu große Ressorucen (z.\,B. ungenutzte VMs) finden
  \end{enumerate}
  \vspace*{\stretch{1}}
\end{flashcard}
