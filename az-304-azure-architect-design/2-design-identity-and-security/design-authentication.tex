\subsubsectioncard{recommend a solution for single-sign on}

\begin{flashcard}[Definition]{Single Sign-on (SSO)}
  \vspace*{\stretch{1}}
  \begin{itemize}
    \item Benutze Azure-Login zur Authentifizierung in anderen Services
    \item OAuth 2.0
    \item Vorteil: 
  \end{itemize}
  \vspace*{\stretch{1}}
\end{flashcard}

\begin{flashcard}[Definition]{Single Sign-on für externe Anwendungen}
  \vspace*{\stretch{1}}
  \begin{itemize}
    \item AD FS
    \item benötigt moderne Protokolle wie SAML, OpenID Connect, \ldots
  \end{itemize}
  \vspace*{\stretch{1}}
\end{flashcard}

\begin{flashcard}[Definition]{Voraussetzungen für SSO}
  \vspace*{\stretch{1}}
  \begin{itemize}
    \item Azure AD Premium P1
  \end{itemize}
  \vspace*{\stretch{1}}
\end{flashcard}

\begin{flashcard}[Definition]{Varianten von SSO}
  \vspace*{\stretch{1}}
  \begin{itemize}
    \item Verbundanmeldung: mehrere Identitätsanbieter werden verwendet
    \item Kennwort: Nutzer definieren Zugriffseinstellungen, die anschließend von Azure AD bereitgestellt werden
    \item verknüpfte Anwendungen: Anmeldung über Links, Benutzer benötigen Konto für den Service
  \end{itemize}
  \vspace*{\stretch{1}}
\end{flashcard}

\subsubsectioncard{recommend a solution for authentication}

\begin{flashcard}[Definition]{Authentifizierung}
  \vspace*{\stretch{1}}
  \begin{itemize}
    \item Ziel: Identität einer Entität ist gültig
    \item Authentifizierung über Azure AD
    \item keine Notwendigkeit, selbst Authentifizierung zu betreiben
  \end{itemize}
  \vspace*{\stretch{1}}
\end{flashcard}

\begin{flashcard}[Definition]{Varianten der Authentifizierung}
  \vspace*{\stretch{1}}
  \begin{itemize}
    \item Passwort: standard, einfach, problematisch
    \item zusätzlicher Faktor (z.\,B. Gegenstand wie Key, OTP, oder biometrisch)
    \begin{itemize}
      \item ok: SMS, Telefon
      \item gut: OTP, Token OTP (Hardware)
      \item ideal: ohne Passwort, Biometrie, FIDO2
    \end{itemize}
  \end{itemize}
  \vspace*{\stretch{1}}
\end{flashcard}

\begin{flashcard}[Definition]{Levels der Sicherheit}
  \vspace*{\stretch{1}}
  \begin{itemize}
    \item Zweifaktor-Authentifizierung bei der Registrierung
    \item für Administratoren
    \item in bestimmten Anwendungsfällen
  \end{itemize}
  \vspace*{\stretch{1}}
\end{flashcard}

\subsubsectioncard{recommend a solution for Conditional Access, including multi-factor authentication}

\begin{flashcard}[Definition]{Mehrfaktor-Authentifizierung}
  \vspace*{\stretch{1}}
  \begin{itemize}
    \item zusätzlich zu Username/Passwort weiterer Faktor
    \item erzwungen, wenn Microsoft Sicherheitsstandard aktiv ist
    \item empfohlen für Administratoren
  \end{itemize}
  \vspace*{\stretch{1}}
\end{flashcard}

\begin{flashcard}[Definition]{Schutz durch Mehrfaktor-Authentifizierung}
  \vspace*{\stretch{1}}
  \begin{itemize}
    \item alle Anwendungen im Azure AD können geschützt werden
    \item auch SaaS-Anwendungen
    \item nicht möglich für Legacyauthentifizierung!\newline
    wichtiges Einfallstor für Angriffe!\newline
    z.\,B. Office 2010, Protokolle wie POP3, IMAP, SMTP, \ldots
  \end{itemize}
  \vspace*{\stretch{1}}
\end{flashcard}

\begin{flashcard}[Definition]{Bedingter Zugriff}
  \vspace*{\stretch{1}}
  \begin{itemize}
    \item nicht gleichzeitig mit Microsoft Sicherheitsstandards
    \item Anforderung von zweitem Faktor abhängig von der Situation
    \item bewertet außer Identität auch:\newline
      Ort, Gerät, Anwendung, Zeit
    \item Zugriff kann automatisch abgelehnt oder mit MFA durchgeführt werden\newline
      (oder zugelassen werden)
    \item Berichtsmodus um Zugriffe zu evaluieren, z.\,B. wenn Zugriffsarten nicht vorhersehbar sind
    \item benötigt Azure AD Premium P1 oder höher
  \end{itemize}
  \vspace*{\stretch{1}}
\end{flashcard}

\begin{flashcard}[Definition]{Eigenschaften von bedingtem Zugriff}
  \vspace*{\stretch{1}}
  \begin{itemize}
    \item Nutzer müssen nicht ihren Flow unterbrechen
    \item Risikobewertung jeder Anmeldung
    \item Complience
    \item Unterstützung für Zero Trust-Umgebungen
  \end{itemize}
  \vspace*{\stretch{1}}
\end{flashcard}

\begin{flashcard}[Definition]{Richtlinien für bedingten Zugriff}
  \vspace*{\stretch{1}}
  \begin{itemize}
    \item Wenn-Dann-Bedingungen
    \item Einteilung nach:\newline
      Nutzer/Gruppen, Abwendungen, Geräte/Plattformen, Regionen
    \item Zugriffssteuerung für den Dann-Teil\newline
      Zugriffstoken wird ausgestellt
    \item Nicht abgedeckte Bedingungen müssen explizit Ausgeschlossen werden!\newline
      z.\,B. nur eine Gruppe erhält zugriff mit Mehrfach-Authentifizierung erfordert:\newline
      nicht-Gruppe wird abgelehnt (sonst erhält nicht-Gruppe Zugriff \emph{ohne MFA})
  \end{itemize}
  \vspace*{\stretch{1}}
\end{flashcard}

\begin{flashcard}[Definition]{Absichern bei bedingtem Zugriff}
  \vspace*{\stretch{1}}
  \begin{itemize}
    \item Es können versehentlich Administratoren ausgeschlossen werden!
    \item Sinnvoll ein Notfallkonto zu haben
    \item Testen! der Zugriffsrichtlinien\newline
      Szenarien, erwartete Ergebnisse, Testnutzer im AD
  \end{itemize}
  \vspace*{\stretch{1}}
\end{flashcard}

\begin{flashcard}[Definition]{Anwendungssicherheit}
  \vspace*{\stretch{1}}
  \begin{itemize}
    \item Datenexfiltration verhindern
    \item Complience bei Dokumenten (Up- und Download)
    \item Zugriff blockieren abhängig vom Risiko (z.\,B. veraltete Zertifikate)
    \item spezielle Aktivitäten der Anwendung verbieten
  \end{itemize}
  \vspace*{\stretch{1}}
\end{flashcard}

\subsubsectioncard{recommend a solution for network access authentication}

\begin{flashcard}[Definition]{Netzwerk-Zugriffssteuerung}
  \vspace*{\stretch{1}}
  \begin{itemize}
    \item Teil des bedingten Zugriffs
    \item auch benannte Orte
    \item Zugriff wird gestattet abhängig vom Netzwerk an dem ein Nutzer eingeloggt ist\newline
      z.\,B. Hauptsitz, etc.
  \end{itemize}
  \vspace*{\stretch{1}}
\end{flashcard}

\begin{flashcard}[Definition]{Netzwerk-Zugriff definieren}
  \vspace*{\stretch{1}}
  \begin{itemize}
    \item Ortsabhängig
    \item IPv4/IPv6-Bereiche
  \end{itemize}
  \vspace*{\stretch{1}}
\end{flashcard}

\subsubsectioncard{recommend a solution for a hybrid identity including Azure AD Connect, Azure AD
Connect cloud sync and Azure AD Connect Health}

\begin{flashcard}[Definition]{Verbinden von Verzeichnissen}
  \vspace*{\stretch{1}}
  \begin{itemize}
    \item lokales Verzeichnis: Azure DS
    \item Cloud-Verzeichnis: Azure AD
    \item synchronisierung möglich (DS => AD), mit Ausnahmen
    \item Azure AD Seamless SSO für gleichzeitiges Login (nicht mit AD FS!)
  \end{itemize}
  \vspace*{\stretch{1}}
\end{flashcard}

\begin{flashcard}[Definition]{Azure AD Connect}
  \vspace*{\stretch{1}}
  \begin{itemize}
    \item Synchronisierung zwischen AD DS und Azure AD
    \item Tool mit Wizard-Unterstützung zur Feature-Auswahl:
    \begin{itemize}
      \item Passwörter zurückschreiben
      \item Synchronisierung von Azure AD
    \end{itemize}
    \item Synchronisierung:
    \begin{itemize}
      \item Nutzer, Gruppen, \ldots werden zu Azure AD hinzugefügt bzw. gelöscht
      \item viele, aber nicht alle Attribute werden synchronisiert
      \item Lizenzen werden weder automatisch hinzugefügt noch deaktiviert (falls Nutzer deaktiviert werden)
    \end{itemize}
  \end{itemize}
  \vspace*{\stretch{1}}
\end{flashcard}

\begin{flashcard}[Definition]{Verwalter der Autorität}
  \vspace*{\stretch{1}}
  \begin{itemize}
    \item nur eine einzige Quelle hat Autorität
    \item nach Synchronisierung mit Azure AD Connect ist die Autorität im lokalen AD DS
    \item das Azure AD hat nur Lesezugriff
  \end{itemize}
  \vspace*{\stretch{1}}
\end{flashcard}

\begin{flashcard}[Definition]{Durchführen der Synchronisierung}
  \vspace*{\stretch{1}}
  \begin{itemize}
    \item benötigt Administratorzugriff auf AD DS und Azure AD\newline
      Es kann nicht der Administrator-Account vom Azure Konto sein!
    \item idealerweise ein dedizierter Account
    \item sollte auf einem eigenen Server laufen
    \item das lokale Verzeichnis sollte fehlerfrei sein. Vorbereiten!\newline
      Überprüfungs-Tools sind verfügbar
  \end{itemize}
  \vspace*{\stretch{1}}
\end{flashcard}

\begin{flashcard}[Definition]{Azure AD Connect Health}
  \vspace*{\stretch{1}}
  \begin{itemize}
    \item Service zur Benachrichtigung bei Schwierigkeiten mit Azure AD Connect
    \item kann Emails verschicken an Verantwortliche
  \end{itemize}
  \vspace*{\stretch{1}}
\end{flashcard}

\subsubsectioncard{recommend a solution for user self-service}

\begin{flashcard}[Definition]{Self-service Passwort-Reset}
  \vspace*{\stretch{1}}
  \begin{enumerate}
    \item Feature von Azure AD
    \item Kostenersparnis (kein Admin oder Service Desk)
    \item Passwort ändern/neu setzen, Account-Freischaltung
    \item einfach für Nutzer
  \end{enumerate}
  \vspace*{\stretch{1}}
\end{flashcard}

\begin{flashcard}[Definition]{Features von self service password reset}
  \vspace*{\stretch{1}}
  \begin{enumerate}
    \item Passwort writeback in lokales Verzeichnis
    \item logging von Passwortänderungen
    \item Passwort ändern/neu setzen, Account-Freischaltung
  \end{enumerate}
  \vspace*{\stretch{1}}
\end{flashcard}

\subsubsectioncard{recommend and implement a solution for B2B integration}

\begin{flashcard}[Definition]{Zusammenarbeit B2B mit Azure Active Directory}
  \vspace*{\stretch{1}}
  \begin{enumerate}
    \item Zugriff auf Ressourcen für externe Partner
    \item externe Partner müssen selbst kein Active Directory betreiben
    \item Zugriff über Einladungen
    \item Beretistellung einer API z.\,B. für Self-Service-Applikationen
    \item Integration in Services wie z.\,B. Conditional Access möglich
  \end{enumerate}
  \vspace*{\stretch{1}}
\end{flashcard}

\begin{flashcard}[Definition]{Anmeldung der Partner bei B2B}
  \vspace*{\stretch{1}}
  \begin{enumerate}
    \item Partner nutzen eigene Identitätslösung
    \item Anmeldung über Geschäfts, Schul- oder sonstiges Konto\newline
    (Email, SAML-, WS-Fed)
    \item keine Synchronisation oder Kennwortverwaltung erforderlich!
    \item eigene Anmeldungs-Workflows möglich\newline
      z.\,B. für benutzerspezifische Genehmigung, \ldots
  \end{enumerate}
  \vspace*{\stretch{1}}
\end{flashcard}

\begin{flashcard}[Definition]{Einladen der Partner}
  \vspace*{\stretch{1}}
  \begin{enumerate}
    \item Gastbenutzer erstellen
    \item Gruppen und Berechtigungen zuweisen
    \item Einladung schicken per Email
    \item Anwendungen (bzw. Besitzer) können Gastzugriff selbst verwalten\newline
    (wenn Berechtigung delegiert wurde)
  \end{enumerate}
  \vspace*{\stretch{1}}
\end{flashcard}

\begin{flashcard}[Definition]{Integration von Identitätslösungen}
  \vspace*{\stretch{1}}  
  \begin{enumerate}
    \item Einige Identitätslösungen können direkt integriert werden:\newline
    Google, Facebook, Microsoft-Konto
    \item Nutzer müssen dann kein neues Konto erzeugen
  \end{enumerate}
  \vspace*{\stretch{1}}
\end{flashcard}

