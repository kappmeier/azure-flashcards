\subsectioncard{Designen kostenoptimaler Lösungen}

\subsubsectioncard{design a solution for cost management and cost reporting}

\begin{flashcard}[Definition]{Kostenschätzung}
  \vspace*{\stretch{1}}
  \begin{itemize}
    \item bevor in der Cloud bereitgestellt wird
    \item Gesamtkosten/Total Cost of Ownership
    \item ermöglicht Vergleich von lokalen und Cloud-Kosten
  \end{itemize}
  \vspace*{\stretch{1}}
\end{flashcard}

\begin{flashcard}[Definition]{Kostenanalyse}
  \vspace*{\stretch{1}}
  \begin{itemize}
    \item Tags für Kostenabrechnung
    \item Leser-Berechtigung für Kostenreports für alle Mitarbeiter
    \item Cloud-Spend in der Abonnement-Übersicht\newline
      ermöglicht tiefreifende Analyse nach Ressource, Service, \ldots
  \end{itemize}
  \vspace*{\stretch{1}}
\end{flashcard}

\begin{flashcard}[Definition]{Systeme zur Unterstützung der Kostenübersicht}
  \vspace*{\stretch{1}}
  \begin{itemize}
    \item Azure Cost Management -> zur detaillierten Analyse der Kostenstruktur
    \item Azure Advisor -> gibt automatische Hinweise
    \item Azure Kostenabrechnungen -> nur Übersicht über Kosten
  \end{itemize}
  \vspace*{\stretch{1}}
\end{flashcard}

\begin{flashcard}[Definition]{Budgets und Alarme}
  \vspace*{\stretch{1}}
  \begin{itemize}
    \item zuweisen von Budgets zu
      \begin{itemize}
        \item Abonnements
        \item Verwaltungsgruppen
        \item Ressourcengruppen
        \item mehrere Ressourcen
      \end{itemize}
    \item Information, sobald Kosten aus dem Rahmen laufen (Email)
    \item Trigger von Azure Monitor Action Groups
  \end{itemize}
  \vspace*{\stretch{1}}
\end{flashcard}

\begin{flashcard}[Definition]{Detaillierte Analyse}
  \vspace*{\stretch{1}}
  \begin{itemize}
    \item über die Standard-Analyse auf der Webseite hinaus
    \item Datenaufbereitung in Power BI
    \item Azure Cost Management Power BI-App (Verbinden)
  \end{itemize}
  \vspace*{\stretch{1}}
\end{flashcard}

\subsubsectioncard{recommend solutions to minimize costs}

\begin{flashcard}[Definition]{Automatische skalierung}
  \vspace*{\stretch{1}}
  Verschiedene Services bieten automatische Skalierungen:
  \begin{itemize}
    \item VM Skalierungsgruppen
    \item App-Services
    \item VMs herunterfahren (Automation, geplanter shutdown, oder auch manuell)
    \item PaaS wie MySQL: elastische Pools zum Abfangen von Spitzen
  \end{itemize}
  \vspace*{\stretch{1}}
\end{flashcard}

\begin{flashcard}[Definition]{Hybrid benefit}
  \vspace*{\stretch{1}}
  Nutzung eigener Lizenzen für Cloud-Workloads
  \begin{itemize}
    \item Betriebssystem (Windows)
    \item Datenbank (SQL Server)
  \end{itemize}
  \vspace*{\stretch{1}}
\end{flashcard}

\begin{flashcard}[Definition]{Günstigere Hardware}
  \vspace*{\stretch{1}}
  \begin{itemize}
    \item kleinere VM sizes
    \item langsemerer Speicher, Disks, archive/kalter Blobspeicher\ldots
  \end{itemize}
  Und cleanup!
  \begin{itemize}
    \item alte Disk images
    \item Snapshots, versteckte Dateien, \ldots
  \end{itemize}
  \vspace*{\stretch{1}}
\end{flashcard}

\begin{flashcard}[Definition]{Beschränkung der erlaubten Resourcen}
  \vspace*{\stretch{1}}
  \begin{itemize}
    \item Azure Policy/Blueprint
    \item nur bestimmte Tier und VM-Größen erlauben
    \item automatischer Shutdown nach Feierabend
    \item Ressourcen in Gruppen und Abonnements einteilen!
    \item Ausnahmen sind möglich
  \end{itemize}
  \vspace*{\stretch{1}}
\end{flashcard}

\begin{flashcard}[Definition]{Reservierungen}
  \vspace*{\stretch{1}}
  \begin{itemize}
    \item Reserved VM-Instanzen (1y, 3y)
    \item Rerservierter Speicher (1y, 3y)
    \item günstige VMs: Spot-Instanzen
    \item Datenbanken: Cosmos DB, SQL
  \end{itemize}
  \vspace*{\stretch{1}}
\end{flashcard}

\begin{flashcard}[Definition]{Azure Advisor}
  \vspace*{\stretch{1}}
  Automatische Vorschläge um Kosten zu sparen
  \begin{itemize}
    \item nicht ausgelastete Ressourcen
    \item Reservierungen
    \item fehlerhafte VM-Größen
    \item Netzwerk: Gateways, Expressrouten
  \end{itemize}
  \vspace*{\stretch{1}}
\end{flashcard}

\begin{flashcard}[Definition]{Automatische Warnungen}
  \vspace*{\stretch{1}}
  
  \begin{itemize}
    \item wenn Budget eines Abonnements überschritten wird (soft limit)
    \item Abteilungs-Limits (nur für Enterprise Agreement-Kunden)
    \item Kredit-Warnung (nur für Enterprise Agreement-Kunden)
  \end{itemize}
  \vspace*{\stretch{1}}
\end{flashcard}
