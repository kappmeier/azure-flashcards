% A flashcard like document containing content relevant to prepare for Microsoft AZ-900 Azure
% fundamentals exam. Supposed to be used by mobile/tablet phones, i.e. each page contains a single
% flash card.
\documentclass{scrartcl}

\usepackage[hidelinks]{hyperref}

% Set up XeLaTex font configuration
\usepackage{fontspec}
%\setmainfont{Latin Modern Sans}
\setmainfont{Myriad Pro}

\usepackage{polyglossia}
\enablehyphenation
\setdefaultlanguage[]{german}
\setotherlanguage[]{english}

% Size of flash cards
\usepackage[a6paper,landscape,margin=1cm]{geometry}

% Vertical alignment also of the title page
\usepackage{titling}
\renewcommand\maketitlehooka{\null\mbox{}\vfill}
\renewcommand\maketitlehookd{\vfill\null}

% Nothing to see on flash cards
\pagestyle{empty}
\thispagestyle{empty}

% Support showing the license as required by CC-BY-SA-4.0
\usepackage[type={CC},modifier={by-nc-sa},version={4.0}]{doclicense}

% Use meta data for creative commons in the produced pdf
\usepackage{xmpincl}
\includexmp{CC-BY-SA-4.0}

% Flashcard environment similar to the flashcards package
% Similar to https://tex.stackexchange.com/a/89347/47921
\newenvironment{flashcard}[2][]{%
    #1
    \vfill
    \centerline{\Large{#2}}
    \vfill
    \newpage
}
{\newpage}

% No intendation for multi line text on cards
\setlength{\parindent}{0cm}

% Use a single card for section
\newcommand{\sectioncard}[1]{
    \vspace*{\stretch{1}}
    \section{#1}
    \vspace*{\stretch{1}}
    \pagebreak
}

% Use a single card for sub section
\newcommand{\subsectioncard}[1]{
    \vspace*{\stretch{1}}
    \subsection{#1}
    \vspace*{\stretch{1}}
    \pagebreak
}

% Other packages required for content
\usepackage{booktabs}
\newcommand{\tabitem}{~~\llap{\textbullet}~~}
\newcommand{\tabitemindent}{~~~~}

\begin{document}

    \title{Microsoft Certified Azure Fundamentals Flashcards}
    \date{2020-05-18}
    \author{Jan-Philipp Kappmeier}

    \clearpage\maketitle
    \thispagestyle{empty}
    \pagebreak

    \sectioncard{Verständnis von Cloud-Konzepten}

    \subsectioncard{Beschreibe die Vorteile und Erwägungen bei Nutzung von Cloud-Services}

    \begin{flashcard}[Definition]{Hochverfügbarkeit}
        \vspace*{\stretch{1}}
        \begin{itemize}
            \item die Möglichkeiten einer Anwendung ohne signifikante Auszeit durchgehend zu laufen
            \item es wird erwartet, dass die Anwendung reagiert und Nutzer mit ihr interagieren können
        \end{itemize}
        \vspace*{\stretch{1}}
    \end{flashcard}

    \begin{flashcard}[Definition]{Kosteneffizienz}
        \vspace*{\stretch{1}}
        Preismodell der nutzungsbasierten Bezahlung:
        \begin{itemize}
            \item keine Vorausszahlung
            \item teure Infrastruktur muss nicht gekauft werden
            \item Resourcen müssen nur bezahlt werden, wenn sie benötigt werden
        \end{itemize}

        \vspace{5mm}
        Alternativ: dedizierte Hardware in der Cloud mieten
        \vspace*{\stretch{1}}
    \end{flashcard}

    \begin{flashcard}[Definition]{Skalierbarkeit}
        \vspace*{\stretch{1}}
        Anzahl der Resourcen und Dienste nach Nachfrage verändern

        \vspace{5mm}
        Vertikale Skalierung (Hochskalieren)
        \begin{itemize}
            \item Leistung von Servern oder Diensten erhöhen\newline
            z.\,B. mehr CPUs, mehr Speicher, schnellere Anbindung
        \end{itemize}

        Horizontale Skalierung
        \begin{itemize}
            \item zusätzliche Server werden hinzugefügt
            \item sämtliche Maschinen funktionieren als Einheit
        \end{itemize}

        \vspace{5mm}
        Skalierung kann automatisch anhand gewisser Kriterien durchgeführt werden
        \vspace*{\stretch{1}}
    \end{flashcard}

    \begin{flashcard}[Definition]{Elastizität der Cloud}
        \vspace*{\stretch{1}}
        \begin{itemize}
            \item automatische Hinzunahme oder Wegnahme von Resourcen bei ändernder Nachfrage
            \item Beispiel: kurzzeitige Lastspitzen, aber auch jahreszeitliche Änderungen oder täglich, wöchentlich
        \end{itemize}

        \vspace*{\stretch{1}}
    \end{flashcard}

    \begin{flashcard}[Definition]{Agilität}
        \vspace*{\stretch{1}}

        \vspace{5mm}
        Schnelle Anpassung an wechselnde Anforderungen eines Unternehmens
        \begin{itemize}
            \item das kann zeitlich schwankender Zugriff sein (Elastizität)
            \item aber auch schnelle Anpassungen an Änderung der Unternehmensstrategie oder Ähnliches
        \end{itemize}

        \vspace*{\stretch{1}}
    \end{flashcard}

    \begin{flashcard}[Definition]{Fehlertoleranz}
        \vspace*{\stretch{1}}

        \begin{itemize}
            \item Datensicherung, Notfallwiederherstellung, Replikation
            \item redundante Architektur
        \end{itemize}
        Diese Maßhnahmen sollen sicherstellen, dass Kunden nicht von auftretenden Fehlern betroffen sind
        \vspace*{\stretch{1}}
    \end{flashcard}

    \begin{flashcard}[Definition]{Notfallwiederherstellung}
        \vspace*{\stretch{1}}
        \begin{itemize}
            \item die Fähigkeit des Systems der Erholung nach einem Zwischenfall\newline
            Zwischenfall = weiträumiger Ausfall (z.\,B. ganze Region)
            \item Datensicherung, Archivierung, Wiederherstellung
        \end{itemize}

        \vspace*{\stretch{1}}
    \end{flashcard}


    \begin{flashcard}[Definition]{Skaleneffekte (Economy of scale)}
        \vspace*{\stretch{1}}
        Arbeit kann pro Einheit günstiger durchgeführt werden, wenn sie in großem Umfang durchgeführt wird.

        \vspace{5mm}
        Bezogen auf cloud:

        Cloudanbieter können Hardware günstiger erwerben, effizienter warten und anderweitig sparen als einzelne Kunden das könnten.
        \vspace*{\stretch{1}}
    \end{flashcard}

    \begin{flashcard}[Definition]{CapEx}
        \vspace*{\stretch{1}}

        Investitionsausgaben (Capital Expenditure)
        \begin{itemize}
            \item Anschaffungskosten physischer Infrastruktur
            \item abschreibbar über die Zeit
            \item Anschaffungskosten, deren Wert über die Zeit geringer wird
        \end{itemize}

        Typisches Kostemodell für lokale Rechenzentren

        \vspace*{\stretch{1}}
    \end{flashcard}

    \begin{flashcard}[Beispiele]{CapEx}
        \vspace*{\stretch{1}}
        \begin{itemize}
            \item Serverkosten: Hardware + Support
            \item Speicher: Storageserver, \ldots, + Support
            \item Netwerkinfrastruktur: inklusive Kabel, WAN, \ldots
            \item Backup + Archivierung
            \item Infrastruktur für das Rechenzentrum
        \end{itemize}
        \vspace*{\stretch{1}}
    \end{flashcard}

    \begin{flashcard}[Definition]{OPex}
        \vspace*{\stretch{1}}
        Betriebskosten (Operational Expenditure)
        \begin{itemize}
            \item laufende Kosten für Dienste und Produkte
            \item keine Anschaffungskosten
            \item Zahlung nur für Dienste, die in Anspruch genommen werden
            \item abschreibbar im laufenden Geschäftsjahr
        \end{itemize}

        Typisches Kostenmodell für Cloud-Computing
        \vspace*{\stretch{1}}
    \end{flashcard}

    \begin{flashcard}[Beispiel]{OPex}
        \vspace*{\stretch{1}}
        Betriebskosten (Operational Expenditure)
        \begin{itemize}
            \item laufende Kosten für Dienste und Produkte
            \item keine Anschaffungskosten
            \item Zahlung nur für Dienste, die in Anspruch genommen werden
            \item abschreibbar im laufenden Geschäftsjahr
        \end{itemize}

        Typisches Kostenmodell für Cloud-Computing
        \vspace*{\stretch{1}}
    \end{flashcard}

    \begin{flashcard}[Vergleiche]{CapEx and OpEx}
        \vspace*{\stretch{1}}
        Vorteile CapEx:
        \begin{itemize}
            \item Ausgaben fix, d.\,h. planbar
            \item praktisch, falls mit begrenzten Budgets geplant werden muss
        \end{itemize}

        Vorteile OpEx:
        \begin{itemize}
            \item schnelle Reaktion auf geänderte Nachfrage und Wachstum
            \item keine hohen Anschaffungskosten bei Projektstart
        \end{itemize}
        $\Rightarrow$ bei unbekannter oder schwankender Nachfrage

        \vspace*{\stretch{1}}
    \end{flashcard}


    \begin{flashcard}[Definition]{Verbrauchsbasierendes Modell}
        \vspace*{\stretch{1}}
            \begin{itemize}
                \item es werden nur benutzte Resourcen bezahlt
                \item bei hoher Nachfrage in einem Monat kann gemäß der Nachfrage skaliert werden (und auch bezahlt)
                \item bei sinkender Auslastung können die Kosten wieder reduziert werden
            \end{itemize}
            $\Rightarrow$ im verbrauchsbasierenden Modell gibt es hauptsächlich OpEx-Kosten

            \vspace{5mm}
            Im dediziertern Modell wird eine bestimmte Hardware (entsprechend der erwarteten Auslastung) angeschafft:
            \begin{itemize}
                \item häufig wird zu viel bezahlt
                \item \emph{trotzdem} können Spitzen nicht bedient werden
            \end{itemize}
        \vspace*{\stretch{1}}
    \end{flashcard}

    \sectioncard{Beschreibe die Unterschiede zwischen Infrastructure-as-a-Service (IaaS), Platform-as-a-Service (Paas) und Software-as-a-Service (SaaS)}

    \begin{flashcard}[Definition]{Kategorien von Cloud-Services}
        \vspace*{\stretch{1}}
        Cloud-Services lassen sich in die folgenden Kategorien einteilen:
        \begin{itemize}
            \item Infrastructure-as-a-Service (IaaS)
            \item Platform-as-a-Service (PaaS)
            \item Software-as-a-Service (SaaS)
        \end{itemize}
        Die Kategorien unterscheiden sich in der Aufteilung der Anforderungen für den Nutzer und den Cloudanbieter.

        Von IaaS zu SaaS sinkt die Flexibilität für Nutzer und der Umfang der bereitgestellten Services steigt.

        \vspace*{\stretch{1}}
    \end{flashcard}

    \begin{flashcard}[Definition]{Modell der gemeinsamen Verantwortung}
        \vspace*{\stretch{1}}
        Cloudanbieter und Nutzer teilen sich die Verantwortung für die Funktionalität eines Services:
        \begin{itemize}
            \item Cloudanbieter stellt sicher, dass Infrastruktur funktioniert
            \item Nutzer ist verantwortlich, dass sein Dienst korrekt funktioniert
        \end{itemize}
        Trifft für IaaS zu. (In geringerem Umfang auch für PaaS).
        \vspace*{\stretch{1}}
    \end{flashcard}

    \begin{flashcard}[Definition]{IaaS}
        \vspace*{\stretch{1}}
        IaaS = Infrastructure-as-a-Service

        \begin{itemize}
            \item Infrastruktur wird vom Cloudanbieter verwaltet:
            \begin{itemize}
                \item Virtuelle Maschinen
                \item Netzwerk
                \item Betriebssysteme (teilweise)
                \item Speicher
            \end{itemize}
            \item Kunden mieten die Infrastructure, anstatt sie zu kaufen.
            \item Kunden sind komplett selbst verantwortlich für ihre eigene Umgebung.
        \end{itemize}
        \vspace*{\stretch{1}}
    \end{flashcard}

    \begin{flashcard}[Definition]{PaaS}
        \vspace*{\stretch{1}}
        PaaS = Platform-as-a-Service

        \begin{itemize}
            \item Vom Cloudanbieter wird eine komplette Umgebung zum Erstellen, Testen und Bereitstellen von Softwareanwendungen angeboten
            \item Kunden müssen sich \emph{nicht} um die Infrastructure kümmern\newline
            (Wartung, Betriebssysteme, Systemupdates, Webserver, \ldots)
            \item Vollständige Entwicklung und Deployment in der Cloud
            \item Sämtliche Vorteile der Cloud (Skalierbarkeit, Hochverfügbarkeit, \ldots) sind\newline automatisch vorhanden
        \end{itemize}
        \vspace*{\stretch{1}}
    \end{flashcard}

    \begin{flashcard}[Definition]{SaaS}
        \vspace*{\stretch{1}}
        SaaS = Software-as-a-Service

        Vollständige vom Cloudanbieter bereitgestellte Software
        \begin{itemize}
            \item Infrastruktur wird vom Cloudanbier verwaltet, gewartet, und bereitgestellt
            \item Software wird für Endkunden bereitgestellt
            \item typischerweise eine Softwareversion für alle Kunden
            \item Abrechnung über Abonnement
        \end{itemize}
        \vspace*{\stretch{1}}
    \end{flashcard}

    \begin{flashcard}[\ ]{Unterschiede der Cloud-Service-Kategorien (verantwortlichkeit)}
        \vspace*{\stretch{1}}
        \begin{tabular}{l|p{34mm}p{34mm}p{34mm}}
                 & \textbf{IaaS}                                             & \textbf{PaaS}                                                                     & \textbf{SaaS} \\
          \hline
          Kosten & Nutzung                                                   & Nutzung                                                                           & Abonnement                                                  \\
          Nutzer & Installation, Verwaltung, Konfiguration eigener Software  & Entwicklung und Bereitstellung eigener Anwendungen bei Nutzung der Cloud-Services & Verwendung der bereitgestellten Software                    \\
          Cloud  & Verantwortlich für Infrastruktur (VM, Netzwerk, Speicher) & Verantwortung für alles \emph{außer} der vom Nutzer entwickelten Software         & Komplette Verantwortung für die Bereitstellung der Software \\
        \end{tabular}
        \vspace*{\stretch{1}}
    \end{flashcard}  

    \subsectioncard{Beschreibe die  Unterschiede zwischen öffentlicher, privater, und Hybrid Cloud}

    \begin{flashcard}[Describe]{Öffentliche Cloud}
        \vspace*{\stretch{1}}
        \begin{itemize}
            \item typische Methode der Cloudnutzung
            \item keine loakle Hardware
            \item \emph{alle} Workloads werden in der Cloud ausgeführt
            \item ggf. können Resourcen für andere Cloudbenutzer freigegeben werden
        \end{itemize}

        \vspace{5mm}
        Azure ist eine öffentliche Cloud
        \vspace*{\stretch{1}}
    \end{flashcard}

    \begin{flashcard}[Describe]{Private Cloud}
        \vspace*{\stretch{1}}
        Aufbau einer Cloudumgebung im lokalen Rechenzentrum

        \begin{itemize}
            \item aus Nutzersicht kein Unterschied zu Cloudbenutzung
            \item Unternehmen ist selbst für Anschaffung und Wartung von Hardware verantwortlich
        \end{itemize}

        \vspace{5mm}
        Von Azure unterstützt durch Azure Stack

        \vspace*{\stretch{1}}
    \end{flashcard}

    \begin{flashcard}[Describe]{Hybrid Cloud}
        \vspace*{\stretch{1}}
        Verbindet öffentliche und private cloud \newline
        $\Rightarrow$ Anwendungen können in der passenden Umgebung ausgeführt werden

        \vspace{5mm}
        Use cases:
        \begin{itemize}
            \item Übergang von lokalen Datencentern zu Cloud
            \item Aufteilung von Daten und Verarbeitung z.\,B. aus rechtlichen Gründen
            \item Abfang von Leistungsspitzen
        \end{itemize}

        \vspace*{\stretch{1}}
    \end{flashcard}

    \begin{flashcard}[Gegenüberstellung]{Drei Cloudmodelle}
        \vspace*{\stretch{1}}
            \begin{tabular}{l|ll}
                Modell     & Vorteile                                     & Nachteile                                  \\
                \hline
                Öffentlich & \tabitem hohe Skalierbarkeit (Agilität)      & \tabitem Sicherheitsanforderungen          \\
                           & \tabitem keine CapEx-Kosten                  & \tabitem rechtliche Anforderungen          \\
                           & \tabitem ausgelagerte Wartung                & \tabitem Legacy-Kompatibilität             \\
                           & \tabitem geringe technische Kenntnisse       & \tabitem kein Eigentum an Diensten +       \\
                           & \tabitemindent der Mitarbeiter erforderlich  & \tabitemindent Hardware (eingeschränkt)    \\
                [.5\normalbaselineskip]
                Private    & \tabitem Kontrolle der Konfiguration         & \tabitem hohe CapEx-Kosten (Anschaffung)   \\
                           & \tabitem Kontrolle über Sicherheit           & \tabitem geringe Skalierbarkeit (Agilität) \\
                           & \tabitem rechtlicher Anforderungen           & \tabitem hohe Fachkenntnis erforderlich    \\
                [.5\normalbaselineskip]
                Hybrid     & \tabitem Eigene Hardware verwendbar          & \tabitem möglicherweise CapEx-Kosten       \\
                           & \tabitem Flexibilität (Lokal + Cloud)        & \tabitem komplizierte Einrichtung          \\
                           & \tabitem Skaleneffekte + lokales sparen      &                                            \\
                           & \tabitem Kontrolle über Sicherheit           &                                            \\
                           & \tabitem rechtliche Anforderungen            &                                            \\
            \end{tabular}
        \vspace*{\stretch{1}}
    \end{flashcard}

    \sectioncard{Understand Core Azure Services}

    \subsectioncard{Understand the Core Azure Architectural Components}

    \begin{flashcard}[Describe]{Regions}
        \vspace*{\stretch{1}}

        \vspace*{\stretch{1}}
    \end{flashcard}

    \begin{flashcard}[Describe]{Availability Zones}
        \vspace*{\stretch{1}}

        \vspace*{\stretch{1}}
    \end{flashcard}

    \begin{flashcard}[Describe]{Resource Groups}
        \vspace*{\stretch{1}}

        \vspace*{\stretch{1}}
    \end{flashcard}

    \begin{flashcard}[Describe]{Azure Resource Manager}
        \vspace*{\stretch{1}}

        \vspace*{\stretch{1}}
    \end{flashcard}

    \begin{flashcard}[Describe]{Benefits and Usage of Core Azure Architectural Components}
        \vspace*{\stretch{1}}

        \vspace*{\stretch{1}}
    \end{flashcard}

    \subsectioncard{Beischreibe einige der in Azure verfügbaren Kernkomponenten}

    \begin{flashcard}[Definition]{Virtuelle Maschine}
        \vspace*{\stretch{1}}
        \begin{itemize}
            \item virtueller Computer, der in der Azure cloud gehostet wird und läuft
            \item verfügbar für Linux (verschiedene Distributionen) und Windows
        \end{itemize}
        \vspace*{\stretch{1}}
    \end{flashcard}

    \begin{flashcard}[Definition]{Skalierungsgruppen für Virtuelle Maschinen}
        \vspace*{\stretch{1}}
        \begin{itemize}
            \item \emph{automatische} Skalierung von in Azure gehosteten virtuellen Maschinen
            \item mehrere virtuelle Maschinen gehören zu einer Gruppe
            \item die Elemente der Skalierungsgruppe sind \emph{identisch}
            \item eine große Zahl an Instanzen kann innerhalb von Minuten verwaltet werden
        \end{itemize}
        $\Rightarrow$ gut geeignet für Big Data-Analysen und Containerworkloads

        \vspace*{\stretch{1}}
    \end{flashcard}

    \begin{flashcard}[Definition]{App Service Functions}
        \vspace*{\stretch{1}}
        \begin{itemize}
            \item PaaS-Service in Azure\newline
            $\Rightarrow$ keine eigene Infrastruktur notwendig
            \item Enterprise-Anwendungen wie Webseiten, (REST)-APIs und für Mobilgeräte
            \item unterstützt Entwicklung und Deployment, CI über Git-Repository
            \item automatische Skalierung
            \item unterstützte Programmiersprachen:
            \begin{itemize}
                \item Python
                \item Java
                \item Ruby
                \item Node.js
                \item .NET/.NET Core
                \item PHP
            \end{itemize}
        \end{itemize}
        \vspace*{\stretch{1}}
    \end{flashcard}

    \begin{flashcard}[\ ]{Azure App Service-Kosten}
        \vspace*{\stretch{1}}
        \begin{itemize}
            \item festgelegt durch App Service-Plan
            \item bestimmt die größe/Anzahl der verfügbaren Hardware
            \item SKU F1 ermöglicht kostenlose Verwendung von kleinen Resourcen im Free-Tarif
        \end{itemize}

        \vspace*{\stretch{1}}
    \end{flashcard}

    \begin{flashcard}[Definition]{Azure Container Instances (ACI)}
        \vspace*{\stretch{1}}
        \begin{itemize}
            \item Container-Anwendungen in Azure
            \item benötigt keine virtuelle Maschine oder Server
        \end{itemize}

        \vspace*{\stretch{1}}
    \end{flashcard}

    \begin{flashcard}[Definition]{Azure Kubernetes Service (AKS)}
        \vspace*{\stretch{1}}
        \begin{itemize}
            \item verwaltet ein Kubernetes-Cluster aus mehreren virtuellen Maschinen
            \item Dienste werden in Containern auf den Maschinen im Cluster ausgeführt
        \end{itemize}

        \vspace*{\stretch{1}}
    \end{flashcard}

    \begin{flashcard}[Definition]{Azure Batch}
        \vspace*{\stretch{1}}
        \begin{itemize}
            \item verwalteter Dienst für parallele Berechnungen
            \item Batch-Dienst unterstüzt viele Batch-Knoten (virtuelle Maschinen oder Skalierungsgruppe)
        \end{itemize}
        \vspace*{\stretch{1}}
    \end{flashcard}

    \begin{flashcard}[Definition]{Virtuelles Netzwerk}
        \vspace*{\stretch{1}}
        \begin{itemize}
            \item Dienst, der virtuelle Maschinen mit eingehenden VPN-Verbindungen (Virtual Private Network) verknüpft
            \item Verbindung von Azure-VPN mit einem lokalen VPN ermöglicht sichere Kommunikation zwischen Azure und lokalem Datacenter
        \end{itemize}

        \vspace*{\stretch{1}}
    \end{flashcard}

    \begin{flashcard}[Definition]{Load Balancer}
        \vspace*{\stretch{1}}
        \begin{itemize}
            \item verteilt ein- und ausgehende Verbindungen
            \item Ziel ist gleichmäßige Auslastung von Anwendungen und Endpunkten
        \end{itemize}

        \vspace*{\stretch{1}}
    \end{flashcard}

    \begin{flashcard}[Definition]{VPN Gateway}
        \vspace*{\stretch{1}}
        \begin{itemize}
            \item ermöglicht den Zugriff auf Azure-VPNs
            \item hochleistungsfähige Gateways
        \end{itemize}

        \vspace*{\stretch{1}}
    \end{flashcard}

    \begin{flashcard}[Definition]{Application Gateway}
        \vspace*{\stretch{1}}
        \begin{itemize}
            \item optimieren die Auslieferung aus Anwendungs-Serverfarmen
            \item erhöht gleichzeitig Anwendungssicherheit
        \end{itemize}

        \vspace*{\stretch{1}}
    \end{flashcard}

    \begin{flashcard}[Definition]{Content Delivery Network (CDN)}
        \vspace*{\stretch{1}}
        \begin{itemize}
            \item Auslieferung von Inhalten mit hoher Bandbreite
            \item schnelle Auslieferung weltweit
        \end{itemize}
        \vspace*{\stretch{1}}
    \end{flashcard}

    \begin{flashcard}[Definition]{Blob-Speicher}
        \vspace*{\stretch{1}}
        \begin{itemize}
            \item Binary Large OBject
            \item sehr große binäre Objekte, z.\,B. Videodateien oder Bilder
            \item Schlüssel-Wert-Speicher
            \item ansprechbar über HTTP-API (Bibliotheken für viele Sprachen verfügbar)
        \end{itemize}

        \vspace*{\stretch{1}}
    \end{flashcard}

    \begin{flashcard}[Definition]{Disk Storage}
        \vspace*{\stretch{1}}
        \begin{itemize}
            \item Festplatten für virtuellle Maschinen, Apps, und Dienste
            \item Managed Disk: verwaltete Disk, wird von Azure erstellt und freigegeben
            \item größe beliebig, bezahlt nach Gigabytes in Zweierpotenzen
        \end{itemize}
        \vspace*{\stretch{1}}
    \end{flashcard}

    \begin{flashcard}[Definition]{File Storage}
        \vspace*{\stretch{1}}
        \begin{itemize}
            \item Dateifreigaben, die übe SMB eingebunden werden können
            \item Zugriff wie auf Dateiserver
            \item ermöglicht Quota
        \end{itemize}
        \vspace*{\stretch{1}}
    \end{flashcard}

    \begin{flashcard}[Definition]{Archivierung}
        \vspace*{\stretch{1}}
        \begin{itemize}
            \item speichern von Daten, die selten benötigt werden
            \item Zugriff langsam
            \item Zugriff teuer
        \end{itemize}
        \vspace*{\stretch{1}}
    \end{flashcard}

    \begin{flashcard}[\ ]{Eigenschaften von Speicherdiensten}
        \vspace*{\stretch{1}}
        \begin{itemize}
            \item dauerhafte Speicherung durch Redundanz und Replikation\newline
            Replikation auf Wunsch
            \item sichere Daten durch Verschlüsselung\newline
            entweder durch Azure, oder durch Kunden
            \item Skalierbarkeit,
            \begin{itemize}
                \item Kapazität: zwar endliche, aber für die meisten Anwendungsfälle unbegrenzte
                \item Zugriffe: kann für zahllose parallele Zugriffe eingerichtet werden
            \end{itemize}
            \item sämtliche Speicherdienste sind verwaltet, d.\,h. keine Wartung und Fehlerbehandlung (des Dienstes selbst) notwendig
            \item Zugriff über HTTP(S) weltweit möglich
        \end{itemize}
        \vspace*{\stretch{1}}
    \end{flashcard}


    \begin{flashcard}[Definition]{Cosmos DB}
        \vspace*{\stretch{1}}
        \begin{itemize}
            \item NoSQL-Datenbank
            \item global verteilt
        \end{itemize}
        \vspace*{\stretch{1}}
    \end{flashcard}

    \begin{flashcard}[Definition]{Azure SQL-Datenbank}
        \vspace*{\stretch{1}}
        \begin{itemize}
            \item verwaltete relationale Datenbank (generisch SQL)
            \item automatische Skalierung
            \item robuste Sicherheit
            \item integrierte intelligente Funktionen?
        \end{itemize}
        \vspace*{\stretch{1}}
    \end{flashcard}

    \begin{flashcard}[Describe]{Azure Database for MySQL}
        \vspace*{\stretch{1}}
        \begin{itemize}
            \item verwaltete relationale Datenbank (MySQL)
            \item hohe Verfügbarkeit
            \item Skalierbarkeit
            \item Sicherheit
        \end{itemize}
        \vspace*{\stretch{1}}
    \end{flashcard}

    \begin{flashcard}[Definition]{Azure Database for PostgreSQL}
        \vspace*{\stretch{1}}
        \begin{itemize}
            \item verwaltete relationale Datenban (PostgreSQL)
            \item hohe Verfügbarkeit
            \item Skalierbarkeit
            \item Sicherheit
        \end{itemize}
        \vspace*{\stretch{1}}
    \end{flashcard}

    \begin{flashcard}[Describe]{Azure Database Migration Service}
        \vspace*{\stretch{1}}
        Unterstützung beim Migrieren von Datenbanken in die Cloud

        $\Rightarrow$ keine Änderungen am Anwendungscode notwendig
        \vspace*{\stretch{1}}
    \end{flashcard}

    \begin{flashcard}[Definition]{Azure Marketplace}
        \vspace*{\stretch{1}}
        \begin{itemize}
            \item Sammlung von Applikationen und Services \newline
            von verschiedenen Anbietern, zertifiziert für Azure-Kompatibilität
            \begin{itemize}
                \item KI + Machine Learning
                \item Disk-Images
                \item Webanwendungen
                \item \ldots
            \end{itemize}
            \item erlaubt das Suchen, Ausprobieren, Kaufen und Bereitstellen mithilfe einer Benutzeroberfläche
            \item Unterstützt die Verbindung von Endnutzern mit
            \begin{itemize}
                \item Microsoft-Parnern
                \item unabhängige Softwareentwickler
                \item Start-Ups, die Azure-Lösungen bereitstellen
            \end{itemize}
        \end{itemize}
        \vspace*{\stretch{1}}
    \end{flashcard}

    \begin{flashcard}[\ ]{Marketplace Anwendungsfälle}
        \vspace*{\stretch{1}}
            \begin{itemize}
                \item Verwenden von WordPress für eine Webseite
                \item Suchen von Diensten und Lösungen für ein Problem
            \end{itemize}
        \vspace*{\stretch{1}}
    \end{flashcard}

    \subsectioncard{Beschreibe einige der in Azure verfügbaren Lösungen}

    \begin{flashcard}[Definition]{Internet der Dinge (IoT)}
        \vspace*{\stretch{1}}
        Nicht primär zur Informationstechnik gehörende Geräte, die Daten sammeln und über das Internet übertragen.
        
        \vspace{5mm}
        Die Gesamtheit der vernetzten Geräte ist das Internet der Dinge'' (IoT)
        
        \vspace{5mm}
        Beispiele:
        \begin{itemize}
            \item Waschmaschine
            \item Smart-Watch
            \item Kühlschrank
            \item Smart-Home-Thermostate
        \end{itemize}

        \vspace*{\stretch{1}}
    \end{flashcard}

    \begin{flashcard}[\ ]{Produkte für IoT in Azure}
        \vspace*{\stretch{1}}
        \begin{itemize}
            \item IoT Central
            \item IoT Hub
            \item IoT Edge
        \end{itemize}

        \vspace*{\stretch{1}}
    \end{flashcard}

    \begin{flashcard}[Definition]{IoT Hub}
        \vspace*{\stretch{1}}
        \begin{itemize}
            \item Messaging-Hub für sichere Kommunikation
            \item ermöglicht Überwachung von millionen IoT-Geräten
        \end{itemize}
        \vspace*{\stretch{1}}
    \end{flashcard}

    \begin{flashcard}[Definition]{IoT Central}
        \vspace*{\stretch{1}}
        \begin{itemize}
            \item skalierbare, verwaltete SaaS-Lösung für IoT
            \item einfache Vernetzung von IoT-Resourcen
            \item Verwaltung, Überwachung
        \end{itemize}
        \vspace*{\stretch{1}}
    \end{flashcard}

    \begin{flashcard}[Definition]{IoT Edge}
        \vspace*{\stretch{1}}
        \begin{itemize}
            \item Datenanalyse-Modelle auf IoT-Geräte schicken
            \item ermöglicht schnelle Reaktion auf Zustandsänderungen
            \item keine cloudbasierte Berechnung notwendig
        \end{itemize}
        \vspace*{\stretch{1}}
    \end{flashcard}

    \begin{flashcard}[Definition]{Big Data Analytics}
        \vspace*{\stretch{1}}
        \begin{itemize}
            \item Lösungen und Dienste zur Analyse und Verarbeitung von großen Datenmengen
            \item ``Big Data'' große Datenvolumen, einzelne Datensätze können groß oder klein sein
            \item Beispiele:
                \begin{itemize}
                    \item Genomanalyse
                    \item Kommunikationssysteme
                    \item \ldots
                \end{itemize}
        \end{itemize}
        \vspace*{\stretch{1}}
    \end{flashcard}

    \begin{flashcard}[Describe]{Big Data Products + Analytics Products}
        \vspace*{\stretch{1}}
        \begin{itemize}
            \item Synapse Analytics
            \item HDInsight
            \item Databricks
        \end{itemize}

        \vspace*{\stretch{1}}
    \end{flashcard}

    \begin{flashcard}[Describe]{SQL Data Warehouse}
        \vspace*{\stretch{1}}
        Umbenannt: nun Azure Synapse.

        \begin{itemize}
            \item Big-Data-Analysedienst
            \item kombiniert Data Warehousing und Big-Data-Analysen
            \item schnelle Ausführung komplexer Abfragen in großen Datenmengen (Petabyte)
            \item nutzt das cloudbasierte Enterprise Data Warehouse (EDW)
            \item Massively Prallel Processing (MPP) mit Unterstützung von freien On-Demand-Resourcen
        \end{itemize}
        \vspace*{\stretch{1}}
    \end{flashcard}

    \begin{flashcard}[Describe]{HDInsight}
        \vspace*{\stretch{1}}
        \begin{itemize}
            \item Verarbeitung großer Datenmengen
            \item verwaltete Hadoop-Cluster
        \end{itemize}
        \vspace*{\stretch{1}}
    \end{flashcard}

    \begin{flashcard}[Describe]{Azure Databricks}
        \vspace*{\stretch{1}}
        \begin{itemize}
            \item Analyseplattform basierend auf Apache Spark
            \item optimiert für Microsoft Cloudplattform
            \item interkative Zusammenarbeit/Kollaboration von Datenspezialisten, Data Engineers und Business Analysts
            \item einfache Einrichtung mit One-Click-Workflow
            \item Erfassung von Rohdaten aus Batch-Prozessierung oder Live-Streaming
        \end{itemize}
        \vspace*{\stretch{1}}
    \end{flashcard}

    \begin{flashcard}[Describe]{Artificial Intelligence (AI)}
        \vspace*{\stretch{1}}
        \begin{itemize}
            \item Sammlung von Diensten, die auf maschinellem Lernen beruhen
            \item Computer lernen aus Daten, zukünftiges Verhalten, Ergebnisse oder Trends vorherzusagen
            \item keine Programmierung wird benötigt, da Algorithmen implizit aus den Daten gelernt werden
        \end{itemize}

        \vspace*{\stretch{1}}
    \end{flashcard}

    \begin{flashcard}[Describe]{Products Available for AI}
        \vspace*{\stretch{1}}
        \begin{itemize}
            \item Azure Machine Learning-Dienst
            \item Azure Machine Learning Studio
        \end{itemize}

        \vspace*{\stretch{1}}
    \end{flashcard}

    \begin{flashcard}[Describe]{Azure Machine Learning-Dienst}
        \vspace*{\stretch{1}}
        \begin{itemize}
            \item Entwicklung, Test, Deployment und Analyse von Machine Learning-Modellen
            \item unterstützt Generierung und Optimiierung von Modellen
            \item Beginn des Trainings auf dem lokalen Rechner
            \item Nutzung von Cloud-Kapazitäten durch horizontale Hochskalierung
        \end{itemize}
        \vspace*{\stretch{1}}
    \end{flashcard}

    \begin{flashcard}[Describe]{Azure Machine Learning Studio}
        \vspace*{\stretch{1}}
        \begin{itemize}
            \item Visuelle Entwiclung von Lösungen für machinelles Lernen
            \item unterstützt Modellentwicklung durch Drag \& Drop
            \item geeignet für Kollaboration mehrerer Entwickler
            \item Auswahl aus verschiedenen vordefinierten Algorithmen und Modulen für maschinelles Lernen
        \end{itemize}
        \vspace*{\stretch{1}}
    \end{flashcard}

    \begin{flashcard}[Describe]{Serverless Computing}
        \vspace*{\stretch{1}}
        \begin{itemize}
            \item direkte Ausführung von Code\newline
            $\Rightarrow$ vollständige Abstraktion der Hostumgebung
            \item Konfiguration und Wartung der Infrastruktur ist weder erforderlich noch zulässig
            \item[$\Rightarrow$] PaaS
            \item hochverfügbar
            \item jede Arbeit muss schnell erledigt werden (Sekunden, maximal Minuten)
            \item gut geeignet für ereignisgesteuerte Berechnungen
            \item Microbilling nur nach ausgeführter Berechnungszeit, evtl. auch nur wenige Sekunden
        \end{itemize}

        \vspace*{\stretch{1}}
    \end{flashcard}

    \begin{flashcard}[\ ]{Azure Serverless-Dienste}
        \vspace*{\stretch{1}}
        \begin{itemize}
            \item Azure Functions\newline
            Ausführung von Code einer beliebigen modernen Programmiersprache
            \item Logic Apps\newline
            graphische Anwendungen ohne Programmcode
            \item Event Grid\newline
        \end{itemize}

        \vspace*{\stretch{1}}
    \end{flashcard}

    \begin{flashcard}[Definition]{Azure Functions}
        \vspace*{\stretch{1}}
        \begin{itemize}
            \item Dienst, der vorgegebene Funktion ausführt
            \item ereignisgesteuert, serverlos
            \item kurze Laufzeit (maximal wenige Minuten)
            \item Varianten: zustandslos (Standard), zustandsbehaftet (\emph{Durable}), die einen Kontext erhalten
        \end{itemize}
        $\Rightarrow$ automatische Skalierung. Gut geeignet für schwankende Anforderungen, z.\,B. aus dem Internet der Dinge
        
        \vspace{5mm}
        \textbf{Vorteile}: keine Kosten während Leerlaufzeiten
        \vspace*{\stretch{1}}
    \end{flashcard}

    \begin{flashcard}[Definition]{Logic Apps}
        \vspace*{\stretch{1}}
        Ähnlich wie Functions, aber nicht programmiert!
        \begin{itemize}
            \item Workflows aus vordefinierten Logikblöcken\newline
            Zusammenstellung im visuellen Designer im Azure-Portal
            \item Orchestrierung von Aufgaben und Business-Prozessen
            \item Beginn der Ausführung mit einem Trigger
            \item Integration von Daten, Anwendungen, und Systemen\newline
            zahllose vordefinierte Module vorhanden um mit anderen Diensten zu kommunizieren
        \end{itemize}

        \vspace*{\stretch{1}}
    \end{flashcard}
    
    \begin{flashcard}[\ ]{Vergleich Azure Functions und Azure Logic Apps}
        \vspace*{\stretch{1}}
        \begin{tabular}{l|p{34mm}p{34mm}}
                 & \textbf{Functions}                                             & \textbf{Logic Apps} \\
          \hline
          Zustand & Standard: Zustandslos, aber auch Durable                  & zustandsbehaftet                                                                         \\
          Entwicklung & imperativ/mit code                                        & deklarativ/grafisch                                                               \\
          Kontext  & Lokal und in der Cloud                                    & Nur in der Cloud                                                                  \\
        \end{tabular}
        \vspace*{\stretch{1}}
    \end{flashcard}

    \begin{flashcard}[Describe]{Event Grid}
        \vspace*{\stretch{1}}
        \begin{itemize}
            \item Aufbau von Anwendungen mit Event-basierer Architektur
            \item verwalteter Dienst
            \item Event-Routing über Publish/Subscribe for Eventverarbeitung
        \end{itemize}

        \vspace*{\stretch{1}}
    \end{flashcard}

    \begin{flashcard}[Describe]{Available DevOps solutions}
        \vspace*{\stretch{1}}

        \vspace*{\stretch{1}}
    \end{flashcard}

    \begin{flashcard}[Describe]{Azure DevOps}
        \vspace*{\stretch{1}}

        \vspace*{\stretch{1}}
    \end{flashcard}

    \begin{flashcard}[Describe]{Azure DevTest Labs}
        \vspace*{\stretch{1}}

        \vspace*{\stretch{1}}
    \end{flashcard}

    \begin{flashcard}[Benefits and Outcomes]{Usage of Azure Solutions}
        \vspace*{\stretch{1}}

        \vspace*{\stretch{1}}
    \end{flashcard}

    \subsectioncard{Understand Azure Management Tools}

    \begin{flashcard}[Understand]{Azure Portal}
        \vspace*{\stretch{1}}

        \vspace*{\stretch{1}}
    \end{flashcard}

    \begin{flashcard}[Describe]{Azure PowerShell}
        \vspace*{\stretch{1}}

        \vspace*{\stretch{1}}
    \end{flashcard}

    \begin{flashcard}[Describe]{Azure CLI}
        \vspace*{\stretch{1}}

        \vspace*{\stretch{1}}
    \end{flashcard}

    \begin{flashcard}[Describe]{Cloud Shell}
        \vspace*{\stretch{1}}

        \vspace*{\stretch{1}}
    \end{flashcard}

    \begin{flashcard}[Describe]{Azure Advisor}
        \vspace*{\stretch{1}}

        \vspace*{\stretch{1}}
    \end{flashcard}

    \sectioncard{Understand Security, Privacy, Compliance, and Truts}

    \subsectioncard{Understand Securing Network Connectivity in Azure}

    \begin{flashcard}[Describe]{Network Security Groups (NSG)}
        \vspace*{\stretch{1}}

        \vspace*{\stretch{1}}
    \end{flashcard}

    \begin{flashcard}[Describe]{Application Security Groups (ASG)}
        \vspace*{\stretch{1}}

        \vspace*{\stretch{1}}
    \end{flashcard}

    \begin{flashcard}[Describe]{User Defined Rules (UDR)}
        \vspace*{\stretch{1}}

        \vspace*{\stretch{1}}
    \end{flashcard}

    \begin{flashcard}[Describe]{Azure Firewall}
        \vspace*{\stretch{1}}

        \vspace*{\stretch{1}}
    \end{flashcard}

    \begin{flashcard}[Describe]{DDos Protection}
        \vspace*{\stretch{1}}

        \vspace*{\stretch{1}}
    \end{flashcard}

    \begin{flashcard}[Choose]{Appropriate Azure security solution}
        \vspace*{\stretch{1}}

        \vspace*{\stretch{1}}
    \end{flashcard}

    \subsectioncard{Describe Core Azure Identity Services}

    \begin{flashcard}[Definition]{Authentifizierung}
        \vspace*{\stretch{1}}
        \begin{itemize}
            \item Überprüfung einer \emph{Person} oder eines \emph{Dienstes} vor Zugriff auf Ressource
            \item Anforderung von Anmeldeinformationen
            \item Nachweis, dass es tatsächlich die Person/der Dienst ist
            \item Grundlage eines Sicherheitsprinzipals-Security Principle
            \item verwaltet über Azure Active Directory
        \end{itemize}
        \vspace*{\stretch{1}}
    \end{flashcard}

    \begin{flashcard}[Definition]{Autorisierung}
        \vspace*{\stretch{1}}
        \begin{itemize}
            \item Überprüfung der Zugriffsebene einer authentifizierten Entität (Person oder Dienst)
            \item auf welche Daten darf zugegriffen werden
            \item welche Aktionen dürfen ausgeführt werden
            \item verwaltet über Azure Active Directory
        \end{itemize}
        \vspace*{\stretch{1}}
    \end{flashcard}

    \begin{flashcard}[Verstehen]{Unterschied zwischen Authentifizierung und Autorisierung}
        \vspace*{\stretch{1}}
        \begin{center}
            Authentifizierung bestimmt \emph{wer ein Nutzer ist}
            \vspace{5mm}

            vs.
            \vspace{5mm}

            Autorisierung bestimmt \emph{was ein Nutzer darf}
        \end{center}
        \vspace*{\stretch{1}}
    \end{flashcard}

    \begin{flashcard}[Definition]{Azure Active Directory}
        \vspace*{\stretch{1}}
        \begin{itemize}
            \item Cloudbasierter Identitätsdienst
            \item Alle Anwendungen können dieselben Anmeldeinformationen verwenden\newline
            unabhängig davon, ob \emph{lokal}, \emph{in der Cloud}, oder \emph{mobil}
            \item \emph{Active Directory-Mandant}: ein Active Directory Instanz in der Organisation
        \end{itemize}
        \vspace*{\stretch{1}}
    \end{flashcard}

    \begin{flashcard}[\ ]{Features von Azure Active Directory}
        \vspace*{\stretch{1}}
        \begin{itemize}
            \item Synchronisierung mit lokalen Active Directory-Instanzen oder LDAP
            \item Zugriffssteuerung zentral über Regeln/Richtlinien\newline
            siehe auch RBAC
        \end{itemize}
        \vspace*{\stretch{1}}
    \end{flashcard}

    \begin{flashcard}[\ ]{Dienste von Azure Active Directory}
        \vspace*{\stretch{1}}
        \begin{itemize}
            \item Authentifizierung
            \item Single Sign-On (SSO)
            \item Verwaltung von Anwendungen
            \item Business-to-Business Identitätsdienste (B2B)
            \item Business-to-Customer Identitätsdienste (B2C)
            \item Geräteverwaltung
        \end{itemize}
        \vspace*{\stretch{1}}
    \end{flashcard}

    \begin{flashcard}[\ ]{Azure Active Directory-Authentifizierung}
        \vspace*{\stretch{1}}
        \begin{itemize}
            \item Überprüfen der Identität
                \newline(siehe Authentifizierung)
            \item Kennwortrücksetzung
            \item Multi-Factor Authentication
            \item Smart Lockout
            \item Gesperrte Kennwörter verwalten
        \end{itemize}
        \vspace*{\stretch{1}}
    \end{flashcard}

    \begin{flashcard}[Definition]{Azure Multi-Factor Authentication}
        \vspace*{\stretch{1}}
        \begin{itemize}
            \item Mindestens zwei Methoden zur Authentifizierung
            \item Drei Kategorien:
                \begin{itemize}
                    \item etwas, das der Nutzer \emph{weiß}: Kennwort oder Antwort auf Sicherheitsfrage
                    \item etwas, das der Nutzer \emph{hat}: App (Microsoft Authenticator, oder Google Authenticator, etc) oder Tokengerät
                    \item persönliche \emph{Merkmale} des Nutzers: Fingerabdruck, Gesichtsscan, \ldots
            \end{itemize}
            \item Erhöhung der Sicherheit, denn: falls eine Anmeldeinformation an einen Angreifer gerät, kann er sich nicht authentifizieren.
            \item Automatisch in Azure Active Directory integriert und kombinierbar mit MFA von Dittanbietern
            \item Globaler Administrator sollte \emph{auf jeden Fall} diese Methode verwenden
        \end{itemize}
        \vspace*{\stretch{1}}
    \end{flashcard}

    \begin{flashcard}[\ ]{Identitäten für Dienste}
        \vspace*{\stretch{1}}
            Problem bei der Identifizierung:
            \begin{itemize}
                \item Anmeldeinformationen sind in Konfigurationsdateien enthalten.
                \item Mangelnde Sicherheitsrichtlinien ermöglichen Zugriff für jeden mit Zugriff auf die entsprechendne Systeme und Repositories
            \end{itemize}
            $\Rightarrow$ Lösung sind \emph{Identitäten für Dienste}

            Zwei Möglichkeiten in Azure:
            \begin{enumerate}
                \item Dinstprinzipale
                \item verwaltete Identitäten
            \end{enumerate}
            \vspace*{\stretch{1}}
    \end{flashcard}

    \begin{flashcard}[Definition]{Dienstprinzipale}
        \vspace*{\stretch{1}}
            \begin{itemize}
                \item \emph{Identität}: eine Entität, die Authentifiziert werden kann \newline
                    (Sowohl Benutzer als auch Anwendungen, Server, …)
                \item \emph{Prinzipal}: eine Identität zusammen mit Rollen \newline
                    (z.\,B. Administrator-Zugriff)
                \item \emph{Dienstprinzipal}: eine Identität, die von einem Dienst oder einer App verwendet wird
                \item[$\Rightarrow $] Dienstprinzipale sind keine Benutzer
                \item haben Rollen wie jede andere Identität
            \end{itemize}
        \vspace*{\stretch{1}}
    \end{flashcard}

    \begin{flashcard}[Definition]{Verwaltete Identität}
        \vspace*{\stretch{1}}
            \begin{itemize}
                \item Konto in der ActivevDirectory-Instanz
                \item Wird automatisch von Azure authentifiziert und verwaltet
                \item Dieses Konto kann wie jedes Azure Active Directory-Konto verwendet werden
                    \begin{itemize}
                        \item z.\,B. Rollen
                    \end{itemize}
            \end{itemize}
            Vorteile gegenüber Dienstprinzipalen:
            \begin{itemize}
                \item Einrichtung von Dienstprinzipal aufwändig
                \item direkt nutzbar von unterstützten Diensten
            \end{itemize}
        \vspace*{\stretch{1}}
    \end{flashcard}

    \subsectioncard{Describe Security Tools and Features of Azure}

    \begin{flashcard}[Describe]{Azure Security Center}
        \vspace*{\stretch{1}}

        \vspace*{\stretch{1}}
    \end{flashcard}

    \begin{flashcard}[Understand]{Azure Security Center usage scenarios}
        \vspace*{\stretch{1}}

        \vspace*{\stretch{1}}
    \end{flashcard}

    \begin{flashcard}[Describe]{Key Vault}
        \vspace*{\stretch{1}}

        \vspace*{\stretch{1}}
    \end{flashcard}

    \begin{flashcard}[Describe]{Azure Information Protection (AIP)}
        \vspace*{\stretch{1}}

        \vspace*{\stretch{1}}
    \end{flashcard}

    \begin{flashcard}[Describe]{Azure Advanced Threat Protection (ATP)}
        \vspace*{\stretch{1}}

        \vspace*{\stretch{1}}
    \end{flashcard}

    \subsectioncard{Describe Azure Governance Methodologies}

    \begin{flashcard}[Describe]{Policies and Initiatives with Azure Policy}
        \vspace*{\stretch{1}}

        \vspace*{\stretch{1}}
    \end{flashcard}

    \begin{flashcard}[Definition]{Role-Based Access Control (RBAC)}
        \vspace*{\stretch{1}}
        \begin{enumerate}
            \item Rollen: bestimmte eingeschränkte Berechtiungen (Schreibgeschützt, Mitwirkender) die einer Identität für bestimmte Azure-Dienstinstanzen zu gewiesen werden
            \item Identitäten erben Rollen über Gruppenmitgliedschaft
            \item Rollen vererben sich von höheren Stufen (gesamtes Abonnement (Subscription) über Resourcengruppen zu einzelnen Resourcen) etc.
        \end{enumerate}
        \vspace*{\stretch{1}}
    \end{flashcard}

    \begin{flashcard}[]{Vorteile von RBAC}
        \vspace*{\stretch{1}}
        \begin{enumerate}
            \item Trennung von Sicherheitsprinzipalen?, Zugriffsrechten und Resourcen
            \item ermöglilcht einfache Verwaltung und differenzierte Steuerung.
            \item Sicherstellung, dass nur erforderliche \emph{Mindestberechtigung} gewährt werden
            \item[$\Rightarrow$] es sollten nie mehr Berechtigungen, als notwendig vergeben werden
        \end{enumerate}
        Fazit: genau bekannt wer auf welche Daten und Infrastruktur zugreifen kann
        \vspace*{\stretch{1}}
    \end{flashcard}

    \begin{flashcard}[Describe]{Locks}
        \vspace*{\stretch{1}}

        \vspace*{\stretch{1}}
    \end{flashcard}

    \begin{flashcard}[Describe]{Azure Advisor Security Assistance}
        \vspace*{\stretch{1}}

        \vspace*{\stretch{1}}
    \end{flashcard}

    \begin{flashcard}[Describe]{Azure Blueprints}
        \vspace*{\stretch{1}}

        \vspace*{\stretch{1}}
    \end{flashcard}

    \subsectioncard{Understand monitoring and Reporting Options in Azure}

    \begin{flashcard}[Describe]{Azure Monitor}
        \vspace*{\stretch{1}}

        \vspace*{\stretch{1}}
    \end{flashcard}

    \begin{flashcard}[Describe]{Azure Service Health}
        \vspace*{\stretch{1}}

        \vspace*{\stretch{1}}
    \end{flashcard}

    \begin{flashcard}[Understand]{Use cases and benefits of Az Monitor and Az Service Health}
        \vspace*{\stretch{1}}

        \vspace*{\stretch{1}}
    \end{flashcard}

    \subsectioncard{Understand Privacy, Complience, and Data Protection Standards in Azure}

    \begin{flashcard}[Understand]{Industry Complience Terms}
        \vspace*{\stretch{1}}

        \vspace*{\stretch{1}}
    \end{flashcard}

    \begin{flashcard}[Understand]{GDPR}
        \vspace*{\stretch{1}}

        \vspace*{\stretch{1}}
    \end{flashcard}

    \begin{flashcard}[Describe]{ISO}
        \vspace*{\stretch{1}}

        \vspace*{\stretch{1}}
    \end{flashcard}

    \begin{flashcard}[Describe]{NIST}
        \vspace*{\stretch{1}}

        \vspace*{\stretch{1}}
    \end{flashcard}

    \begin{flashcard}[Understand]{Microsoft Privacy Statement}
        \vspace*{\stretch{1}}

        \vspace*{\stretch{1}}
    \end{flashcard}

    \begin{flashcard}[Describe]{Trust Center}
        \vspace*{\stretch{1}}

        \vspace*{\stretch{1}}
    \end{flashcard}

    \begin{flashcard}[Describe]{Service Trust Portal}
        \vspace*{\stretch{1}}

        \vspace*{\stretch{1}}
    \end{flashcard}

    \begin{flashcard}[Describe]{Complieance Manager}
        \vspace*{\stretch{1}}

        \vspace*{\stretch{1}}
    \end{flashcard}

    \begin{flashcard}[Determine]{if Azure is complieant for a business need}
        \vspace*{\stretch{1}}

        \vspace*{\stretch{1}}
    \end{flashcard}

    \begin{flashcard}[Understand]{Azure Government cloud services}
        \vspace*{\stretch{1}}

        \vspace*{\stretch{1}}
    \end{flashcard}

    \begin{flashcard}[Describe]{Azure China cloud services}
        \vspace*{\stretch{1}}

        \vspace*{\stretch{1}}
    \end{flashcard}

    \sectioncard{Understand Azure Pricing and Support}

    \begin{flashcard}[Describe]{Azure subscription}
        \vspace*{\stretch{1}}

        \vspace*{\stretch{1}}
    \end{flashcard}

    \begin{flashcard}[Understand]{Uses and options with Azure subscriptions}
        \vspace*{\stretch{1}}

        \vspace*{\stretch{1}}
    \end{flashcard}

    \begin{flashcard}[Understand]{Subscription access control}
        \vspace*{\stretch{1}}

        \vspace*{\stretch{1}}
    \end{flashcard}

    \begin{flashcard}[Understand]{Subscription offer types}
        \vspace*{\stretch{1}}

        \vspace*{\stretch{1}}
    \end{flashcard}

    \begin{flashcard}[Understand]{Subscription management using Management groups}
        \vspace*{\stretch{1}}

        \vspace*{\stretch{1}}
    \end{flashcard}

    \subsectioncard{Understand Planning and Management of Costs}

    \begin{flashcard}[Understand]{Options for purchasing Azure products and services}
        \vspace*{\stretch{1}}

        \vspace*{\stretch{1}}
    \end{flashcard}

    \begin{flashcard}[Understand]{Options around Azure Free Account}
        \vspace*{\stretch{1}}

        \vspace*{\stretch{1}}
    \end{flashcard}

    \begin{flashcard}[Understand]{Factors affecting costs}
        \vspace*{\stretch{1}}

        \vspace*{\stretch{1}}
    \end{flashcard}

    \begin{flashcard}[Understand]{Cost influence of resource types}
        \vspace*{\stretch{1}}

        \vspace*{\stretch{1}}
    \end{flashcard}

    \begin{flashcard}[Understand]{Cost influence of services}
        \vspace*{\stretch{1}}

        \vspace*{\stretch{1}}
    \end{flashcard}

    \begin{flashcard}[Understand]{Cost influence of locations}
        \vspace*{\stretch{1}}

        \vspace*{\stretch{1}}
    \end{flashcard}

    \begin{flashcard}[Understand]{Cost influence of ingress traffic}
        \vspace*{\stretch{1}}

        \vspace*{\stretch{1}}
    \end{flashcard}

    \begin{flashcard}[Understand]{Cost influence of egress traffic}
        \vspace*{\stretch{1}}

        \vspace*{\stretch{1}}
    \end{flashcard}

    \begin{flashcard}[Understand]{Zones for billing purposes}
        \vspace*{\stretch{1}}

        \vspace*{\stretch{1}}
    \end{flashcard}

    \begin{flashcard}[Understand]{Pricing calculator}
        \vspace*{\stretch{1}}

        \vspace*{\stretch{1}}
    \end{flashcard}

    \begin{flashcard}[Understand]{Total Cost of Ownership (TCO) calculator}
        \vspace*{\stretch{1}}

        \vspace*{\stretch{1}}
    \end{flashcard}

    \begin{flashcard}[Understand]{Best practices for minimizing Azure costs}
        \vspace*{\stretch{1}}

        \vspace*{\stretch{1}}
    \end{flashcard}

    \begin{flashcard}[Understand]{Performing cost analysis}
        \vspace*{\stretch{1}}

        \vspace*{\stretch{1}}
    \end{flashcard}

    \begin{flashcard}[Understand]{Creating spending limits, quotas}
        \vspace*{\stretch{1}}

        \vspace*{\stretch{1}}
    \end{flashcard}

    \begin{flashcard}[Understand]{Tags to identify cost owners}
        \vspace*{\stretch{1}}

        \vspace*{\stretch{1}}
    \end{flashcard}

    \begin{flashcard}[Understand]{Azure reservations}
        \vspace*{\stretch{1}}

        \vspace*{\stretch{1}}
    \end{flashcard}

    \begin{flashcard}[Understand]{Azure Advisor recommendations}
        \vspace*{\stretch{1}}

        \vspace*{\stretch{1}}
    \end{flashcard}

    \begin{flashcard}[Understand]{Azure Cost Management}
        \vspace*{\stretch{1}}

        \vspace*{\stretch{1}}
    \end{flashcard}

    \subsectioncard{Understand the Support Options Available with Azure}

    \begin{flashcard}[Understand]{Support Plans}
        \vspace*{\stretch{1}}
        
        \vspace*{\stretch{1}}
    \end{flashcard}

    \begin{flashcard}[Understand]{Dev support plan}
        \vspace*{\stretch{1}}

        \vspace*{\stretch{1}}
    \end{flashcard}

    \begin{flashcard}[Understand]{Standard support plan}
        \vspace*{\stretch{1}}

        \vspace*{\stretch{1}}
    \end{flashcard}

    \begin{flashcard}[Understand]{Professional Direct support plan}
        \vspace*{\stretch{1}}

        \vspace*{\stretch{1}}
    \end{flashcard}

    \begin{flashcard}[Understand]{Premiur support plan}
        \vspace*{\stretch{1}}

        \vspace*{\stretch{1}}
    \end{flashcard}

    \begin{flashcard}[Understand]{How to open a support ticket}
        \vspace*{\stretch{1}}

        \vspace*{\stretch{1}}
    \end{flashcard}

    \begin{flashcard}[Understand]{Available support channels (outside of support plan)}
        \vspace*{\stretch{1}}

        \vspace*{\stretch{1}}
    \end{flashcard}

    \begin{flashcard}[Describe]{Knowledge Center}
        \vspace*{\stretch{1}}

        \vspace*{\stretch{1}}
    \end{flashcard}

    \subsectioncard{Beschreibe Vereinbarungen zum Servicelevel für Azure (SLAs)}

    \begin{flashcard}[Definition]{Vereinbarungen zum Servicelevel (SLA)}
        \vspace*{\stretch{1}}
        \begin{itemize}
            \item formales Dokument/Vertrag mit spezifischen Bedinungen/Verpflichtungen, die Microsoft eingeht
            \item Leistungsstandards, die eingehalten werden
            \item spezifisch für einzelne Azure-Produkte und -Dienste
            \item Bedingungen für nicht einhalten der Vereinbarung
            \item
                \begin{itemize}
                    \item Leistungsziele
                    \item Betriebszeit und Konnektivität
                    \item Dienstguthaben
                \end{itemize}

        \end{itemize}
        Keine SLAs für Free- und Shared-Tarife.
        \vspace*{\stretch{1}}
    \end{flashcard}

    \begin{flashcard}[\ ]{Auswirkungen der Betriebszeit}
        \vspace*{\stretch{1}}
        \begin{tabular}{l|lll}
            SLA      &  Woche  & Monat  & Jahr    \\
            \hline
            99\%     &  1,68 h & 7,2 h  & 3,65 d  \\
            99,9\%   &  10,1 M & 43,2 M & 8,76 h  \\
            99,95\%  &  5 M    & 21,6 M & 4,38 h  \\
            99,99\%  &  1,01 M & 4,32 M & 52,56 M \\
            99,999\% &  6 s    & 25,9 s & 5,26 M  \\
        \end{tabular}
        
        \vspace{5mm}
        Wenn die Betriebszeit in den Servicelevel-Vereinbarungen zu klein ist, kann sie schwer garantiert werden
        \vspace*{\stretch{1}}
    \end{flashcard}


    \begin{flashcard}[\ ]{Zusammengesetzte SLAs}
        \vspace*{\stretch{1}}
        \begin{itemize}
            \item Kombination von mehreren Azure-Diensten
            \item kann je nach Architektur \emph{niedrigere} oder \emph{höhere} Betriebszeiten ergeben
        \end{itemize}
        ``UND''
        \begin{itemize}
            \item eine Anwendung benötigt zwei (oder mehr) Dienste zusammen
            \item zusammengesetzte Fehlerwahrscheinlichkeit \emph{höher}
        \end{itemize}
        ``ODER''
        \begin{itemize}
            \item eine Anwendung benötigt mindestens einer von zwei (oder mehreren) Diensten
            \item zusammengesetzte Fehlerwahrscheinlichkeit \emph{niedriger}
            \item teurer und komplexere Architektur
        \end{itemize}
        \vspace*{\stretch{1}}
    \end{flashcard}

    
    
    \begin{flashcard}[Beschreibe]{Ein angemessenes SLA für eine Anwendung bestimmen}
        \vspace*{\stretch{1}}
            \begin{itemize}
                \item bestimmen der Anforderungen an die Anwendung
                \item die passenden Azure-Dienste auswählen, sollten keine kleinere Verfügbarkeit haben als das angestrebte Ziel
                \item höhere Verfügbarkeit erfordert komplexere Systeme\newline
                Verfügbarkeitsklasse 4 (99,99\%) nicht mit manuellem Eingriff möglich
                \item Zeitfenster für Betriebszeit enorm wichtig\newline
                $\Rightarrow$ stündliche oder tägliche Betriebszeit führt zu unerreichbar kleinen Toleranzen
            \end{itemize}
        \vspace*{\stretch{1}}
    \end{flashcard}

    \subsectioncard{Verstehen des Azure Service-Lebenszyklus}

    \begin{flashcard}[Definition]{Öffentliche and private Previewfeatures}
        \vspace*{\stretch{1}}
        \begin{itemize}
            \item Betaversionen oder Vorabversionen zum Testen
            \item private Vorschau: nur für einge Kunden verfügbar\newline
            benötigt Einladung
            \item öffentliche Vorschau: für alle Kunden verfügbar
            \item eingeschränkte Gechäftsbedingungen: kein Service oder keine Verfügbarkeitsgarantie
            \item Das Azure-Portal hat auch eine Vorschau\newline
            (\href{https://preview.portal.azure.com/}{https://preview.portal.azure.com/})
        \end{itemize}
        \vspace*{\stretch{1}}
    \end{flashcard}
    
    \begin{flashcard}[\ ]{Benutzung von öffentlichen Previewfeatures}
        \vspace*{\stretch{1}}
        \begin{itemize}
            \item Im Azure-Portal: Resourcen mit namen ``Preview''
            \item RSS-Feed
            \item Webseite: \href{https://azure.microsoft.com/services/preview/}{https://azure.microsoft.com/services/preview/}
        \end{itemize}
        
        \vspace{5mm}
        Feedback: Über das Portal
        \vspace*{\stretch{1}}
    \end{flashcard}
    
    

    \begin{flashcard}[Definition]{General Availability (GA)}
        \vspace*{\stretch{1}}
        Produkte und Dienste die mit Support für alle Kunden verfügbar sind
        \vspace*{\stretch{1}}
    \end{flashcard}

    \begin{flashcard}[\ ]{Wie können Produktänderungen und Features verfolgt werden}
        \vspace*{\stretch{1}}
        \begin{itemize}
            \item Webseite \href{https://azure.microsoft.com/updates/}{https://azure.microsoft.com/updates/}
            \item Eintrag ``Neuheiten'' im Azure-Portal
        \end{itemize}

        \vspace*{\stretch{1}}
    \end{flashcard}

    \vspace*{\stretch{1}}
    \doclicenseThis
    \vspace*{\stretch{1}}
    \pagebreak
\end{document}
