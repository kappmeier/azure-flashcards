\subsectioncard{Beschreibe die  Unterschiede zwischen öffentlicher, privater, und Hybrid Cloud}

\begin{flashcard}[Describe]{Öffentliche Cloud}
    \vspace*{\stretch{1}}
    \begin{itemize}
        \item typische Methode der Cloudnutzung
        \item keine loakle Hardware
        \item \emph{alle} Workloads werden in der Cloud ausgeführt
        \item ggf. können Ressourcen für andere Cloudbenutzer freigegeben werden
    \end{itemize}

    \vspace{5mm}
    Azure ist eine öffentliche Cloud
    \vspace*{\stretch{1}}
\end{flashcard}

\begin{flashcard}[Describe]{Private Cloud}
    \vspace*{\stretch{1}}
    Aufbau einer Cloudumgebung im lokalen Rechenzentrum

    \begin{itemize}
        \item aus Nutzersicht kein Unterschied zu Cloudbenutzung
        \item Unternehmen ist selbst für Anschaffung und Wartung von Hardware verantwortlich
    \end{itemize}

    \vspace{5mm}
    Von Azure unterstützt durch Azure Stack

    \vspace*{\stretch{1}}
\end{flashcard}

\begin{flashcard}[Describe]{Hybrid Cloud}
    \vspace*{\stretch{1}}
    Verbindet öffentliche und private cloud\newline
    $\Rightarrow$ Anwendungen können in der passenden Umgebung ausgeführt werden

    \vspace{5mm}
    Use cases:
    \begin{itemize}
        \item Übergang von lokalen Datencentern zu Cloud
        \item Aufteilung von Daten und Verarbeitung z.\,B. aus rechtlichen Gründen
        \item Abfang von Leistungsspitzen
    \end{itemize}

    \vspace*{\stretch{1}}
\end{flashcard}

\begin{flashcard}[Gegenüberstellung]{Drei Cloudmodelle}
    \vspace*{\stretch{1}}
        \begin{tabular}{l|ll}
            Modell     & Vorteile                                     & Nachteile                                  \\
            \hline
            Öffentlich & \tabitem hohe Skalierbarkeit (Agilität)      & \tabitem Sicherheitsanforderungen          \\
                        & \tabitem keine CapEx-Kosten                  & \tabitem rechtliche Anforderungen          \\
                        & \tabitem ausgelagerte Wartung                & \tabitem Legacy-Kompatibilität             \\
                        & \tabitem geringe technische Kenntnisse       & \tabitem kein Eigentum an Diensten +       \\
                        & \tabitemindent der Mitarbeiter erforderlich  & \tabitemindent Hardware (eingeschränkt)    \\
            [.5\normalbaselineskip]
            Private    & \tabitem Kontrolle der Konfiguration         & \tabitem hohe CapEx-Kosten (Anschaffung)   \\
                        & \tabitem Kontrolle über Sicherheit           & \tabitem geringe Skalierbarkeit (Agilität) \\
                        & \tabitem rechtlicher Anforderungen           & \tabitem hohe Fachkenntnis erforderlich    \\
            [.5\normalbaselineskip]
            Hybrid     & \tabitem Eigene Hardware verwendbar          & \tabitem möglicherweise CapEx-Kosten       \\
                        & \tabitem Flexibilität (Lokal + Cloud)        & \tabitem komplizierte Einrichtung          \\
                        & \tabitem Skaleneffekte + lokales sparen      &                                            \\
                        & \tabitem Kontrolle über Sicherheit           &                                            \\
                        & \tabitem rechtliche Anforderungen            &                                            \\
        \end{tabular}
    \vspace*{\stretch{1}}
\end{flashcard}
