\subsectioncard{Beschreibe die Unterschiede zwischen Infrastructure-as-a-Service (IaaS), Platform-as-a-Service (Paas) und Software-as-a-Service (SaaS)}

\begin{flashcard}[Definition]{Kategorien von Cloud-Services}
    \vspace*{\stretch{1}}
    Cloud-Services lassen sich in die folgenden Kategorien einteilen:
    \begin{itemize}
        \item Infrastructure-as-a-Service (IaaS)
        \item Platform-as-a-Service (PaaS)
        \item Software-as-a-Service (SaaS)
    \end{itemize}
    Die Kategorien unterscheiden sich in der Aufteilung der Anforderungen für den Nutzer und den Cloudanbieter.

    Von IaaS zu SaaS sinkt die Flexibilität für Nutzer und der Umfang der bereitgestellten Services steigt.

    \vspace*{\stretch{1}}
\end{flashcard}

\begin{flashcard}[Definition]{Modell der gemeinsamen Verantwortung}
    \vspace*{\stretch{1}}
    Cloudanbieter und Nutzer teilen sich die Verantwortung für die Funktionalität eines Services:
    \begin{itemize}
        \item Cloudanbieter stellt sicher, dass Infrastruktur funktioniert
        \item Nutzer ist verantwortlich, dass sein Dienst korrekt funktioniert
    \end{itemize}
    Trifft für IaaS zu. (In geringerem Umfang auch für PaaS).
    \vspace*{\stretch{1}}
\end{flashcard}

\begin{flashcard}[Definition]{IaaS}
    \vspace*{\stretch{1}}
    IaaS = Infrastructure-as-a-Service

    \begin{itemize}
        \item Infrastruktur wird vom Cloudanbieter verwaltet:
        \begin{itemize}
            \item Virtuelle Maschinen
            \item Netzwerk
            \item Betriebssysteme (teilweise)
            \item Speicher
        \end{itemize}
        \item Kunden mieten die Infrastructure, anstatt sie zu kaufen.
        \item Kunden sind komplett selbst verantwortlich für ihre eigene Umgebung.
    \end{itemize}
    \vspace*{\stretch{1}}
\end{flashcard}

\begin{flashcard}[Definition]{PaaS}
    \vspace*{\stretch{1}}
    PaaS = Platform-as-a-Service

    \begin{itemize}
        \item Vom Cloudanbieter wird eine komplette Umgebung zum Erstellen, Testen und Bereitstellen von Softwareanwendungen angeboten
        \item Kunden müssen sich \emph{nicht} um die Infrastructure kümmern\newline
        (Wartung, Betriebssysteme, Systemupdates, Webserver, \ldots)
        \item Vollständige Entwicklung und Deployment in der Cloud
        \item Sämtliche Vorteile der Cloud (Skalierbarkeit, Hochverfügbarkeit, \ldots) sind\newline automatisch vorhanden
    \end{itemize}
    \vspace*{\stretch{1}}
\end{flashcard}

\begin{flashcard}[Definition]{SaaS}
    \vspace*{\stretch{1}}
    SaaS = Software-as-a-Service

    Vollständige vom Cloudanbieter bereitgestellte Software
    \begin{itemize}
        \item Infrastruktur wird vom Cloudanbier verwaltet, gewartet, und bereitgestellt
        \item Software wird für Endkunden bereitgestellt
        \item typischerweise eine Softwareversion für alle Kunden
        \item Abrechnung über Abonnement
    \end{itemize}
    \vspace*{\stretch{1}}
\end{flashcard}

\begin{flashcard}[\ ]{Unterschiede der Cloud-Service-Kategorien (verantwortlichkeit)}
    \vspace*{\stretch{1}}
    \begin{tabular}{l|p{34mm}p{34mm}p{34mm}}
                & \textbf{IaaS}                                             & \textbf{PaaS}                                                                     & \textbf{SaaS} \\
        \hline
        Kosten & Nutzung                                                   & Nutzung                                                                           & Abonnement                                                  \\
        Nutzer & Installation, Verwaltung, Konfiguration eigener Software  & Entwicklung und Bereitstellung eigener Anwendungen bei Nutzung der Cloud-Services & Verwendung der bereitgestellten Software                    \\
        Cloud  & Verantwortlich für Infrastruktur (VM, Netzwerk, Speicher) & Verantwortung für alles \emph{außer} der vom Nutzer entwickelten Software         & Komplette Verantwortung für die Bereitstellung der Software \\
    \end{tabular}
    \vspace*{\stretch{1}}
\end{flashcard}
