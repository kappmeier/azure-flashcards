\subsectioncard{Beschreibe die Vorteile und Erwägungen bei Nutzung von Cloud-Services}

\begin{flashcard}[Definition]{Hochverfügbarkeit}
    \vspace*{\stretch{1}}
    \begin{itemize}
        \item die Möglichkeiten einer Anwendung ohne signifikante Auszeit durchgehend zu laufen
        \item es wird erwartet, dass die Anwendung reagiert und Nutzer mit ihr interagieren können
    \end{itemize}
    \vspace*{\stretch{1}}
\end{flashcard}

\begin{flashcard}[Definition]{Kosteneffizienz}
    \vspace*{\stretch{1}}
    Preismodell der nutzungsbasierten Bezahlung:
    \begin{itemize}
        \item keine Vorausszahlung
        \item teure Infrastruktur muss nicht gekauft werden
        \item Ressourcen müssen nur bezahlt werden, wenn sie benötigt werden
    \end{itemize}

    \vspace{5mm}
    Alternativ: dedizierte Hardware in der Cloud mieten
    \vspace*{\stretch{1}}
\end{flashcard}

\begin{flashcard}[Definition]{Skalierbarkeit}
    \vspace*{\stretch{1}}
    Anzahl der Ressourcen und Dienste nach Nachfrage verändern

    \vspace{5mm}
    Vertikale Skalierung (Hochskalieren)
    \begin{itemize}
        \item Leistung von Servern oder Diensten erhöhen\newline
        z.\,B. mehr CPUs, mehr Speicher, schnellere Anbindung
    \end{itemize}

    Horizontale Skalierung
    \begin{itemize}
        \item zusätzliche Server werden hinzugefügt
        \item sämtliche Maschinen funktionieren als Einheit
    \end{itemize}

    \vspace{5mm}
    Skalierung kann automatisch anhand gewisser Kriterien durchgeführt werden
    \vspace*{\stretch{1}}
\end{flashcard}

\begin{flashcard}[Definition]{Elastizität der Cloud}
    \vspace*{\stretch{1}}
    \begin{itemize}
        \item automatische Hinzunahme oder Wegnahme von Ressourcen bei ändernder Nachfrage
        \item Beispiel: kurzzeitige Lastspitzen, aber auch jahreszeitliche Änderungen oder täglich, wöchentlich
    \end{itemize}

    \vspace*{\stretch{1}}
\end{flashcard}

\begin{flashcard}[Definition]{Agilität}
    \vspace*{\stretch{1}}

    \vspace{5mm}
    Schnelle Anpassung an wechselnde Anforderungen eines Unternehmens
    \begin{itemize}
        \item das kann zeitlich schwankender Zugriff sein (Elastizität)
        \item aber auch schnelle Anpassungen an Änderung der Unternehmensstrategie oder Ähnliches
    \end{itemize}

    \vspace*{\stretch{1}}
\end{flashcard}

\begin{flashcard}[Definition]{Fehlertoleranz}
    \vspace*{\stretch{1}}

    \begin{itemize}
        \item Datensicherung, Notfallwiederherstellung, Replikation
        \item redundante Architektur
    \end{itemize}
    Diese Maßhnahmen sollen sicherstellen, dass Kunden nicht von auftretenden Fehlern betroffen sind
    \vspace*{\stretch{1}}
\end{flashcard}

\begin{flashcard}[Definition]{Notfallwiederherstellung}
    \vspace*{\stretch{1}}
    \begin{itemize}
        \item die Fähigkeit des Systems der Erholung nach einem Zwischenfall\newline
        Zwischenfall = weiträumiger Ausfall (z.\,B. ganze Region)
        \item Datensicherung, Archivierung, Wiederherstellung
    \end{itemize}

    \vspace*{\stretch{1}}
\end{flashcard}

\begin{flashcard}[Definition]{Skaleneffekte (Economy of scale)}
    \vspace*{\stretch{1}}
    Arbeit kann pro Einheit günstiger durchgeführt werden, wenn sie in großem Umfang durchgeführt wird.

    \vspace{5mm}
    Bezogen auf cloud:

    Cloudanbieter können Hardware günstiger erwerben, effizienter warten und anderweitig sparen als einzelne Kunden das könnten.
    \vspace*{\stretch{1}}
\end{flashcard}

\begin{flashcard}[Definition]{CapEx}
    \vspace*{\stretch{1}}

    Investitionsausgaben (Capital Expenditure)
    \begin{itemize}
        \item Anschaffungskosten physischer Infrastruktur
        \item abschreibbar über die Zeit
        \item Anschaffungskosten, deren Wert über die Zeit geringer wird
    \end{itemize}

    Typisches Kostemodell für lokale Rechenzentren

    \vspace*{\stretch{1}}
\end{flashcard}

\begin{flashcard}[Beispiele]{CapEx}
    \vspace*{\stretch{1}}
    \begin{itemize}
        \item Serverkosten: Hardware + Support
        \item Speicher: Storageserver, \ldots, + Support
        \item Netwerkinfrastruktur: inklusive Kabel, WAN, \ldots
        \item Backup + Archivierung
        \item Infrastruktur für das Rechenzentrum
    \end{itemize}
    \vspace*{\stretch{1}}
\end{flashcard}

\begin{flashcard}[Definition]{OPex}
    \vspace*{\stretch{1}}
    Betriebskosten (Operational Expenditure)
    \begin{itemize}
        \item laufende Kosten für Dienste und Produkte
        \item keine Anschaffungskosten
        \item Zahlung nur für Dienste, die in Anspruch genommen werden
        \item abschreibbar im laufenden Geschäftsjahr
    \end{itemize}

    Typisches Kostenmodell für Cloud-Computing
    \vspace*{\stretch{1}}
\end{flashcard}

\begin{flashcard}[Beispiel]{OPex}
    \vspace*{\stretch{1}}
    Betriebskosten (Operational Expenditure)
    \begin{itemize}
        \item laufende Kosten für Dienste und Produkte
        \item keine Anschaffungskosten
        \item Zahlung nur für Dienste, die in Anspruch genommen werden
        \item abschreibbar im laufenden Geschäftsjahr
    \end{itemize}

    Typisches Kostenmodell für Cloud-Computing
    \vspace*{\stretch{1}}
\end{flashcard}

\begin{flashcard}[Vergleiche]{CapEx and OpEx}
    \vspace*{\stretch{1}}
    Vorteile CapEx:
    \begin{itemize}
        \item Ausgaben fix, d.\,h. planbar
        \item praktisch, falls mit begrenzten Budgets geplant werden muss
    \end{itemize}

    Vorteile OpEx:
    \begin{itemize}
        \item schnelle Reaktion auf geänderte Nachfrage und Wachstum
        \item keine hohen Anschaffungskosten bei Projektstart
    \end{itemize}
    $\Rightarrow$ bei unbekannter oder schwankender Nachfrage

    \vspace*{\stretch{1}}
\end{flashcard}

\begin{flashcard}[Definition]{Verbrauchsbasierendes Modell}
    \vspace*{\stretch{1}}
        \begin{itemize}
            \item es werden nur benutzte Ressourcen bezahlt
            \item bei hoher Nachfrage in einem Monat kann gemäß der Nachfrage skaliert werden (und auch bezahlt)
            \item bei sinkender Auslastung können die Kosten wieder reduziert werden
        \end{itemize}
        $\Rightarrow$ im verbrauchsbasierenden Modell gibt es hauptsächlich OpEx-Kosten

        \vspace{5mm}
        Im dediziertern Modell wird eine bestimmte Hardware (entsprechend der erwarteten Auslastung) angeschafft:
        \begin{itemize}
            \item häufig wird zu viel bezahlt
            \item \emph{trotzdem} können Spitzen nicht bedient werden
        \end{itemize}
    \vspace*{\stretch{1}}
\end{flashcard}
