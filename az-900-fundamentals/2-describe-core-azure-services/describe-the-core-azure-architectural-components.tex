\subsectioncard{Understand the Core Azure Architectural Components}

\begin{flashcard}[Definition]{Regionen}
    \vspace*{\stretch{1}}
    \begin{itemize}
        \item mehrere Datencenter, die nahe beieinander sind (geringe Latenz)
        \item viele (>42) ermöglichen Bereitstellung von Services, wo sie genutzt werden
        \item Partner-Region in der gleichen Geographie:\newline
        $\Rightarrow$ ermöglichen geordnetes herunterfahren von Datencentern
        \item Priorisierung einer Region eines Paares im Fall von Ausfällen
    \end{itemize}
    $\Rightarrow$ es sollten nach Möglichkeit beide Regionen eines Paares genutzt werden

    \vspace{5mm}
    Geographien:
    \begin{itemize}
        \item geographische Regionen, die mindestens eine Region enthalten
    \end{itemize}

    \vspace*{\stretch{1}}
\end{flashcard}

\begin{flashcard}[Definition]{Verfügbarkeitszone}
    \vspace*{\stretch{1}}
    \begin{itemize}
        \item ermöglicht Hochverfügbarkeit
        \item physikalisch getrennte Bereiche innerhalb einer Region
        \item bestehen aus Datencentern, die ihrerseits eigene Versorgung haben
        \item 3 oder mehr Zonen pro Region ermöglichen Resilienz
        \item Verfügbarkeitszonen haben eine Servicelevel-Vereinbarung von 99,99\% zur Verfügbarkeit von VMs
    \end{itemize}
    \vspace*{\stretch{1}}
\end{flashcard}

\begin{flashcard}[Definition]{Fehler- und Updatedomänen}
    \vspace*{\stretch{1}}
    \begin{itemize}
        \item Bereiche innerhalb einer Region, verbunden mit Verfügbarkeitszonen
        \item falls mehr als 3 VMs in den 3 Verfügbarkeitszonen erstellt werden
        \item Maschinen werden automatisch auf Fehler- und Updatedomänen verteilt
        \item Fehlerdomänen:\newline
            Verteilung der VMs um Verfügbarkeit zu erhöhen
        \item Updatedomänen:\newline
            VMs in verschiedenen Verfügbarkeitszonen werden nicht gleichzeitig Aktualisiert
    \end{itemize}
    \vspace*{\stretch{1}}
\end{flashcard}

\begin{flashcard}[Definition]{Ressourcengrupps}
    \vspace*{\stretch{1}}
    \begin{itemize}
        \item logischer Container für Azure-Ressourcen:\newline
        $\Rightarrow$ dient der Ordnung und Gruppierung (z.\,B. regional, nach Typ, Verwendungszweck)
        \item \emph{jede} Ressource muss sich in einer Ressourcengruppe befinden (1:$n$)
        \item Ressourcen können in andere Gruppe verschoben werden
        \item Ressourcengruppen können \emph{keine} Ressourcengruppen beinhalten (flache Hierarchie)
        \item beim Löschen werden alle enthaltenen Ressourcen ebenfalls gelöscht
        \item Berechtigungen der RBAC werden vererbt
    \end{itemize}

    \vspace*{\stretch{1}}
\end{flashcard}

\begin{flashcard}[\ ]{Nutzung von Ressourcengruppen}
    \vspace*{\stretch{1}}
    \begin{itemize}
        \item Organisation von Ressourcen in großen Unternehmen
        \item z.\,B. durch geeignete Benennungskonvention
        \item Organisationskriterien:
            \begin{itemize}
                \item Ressourcentyp (Virtuelle Maschine, Datenbank, \ldots)
                \item Umgebung (Prod, Staging, Dev)
                \item Abteilung (Research, HR, \ldots)
            \end{itemize}
            oder beliebige Kombination\newline
            $\Rightarrow$ je nach Größe z.\,B. auch Abonemment (z.\,B. für Umgebung)
        \item Kriterien: Zugriffssteuerung (Ownership), Abrechnung, Ressourcenlebenszyklus
    \end{itemize}

    \vspace*{\stretch{1}}
\end{flashcard}

\begin{flashcard}[Definition]{Azure Ressourcenmanager}
    \vspace*{\stretch{1}}
    \begin{itemize}
        \item von Azure bereitgestellte Management-Plattform, um Resourcen zu verwalten:\newline
        Bereitstellung, Modifikation, und Freigabe
        \item kann Ressourcen zusammen als Gruppe verwalten
        \item ermöglicht Templates um gleiche Ressourcen in unterschiedlichen Umgebungen zu erstellen
        \item stellt Verwaltungsfeatures für Sicherheit, Audit, und Tagging\newline
        z.\,B. RBAC
    \end{itemize}

    \vspace*{\stretch{1}}
\end{flashcard}

\begin{flashcard}[\ ]{Benefits and Usage of Core Azure Architectural Components}
    \vspace*{\stretch{1}}
    Grundlegende Komponenten von Azure um Infrastruktur in bereitzustellen und die Cloud-Vorteile auszunutzen:
    \begin{itemize}
        \item Regionen $\Rightarrow$ ermöglichen die Bereitstellung von Ressourcen an verschiedenen Orten
        \item Verfügbarkeitszonen $\Rightarrow$ ermöglichen Redundanz von Ressourcen
        \item Ressourcengruppen $\Rightarrow$ logische Verwaltung von Ressourcen
    \end{itemize}

    \vspace{5mm}
    Ressourcen Manager ist die von Azure bereitgestellte Plattform, um Ressourcen in Azure bereitzustellen und zu verwalten

    \vspace*{\stretch{1}}
\end{flashcard}
