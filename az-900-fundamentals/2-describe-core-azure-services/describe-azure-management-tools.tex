\subsectioncard{Understand Azure Management Tools}

\begin{flashcard}[\ ]{Azure Portal}
    \vspace*{\stretch{1}}
    \begin{itemize}
        \item Eine von drei möglichen Azure Management tools
        \item Web-Frontend zur Verwaltung von Azure Services und Ressourcen
        \item Kontaktaufnahme zu Microsoft, z.\,B. über Service-Tickets
        \item Anzeige von Logs, Dashboards, Fehlermeldungen
        \item nicht so gut geeignet um Ressourcen/Infrastruktur tatsächlich zu erstellen und Verwalten
        \item keine Automatisierung
    \end{itemize}
    $\Rightarrow$ Infrastructur as Code besser geeignet

    \vspace*{\stretch{1}}
\end{flashcard}

\begin{flashcard}[\ ]{Azure PowerShell}
    \vspace*{\stretch{1}}
    \begin{itemize}
        \item Eine von drei möglichen Azure Management tools
        \item Modul dasin Windows PowerShell oder PowerShell Core eingebunden werden kann
        \item verfügbar als \emph{Azure Cloud Shell}, oder lokal
        \item unterstützt Automatisierung
    \end{itemize}
    \vspace*{\stretch{1}}
\end{flashcard}

\begin{flashcard}[Definition]{Azure CLI}
    \vspace*{\stretch{1}}
    \begin{itemize}
        \item Eine von drei möglichen Azure Management tools
        \item Kommandozeilen-Programm um mit Azure zu Verbinden und Befehle abzusetzen
        \item verwaltet Azure Ressourcen
        \item verfügbar als \emph{Azure Cloud Shell}, oder lokal
        \item unterstützt Automatisierung
    \end{itemize}
    \vspace*{\stretch{1}}
\end{flashcard}

\begin{flashcard}[Definition]{Cloud Shell}
    \vspace*{\stretch{1}}
    \begin{itemize}
        \item Im webbrowser laufende Shell, die sich mit Azure verbinden kann
        \item Unterstütz Azure CLI und Azure PowerShell
        \item pro Benutzer werden Daten in Azure Blob gespeichert
    \end{itemize}

    \vspace{5mm}
    Um auf Azure-Ressourcen zugreifen zu können, muss die IP des cloud shell servers freigegeben sein!
    \vspace*{\stretch{1}}
\end{flashcard}

\begin{flashcard}[Describe]{Azure Advisor}
    \vspace*{\stretch{1}}
    \begin{itemize}
        \item von Microsoft bereitgstelltes Tool
        \item analysiert Cloud-Nutzung und erkennt Probleme in verschiedenen Kategorieen
        \item Sicherheit, Kosten, \ldots
        \item markiert in Kategorien von unwichtig bis wichtig
        \item grundversion Kostenlos
    \end{itemize}

    \vspace*{\stretch{1}}
\end{flashcard}
