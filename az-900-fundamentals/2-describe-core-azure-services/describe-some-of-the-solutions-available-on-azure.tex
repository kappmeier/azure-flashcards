\subsectioncard{Beschreibe einige der in Azure verfügbaren Lösungen}

\begin{flashcard}[Definition]{Internet der Dinge (IoT)}
    \vspace*{\stretch{1}}
    Nicht primär zur Informationstechnik gehörende Geräte, die Daten sammeln und über das Internet übertragen.

    \vspace{5mm}
    Die Gesamtheit der vernetzten Geräte ist das Internet der Dinge'' (IoT)

    \vspace{5mm}
    Beispiele:
    \begin{itemize}
        \item Waschmaschine
        \item Smart-Watch
        \item Kühlschrank
        \item Smart-Home-Thermostate
    \end{itemize}

    \vspace*{\stretch{1}}
\end{flashcard}

\begin{flashcard}[\ ]{Produkte für IoT in Azure}
    \vspace*{\stretch{1}}
    \begin{itemize}
        \item IoT Central
        \item IoT Hub
        \item IoT Edge
    \end{itemize}

    \vspace*{\stretch{1}}
\end{flashcard}

\begin{flashcard}[Definition]{IoT Hub}
    \vspace*{\stretch{1}}
    \begin{itemize}
        \item Messaging-Hub für sichere Kommunikation
        \item ermöglicht Überwachung von millionen IoT-Geräten
    \end{itemize}
    \vspace*{\stretch{1}}
\end{flashcard}

\begin{flashcard}[Definition]{IoT Central}
    \vspace*{\stretch{1}}
    \begin{itemize}
        \item skalierbare, verwaltete SaaS-Lösung für IoT
        \item einfache Vernetzung von IoT-Ressourcen
        \item Verwaltung, Überwachung
    \end{itemize}
    \vspace*{\stretch{1}}
\end{flashcard}

\begin{flashcard}[Definition]{IoT Edge}
    \vspace*{\stretch{1}}
    \begin{itemize}
        \item Datenanalyse-Modelle auf IoT-Geräte schicken
        \item ermöglicht schnelle Reaktion auf Zustandsänderungen
        \item keine cloudbasierte Berechnung notwendig
    \end{itemize}
    \vspace*{\stretch{1}}
\end{flashcard}

\begin{flashcard}[Definition]{Big Data Analytics}
    \vspace*{\stretch{1}}
    \begin{itemize}
        \item Lösungen und Dienste zur Analyse und Verarbeitung von großen Datenmengen
        \item ``Big Data'' große Datenvolumen, einzelne Datensätze können groß oder klein sein
        \item Beispiele:
            \begin{itemize}
                \item Genomanalyse
                \item Kommunikationssysteme
                \item \ldots
            \end{itemize}
    \end{itemize}
    \vspace*{\stretch{1}}
\end{flashcard}

\begin{flashcard}[Describe]{Big Data Products + Analytics Products}
    \vspace*{\stretch{1}}
    \begin{itemize}
        \item Synapse Analytics
        \item HDInsight
        \item Databricks
    \end{itemize}

    \vspace*{\stretch{1}}
\end{flashcard}

\begin{flashcard}[Describe]{SQL Data Warehouse}
    \vspace*{\stretch{1}}
    Umbenannt: nun Azure Synapse.

    \begin{itemize}
        \item Big-Data-Analysedienst
        \item kombiniert Data Warehousing und Big-Data-Analysen
        \item schnelle Ausführung komplexer Abfragen in großen Datenmengen (Petabyte)
        \item nutzt das cloudbasierte Enterprise Data Warehouse (EDW)
        \item Massively Prallel Processing (MPP) mit Unterstützung von freien On-Demand-Ressourcen
    \end{itemize}
    \vspace*{\stretch{1}}
\end{flashcard}

\begin{flashcard}[Describe]{HDInsight}
    \vspace*{\stretch{1}}
    \begin{itemize}
        \item Verarbeitung großer Datenmengen
        \item verwaltete Hadoop-Cluster
    \end{itemize}
    \vspace*{\stretch{1}}
\end{flashcard}

\begin{flashcard}[Describe]{Azure Databricks}
    \vspace*{\stretch{1}}
    \begin{itemize}
        \item Analyseplattform basierend auf Apache Spark
        \item optimiert für Microsoft Cloudplattform
        \item interkative Zusammenarbeit/Kollaboration von Datenspezialisten, Data Engineers und Business Analysts
        \item einfache Einrichtung mit One-Click-Workflow
        \item Erfassung von Rohdaten aus Batch-Prozessierung oder Live-Streaming
    \end{itemize}
    \vspace*{\stretch{1}}
\end{flashcard}

\begin{flashcard}[Describe]{Artificial Intelligence (AI)}
    \vspace*{\stretch{1}}
    \begin{itemize}
        \item Sammlung von Diensten, die auf maschinellem Lernen beruhen
        \item Computer lernen aus Daten, zukünftiges Verhalten, Ergebnisse oder Trends vorherzusagen
        \item keine Programmierung wird benötigt, da Algorithmen implizit aus den Daten gelernt werden
    \end{itemize}

    \vspace*{\stretch{1}}
\end{flashcard}

\begin{flashcard}[Describe]{Products Available for AI}
    \vspace*{\stretch{1}}
    \begin{itemize}
        \item Azure Machine Learning-Dienst
        \item Azure Machine Learning Studio
    \end{itemize}

    \vspace*{\stretch{1}}
\end{flashcard}

\begin{flashcard}[Describe]{Azure Machine Learning-Dienst}
    \vspace*{\stretch{1}}
    \begin{itemize}
        \item Entwicklung, Test, Deployment und Analyse von Machine Learning-Modellen
        \item unterstützt Generierung und Optimiierung von Modellen
        \item Beginn des Trainings auf dem lokalen Rechner
        \item Nutzung von Cloud-Kapazitäten durch horizontale Hochskalierung
    \end{itemize}
    \vspace*{\stretch{1}}
\end{flashcard}

\begin{flashcard}[Describe]{Azure Machine Learning Studio}
    \vspace*{\stretch{1}}
    \begin{itemize}
        \item Visuelle Entwiclung von Lösungen für machinelles Lernen
        \item unterstützt Modellentwicklung durch Drag \& Drop
        \item geeignet für Kollaboration mehrerer Entwickler
        \item Auswahl aus verschiedenen vordefinierten Algorithmen und Modulen für maschinelles Lernen
    \end{itemize}
    \vspace*{\stretch{1}}
\end{flashcard}

\begin{flashcard}[Describe]{Serverless Computing}
    \vspace*{\stretch{1}}
    \begin{itemize}
        \item direkte Ausführung von Code\newline
        $\Rightarrow$ vollständige Abstraktion der Hostumgebung
        \item Konfiguration und Wartung der Infrastruktur ist weder erforderlich noch zulässig
        \item[$\Rightarrow$] PaaS
        \item hochverfügbar
        \item jede Arbeit muss schnell erledigt werden (Sekunden, maximal Minuten)
        \item gut geeignet für ereignisgesteuerte Berechnungen
        \item Microbilling nur nach ausgeführter Berechnungszeit, evtl. auch nur wenige Sekunden
    \end{itemize}

    \vspace*{\stretch{1}}
\end{flashcard}

\begin{flashcard}[\ ]{Azure Serverless-Dienste}
    \vspace*{\stretch{1}}
    \begin{itemize}
        \item Azure Functions\newline
        Ausführung von Code einer beliebigen modernen Programmiersprache
        \item Logic Apps\newline
        graphische Anwendungen ohne Programmcode
        \item Event Grid\newline
    \end{itemize}

    \vspace*{\stretch{1}}
\end{flashcard}

\begin{flashcard}[Definition]{Azure Functions}
    \vspace*{\stretch{1}}
    \begin{itemize}
        \item Dienst, der vorgegebene Funktion ausführt
        \item ereignisgesteuert, serverlos
        \item kurze Laufzeit (maximal wenige Minuten)
        \item Varianten: zustandslos (Standard), zustandsbehaftet (\emph{Durable}), die einen Kontext erhalten
    \end{itemize}
    $\Rightarrow$ automatische Skalierung. Gut geeignet für schwankende Anforderungen, z.\,B. aus dem Internet der Dinge

    \vspace{5mm}
    \textbf{Vorteile}: keine Kosten während Leerlaufzeiten
    \vspace*{\stretch{1}}
\end{flashcard}

\begin{flashcard}[Definition]{Logic Apps}
    \vspace*{\stretch{1}}
    Ähnlich wie Functions, aber nicht programmiert!
    \begin{itemize}
        \item Workflows aus vordefinierten Logikblöcken\newline
        Zusammenstellung im visuellen Designer im Azure-Portal
        \item Orchestrierung von Aufgaben und Business-Prozessen
        \item Beginn der Ausführung mit einem Trigger
        \item Integration von Daten, Anwendungen, und Systemen\newline
        zahllose vordefinierte Module vorhanden um mit anderen Diensten zu kommunizieren
    \end{itemize}

    \vspace*{\stretch{1}}
\end{flashcard}

\begin{flashcard}[\ ]{Vergleich Azure Functions und Azure Logic Apps}
    \vspace*{\stretch{1}}
    \begin{tabular}{l|p{34mm}p{34mm}}
                & \textbf{Functions}                                             & \textbf{Logic Apps} \\
        \hline
        Zustand & Standard: Zustandslos, aber auch Durable                  & zustandsbehaftet                                                                         \\
        Entwicklung & imperativ/mit code                                        & deklarativ/grafisch                                                               \\
        Kontext  & Lokal und in der Cloud                                    & Nur in der Cloud                                                                  \\
    \end{tabular}
    \vspace*{\stretch{1}}
\end{flashcard}

\begin{flashcard}[Describe]{Event Grid}
    \vspace*{\stretch{1}}
    \begin{itemize}
        \item Aufbau von Anwendungen mit Event-basierer Architektur
        \item verwalteter Dienst
        \item Event-Routing über Publish/Subscribe for Eventverarbeitung
    \end{itemize}

    \vspace*{\stretch{1}}
\end{flashcard}

\begin{flashcard}[Describe]{Available DevOps solutions}
    \vspace*{\stretch{1}}

    \vspace*{\stretch{1}}
\end{flashcard}

\begin{flashcard}[Describe]{Azure DevOps}
    \vspace*{\stretch{1}}
    \begin{itemize}
        \item Azure-Service mit Tools zur Unterstützung von DevOps-Prozessen
        \item Azure Pipelines:
        \begin{itemize}
            \item Continuous Delivery
            \item Continuous Integration
        \end{itemize}
        \item Azure Boards: Issue tracking, Organisation, \ldots
        \item Azure Test Plans: Test-Tools für manuelles Testen
        \item Azure Repos: source code Verwaltung mit Git
        \item Azure Artifacts: Binary repository
    \end{itemize}

    \vspace*{\stretch{1}}
\end{flashcard}

\begin{flashcard}[Describe]{Azure DevTest Labs}
    \vspace*{\stretch{1}}
    \begin{itemize}
        \item Umgebung für Test/Entwicklung mit virtuellen Maschinen und PaaS Services
        \item Richtlinien ermöglichen Nutzung von Ressourcen ohne Genehmigung
        \begin{itemize}
            \item vorkonfigurierte und vor-erstellte Systeme
            \item automatischer Shutdown
        \end{itemize}
        \item vorkonfigurierte Systeme basierend auf Resource Manager-Vorlagen
        \item Kontrolle über verschwendete Ressourcen, Kosten, Budgests
    \end{itemize}
    \vspace*{\stretch{1}}
\end{flashcard}

\begin{flashcard}[Benefits and Outcomes]{Usage of Azure Solutions}
    \vspace*{\stretch{1}}

    \vspace*{\stretch{1}}
\end{flashcard}
