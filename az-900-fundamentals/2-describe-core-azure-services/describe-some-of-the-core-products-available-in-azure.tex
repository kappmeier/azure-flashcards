\subsectioncard{Beschreibe einige der in Azure verfügbaren Kernkomponenten}

\begin{flashcard}[Definition]{Virtuelle Maschine}
    \vspace*{\stretch{1}}
    \begin{itemize}
        \item virtueller Computer, der in der Azure cloud gehostet wird und läuft
        \item verfügbar für Linux (verschiedene Distributionen) und Windows
    \end{itemize}
    \vspace*{\stretch{1}}
\end{flashcard}

\begin{flashcard}[Definition]{Skalierungsgruppen für Virtuelle Maschinen}
    \vspace*{\stretch{1}}
    \begin{itemize}
        \item \emph{automatische} Skalierung von in Azure gehosteten virtuellen Maschinen
        \item mehrere virtuelle Maschinen gehören zu einer Gruppe
        \item die Elemente der Skalierungsgruppe sind \emph{identisch}
        \item eine große Zahl an Instanzen kann innerhalb von Minuten verwaltet werden
    \end{itemize}
    $\Rightarrow$ gut geeignet für Big Data-Analysen und Containerworkloads

    \vspace*{\stretch{1}}
\end{flashcard}

\begin{flashcard}[Definition]{App Service Functions}
    \vspace*{\stretch{1}}
    \begin{itemize}
        \item PaaS-Service in Azure\newline
        $\Rightarrow$ keine eigene Infrastruktur notwendig
        \item Enterprise-Anwendungen wie Webseiten, (REST)-APIs und für Mobilgeräte
        \item unterstützt Entwicklung und Deployment, CI über Git-Repository
        \item automatische Skalierung
        \item unterstützte Programmiersprachen:
        \begin{itemize}
            \item Python
            \item Java
            \item Ruby
            \item Node.js
            \item .NET/.NET Core
            \item PHP
        \end{itemize}
    \end{itemize}
    \vspace*{\stretch{1}}
\end{flashcard}

\begin{flashcard}[\ ]{Azure App Service-Kosten}
    \vspace*{\stretch{1}}
    \begin{itemize}
        \item festgelegt durch App Service-Plan
        \item bestimmt die größe/Anzahl der verfügbaren Hardware
        \item SKU F1 ermöglicht kostenlose Verwendung von kleinen Ressourcen im Free-Tarif
    \end{itemize}

    \vspace*{\stretch{1}}
\end{flashcard}

\begin{flashcard}[Definition]{Azure Container Instances (ACI)}
    \vspace*{\stretch{1}}
    \begin{itemize}
        \item Container-Anwendungen in Azure
        \item benötigt keine virtuelle Maschine oder Server
    \end{itemize}

    \vspace*{\stretch{1}}
\end{flashcard}

\begin{flashcard}[Definition]{Azure Kubernetes Service (AKS)}
    \vspace*{\stretch{1}}
    \begin{itemize}
        \item verwaltet ein Kubernetes-Cluster aus mehreren virtuellen Maschinen
        \item Dienste werden in Containern auf den Maschinen im Cluster ausgeführt
    \end{itemize}

    \vspace*{\stretch{1}}
\end{flashcard}

\begin{flashcard}[Definition]{Azure Batch}
    \vspace*{\stretch{1}}
    \begin{itemize}
        \item verwalteter Dienst für parallele Berechnungen
        \item Batch-Dienst unterstüzt viele Batch-Knoten (virtuelle Maschinen oder Skalierungsgruppe)
    \end{itemize}
    \vspace*{\stretch{1}}
\end{flashcard}

\begin{flashcard}[Definition]{Virtuelles Netzwerk}
    \vspace*{\stretch{1}}
    \begin{itemize}
        \item Dienst, der virtuelle Maschinen mit eingehenden VPN-Verbindungen (Virtual Private Network) verknüpft
        \item Verbindung von Azure-VPN mit einem lokalen VPN ermöglicht sichere Kommunikation zwischen Azure und lokalem Datacenter
        \item ExpressRoute ermöglicht dedizierte, sichere, \textbf{private} Verbindung zwischen Netzwerk und Azure\newline
        $\Rightarrow$ erhöht Sicherheit (nicht ür Endnutzer zugänglich!)
    \end{itemize}

    \vspace*{\stretch{1}}
\end{flashcard}

\begin{flashcard}[Definition]{Load Balancer}
    \vspace*{\stretch{1}}
    \begin{itemize}
        \item verteilt ein- und ausgehende Verbindungen
        \item Ziel ist gleichmäßige Auslastung von Anwendungen und Endpunkten
    \end{itemize}

    \vspace*{\stretch{1}}
\end{flashcard}

\begin{flashcard}[Definition]{VPN Gateway}
    \vspace*{\stretch{1}}
    \begin{itemize}
        \item ermöglicht den Zugriff auf Azure-VPNs
        \item hochleistungsfähige Gateways
    \end{itemize}

    \vspace*{\stretch{1}}
\end{flashcard}

\begin{flashcard}[Definition]{Application Gateway}
    \vspace*{\stretch{1}}
    \begin{itemize}
        \item optimieren die Auslieferung aus Anwendungs-Serverfarmen
        \item erhöht gleichzeitig Anwendungssicherheit:
        \begin{itemize}
            \item fungiert als Web Application Firewal (WAF)
            \item[$\Rightarrow$] Schutz vor bekannten Sicherheitsrisiken
        \end{itemize}
    \end{itemize}

    \vspace*{\stretch{1}}
\end{flashcard}

\begin{flashcard}[Definition]{Content Delivery Network (CDN)}
    \vspace*{\stretch{1}}
    \begin{itemize}
        \item Auslieferung von Inhalten mit hoher Bandbreite
        \item schnelle Auslieferung weltweit
    \end{itemize}
    \vspace*{\stretch{1}}
\end{flashcard}

\begin{flashcard}[Definition]{Blob-Speicher}
    \vspace*{\stretch{1}}
    \begin{itemize}
        \item Binary Large OBject
        \item sehr große binäre Objekte, z.\,B. Videodateien oder Bilder
        \item Schlüssel-Wert-Speicher
        \item ansprechbar über HTTP-API (Bibliotheken für viele Sprachen verfügbar)
    \end{itemize}

    \vspace*{\stretch{1}}
\end{flashcard}

\begin{flashcard}[Definition]{Disk Storage}
    \vspace*{\stretch{1}}
    \begin{itemize}
        \item Festplatten für virtuellle Maschinen, Apps, und Dienste
        \item Managed Disk: verwaltete Disk, wird von Azure erstellt und freigegeben
        \item größe beliebig, bezahlt nach Gigabytes in Zweierpotenzen
    \end{itemize}
    \vspace*{\stretch{1}}
\end{flashcard}

\begin{flashcard}[Definition]{File Storage}
    \vspace*{\stretch{1}}
    \begin{itemize}
        \item Dateifreigaben, die übe SMB eingebunden werden können
        \item Zugriff wie auf Dateiserver
        \item ermöglicht Quota
    \end{itemize}
    \vspace*{\stretch{1}}
\end{flashcard}

\begin{flashcard}[Definition]{Archivierung}
    \vspace*{\stretch{1}}
    \begin{itemize}
        \item speichern von Daten, die selten benötigt werden
        \item Zugriff langsam
        \item Zugriff teuer
    \end{itemize}
    \vspace*{\stretch{1}}
\end{flashcard}

\begin{flashcard}[\ ]{Eigenschaften von Speicherdiensten}
    \vspace*{\stretch{1}}
    \begin{itemize}
        \item dauerhafte Speicherung durch Redundanz und Replikation\newline
        Replikation auf Wunsch
        \item sichere Daten durch Verschlüsselung\newline
        entweder durch Azure, oder durch Kunden
        \item Skalierbarkeit,
        \begin{itemize}
            \item Kapazität: zwar endliche, aber für die meisten Anwendungsfälle unbegrenzte
            \item Zugriffe: kann für zahllose parallele Zugriffe eingerichtet werden
        \end{itemize}
        \item sämtliche Speicherdienste sind verwaltet, d.\,h. keine Wartung und Fehlerbehandlung (des Dienstes selbst) notwendig
        \item Zugriff über HTTP(S) weltweit möglich
    \end{itemize}
    \vspace*{\stretch{1}}
\end{flashcard}

\begin{flashcard}[Definition]{Cosmos DB}
    \vspace*{\stretch{1}}
    \begin{itemize}
        \item NoSQL-Datenbank
        \item global verteilt
    \end{itemize}
    \vspace*{\stretch{1}}
\end{flashcard}

\begin{flashcard}[Definition]{Azure SQL-Datenbank}
    \vspace*{\stretch{1}}
    \begin{itemize}
        \item verwaltete relationale Datenbank (generisch SQL)
        \item automatische Skalierung
        \item robuste Sicherheit
        \item integrierte intelligente Funktionen?
    \end{itemize}
    \vspace*{\stretch{1}}
\end{flashcard}

\begin{flashcard}[Describe]{Azure Database for MySQL}
    \vspace*{\stretch{1}}
    \begin{itemize}
        \item verwaltete relationale Datenbank (MySQL)
        \item hohe Verfügbarkeit
        \item Skalierbarkeit
        \item Sicherheit
    \end{itemize}
    \vspace*{\stretch{1}}
\end{flashcard}

\begin{flashcard}[Definition]{Azure Database for PostgreSQL}
    \vspace*{\stretch{1}}
    \begin{itemize}
        \item verwaltete relationale Datenban (PostgreSQL)
        \item hohe Verfügbarkeit
        \item Skalierbarkeit
        \item Sicherheit
    \end{itemize}
    \vspace*{\stretch{1}}
\end{flashcard}

\begin{flashcard}[Describe]{Azure Database Migration Service}
    \vspace*{\stretch{1}}
    Unterstützung beim Migrieren von Datenbanken in die Cloud

    $\Rightarrow$ keine Änderungen am Anwendungscode notwendig
    \vspace*{\stretch{1}}
\end{flashcard}

\begin{flashcard}[Definition]{Azure Marketplace}
    \vspace*{\stretch{1}}
    \begin{itemize}
        \item Sammlung von Applikationen und Services\newline
        von verschiedenen Anbietern, zertifiziert für Azure-Kompatibilität
        \begin{itemize}
            \item KI + Machine Learning
            \item Disk-Images
            \item Webanwendungen
            \item \ldots
        \end{itemize}
        \item erlaubt das Suchen, Ausprobieren, Kaufen und Bereitstellen mithilfe einer Benutzeroberfläche
        \item Unterstützt die Verbindung von Endnutzern mit
        \begin{itemize}
            \item Microsoft-Parnern
            \item unabhängige Softwareentwickler
            \item Start-Ups, die Azure-Lösungen bereitstellen
        \end{itemize}
    \end{itemize}
    \vspace*{\stretch{1}}
\end{flashcard}

\begin{flashcard}[\ ]{Marketplace Anwendungsfälle}
    \vspace*{\stretch{1}}
        \begin{itemize}
            \item Verwenden von WordPress für eine Webseite
            \item Suchen von Diensten und Lösungen für ein Problem
        \end{itemize}
    \vspace*{\stretch{1}}
\end{flashcard}

