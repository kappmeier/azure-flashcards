\subsectioncard{Describe Azure Governance Methodologies}

\begin{flashcard}[Definition]{Richtlinien mit Azure Policy}
    \vspace*{\stretch{1}}
    \begin{itemize}
        \item Richtlinien und Regeln für die Ressourcenerstellung\newline
        $\Rightarrow$ erzwingt Konformität von Ressourcen
        \item Mögliche Kriterien:
        \begin{itemize}
            \item mögliche Regionen für Ressourcenerstellung
            \item Benennungskonvention
            \item Typ von erlaubten Ressourcen
            \item Einforderung bestimmter Tags
        \end{itemize}
    \end{itemize}
    
    \vspace{5mm}
    Richtlinien beschreiben \emph{explizite Verbote}, während standardmäßíg alles zugelassen ist
    
    $\Rightarrow$ dies ist das Gegenteil von RBAC
    \vspace*{\stretch{1}}
\end{flashcard}

\begin{flashcard}[\ ]{Verwenden von Richtlinien}
    \vspace*{\stretch{1}}
    \begin{enumerate}
        \item Erstellen einer \emph{Richtliniendefinition}\newline
        z.\,B. im JSON-Format
        \item Registrierung einer Richtlinie
        \item Zuweisen der Richtliniendefinition\newline
        z.\,B. für Resourcengruppen, ein gesamtes Abbonement, \ldots
        \item Auswirkungen bei verletzten Richtlinien\newline
        z.\,B. Ressourcenerstellung abbrechen, Audit, \ldots
    \end{enumerate}
    
    \vspace{5mm}
    Ergebnisse von Richtlinien können in der \emph{Richtlinienkonformität} im Protal angeschaut werden
\end{flashcard}

\begin{flashcard}[Definition]{Initiativen}
    \vspace*{\stretch{1}}
    \begin{itemize}
        \item Initiativendefinition:\newline
        Sammlung von Richtliniendefinitionen (aus Azure Policy)
        \item Initiatives bündeln Richtlinien für ein bestimmtes Ziel, z.\,B. eine spezielle Compliance-Anforderung
        \item \emph{Initiativenzuweisung} analog zu Richtlinienzuweisung
    \end{itemize}
    
    \vspace{5mm}
    $\Rightarrow$ meistens ist es sinnvoll Initiativen anstatt Richtlinien zu verwenden:
    \begin{itemize}
        \item einfachere Verwaltung
        \item oft mehrere Richtlinien für ein Ziel notwendig
        \item sogar für einzelne Richtlinien!
    \end{itemize}

    \vspace*{\stretch{1}}
\end{flashcard}

\begin{flashcard}[Definition]{Role-Based Access Control (RBAC)}
    \vspace*{\stretch{1}}
    \begin{enumerate}
        \item Rollen: bestimmte eingeschränkte Berechtigungen (Schreibgeschützt, Mitwirkender) die einer Identität für bestimmte Azure-Dienstinstanzen zu gewiesen werden
        \item Identitäten erben Rollen über Gruppenmitgliedschaft
        \item Rollen vererben sich von höheren Stufen (gesamtes Abonnement (Subscription) über Ressourcengruppen zu einzelnen Ressourcen) etc.
        \item[$\Rightarrow$] \emph{Modell mit Zulassung}
    \end{enumerate}
    \vspace*{\stretch{1}}
\end{flashcard}

\begin{flashcard}[]{Vorteile von RBAC}
    \vspace*{\stretch{1}}
    \begin{enumerate}
        \item Trennung von Sicherheitsprinzipalen?, Zugriffsrechten und Ressourcen
        \item ermöglilcht einfache Verwaltung und differenzierte Steuerung.
        \item Sicherstellung, dass nur erforderliche \emph{Mindestberechtigung} gewährt werden
        \item[$\Rightarrow$] es sollten nie mehr Berechtigungen, als notwendig vergeben werden
    \end{enumerate}
    Fazit: genau bekannt wer auf welche Daten und Infrastruktur zugreifen kann
    \vspace*{\stretch{1}}
\end{flashcard}

\begin{flashcard}[Describe]{Locks}
    \vspace*{\stretch{1}}
    \begin{itemize}
        \item Sperre von Ressourcen um Änderungen zu verhindern (auch \emph{Löschung})
        \begin{itemize}
            \item Löschen:\newline Ressource kann verändert werden, \emph{aber nicht gelöscht}
            \item Schreibgeschützt:\newline Ressource kann \emph{nur gelesen} werden, nicht verändert\newline
            Achtung: manche Lesevorgänge sind doch Schreibvorgänge!
        \end{itemize}
        \item verschiedene Ebenen: Abonnement, Ressourcengruppe, einzelne Ressourcen
        \item muss explizit gelöscht werden, bevor Resource gelöscht/verändert werden kann
        \item nur Besitzer und Administratoren können Locks entfernen
    \end{itemize}
    
    Wird benutzt um versehentliche Löschung von Ressourcen zu verhindern. \newline
    $\Rightarrow$ nutzen, wenn Löschen weitreichende Auswirkungen auf produktionskritische Infrastruktur hat

    \vspace*{\stretch{1}}
\end{flashcard}

\begin{flashcard}[Describe]{Azure Advisor Security Assistance}
    \vspace*{\stretch{1}}

    \vspace*{\stretch{1}}
\end{flashcard}

\begin{flashcard}[Describe]{Azure Blaupausen}
    \vspace*{\stretch{1}}
    \begin{itemize}
        \item Unterstützung von Sicherheits- und Complience-Anforderungen durch Vorlagen
        \item ermöglicht wiederholbare Erstellung von Ressourcen
        \item Vorlage ermöglicht schnelle Bereitstellung von ähnlichen und konformen Umgebungen
        \item benutzbar auf allen Levels, bereits von Verwaltungsgruppen abwärts
        \item Blaupausen werden an Azure Ressource Manager übergeben um Ressourcen zu erstellen
        \item eingebaute Versionierung (direkt in Azure, im Unterschied zu Ressource-Manager-Vorlagen)
    \end{itemize}
    $\Rightarrow$ Richtlinien und Rollen auf Verwaltungsgruppen-Ebene sind der größte Vorteil von Blaupausen gegenüber Tools wie Terraform.
    Für kleinteiligere Erstellung von Ressourcen weniger geeignet

    \vspace*{\stretch{1}}
\end{flashcard} 
