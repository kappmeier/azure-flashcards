\subsectioncard{Describe Core Azure Identity Services}

\begin{flashcard}[Definition]{Authentifizierung}
    \vspace*{\stretch{1}}
    \begin{itemize}
        \item Überprüfung einer \emph{Person} oder eines \emph{Dienstes} vor Zugriff auf Ressource
        \item Anforderung von Anmeldeinformationen
        \item Nachweis, dass es tatsächlich die Person/der Dienst ist
        \item Grundlage eines Sicherheitsprinzipals-Security Principle
        \item verwaltet über Azure Active Directory
    \end{itemize}
    \vspace*{\stretch{1}}
\end{flashcard}

\begin{flashcard}[Definition]{Autorisierung}
    \vspace*{\stretch{1}}
    \begin{itemize}
        \item Überprüfung der Zugriffsebene einer authentifizierten Entität (Person oder Dienst)
        \item auf welche Daten darf zugegriffen werden
        \item welche Aktionen dürfen ausgeführt werden
        \item verwaltet über Azure Active Directory
    \end{itemize}
    \vspace*{\stretch{1}}
\end{flashcard}

\begin{flashcard}[Verstehen]{Unterschied zwischen Authentifizierung und Autorisierung}
    \vspace*{\stretch{1}}
    \begin{center}
        Authentifizierung bestimmt \emph{wer ein Nutzer ist}
        \vspace{5mm}

        vs.
        \vspace{5mm}

        Autorisierung bestimmt \emph{was ein Nutzer darf}
    \end{center}
    \vspace*{\stretch{1}}
\end{flashcard}

\begin{flashcard}[Definition]{Azure Active Directory}
    \vspace*{\stretch{1}}
    \begin{itemize}
        \item Cloudbasierter Identitätsdienst
        \item Alle Anwendungen können dieselben Anmeldeinformationen verwenden\newline
        unabhängig davon, ob \emph{lokal}, \emph{in der Cloud}, oder \emph{mobil}
        \item \emph{Active Directory-Mandant}: ein Active Directory Instanz in der Organisation
    \end{itemize}
    \vspace*{\stretch{1}}
\end{flashcard}

\begin{flashcard}[\ ]{Features von Azure Active Directory}
    \vspace*{\stretch{1}}
    \begin{itemize}
        \item Synchronisierung mit lokalen Active Directory-Instanzen oder LDAP
        \item Zugriffssteuerung zentral über Regeln/Richtlinien\newline
        siehe auch RBAC
    \end{itemize}
    \vspace*{\stretch{1}}
\end{flashcard}

\begin{flashcard}[\ ]{Dienste von Azure Active Directory}
    \vspace*{\stretch{1}}
    \begin{itemize}
        \item Authentifizierung
        \item Single Sign-On (SSO)
        \item Verwaltung von Anwendungen
        \item Business-to-Business Identitätsdienste (B2B)
        \item Business-to-Customer Identitätsdienste (B2C)
        \item Geräteverwaltung
    \end{itemize}
    \vspace*{\stretch{1}}
\end{flashcard}

\begin{flashcard}[\ ]{Azure Active Directory-Authentifizierung}
    \vspace*{\stretch{1}}
    \begin{itemize}
        \item Überprüfen der Identität
\newline(siehe Authentifizierung)
        \item Kennwortrücksetzung
        \item Multi-Factor Authentication
        \item Smart Lockout
        \item Gesperrte Kennwörter verwalten
    \end{itemize}
    \vspace*{\stretch{1}}
\end{flashcard}

\begin{flashcard}[Definition]{Azure Multi-Factor Authentication}
    \vspace*{\stretch{1}}
    \begin{itemize}
        \item Mindestens zwei Methoden zur Authentifizierung
        \item Drei Kategorien:
            \begin{itemize}
                \item etwas, das der Nutzer \emph{weiß}: Kennwort oder Antwort auf Sicherheitsfrage
                \item etwas, das der Nutzer \emph{hat}: App (Microsoft Authenticator, oder Google Authenticator, etc) oder Tokengerät
                \item persönliche \emph{Merkmale} des Nutzers: Fingerabdruck, Gesichtsscan, \ldots
        \end{itemize}
        \item Erhöhung der Sicherheit, denn: falls eine Anmeldeinformation an einen Angreifer gerät, kann er sich nicht authentifizieren.
        \item Automatisch in Azure Active Directory integriert und kombinierbar mit MFA von Dittanbietern
        \item Globaler Administrator sollte \emph{auf jeden Fall} diese Methode verwenden
    \end{itemize}
    \vspace*{\stretch{1}}
\end{flashcard}

\begin{flashcard}[\ ]{Identitäten für Dienste}
    \vspace*{\stretch{1}}
        Problem bei der Identifizierung:
        \begin{itemize}
            \item Anmeldeinformationen sind in Konfigurationsdateien enthalten.
            \item Mangelnde Sicherheitsrichtlinien ermöglichen Zugriff für jeden mit Zugriff auf die entsprechendne Systeme und Repositories
        \end{itemize}
        $\Rightarrow$ Lösung sind \emph{Identitäten für Dienste}

        Zwei Möglichkeiten in Azure:
        \begin{enumerate}
            \item Dinstprinzipale
            \item verwaltete Identitäten
        \end{enumerate}
        \vspace*{\stretch{1}}
\end{flashcard}

\begin{flashcard}[Definition]{Dienstprinzipale}
    \vspace*{\stretch{1}}
        \begin{itemize}
            \item \emph{Identität}: eine Entität, die Authentifiziert werden kann\newline
                (Sowohl Benutzer als auch Anwendungen, Server, …)
            \item \emph{Prinzipal}: eine Identität zusammen mit Rollen\newline
                (z.\,B. Administrator-Zugriff)
            \item \emph{Dienstprinzipal}: eine Identität, die von einem Dienst oder einer App verwendet wird
            \item[$\Rightarrow $] Dienstprinzipale sind keine Benutzer
            \item haben Rollen wie jede andere Identität
        \end{itemize}
    \vspace*{\stretch{1}}
\end{flashcard}

\begin{flashcard}[Definition]{Verwaltete Identität}
    \vspace*{\stretch{1}}
        \begin{itemize}
            \item Konto in der ActivevDirectory-Instanz
            \item Wird automatisch von Azure authentifiziert und verwaltet
            \item Dieses Konto kann wie jedes Azure Active Directory-Konto verwendet werden
                \begin{itemize}
                    \item z.\,B. Rollen
                \end{itemize}
        \end{itemize}
        Vorteile gegenüber Dienstprinzipalen:
        \begin{itemize}
            \item Einrichtung von Dienstprinzipal aufwändig
            \item direkt nutzbar von unterstützten Diensten
        \end{itemize}
    \vspace*{\stretch{1}}
\end{flashcard}
