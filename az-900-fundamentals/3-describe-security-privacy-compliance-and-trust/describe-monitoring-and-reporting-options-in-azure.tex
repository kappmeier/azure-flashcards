\subsectioncard{Verstehen der Überwachungs-Optionen in Azure}

\begin{flashcard}[Describe]{Azure Monitor}
    \vspace*{\stretch{1}}
    \begin{itemize}
        \item Dienst für das Sammeln, Analysieren und Reagieren auf Telemetriedaten
        \item ermöglicht Verständnis der Leistung von eigenen Anwendungen in Azure
        \item detektierung von Problemen, die auftreten können (proaktiv)
        \item Datengewinnunga aus verschiedensten Datenquellen:\newline
        Anwendung, Betriebssystem, Ressourcen, Abbonement, Active Directory, \ldots
        \item sammelt automatisch und im Hintergrund Daten und speichert in \emph{Aktivitätsprotokollen}\newline
        $\Rightarrow$ Bereitstellung im Portal als Metriken
        \item Datenabruf für spezielle Ressourcen: Webanwendungen/Azure Apps (Application Insights), Container/AKS, Virtuelle Maschinen
        \item Integration der Überwachungsdienste in Azure Service Health
        \item automatische Benachrichtigung im Problemfall
    \end{itemize}

    \vspace*{\stretch{1}}
\end{flashcard}

\begin{flashcard}[Describe]{Azure Service Health}
    \vspace*{\stretch{1}}
    \begin{itemize}
        \item Sammlung mit Anleitungen und Unterstützung bei Problemen mit Azure-Diensten
        \item Azure Status:\newline
            Globale Übersicht (d.\,h. nicht für einzelne Azure-Kunden)
        \item Service Health:\newline
            Dashboard mit Stati der genutzten Azure-Dienste, Ereignisübersicht, Regionsübersicht
        \item Ressource Health:\newline
            Diagnoseunterstützung bei Problemen von Ressourcen. Personalisiert.
    \end{itemize}

    \vspace*{\stretch{1}}
\end{flashcard}

\begin{flashcard}[Understand]{Use cases and benefits of Az Monitor and Az Service Health}
    \vspace*{\stretch{1}}

    \vspace*{\stretch{1}}
\end{flashcard}
