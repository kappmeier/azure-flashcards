\subsectioncard{Sichere Netzwerkverbindungen in Azure}

\begin{flashcard}[Definition]{Netzwerksicherheitsgruppe (NSG)}
    \vspace*{\stretch{1}}
    \begin{itemize}
        \item stellen Richtlinien bereit, wie bestimmte IaaS-Infrastruktur im Netzwerk abgesichert sein soll
        \item Integration in Azure Security Center: jede Ressource mit externem Internetzugriff soll in einer NSG sein
        \item filtern Datenverkehr (ein- und ausgehend) zu Ressourcen, z.\,B. nach IP-Adresse, Port, Protokoll, Subnetzen, \ldots
        \item öffentlicher Zugriff kann komplett gesperrt werden
        \item NSGs befinden sich innerhalb von virtuellen Netzwerken\newline
        $\Rightarrow$ schränken Kommunikation innerhalb eines virtuellen Netzwerks ein
    \end{itemize}

    \vspace*{\stretch{1}}
\end{flashcard}

\begin{flashcard}[Describe]{Application Security Groups (ASG)}
    \vspace*{\stretch{1}}
    \begin{itemize}
        \item Sicherheit auf Netzwerkebene nicht auf IP adressen, sondern Anwendungen
        \item Virtuelle Maschinen können zu Gruppen zusammengefasst werden
        \item Virtuelle Maschinen können teil mehrerer Application Security Groups sein
        \item Zugriffskontrolle basierend auf Workloads
        \item Spezifikation von Zugriffen auf Anwendungs-Ebene ermöglicht, die Anzahl der Netzwerksicherheitsgruppen zu reduzieren
    \end{itemize}
    \vspace*{\stretch{1}}
\end{flashcard}

\begin{flashcard}[Definition]{User Defined Rules (UDR)}
    \vspace*{\stretch{1}}
    \begin{itemize}
        \item zusätzliche Netzwerkrouten in einem Subnetzen
        \item benutzerdefinierte Routen können Systemrouten überschreiben
        \item benutzerdefinierte Routen haben höchste Priorität
        \item einzelne Routentabellen können mehreren virtuellen Subnetzen zugefügt werden
        \item unterstützte nächste Hops:
        \begin{itemize}
            \item virtuelles Gerät
            \item Gateway für virtuelle Netzwerke, \emph{außer ExpressRoute}
            \item keine: falls Netzwerkverkehr verworfen werden soll
            \item virtuelles Netzwerk: zum Überschreiben des Standardroutings
            \item Internet
        \end{itemize}

    \end{itemize}

    \vspace*{\stretch{1}}
\end{flashcard}

\begin{flashcard}[Definition]{Azure Firewall}
    \vspace*{\stretch{1}}
    \begin{itemize}
        \item verwalteter cloudbasierter Dienst (Firewall-as-a-Service)
        \item zustandsbehaftete Firewall
        \item viele Protokolle: RDP, SSH, FTP, \ldots
        \item Schutz für alle Protokolle auf Netzwerkebene für eingehende und ausgehende Verbindungen
        \item Anwendungsebene: ebenfalls Schutz für HTTP(S)
        \item globaler Schutz vor mehreren virtuellen Netzwerken
    \end{itemize}

    \vspace*{\stretch{1}}
\end{flashcard}

\begin{flashcard}[Beschreibung]{Schutz vor DDoS-Angriffen}
    \vspace*{\stretch{1}}
    Denial-of-Service-Angriffe: Überlastung einer (Netwzerk-)Ressource durch zahlreiche Anfragen
    \begin{itemize}
        \item nutzt Umfang und Elastizität des globalen Microsoft-Netzwerks
        \item Überwachung des Datenverkehrs im Edge-Bereich
        \item automatische Benachrichtigung eines Angriffs (in Minuten)
        \item es wird Verhindert, dass illegitimer Datenverkehr die Azure-Dienste erreicht
        \item Datenverkehr des Kunden erreicht Azure-Dienste ungehindert
    \end{itemize}
    Zwei Tarife: Basic und Starndard-Tarif ($\rightarrow$ erweiterte Schutzfunktionen)
    \vspace*{\stretch{1}}
\end{flashcard}

\begin{flashcard}[Beschreibung]{DDos Protection Standard-Tarif}
    \vspace*{\stretch{1}}
    Basic:
    \begin{itemize}
        \item ständig aktiviert
        \item Schutzmaßnahmen wie für Microsoft-Dienste (global)
    \end{itemize}

    Erweiterte Funktionen im Standard-Tarif
    \begin{itemize}
        \item optimierte Funktionen für Azure-Netzwerkressourcen
        \item einfache Aktivierung
        \item dedizierte Datenüberwachung + Machine Learning
        \item Angriffsszenarien:
        \begin{itemize}
            \item Volumenangriffe: Überlastung des Netzwerks mit Datenverkehr
            \item Protokollangriffe: Ausnutzen von Schwachstellen auf OSI Schichten 3 und 4
            \item direkte Angriffe auf Anwendungsebene (7): Angriffe auf Web-Applikationen
        \end{itemize}

    \end{itemize}
    \vspace*{\stretch{1}}
\end{flashcard}

\begin{flashcard}[\ ]{Geeignete Lösung für Sicherheit}
    \vspace*{\stretch{1}}
    \begin{itemize}
        \item mehrstufiges Sicherheitskonzept
        \item abhängig von Anforderungen (z.\,B. rechtlich, \ldots)
        \item mehrere stufen:\newline
        \begin{enumerate}
            \item sichtbar für Endkunden (im öffentlichen Netz) (z.\,B. Firewall)
            \item Netzwerksicherheit innerhalb des Accounts (z\,B. NSG)
            \item sichere Daten (z.\,B. Verschlüsselung)
        \end{enumerate}
        \item falls gewünscht, können Sicherheitskonzepte kombiniert werden
    \end{itemize}
    Immer Trade-Off zwischen Anforderungen, Sicherheit, und Produktivität
    \vspace*{\stretch{1}}
\end{flashcard}
