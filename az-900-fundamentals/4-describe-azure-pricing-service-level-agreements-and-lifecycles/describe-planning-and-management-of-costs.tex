\subsectioncard{Understand Planning and Management of Costs}

\begin{flashcard}[\ ]{Azure kaufen}
    \vspace*{\stretch{1}}
    \begin{itemize}
        \item \href{azure.com}{azure.com}\newline
        einfachste Möglichkeit für Kunden aller Größen
        \item Microsoft-Vertreter\newline
        Für Großkunden, die direkt mit Microsoft zusammenarbeiten
        \item Microsoft-Partner\newline
        erwerb von verwalteten Diensten über Partner
    \end{itemize}
    \vspace*{\stretch{1}}
\end{flashcard}

\begin{flashcard}[\ ]{Möglichkeiten Azure-Produkte und Services zu erwerben}
    \vspace*{\stretch{1}}
    \begin{itemize}
        \item Azure Portal: Aus dem Menü auswählen, Ressourcen erstellen
        \item Ressourcen können aus Cloud Shell oder Azure CLI erstellt werden
        \item Infrakstruktur automatisch erstellen, z.\,B. Terraform
    \end{itemize}

    \vspace{5mm}
    Achtung: Sobald eine Ressource erstellt wird, muss eine Verbrauchseinheit bezahlt werden
    \vspace*{\stretch{1}}
\end{flashcard}

\begin{flashcard}[\ ]{Azure Free Account}
    \vspace*{\stretch{1}}
    \begin{itemize}
        \item kostenloses Konto zum ausprobieren
        \item 12 Monate kostenloser Zugriff auf einige Azure-Produkte
        \item Guthaben für die ersten 30 Tage
        \item benötigt Microsoft- oder GitHub-Konto
    \end{itemize}

    \vspace*{\stretch{1}}
\end{flashcard}

\begin{flashcard}[Understand]{Factors affecting costs}
    \vspace*{\stretch{1}}
    \begin{itemize}
        \item Ressourcentyp
        \item Dienste
        \item Standord
        \item Abrechnungszone
    \end{itemize}

    \vspace*{\stretch{1}}
\end{flashcard}

\begin{flashcard}[\ ]{Einfluss des Ressourcentyps auf Kosten}
    \vspace*{\stretch{1}}
    \begin{itemize}
        \item Kosten sind ressourcenspezifisch\newline
        Ressourcentyp bestimmt preis pro Verbrauchseinheit
        \item werden Bestimmt durch die Anzahl der Verbrauchseinheiten
        \item und der Nutzungsart
    \end{itemize}

    \vspace*{\stretch{1}}
\end{flashcard}

\begin{flashcard}[\ ]{Kosteneinfluss von Diensten}
    \vspace*{\stretch{1}}
    \begin{itemize}
        \item unterscheien sich abhängig vom Konto
        \begin{itemize}
            \item Enterprise
            \item Web Direct
            \item CSP
        \end{itemize}
        \item vorhandene Kontingente
        \item verschiedene Abrechnungsstrukturen für Drittanbieterdienste
    \end{itemize}
    \vspace*{\stretch{1}}
\end{flashcard}

\begin{flashcard}[\ ]{Einfluss des Standorts auf Kosten}
    \vspace*{\stretch{1}}
    \begin{itemize}
        \item Variierende Kosten je nach Datencenter wegen
        \begin{itemize}
            \item Nachfrage
            \item lokalen Infrastrukturkosten
            \item Beliebtheit
        \end{itemize}
    \end{itemize}

    \vspace*{\stretch{1}}
\end{flashcard}

\begin{flashcard}[\ ]{Einfluss von ausgehendem Datenverkehr}
    \vspace*{\stretch{1}}
    Daten, die aus Azure-Datencentern fließen, werden entsprechend der Abrechungszone abgerechnet

    \vspace{5mm}
    \emph{Meistens} sind 5 Gigabypte \emph{pro Zone} kostenfrei
    \vspace*{\stretch{1}}
\end{flashcard}

\begin{flashcard}[\ ]{Kosten von eingehendem Datenverkehr}
    \vspace*{\stretch{1}}
    Daten, die in Azure-Datenzenter fließen sind kostenlos.
    \vspace*{\stretch{1}}
\end{flashcard}

\begin{flashcard}[Definition]{Abrechnungszonen}
    \vspace*{\stretch{1}}
    \begin{itemize}
        \item geografische Gruppierung von Azure-Regionen
        \item \emph{nur} zu Abrechnungszwecken
        \item wird für ausgehenden Datenverkehr genutzt
        \item üblicherweise 5 Gigabyte pro Monat kostenlos
    \end{itemize}
    Nicht zu verwechseln mit \emph{Verfügbarkeitszonen}

    \begin{tabular}{l|lll}
        Zone        &  Bereich\\
        \hline
        Zone 1      &  Nordamerika, US Government, UK\\
                    & Europa, Kanada, Frankreich, Schweiz\\
        Zone 2      &  Asien (Osten, Südosten), Japan, Australien, Indien, Südkorea\\
        Zone 3      &  Brasilien, Südafrika, VAE\\
        Deutschland & Deutschland\\
    \end{tabular}

    \vspace*{\stretch{1}}
\end{flashcard}

\begin{flashcard}[Understand]{Preisrechner}
    \vspace*{\stretch{1}}
    \begin{itemize}
        \item Web-Werkzeug zur Kostenschätzung von Azure-Diensten
        \item erlaubt verschiedene Eigenschaften und Optionen auszuwählen
        \item unterstützt verschiedene Standorte
        \item Auswahl des Azure-Tarifs (Free, Basic, \ldots)
        \item Supportoptionen
    \end{itemize}

    \vspace*{\stretch{1}}
\end{flashcard}

\begin{flashcard}[Understand]{Total Cost of Ownership (TCO) calculator}
    \vspace*{\stretch{1}}
    \begin{itemize}
        \item Tool zum Vorhersagen der Gesamtkosten einer existierenden Umgebung in Azure
        \item Daten der existierenden lokalen Struktur als Eingabe
        \item Benötigt spezifikation von
        \begin{itemize}
            \item Server
            \item Datenbanken
            \item Speicher
            \item Netzwerk
        \end{itemize}
        \item Konfiguration von zahlreichen Parametern:\newline
        Strom, Speicher, IT, Software, Hardware, Datencenter, \ldots
    \end{itemize}

    \vspace*{\stretch{1}}
\end{flashcard}

\begin{flashcard}[Understand]{Best practices for minimizing Azure costs}
    \vspace*{\stretch{1}}

    \vspace*{\stretch{1}}
\end{flashcard}

\begin{flashcard}[Understand]{Performing cost analysis}
    \vspace*{\stretch{1}}

    \vspace*{\stretch{1}}
\end{flashcard}

\begin{flashcard}[Understand]{Creating spending limits, quotas}
    \vspace*{\stretch{1}}

    \vspace*{\stretch{1}}
\end{flashcard}

\begin{flashcard}[Understand]{Tags to identify cost owners}
    \vspace*{\stretch{1}}

    \vspace*{\stretch{1}}
\end{flashcard}

\begin{flashcard}[Understand]{Azure reservations}
    \vspace*{\stretch{1}}

    \vspace*{\stretch{1}}
\end{flashcard}

\begin{flashcard}[Understand]{Azure Advisor recommendations}
    \vspace*{\stretch{1}}

    \vspace*{\stretch{1}}
\end{flashcard}

\begin{flashcard}[Understand]{Azure Cost Management}
    \vspace*{\stretch{1}}

    \vspace*{\stretch{1}}
\end{flashcard}
