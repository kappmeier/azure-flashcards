\subsectioncard{Verstehen des Azure Service-Lebenszyklus}

\begin{flashcard}[Definition]{Öffentliche and private Previewfeatures}
    \vspace*{\stretch{1}}
    \begin{itemize}
        \item Betaversionen oder Vorabversionen zum Testen
        \item private Vorschau: nur für einge Kunden verfügbar\newline
        benötigt Einladung
        \item öffentliche Vorschau: für alle Kunden verfügbar
        \item eingeschränkte Gechäftsbedingungen: kein Service oder keine Verfügbarkeitsgarantie
        \item Das Azure-Portal hat auch eine Vorschau\newline
        (\href{https://preview.portal.azure.com/}{https://preview.portal.azure.com/})
    \end{itemize}
    \vspace*{\stretch{1}}
\end{flashcard}

\begin{flashcard}[\ ]{Benutzung von öffentlichen Previewfeatures}
    \vspace*{\stretch{1}}
    \begin{itemize}
        \item Im Azure-Portal: Ressourcen mit namen ``Preview''
        \item RSS-Feed
        \item Webseite: \href{https://azure.microsoft.com/services/preview/}{https://azure.microsoft.com/services/preview/}
    \end{itemize}

    \vspace{5mm}
    Feedback: Über das Portal
    \vspace*{\stretch{1}}
\end{flashcard}

\begin{flashcard}[Definition]{General Availability (GA)}
    \vspace*{\stretch{1}}
    Produkte und Dienste die mit Support für alle Kunden verfügbar sind
    \vspace*{\stretch{1}}
\end{flashcard}

\begin{flashcard}[\ ]{Wie können Produktänderungen und Features verfolgt werden}
    \vspace*{\stretch{1}}
    \begin{itemize}
        \item Webseite \href{https://azure.microsoft.com/updates/}{https://azure.microsoft.com/updates/}
        \item Eintrag ``Neuheiten'' im Azure-Portal
    \end{itemize}

    \vspace*{\stretch{1}}
\end{flashcard}
