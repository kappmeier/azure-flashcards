\subsectioncard{Beschreibe Vereinbarungen zum Servicelevel für Azure (SLAs)}

\begin{flashcard}[Definition]{Vereinbarungen zum Servicelevel (SLA)}
    \vspace*{\stretch{1}}
    \begin{itemize}
        \item formales Dokument/Vertrag mit spezifischen Bedinungen/Verpflichtungen, die Microsoft eingeht
        \item Leistungsstandards, die eingehalten werden
        \item spezifisch für einzelne Azure-Produkte und -Dienste
        \item Bedingungen für nicht einhalten der Vereinbarung
        \item
            \begin{itemize}
                \item Leistungsziele
                \item Betriebszeit und Konnektivität
                \item Dienstguthaben
            \end{itemize}

    \end{itemize}
    Keine SLAs für Free- und Shared-Tarife.
    \vspace*{\stretch{1}}
\end{flashcard}

\begin{flashcard}[\ ]{Auswirkungen der Betriebszeit}
    \vspace*{\stretch{1}}
    \begin{tabular}{l|lll}
        SLA      &  Woche  & Monat  & Jahr    \\
        \hline
        99\%     &  1,68 h & 7,2 h  & 3,65 d  \\
        99,9\%   &  10,1 M & 43,2 M & 8,76 h  \\
        99,95\%  &  5 M    & 21,6 M & 4,38 h  \\
        99,99\%  &  1,01 M & 4,32 M & 52,56 M \\
        99,999\% &  6 s    & 25,9 s & 5,26 M  \\
    \end{tabular}

    \vspace{5mm}
    Wenn die Betriebszeit in den Servicelevel-Vereinbarungen zu klein ist, kann sie schwer garantiert werden
    \vspace*{\stretch{1}}
\end{flashcard}

\begin{flashcard}[\ ]{Zusammengesetzte SLAs}
    \vspace*{\stretch{1}}
    \begin{itemize}
        \item Kombination von mehreren Azure-Diensten
        \item kann je nach Architektur \emph{niedrigere} oder \emph{höhere} Betriebszeiten ergeben
    \end{itemize}
    ``UND''
    \begin{itemize}
        \item eine Anwendung benötigt zwei (oder mehr) Dienste zusammen
        \item zusammengesetzte Fehlerwahrscheinlichkeit \emph{höher}
    \end{itemize}
    ``ODER''
    \begin{itemize}
        \item eine Anwendung benötigt mindestens einer von zwei (oder mehreren) Diensten
        \item zusammengesetzte Fehlerwahrscheinlichkeit \emph{niedriger}
        \item teurer und komplexere Architektur
    \end{itemize}
    \vspace*{\stretch{1}}
\end{flashcard}

\begin{flashcard}[Beschreibe]{Ein angemessenes SLA für eine Anwendung bestimmen}
    \vspace*{\stretch{1}}
        \begin{itemize}
            \item bestimmen der Anforderungen an die Anwendung
            \item die passenden Azure-Dienste auswählen, sollten keine kleinere Verfügbarkeit haben als das angestrebte Ziel
            \item höhere Verfügbarkeit erfordert komplexere Systeme\newline
            Verfügbarkeitsklasse 4 (99,99\%) nicht mit manuellem Eingriff möglich
            \item Zeitfenster für Betriebszeit enorm wichtig\newline
            $\Rightarrow$ stündliche oder tägliche Betriebszeit führt zu unerreichbar kleinen Toleranzen
        \end{itemize}
    \vspace*{\stretch{1}}
\end{flashcard}
